\documentclass{llncs}

% -----------------------------------------------------------------------------
\begin{document}
\title          {Data Parallel Data Flow with Datum \\
                        System Demonstration }
\author         {Ben Lippmeier and Fil Mackay}
\institute      { Vertigo Technology \\
                  \email{ \{benl, fil\}@vergo.co }}
\maketitle


% -----------------------------------------------------------------------------
\begin{abstract}
Repa Flow is a library for data parallel data flow programming in Haskell. A flow is a bundle of independent streams, and the library provides operators such as map, fold and filter that apply to all streams in a bundle. Data parallelism is introduced by evaluating each stream in a separate thread on a multi-core machine. Like a souped-up version of the Haskell conduit or pipes library, Repa Flow adds support for efficient chunked streams of unboxed data; bucketed files; and analytic operators such as the SQL-like groupBy and the Hadoop-like shuffle. Repa Flow uses three separate array fusion methods to gain good numeric performance, all while maintaining a pleasant user-facing API.
\end{abstract}


% -----------------------------------------------------------------------------
%!TEX root = ../Main.tex
\section{Introdution}

The Haskell library ecosystem is blessed with a multitude of libraries for writing streaming data flow programs. Stand out examples include iteratee CITE, enumerator CITE, conduit CITE and pipes CITE. These libraries are based around ... and more recent examples such as pipes provide a useful set of algebraic equivalences that give a clean mathematical structure to the provided mathemetical structure.

Libraries such as iteratee and enumerator are typically used to deal with data sets that do not fit in main memory, as the constant space guarantee ensures that the program will run to completion without suffering an out-of-memory error. However, current computing platforms use multi-core processors, the programming models provided by such streaming libraries do not also provide a notion of \emph{parallelism} to help deal with the implied amount of data. They also lack support for branching data flows where produced streams can be consumed by several consumers without the programmer needing to had fuse them.

We provide several techniques that increase the scope of programs that can be written in such libraries. Our target applications concern \emph{medium data}, meaning data that is large enough that it does not fit in the main memory of a normal desktop machine, but not so large that we require a cluster of multiple physical machines. For a lesser amount of data one could simply load the data into main memory and use an in-memory array library such as CITE or CITE. For greater data one needs to turn to a distributed system such as Hadoop or Spark and deal with the unreliable network and lack of shared memory. Repa Flow targets the sweet middle ground.

We make the following contributions:

\begin{itemize}
\item Our parallel data flows consist of a bundle of streams, where each stream can process a separate partition of a data set on a separate processor core.

\item Our API uses polarised flow endpoints (@Sources@ and @Sinks@) to ensure that programs run in constant space. We demonstrate how this standard technique can be extended to branching data flows, where produced flows are consumed by multiple consumers.

\item The data processed by our streams is chunked so that each operation processes several elements at a time. We show how to design the core API in a generic fashion so that chunk-at-a-time operators can interoperate smoothly with element-at-a-time operators.
\TODO{We don't support leftovers}

\item We show how to use Continuation Passing Style to provoke the Glasgow Haskell Compiler into applying stream fusion across chunks processed by independent flow operators. For example, the map-map fusion on flows arises naturally from map-map fusion rule on chunks (arrays) of elements.
\end{itemize}

Our work is embodied in Repa Flow, which is available on Hackage. \TODO{Specify the relationship to previous work on Repa}. This is a new layer on the original delayed arrays of our original Repa library.




\end{document}

