\documentclass{llncs}

% -----------------------------------------------------------------------------
\begin{document}
\title          {Data Parallel Data Flow with Datum \\
                        System Demonstration }
\author         {Ben Lippmeier and Fil Mackay}
\institute      { Vertigo Technology \\
                  \email{ \{benl, fil\}@vergo.co }}
\maketitle


% -----------------------------------------------------------------------------
\begin{abstract}
Repa Flow is a library for data parallel data flow programming in Haskell. A flow is a bundle of independent streams, and the library provides operators such as map, fold and filter that apply to all streams in a bundle. Data parallelism is introduced by evaluating each stream in a separate thread on a multi-core machine. Like a souped-up version of the Haskell conduit or pipes library, Repa Flow adds support for efficient chunked streams of unboxed data; bucketed files; and analytic operators such as the SQL-like groupBy and the Hadoop-like shuffle. Repa Flow uses three separate array fusion methods to gain good numeric performance, all while maintaining a pleasant user-facing API.
\end{abstract}


% -----------------------------------------------------------------------------

\section{Introdution}

This is some stuff.



\end{document}

