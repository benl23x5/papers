
\documentclass[preprint]{sigplanconf}
\usepackage{pdf14}
\usepackage{graphicx}
\usepackage{alltt}
\usepackage{style/code}
\usepackage{style/utils}

% % -----------------------------------------------------------------------------
\begin{document}
\title
{       Polarized Data Parallel Data Flow}
 
\authorinfo
{       Ben Lippmeier$^\alpha$ \and 
        Fil Mackay$^\beta$ \and
        Amos Robinson$^\gamma$
}
{ 
  \vspace{5pt}
  \shortstack{
    $^{\alpha,\beta}$Vertigo Technology \\
    \\[2pt]
    \textsf{\{benl,fil\}@vergo.co}
  }
  \shortstack{
    ~~~$^{\alpha,\gamma}$UNSW Australia \\
    \\[2pt]
    ~~~~~~ \textsf{\{benl,amosr\}@cse.unsw.edu.au}
  }
  \shortstack{
    $^\gamma$Ambiata \\
    \\[2pt]
    ~~~~~ \textsf{amos.robinson@ambiata.com}
  }
}

\maketitle
\makeatactive

% -----------------------------------------------------------------------------
\begin{abstract}
We present an approach to writing fused data parallel data flow programs where the library API guarantees that the client programs run in constant space. Our constant space guarantee is achieved by observing that binary stream operators can be provided in several \mbox{\emph{polarity versions}}. Each polarity version uses a different combination of stream sources and sinks, and some versions allow constant space execution while others do not. Our approach is embodied in the Repa Flow Haskell library, which we are currently using for production workloads at Vertigo.
\end{abstract}

% -----------------------------------------------------------------------------
%!TEX root = ../Main.tex
\section{Introdution}

The Haskell library ecosystem is blessed with a multitude of libraries for writing streaming data flow programs. Stand out examples include iteratee CITE, enumerator CITE, conduit CITE and pipes CITE. These libraries are based around ... and more recent examples such as pipes provide a useful set of algebraic equivalences that give a clean mathematical structure to the provided mathemetical structure.

Libraries such as iteratee and enumerator are typically used to deal with data sets that do not fit in main memory, as the constant space guarantee ensures that the program will run to completion without suffering an out-of-memory error. However, current computing platforms use multi-core processors, the programming models provided by such streaming libraries do not also provide a notion of \emph{parallelism} to help deal with the implied amount of data. They also lack support for branching data flows where produced streams can be consumed by several consumers without the programmer needing to had fuse them.

We provide several techniques that increase the scope of programs that can be written in such libraries. Our target applications concern \emph{medium data}, meaning data that is large enough that it does not fit in the main memory of a normal desktop machine, but not so large that we require a cluster of multiple physical machines. For a lesser amount of data one could simply load the data into main memory and use an in-memory array library such as CITE or CITE. For greater data one needs to turn to a distributed system such as Hadoop or Spark and deal with the unreliable network and lack of shared memory. Repa Flow targets the sweet middle ground.

We make the following contributions:

\begin{itemize}
\item Our parallel data flows consist of a bundle of streams, where each stream can process a separate partition of a data set on a separate processor core.

\item Our API uses polarised flow endpoints (@Sources@ and @Sinks@) to ensure that programs run in constant space. We demonstrate how this standard technique can be extended to branching data flows, where produced flows are consumed by multiple consumers.

\item The data processed by our streams is chunked so that each operation processes several elements at a time. We show how to design the core API in a generic fashion so that chunk-at-a-time operators can interoperate smoothly with element-at-a-time operators.
\TODO{We don't support leftovers}

\item We show how to use Continuation Passing Style to provoke the Glasgow Haskell Compiler into applying stream fusion across chunks processed by independent flow operators. For example, the map-map fusion on flows arises naturally from map-map fusion rule on chunks (arrays) of elements.
\end{itemize}

Our work is embodied in Repa Flow, which is available on Hackage. \TODO{Specify the relationship to previous work on Repa}. This is a new layer on the original delayed arrays of our original Repa library.


%!TEX root = ../Main.tex

\clearpage
\section{Streams and Flows}

Our library is based around polarised streams. A stream is the usual list like structure which allows the next element to be taken from the front, but does not support random access of other elements. We imagine a stream as an array where the indexing dimension is time --- as each element is read it exists only for a moment, then is gone. In the library we manipulate stream endpoints rather than the streams themselves. The endpoints are polarised, meaning that we push (write) data in to stream sinks, but pull (read) data from stream sources. We name a \emph{bundle} several related streams a \emph{flow}. Typically, a flow consists of streams that carry separate partitions of a single large data set. Fig \ref{f:GenericFlows} shows the types we use to represent the the bundle of stream endpoints for an overall flow.

In the type @Sources i m e@, the @i@ parameter stands for type that indexes the individual streams, @m@ is a monadic constructor that sets the computational fabric, and @e@ is the type of elements pushed to or pulled from the endpoints. The paramters of the @Sinks@ constructor are similar. 

Mention conduit and pipes, manpiulates whole streams.

\begin{figure}
\begin{code}
data Sources i m e 
   = Sources { arity :: i
             , pull  :: i -> (e -> m ()) -> m () 
                                         -> m () }
data Sinks   i m e 
   = Sinks   { arity :: i
             , push  :: i -> e -> m ()
             , eject :: i -> m () }
\end{code}
\label{f:GenericFlows}
\caption{Generic Flow definitions and conversions}
\end{figure}

\begin{figure}
\begin{code}
zipWith_ii :: Monad m => (a -> b -> c)
           -> Sources i m a -> Sources i m b -> m (Sources i m c)
zipWith_ii f (Sources nA pullA) (Sources nB pullB)
 = return \$ Sources (min nA nB) pullC
 where  pullC i eatC ejectC
         = pullA i eatA ejectC
         where  eatA xA = pullB i eatB ejectC
                 where  eatB xB = eatC (f xA xB)

zipWith_io :: (Ord i, Monad m) => (a -> b -> c)
           -> Sinks i m c -> Sources i m a -> m (Sinks i m b)
zipWith_io f (Sinks nC pushC ejectC) (Sources nA pullA)
 = return \$ Sinks nB pushB ejectC
 where  nB = min nC nA
        pushB i xB 
         | i > nB       = return ()
         | otherwise    = pullA i eatA (ejectC i)
         where  eatA xA = pushC i (f xA xB)
\end{code}
\caption{Continuation style implementation of zipWith functions}
\end{figure}

\begin
{figure*}
\begin{code}
-- Conversion
fromList   :: i -> [a] -> m (Sources i m a)
toList1    :: i -> Sources i m a -> m [a]

fromLists  :: [[a]] -> m (Sources Int m a)
toLists    :: Sources Int m a -> m [[a]]

-- Computation
drainS     :: Sources i   m  a -> Sinks i   m  a -> m  ()
drainP     :: Sources Int IO a -> Sinks Int IO a -> IO ()

-- Mapping
map_i      :: (a -> b) -> Sources i m a  -> m (Sources i m b)
map_o      :: (b -> a) -> Sinks   i m b  -> m (Sinks   i m a)

zipWith_ii ::  (a -> b -> c) 
                  -> Sources i m a -> Sources i m b -> m (Sources i m c)
zipWith_io :: ... -> Sources i m a -> Sinks   i m c -> m (Sinks   i m b)
zipWith_oi :: ... -> Sinks   i m a -> Sources i m b -> m (Sinks   i m a)

-- Connection
dup_oo     ::        Sinks   i m a -> Sinks   i m a -> m (Sinks   i m a)
dup_io     ::        Sources i m a -> Sinks   i m a -> m (Sinks   i m a)
dup_oi     ::        Sinks   i m a -> Sources i m a -> m (Sinks   i m a)

connect_i  ::        Sources i m a -> m (Sources i m a, Sources i m a)

-- Projection
project_i  :: i ->   Sources i m a -> m (Sources () m a)
project_o  :: i ->   Sinks   i m a -> m (Sinks   () m a)

-- Funneling
funnel_i   ::        Sources i m a -> m (Sources () m a)
funnel_o   ::        Sinks  () m a -> m (Sinks   i  m a)

-- Elided constraints: (Monad m, States i m) => ...
\end{code}
\caption{Generic Flow operators}
\end{figure*}



%!TEX root = ../Main.tex
\clearpage{}
\section{Chunked streams}

\begin{itemize}
\item Chunked streams to reduce overhead.
\end{itemize}

%!TEX root = ../Main.tex
\section{Related Work}
Our work is embodied in Repa Flow, which is available on Hackage. Repa Flow is a new layer on top of the existing Repa library for delayed arrays~\cite{Lippmeier:Guiding}, and performs fusion via the GHC simplifier rather than using a custom program transformation based on series expressions~\cite{Lippmeier:DataFlow} as in our prior work.

Our system ensures that programs run in constant space, without requiring buffering or backpressure as in Akka~\cite{github:akka} and Heron~\cite{Kulkarn:Heron}. Our stream programs are fused into nested loops that read the input data from the source files, process it, and write the results immediately without blocking. The only intermediate space required is for aggregators --- for example if we had a stream of text and were counting the number of occurrences of each word we would need a map of words to the number of occurrences. 

We use stream processing to deal with large data sets that do not fit in memory. Real time streaming applications, such as to process click streams generated from websides, are the domain of synchronous data flow languages such as Lucy~\cite{Mandel:Lucy}, and reactive stream processing systems such as Heron~\cite{Kulkarn:Heron} and S4~\cite{Neumeyer:S4}. Lucy uses a clock analysis to determine where (finite) buffers must be introduced into the data flow graph. Heron and S4 use fixed size buffers with runtime back-pressure to match differing rates of production and consumption.

The main difference between Repa Flow and Iteratee based Haskell libraries 
\cite{Kiselyov:iteratee, hackage:enumerator, hackage:conduit, hackage:pipes} is that Repa Flow uses the separate @Sources@ and @Sinks@ types to express the \emph{endpoints} of flows, whereas an Iteratee is better thought of as a \emph{computation} as it is given a monadic interface. The advantage of the @Iteratee@ approach is that pleasing algebraic identities arise between iteratee computations. The disadvantage is that consuming data from two separate sources is awkward because each source is represented by its own monadic computation, and multiple computations must be layered using monad transformers. Repa Flow lacks the convenience of a uniform monadic interface, though writing programs that deal with many sources and sinks is straightforward by design.

The idea that parallelism can be introduced into a data flow graph via a single operator is well known in the databases community. The Volcano~\cite{Graefe:Volcano} parallel database inserts an @exchange@ operator into its query plans, which forks a child thread for the producer of some data, leaving the master thread as the consumer. The implementation of @exchange@ also introduces buffering and uses back pressure to handle mismatch between rates of production and consumption. In Repa Flow we use @drainP@ to introduce parallelism, and @drainP@ itself introduces no extra buffering. In Volcano and other database systems, communication between operators is performed with a uniform @open@, @next@, @close@ interface, similar to a streaming file API. In Repa Flow the API between operators consists of the @Sources@ and @Sinks@ type, where the next element in a given stream can be uniformly acquired via the @pull@ function.

In itself the duality between source and sink, push and pull, is folklore, and has previously been used for code generation in array processing languages~\cite{Claessen:ExpressiveArray,Svensson:Defunctionalizing} and XML processing pipelines~\cite{Kay:YouPull}. More recently, Bernardy and Svenningsson describe a library~\cite{Bernardy:Duality} that defines streams with sources and sinks, where each is defined as if it were the logical negation of the other. They also define co-sources and co-sinks, where a co-source is a sink that accepts element consumers and a co-sink is a source that produces element consumers. In related work Bernardy~\emph{et al} describe a core calculus \cite{Bernardy:Composable} based on Linear Logic which guarantees fusion does not increase the cost of program execution. The system is based fundamentally around linear logic rather than lambda calculus, with evaluation being driven by cut elimiation rather than function application. They describe a compiler targeting C and encouraging benchmark results.




% -----------------------------------------------------------------------------
\bibliographystyle{plain}
\bibliography{Main}


\end{document}
