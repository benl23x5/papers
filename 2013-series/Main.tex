
\documentclass[preprint]{sigplanconf}
\usepackage{pdf14}
\usepackage{version}
\usepackage{amssymb}
\usepackage{graphicx}
\usepackage{amsmath}
\usepackage{mathptmx}
\usepackage{hyperref}
\usepackage{alltt}
\usepackage{url}
\usepackage{float}
\usepackage{style/utils}
\usepackage{style/proof}
\usepackage{style/code}


% -----------------------------------------------------------------------------
\begin{document}

\title	{Data Flow Fusion with Series Expressions in Haskell}

\authorinfo{ 
  Ben Lippmeier$^\dagger$
  \and Manuel M. T. Chakravarty$^\dagger$
  \and Gabriele Keller$^\dagger$ 
  \and Amos Robinson$^\dagger$ 
}{
  \vspace{5pt}
  \shortstack{
    $^\dagger$Computer Science and Engineering \\
    University of New South Wales, Australia \\[2pt]
    \textsf{\{benl,chak,keller,amosr\}@cse.unsw.edu.au}
  }
}

\exclusivelicense
\conferenceinfo{Haskell~'13}{September 23--24, 2013, Boston, MA, USA}
\copyrightyear{2013} 
\copyrightdata{978-1-4503-2383-3/13/09} 
\doi{2503778.2503782} 
\maketitle
\makeatactive


% -----------------------------------------------------------------------------
\begin{abstract}
Existing approaches to array fusion can deal with straight-line producer consumer pipelines, but cannot fuse branching data flows where a generated array is consumed by several different consumers. Branching data flows are common and natural to write, but a lack of fusion leads to the creation of an intermediate array at every branch point. We present a new array fusion system that handles branches, based on Waters's series expression framework, but extended to work in a functional setting. Our system also solves a related problem in stream fusion, namely the introduction of duplicate loop counters. We demonstrate speedup over existing fusion systems for several key examples.
\end{abstract}

\category
	{D.3.3}
	{Programming Languages}
	{Language Constructs and Features---Concurrent programming structures; Control structures; Abstract data types}

\keywords
	Arrays; Fusion; Haskell


% -----------------------------------------------------------------------------

\section{Introdution}

This is some stuff.

%!TEX root = ../Main.tex

\clearpage
% ---------------------------------------------------------
\section{Streams and Flows}

A \emph{stream} is an array of elements where the indexing dimension is time. As each element is read from the stream it is available only in that moment, and if the consumer wants to re-use the element at a later time it must save it itself. A \emph{flow} is a bundle of related streams, where each stream typically carries data from a single partition of a larger data set --- we might create a flow consisting of 8 streams where each stream carries data from a 1GB partition of a 8GB data set.

In our API we manipulate flow endpoints rather than the flows themselves. We use the following data types:

\begin{code}
data Sources i m e 
   = Sources
   { arity :: i
   , pull  :: i -> (e -> m ()) -> m () -> m () }

data Sinks   i m e 
   = Sinks   
   { arity :: i
   , push  :: i -> e -> m ()
   , eject :: i -> m () }
\end{code}

Type @Sources i m e@ classifies flow sources which can produce elements of type @e@, using some monad @m@, where the individual streams in the bundle are indexed by values of type @i@. Likewise @Sinks i m e@ classifies flow sinks which can consume elements of type @e@.

In the @Sources@ type, field @arity@ stores the number of individual streams in the bundle. To receive data from a flow producer we use the function in the @pull@ field, passing the index of type @i@ for the desired stream, an \emph{eat} function of type @(e -> m ())@ to consume an element if one is available, and an \emph{eject} function of type @(m ())@ which will be called when no more elements will ever be available for the specified stream. The pull function will then perform an @(m ())@ computation before either calling our eat or eject function, depending on whether data is available from the source.

In the @Sinks type@, field @arity@ stores the number of individual streams as before. To send data to a flow consumer we call the @push@ function, passing the stream index of type @i@, and the element of type @e@. The @push@ function will then perform an @(m ())@ computation to consume the provided element. If no more data is available we instead call the @eject@ function, passing the stream index of type @i@, and this function will perform an @(m ())@ computation to shut down the flow sink --- possibly closing files or disconnecting network sockets.

Consuming data from a @Sources@ and producing data to a @Sinks@ is synchronous, meaning that at runtime the computation will block until either an is produced or no more elements are available (when consuming); or an element is consumed or the endpoint is shut down (when producing). Finally, note that the @eject@ functions used in both the @Sources@ and @Sinks@ type are associated with a single stream only. If our flow consists of 8 streams attached to 8 separate files then ejecting a single stream will close a single file.

\subsection{Sourcing, Sinking and Draining}
Figure~\ref{f:Draining} gives the definitions of @sourceFs@, @sinkFs@ which create flow sources and sinks based on a list of files, as well as @drainP@ which pulls data from a flow source and pushes it to a sink. 

Given the definition of the @Sources@ and @Sinks@ types, writing @sourceFs@ and @sinkFs@ is straightforward. In @sourceFs@ we first open all the provided files, yielding file handles for each one, and the @pull@ function for each stream source reads data from the corresponding file handle. When we reach the end of one of the files we eject the correponding stream. In @sinkFs@ the @push@ function from the stream simply writes the provded element to the corredponding file, and the @eject@ function closes it.

The @drainP@ function takes a bundle of stream @Sources@, a bundle of stream @Sinks@, and drains all the data from each source into the corresponding sink. Importantly, @drainP@ forks a separate thread to drain each stream, making the system data parallel. We use intermediate Haskell MVars to communicate between the worker threads and the main threads, so after forking all the workers the main threads waits until they are all finished. Now that we have the @drainP@ function, we can copy a partitioned data set from one set of files to another, in parallel:

\begin{code}
 copySetP :: [FilePath] -> [FilePath] -> IO ()
 copySetP srcs dsts
  = do  ss <- sourceFs srcs
        sk <- sinkFs   dsts
        drainP ss sk
\end{code}


\begin{figure}
\begin{code}
sourceFs :: [FilePath] -> IO (Sources Int IO Char)
sourceFs names = do
 hs <- mapM (\n -> openFile n ReadMode) names
 let pulls i ieat ieject
      = do let h = hs !! i
           eof <- hIsEOF h
           if eof then hClose   h >> ieject
                  else hGetChar h >>= ieat
 return (Sources (length names) pulls)

sinkFs  :: [FilePath] -> IO (Sinks Int IO Char)
sinkFs names = do
 hs <- mapM (\n -> openFile n WriteMode) names
 let pushs  i e = hPutChar (hs !! i) e
 let ejects i   = hClose   (hs !! i)
 return (Sinks (length names) pushs ejects)

drainP :: Sources Int IO a -> Sinks Int IO a -> IO ()
drainP (Sources i1 ipull) 
       (Sinks   i2 opush oeject) = do
 let drainStream i
      = ipull i eats ejects 
      where eats   v = opush  i v >> drainStream i
            ejects   = oeject i
 let makeDrainer i = do
        mv <- newEmptyMVar 
        forkFinally (drainStream i) 
                    (\_ -> putMVar mv ())
        return mv 
 mvs <- mapM makeDrainer [0 .. min i1 i2]
 mapM_ readMVar mvs
\end{code}

\caption{Sourcing, Sinking and Draining}
\label{f:Draining}
\end{figure}


% ---------------------------------------------------------
\subsection{Stateful Streams, Branching and Linearity}

Suppose we wish to copy our set of files to \emph{two} different destination sets instead of a single one. Trying to drain the source flow into two separate sinks does not work, which is a key part of the design of our system. 

\eject 
We might try something like:

\begin{code}
 badCopyMultiple 
  :: [FilePath] -> [FilePath] -> [FilePath] -> IO ()
 badCopyMultiple srcs dsts1 dsts2
  = do  ss  <- sourceFs srcs
        sk1 <- sinksFs  dsts1
        sk2 <- sinksFs  dsts2
        drainP ss sk1
        drainP ss sk2
\end{code}

This cannot work, nor should it. The @Sinks@ endpoint created by @sourceFs@ is a stateful object -- it represents the current position in each of the source files being read. After we have applied the first @drainP@, we have already finished reading through all the source files, so draining the associated flow again does not yield more data. 

Each of the drain computations is performed in the IO monad, which enforces an explicit notion of sequence. The second drain happens after the first one. To copy the source data using two separate drain stages we would need to either rewind the file handles attached to the sources, or buffer the data read from the files in memory. Both options are bad. An object of type @Sources@ is an abstract producer of data, and in general it may not be possible to rewind it to a previous state --- suppose it was connected to a stream of sensor readings. On the other hand, buffering all the data read from the flow source so that we can write it one set of files, then another, does not work when the source contains more data than will fit in memory. Instead, we introduce an explicit combinator which branches our flow:

\begin{code}
dup_ooo :: (Ord i, Monad m)
       => Sinks i m a -> Sinks i m a -> Sinks i m a
dup_ooo (Sinks n1 push1 eject1) 
        (Sinks n2 push2 eject2)
 = let pushs  i x = push1 i x >> push2 i x
       ejects i   = eject1 i  >> eject2 i
   in  Sinks (min n1 n2) pushs ejects
\end{code}

The @dup_ooo@ combinator takes two separate sinks and creates a new one. When we push an element to the new sink it will push that element to the two original sinks. Likewise, when we eject the new sink it will eject the two original sinks. We can use this new combinator to write a working @copyMultiple@ function:

\begin{code}
copyMultipleP 
 :: [FilePath] -> [FilePath] -> [FilePath] -> IO ()
copyMultipleP srcs dsts1 dsts2
 = do  ss  <- sourceFs srcs
       sk1 <- sinkFs   dsts1
       sk2 <- sinkFs   dsts2 
       drainP ss (dup_oo sk1 sk2)
\end{code}

This new @copyMultiple@ function runs in constant space, which is the behaviour we wanted. Importantly, note that in the new function definition there is only a single occurrence of each of the variables bound to sources and sinks: @ss@, @sk1@, @sk2@. Each source and sink is used linearly. Our program expresses a data flow graph, where the functions @sourceFs@, @sinkFs@ and @dup_ooo@ create nodes in the graph, and the use-def relation of variables connects the nodes with edges.


% ---------------------------------------------------------
\subsection{Polarity and buffering}
In the previous section, @dup_ooo@ branches a flow by taking two existing sinks and produces a new one. The @_ooo@ suffix stands for ``output, output, output'', referring to the three sinks. It turns out that the converse @dup_iii@ combinator is not implementable without requiring unbounded buffering. Such a combinator would have the following type:
\begin{code}
 dup_iii :: (Ord i, Monad m)
         =>  Sources i m a 
         -> (Sources i m a, Source i m a)
\end{code}

Consider how this would work. The @dup_iii@ combinator takes an argument source and produces two result sources. Now suppose we pull data from the left result source. The combinator would need to pull from its argument source to retrieve this data, then when we pull from the right result source we want the same data. The problem is that there is nothing stopping us from pulling the entire stream via the left result source before pulling any elements from the right result source, so @dup_iii@ would need to introduce an unbounded buffer to store all the elements in the interim.

Interestingly, although @dup_iii@ cannot work without an unbounded buffer, a hybrid combinator @dup_ioi@ can. This combinator has the following definition:
\begin{code}
dup_ioi :: (Ord i, Monad m)
        => Sources i m a -> Sinks i m a 
        -> Sources i m a
dup_ioi (Sources n1 pull1) (Sinks n2 push2 eject2)
 = let pull3 i eat3 eject3
        = pull1 i eat1 eject1
        where eat1   x = eat3 x >> push2  i x
              eject1   = eject3 >> eject2 i
   in  Sources (min n1 n2) pull3
\end{code}

The @dup_ioi@ combinator takes an argument flow source, an argument flow sink, and returns a result flow source. When we pull data from the result source the combiantor pulls from its argument source and then \emph{pushes} the same data to the argument sink. Similarly to @dup_ooo@, we can use @dup_ioi@ to introduce a branch into the data flow graph \emph{without} requiring unbounded buffering of the input flow. Actually, the same approach works for any number of argument sinks, for example:
\begin{code}
dup_iooi
 :: (Ord i, Monad m)
 => Sources i m a -> Sinks i m a -> Sinks i m a
 -> Sources i m a
\end{code}

With @dup_iooi@ when we pull from the result source the combinators pulls from its argument source and pushes the same data to its argument sinks. 

Figure~\ref{f:Polarity} the various options for assigning polarities to the argument and result endpoints of a flow duplication combinator. Sources are indicated with a $\bullet$ and Sinks with a $\circ$. We use the mnemonic that the filled $\bullet$ can always produce data (being a source) while the empty $\circ$ can always accept data (being a sink). In the figure the greyed out diagrams are for polarity assignments that would require unbounded buffering, while the solid ones require no buffering.

\begin{figure}
\includegraphics[scale=0.7]{figures/polarity.pdf}

\caption{Possible polarities for the flow duplication combinator}
\label{f:Polarity}
\end{figure}

In the right of Figure~\ref{f:Polarity} we have also included the corresponding polarity diagram for the @drainP@ combinator from Figure~\ref{f:Draining}. The @drainP@ combinator requires no buffering because all data pulled from the source is immediately pushed to the sink. On the other hand, if we invert the polarities of @drainP@ we arrive at the natural assignment for a primitive buffering combinator. A combinator that consumes data via a single argument sink and produces the same data via a result source \emph{must} introduce a buffer as there is no guarantee that new elements are pushed to the sink at the same rate they are pulled from the source. 

Repa library makes it easy to guarantee that data flow programs written with it run in constant space, by only providing combinators with the polarity assignments that do so. This guarantee also requires that variables bound to sources and sinks are used linearly, which alas we cannot check with Haskell as it does not support linear types. However, linearity is an easy constraint to understand and check manually, so for now we we leave the job of enforcing it to the client programmer.

In future work would be interesting to implement a system to take a data flow graph without a polarity assignment, and assign polarities to the combinators such that the entire graph can be executed without buffering (if possible). However, so far the largest programs we have written using the library have included only 10-15 combinators in a single function, and assigning the polarities has not been a burden. In practice there is usually a natural distinction between the \emph{input network} written using sources and the \emph{output network} using sinks, and hybrid combinators such as @dup_iio@ are used infrequently.


% ---------------------------------------------------------
\subsection{Mapping, Folding and Grouping}
Map and co-map, folding, grouping and other basic combinators. Given an example that folds the input stream while passing it through to another combinator.

\subsection{Stream projection and funneling}

\begin
{figure*}
\begin{code}
-- Conversion
fromList   :: i -> [a] -> m (Sources i m a)
toList1    :: i -> Sources i m a -> m [a]

fromLists  :: [[a]] -> m (Sources Int m a)
toLists    :: Sources Int m a -> m [[a]]

-- Computation
drainS     :: Sources i   m  a -> Sinks i   m  a -> m  ()
drainP     :: Sources Int IO a -> Sinks Int IO a -> IO ()

-- Mapping
map_i      :: (a -> b) -> Sources i m a  -> m (Sources i m b)
map_o      :: (b -> a) -> Sinks   i m b  -> m (Sinks   i m a)

zipWith_ii ::  (a -> b -> c) 
                  -> Sources i m a -> Sources i m b -> m (Sources i m c)
zipWith_io :: ... -> Sources i m a -> Sinks   i m c -> m (Sinks   i m b)
zipWith_oi :: ... -> Sinks   i m a -> Sources i m b -> m (Sinks   i m a)

-- Connection
dup_oo     ::        Sinks   i m a -> Sinks   i m a -> m (Sinks   i m a)
dup_io     ::        Sources i m a -> Sinks   i m a -> m (Sinks   i m a)
dup_oi     ::        Sinks   i m a -> Sources i m a -> m (Sinks   i m a)

connect_i  ::        Sources i m a -> m (Sources i m a, Sources i m a)

-- Projection
project_i  :: i ->   Sources i m a -> m (Sources () m a)
project_o  :: i ->   Sinks   i m a -> m (Sinks   () m a)

-- Funneling
funnel_i   ::        Sources i m a -> m (Sources () m a)
funnel_o   ::        Sinks  () m a -> m (Sinks   i  m a)

-- Elided constraints: (Monad m, States i m) => ...
\end{code}
\caption{Generic Flow operators}
\end{figure*}



%!TEX root = ../Main.tex

\section{Rates and Contexts}
\label{s:RatesAndContexts}

% -----------------------------------------------------------------------------
\begin{figure}
\begin{small}
\begin{alltt}
  -- Haskell source after rate inference
  filterMax :: Vector Int -> (Vector Int, Int)
  filterMax vec
   = runSeries vec go1
   where go1 s1     = let s2    = map (+ 1) s1
                          flags = map (> 0) s2
                      in  mkSel flags (go2 s2)

         go2 s2 sel = let s4 = pack sel s2
                      in  ( create s4
                          , fold max 0 s4)

  -- With explicit type abstraction and application
  filterMax :: Vector Int -> (Vector Int, Int)
  filterMax vec
   = runSeries @Int @(Vector Int, Int) vec go1
   where go1 = \(\Lambda\)(k1 : &). \(\lambda\)(s1 : Series k1 Int). 
               let s2    : Series k1 Int   
                         = map @k1 @Int @Int  (+ 1) s1
                   flags : Series k1 Bool
                         = map @k1 @Int @Bool (> 0) s2
               in  mkSel @k1 @(Vector Int, Int)
                          flags (go2 @k1 s2)

         go2 = \(\Lambda\)(k1 : &). \(\lambda\)(s2  : Series k1 Int). 
               \(\Lambda\)(k2 : &). \(\lambda\)(sel : Sel k1 k2).
               let s4    : Series k2 Int 
                         = pack @k1 @k2 @Int sel s2
               in  ( create @k2 @Int s4
                   , fold   @k2 @Int @Int max 0 s4)
\end{alltt}
\end{small}
\caption{The \texttt{filterMax} example after rate inference}
\label{f:new-filterMax}
\end{figure}


\begin{figure}
\begin{small}
\begin{alltt}
 runSeries  :: \(\forall\)(a b : *). Vector a 
            -> (\(\forall\)(k : &). Series k a -> b) -> b
 
 runSeries2 :: \(\forall\)(a b c : *). Vector a -> Vector b
            -> (\(\forall\)(k : &). Series k a -> Series k b -> c)
            -> Maybe c

 generate   :: \(\forall\)(a : *). Int -> (Int -> a)
            -> (\(\forall\)(k : &). Series k a -> b) -> b

 create     :: \(\forall\)(k : &). \(\forall\)(a : *)
            .  Series k a -> Vector a

 map        :: \(\forall\)(k : &). \(\forall\)(a b : *)
            .  (a -> b)  
            -> Series k a -> Series k b

 map2       :: \(\forall\)(k : &). \(\forall\)(a b c : *)
            .  (a -> b -> c) 
            -> Series k a -> Series k b -> Series k c

 fold       :: \(\forall\)(k : &). \(\forall\)(a b : *)
            .  (a -> b -> a) -> a -> Series k b -> a

 mkSel      :: \(\forall\)(k1 : &). \(\forall\)(a : *)
            .  Series k1 Bool
            -> (\(\forall\)(k2 : &). Sel k1 k2 -> a)   -> a

 pack       :: \(\forall\)(k1 k2 : &). \(\forall\)(a : *)
            .  Sel k1 k2 -> Series k1 a -> Series k2 a
\end{alltt}
\end{small}
\caption{Series Operators}
\label{f:SeriesOperators}
\end{figure}

The first definition in Figure~\ref{f:new-filterMax} shows the code of @filterMax@ after performing \emph{rate inference} as a pre-processing step. Rate inference identifies regions of code amenable to flow fusion. Moreover, it decomposes rate changing operations such as @filter@, into primitives, such as @mkSel@ and @pack@. As part of this decomposition, we have also introduced two intermediate @go@ bindings to clarify data dependencies. We will explain rate inference in \S\ref{s:rate-inference}, but first we discuss the importance of rates for flow fusion. 

The second definition in Figure~\ref{f:new-filterMax} is the same function as before, but with explicit type abstractions and applications. Here $\Lambda$ indicates type abstraction and \verb|@| type application. We use @&@ as the \emph{kind of rate types}. 


% -----------------------------------------------------------------------------
\subsection{Vectors and Series}
\label{s:VectorsAndSeries}

Rate inference ensures that subexpressions subjected to flow fusion contain only the array combinators shown in Figure~\ref{f:SeriesOperators}. In those signatures, @Vector a@ is the type of \emph{manifest linear arrays}, represented by contiguous blocks of machine words in memory. In contrast, @Series k a@ is the type of \emph{abstract representations of sequences} of @a@-values produced at rate @k@, such that they may participate in fusion. In this respect a series is similar to a \emph{delayed array} from Repa~\cite{Keller:Repa}, except that series do not support random access indexing.

The \emph{rate} @k@ of a series is a type level representation of its length, with the following key invariant: \emph{two series of the same rate are guaranteed to have the same length}. We use @&@ as the kind of rate types, leaving @*@ as the kind of value types as in standard Haskell. This distinction is important because rates are purely type-level information. There are no values that have a type of kind @&@.

The rate of a series is similar to a \emph{clock type}, as used by the clock typing systems of synchronous data flow languages such as Lustre~\cite{Caspi:Lustre} and Lucid Synchrone~\cite{Caspi:kahn-networks, Pouzet:lucid-synchrone}. We compare these systems further in \S\ref{s:Related} 


% -----------------------------------------------------------------------------
\subsection{Running Series Expressions}

Rate inference encapsulates regions of fusible code in an abstraction @fn@, that is then passed as an argument to the function @runSeries@, @runSeries2@, or @generate@. The types for these functions are in Figure~\ref{f:SeriesOperators}. The application @(runSeries v fn)@ converts the vector @v@ to a @Series@, which is then consumed by @fn@. In the type of @runSeries@, the rate variable @k@ (which is a type-level representation of the length of @v@) is universally quantified in the type of @fn@. The inner quantification ensures that the rate information cannot escape, and that multiple series of differing lengths can never have the same rate variable. This is much like the use of a rank-2 type to encapsulate state in the @runST@ function of the @ST@ monad~\cite{Launchbury:state-threads}.

The inner quantification of a rate variable logically creates a \emph{rate context}, where array data is processed at the specified rate. For example, the body of @go1@ in Figure~\ref{f:new-filterMax} is a rate context in which we construct two new @Series@ values, @s2@ and @flags@, both with the same rate @k1@ derived from the incoming series @s1@. 

Flow fusion can merge all computations producing and consuming series at the same rate into a single loop. For example, fused code that performs multiple folds over multiple series at the same rate will proceed as follows: first, the code initializes an accumulator for each fold; then, in a single loop, it reads elements from each series in lock step and updates the accumulators appropriately.

The @runSeries2@ function is like @runSeries@, but accepts two @Vector@s \emph{of the same length} as inputs and converts them both to @Series@ before passing them to the worker. If the input vectors do not have the same length then @runSeries2@ yields @Nothing@. This dynamic check justifies our use of the same rate variable @k@ for both @Series@. The @generate@ function is like @runSeries@, but generating a series from a function producing array elements, instead of from a manifest vector.

Finally, @create@ materializes a @Series@ into a @Vector@. The latter is free of an associated rate variable, and hence can be passed outside the current rate context.


% -----------------------------------------------------------------------------
\subsection{Maps and Folds}

The types of @map@ and @fold@ in Figure~\ref{f:SeriesOperators} are standard, except for the inclusion of the rate variable @k@. The function @map2@ is much like Haskell's standard @zipWith@, but requires both series to have the same rate (length). In our full implementation we have an entire family of functions @map3@, @map4@ and so on.

In the explicitly typed code of Figure~\ref{f:new-filterMax}, the rate variables used as type arguments identify which rate context each operator belongs to. We will see in \S\ref{s:Normalizing} that these type arguments indicate the set of series that ought to be evaluated in a single loop. 


% -----------------------------------------------------------------------------
\subsection{Selectors and Packing}
\label{s:SelectorsAndPacking}

In the expression @(pack sel s2)@, the \emph{selector} @sel@ identifies the values in series @s2@ that should be included in the shorter result series. For example, suppose that during evaluation of @go2@ from Figure~\ref{f:new-filterMax}, we have the following values for @s2@ and @flags@, where @flags@ identifies the positive elements of @s2@:
%
\begin{code}
  s2    :: Series k1 Int
  s2    = [+4 -1 +5 +3 +8 -4 +2 +1 -5]

  flags :: Series k1 Bool
  flags = [ T  F  T  T  T  F  T  T  F]
\end{code}
%
The application @(mkSel flags fn)@ converts @flags@ into a \emph{selector}, which is passed to @fn@, just as with @runSeries@ before. The selector is an abstract representation of the vector containing the indices of all the @T@ (True) values in that series: 
%
\begin{code}
  sel :: Sel k1 k2
  sel = [0 2 3 4 6 7]
\end{code}
%
The selector is then used by @pack@ to select just those elements of @s2@ that had their corresponding flag set: 
%
\begin{code}
  s4  :: Series k2 Int
  s4  = pack sel s2  =  [+4 +5 +3 +8 +2 +1]
\end{code}

Importantly, because the selector maps a series of one length onto a series of another, we tag the selector type with two different rate variables. For @filterMax@, we have @sel :: Sel k1 k2@ where @k1@ is the rate of @s2@ and @flags@, and @k2@ is the rate of @s4@ --- the series resulting from @pack@. Additionally, because selectors are always produced from a series of flags, we know that the length (rate) of the selector is no greater than the length (rate) of the original series. To put this another way, the rate context @k2@ is \emph{contained by} the rate context @k1@, and the selector is \emph{evidence} and a \emph{witness} for this relationship. Figure~\ref{f:nested-contexts} shows this graphically.


% -----------------------------------------------------------------------------
\subsection{Data Flow Languages and Clock Calculi}
\label{s:DataFlowLanguages}

Figure~\ref{f:nested-contexts} shows that @filterMax@ is a first order, non-recursive data-flow program, as one might expect. The expressions that flow fusion can fuse all have this form: they consist of a number of manifest data sources, and a hierarchy of well nested rate contexts containing a directed acyclic graph of data flow operators terminated by manifest data sinks. Any program of this form can be completely fused by flow fusion.

The programs that we handle constitute a fragment of a more general data flow language such as Lustre~\cite{Caspi:Lustre} or Lucid Synchrone~\cite{Pouzet:lucid-synchrone}. These larger languages work over infinite streams with recursion and delay elements, and prior work on compiling them has focused on dealing with these extra features~\cite{Halbwachs:data-flow-compilation}. These languages come with clock typing systems that ensure the program can be evaluated synchronously, and without unbounded buffering. In contrast, the fragment that we compile is defined by the API of Figure~\ref{f:SeriesOperators}, which only provides finite series. We do not have delay elements or recursion. We use rate variables to express relationships between different series, but as we start with a simpler language, we can get by with a simpler rate analysis as described next.


% -----------------------------------------------------------------------------
\subsection{Rate Inference}
\label{s:rate-inference}

Rate inference identifies non-recursive data flow expressions that are amenable to flow fusion, and turns them into applications of @runSeries@. These expressions may only contain the following operators: (1) the vector operators from Figure~\ref{f:SeriesOperators} and (2) operators with scalar argument and result types. All higher-level array operators must be implemented in terms of these primitives to participate in flow fusion. Before rate inference, we assume the definitions of these composite operators have been inlined and the resulting code converted to a-normal form (administrative-normal form).

Rate inference proceeds in two phases: first, we identify, and if necessary, rearrange vector-valued subexpressions that we can fuse into single loops. For this we adapt the existing \emph{size inference and scheduling algorithm} described by Chatterjee et al~\cite{Chatterjee:size-access-inference}.

After solving the constraints as in \cite{Chatterjee:size-access-inference}, we proceed to the second phase. We rewrite the expression using operators on @Vector@s to use operators on @Series@, and wrap it in a @runSeries@. For this we replace all free vector-valued variables $\tt{v}_1 \tt{::} \tt{Vector}~ \tt{a}_1$ to $\tt{v}_n \tt{::} \tt{Vector}~ \tt{a}_n$ with fresh variables $\tt{s} \tt{::} \tt{Series}~ k~ \tt{a}_i$. The rate inference algorithm ensures that all such variables have the same rate @k@. Rewriting the expression to use series operators is mostly trivial. Only @filter@ needs special handling, to expand each occurrence into a use of @mkSel@ and @pack@. As the input code is already in ANF, @filter@ can only occur in bindings of the form:
%
\begin{code}
 let s1 = filter fn s0 
 in  e2
\end{code}  
%
which we rewrite to
%
\begin{alltt}
 let flags = map fn s0
 in  mkSel flags (\(\lambda\)sel. let s1 = pack sel s0 in e2)
\end{alltt}  
%
Finally, we wrap the series expression @e'@ obtained this way with a @runSeries@:
%
\begin{alltt}
 runSeriesN v\_1 \(\ldots\) v_n (\(\lambda\) s\_1 \ldots s\_n. e')
\end{alltt}


%!TEX root = ../Main.tex

\section{Loop Generation}
\label{s:LoopGeneration}
After rate inference, our compilation method is as follows:

\begin{enumerate}
\item   Type check and desugar Haskell source code to GHC Core.
\item   A-normalize and eta-expand intermediate code.
\item   \emph{Slurp} out a data flow graph for each series process.
\item   \emph{Schedule} the operators in this graph into an abstract loop nest.
\item   \emph{Concretize} rate variables into loop counters.
\item   \emph{Extract}    new Core code from the loop nest.
\item   Inline library functions into the extracted code.
\item   Complete compilation with GHC's standard pipeline.
\end{enumerate}

A \emph{series process} is a computation that can be expressed as a static, first order, non-recursive data flow graph like that of Figure~\ref{f:nested-contexts}. Data flow graphs are represented by the @Process@ language shown in Figure~\ref{f:Process}. Abstract loop nests are represented by the @Procedure@ language of Figure~\ref{f:Procedure}. In our current implementation stages 1, 7 and 8 are performed by GHC proper using its internal Core language, while stages 2-6 are performed by our GHC plugin. Note that the \emph{Schedule} phase (described in \S\ref{s:SchedulingLoops}) is really the core of our method, with the other phases performing impedance matching between the input and output languages. 


% -----------------------------------------------------------------------------
%!TEX root = ../Main.tex

% -----------------------------------------------------------------------------
\begin{figure}

\begin{tabbing}
MMMM           \= MM \= \kill
$name$  \> $\to$ \> (process name)      \\
$x,~ s$ \> $\to$ \> (value variable)    \\
$a,~ k$ \> $\to$ \> (type variable)
\\[2ex]

$kind$          \> \tt{::=}
                \>      $\tt{*} ~~|~~ \tt{\&}$

\\[2ex]
$type$          \> \tt{::=}
                \>      $a ~~|~~ \tt{Int} ~~|~~ \tt{Float} ~~|~~ ...$ 
\end{tabbing}

\begin{tabbing}
x           \= MM \= MMMMMM \= \kill
$process$ \\
 \> \tt{::=}     
          \> $\tt{process}~~ name~~ (k_{in} : \&)~~ \ov{(a : kind)}$ \\
 \>       \> $~~~~~~~~~~~~~~~~~~~~~~~~~~~~
                        \ov{(x : type)}~~
                        \ov{(s : \tt{Series}~~ k_{in}~~ type)}$      \\
 \>       \> ~~~ $\tt{with}~~     \ov{operator~}~~ 
                \tt{yields}~~       exp$

\\[2ex]
$operator$  \\
 \> \tt{::=}
 \> $\tt{mkSel}
                        ~~ (k_{inner} : \tt{\&})
                        ~~ (x_{sel} : \tt{Sel}~~ k_{outer}~~ k_{inner})$ \\
 \>       \> ~~~ $\tt{from}
                        ~~ k_{outer}
                        ~~ s_{flags}
                        ~~ \tt{in}
                        ~~ \ov{operator}$

 \\[1ex]
 \> $~~|$ \> $s_{out} ~~~~~\tt{<-}
                ~~~ \tt{map}^{\;n}
                        ~~ k_{in}
                        ~~ \ov{type}^{\;n}
                        ~~ exp_{work}
                        ~~ \ov{s_{in}}^{\;n}$ 

 \\[1ex]
 \> $~~|$ \> $s_{out} ~~~~~\tt{<-}
                ~~~ \tt{pack} 
                        ~~ k_{out}
                        ~~ k_{in}
                        ~~ type_{in}
                        ~~ x_{sel}      ~~ s_{in}$

 \\[1ex]
 \> $~~|$ \> $x_{result}~~  
                        \tt{<= fold}~~ k_{in}~~ type_{in}~~ type_{result}~~ s_{in}$ \\
 \>       \> \> $\tt{with}   ~~ exp_{work}
                  ~~    \tt{and}    ~~ exp_{zero}$

 \\[1ex]
 \> $~~|$ \> $x_{vec}~~~~~
                        \tt{<= create}~~ k_{in}~~ type_{in}~~ s_{in}$
\end{tabbing}

\begin{tabbing}
MMMM           \= MM \= \kill
$exp$   
 \> \tt{::=} \>  ... Haskell expressions ...
\end{tabbing} 
\caption{Data Flow Process Description}
\label{f:Process}
\end{figure}



% -----------------------------------------------------------------------------
\begin{figure*}[t]
\begin{center}
\includegraphics[scale=0.8]{figures/flow-contexts.pdf}
\end{center}
\caption{Nested Rate Contexts for \texttt{filterMax}}
\label{f:nested-contexts}
\end{figure*}


% -----------------------------------------------------------------------------
\subsection{Slurping Processes}
\label{s:Slurping}
The Slurp phase takes a normalized Core module and produces a list of fusible series processes.
\begin{code}
    slurp  :: Module -> [Process]
\end{code}

We supply the Core version of each series process to be fused as a top-level binding in the @Module@. During normalization (stage~2) the application of @runSeries@ that creates the outer-most rate context is also split from the rest of the input code and floated to top-level. The @runSeries@ function itself is implemented in an external Haskell library, and is not part of the Core program given to the loop generator. For our @filterMax@ example, we would then have a binding of the following type:

%
\begin{alltt}
  filterMax_series 
   :: \(\forall\)(k : &). Series k Int -> (Vector Int, Int)
\end{alltt}

This @filterMax_series@ function is the same as @go1@ from Figure~\ref{f:new-filterMax}, after it has been floated to top-level.

The @Process@ language represents the data flow graph for a series process directly, without admitting language features that may be supported by the source language (Haskell) but not representable as a static, first-order data flow graph. If the Core version of the series process cannot be converted to our internal @Process@ language, then the user gets a compile-time warning and the program is compiled via the fallback library discussed in \S\ref{s:Benchmarks}. An example of this is in \S\ref{s:Normalizing}. On the other hand, if we \emph{can} convert the source program to a @Process@, then we guarantee it will be completely fused.


\eject
% -----------------------------------------------------------------------------
The grammar for @Process@ is shown in Figure~\ref{f:Process}, and here is the process description for our @filterMax@ example which encodes the graph in Figure~\ref{f:nested-contexts}:

\begin{code}
process filterMax_s (k1 : &) (s1 : Series k1 Int)
 with { s2 <- map k1 Int (+ 1) s1
      ; s3 <- map k1 Int (> 0) s2
      ; mkSel (k2 : Rate) (sel : Sel k1 k2)
          from  k1 s3  in
        { s4   <- pack   k1 k2 Int sel s2
        ; vec' <= create k2 Int s4
        ; mx   <= fold k2 Int s4 with (+) and 0 } }
 yields (vec', mx)
\end{code}


A $name$ is a process name like @filterMax_s@ (where @_s@ indicates the @Process@ version of this function). We use $x$ and $s$ as (meta) value variables, and by convention use $s$ for series and $x$ for non-series variables. We use $a$ and $k$ as (meta) type variables, where $a$ indicates an element type variable and $k$ indicates a rate.

We use $type$ for element types, with the full set being defined by the host language (Haskell). Our program transformations treat element types abstractly.

A $process$ consists of its name, its type and value parameters, a list of series operators, and an expression that yields the result of the overall process. We have left $exp$ unspecified as this represents expressions from the source language --- Haskell in our case. The rates of all input series must be identical to the first type variable $k_{in}$. This is the rate of the loop that we will generate for this process.

An $operator$ can introduce a new rate context (@mkSel@), be a transformation that converts some series into another one (@map@ and @pack@), or a \emph{sink} that consumes some series and produces a non-series result (@fold@ and @create@). Our operators are explicitly typed, being applied to rate variables that describe the context of each operator, and type arguments that give the element types of each series. Each operator defines a node in the data flow graph, where the binding symbols @<-@ and @<=@ represent the edges. The bindings in an operator list are non-recursive, and variables must be bound before they are used.

The @mkSel@ construct binds the new variables $k_{inner}$ and $x_{sel}$ which are added to the environment of the enclosed list of operators. The @mkSel@ operator itself defines a new rate context $k_{inner}$, inside an outer context $k_{outer}$. It consumes a series of flags $s_{flags}$ and produces a selector $x_{sel}$. In the enclosed operator list, all new series bound at that level must have rate $k_{inner}$. In Figure~\ref{f:nested-contexts} we have drawn @mkSel@ over the dotted line separating the rate contexts @k1@ and @k2@ to indicate that this operator defines the inner context.

The @map@$^{n}$ operator combines several input series $\ov{s_{in}}$ at rate $k_{in}$ using the worker function defined by $exp_{work}$.  The $n$ variable sets the number of input series, though we write just @map@ for @map1@. 

The @pack@ operator takes a series $s_{in}$ of rate $k_{in}$ to a series $s_{out}$ of rate $k_{out}$ using the selector $x_{sel}$. 

The @fold@ operator binds a new variable $x_{result}$ of $type_{result}$ which is the result of reducing the elements of series $s_{in}$ at rate $k_{in}$ using worker function $exp_{work}$ and neutral element defined by $exp_{zero}$. The $type_{in}$ argument is the type of the elements. 

The @create@ operator binds a new variable $x_{vec}$ which is the vector created from elements of the series $s_{in}$, at rate $k_{in}$ and element type $type_{in}$. In Figure~\ref{f:nested-contexts} we have drawn the results produced by @fold@ and @create@ in square boxes to indicate that these are manifest values and not fusible series. 


% -----------------------------------------------------------------------------
\subsubsection{Scoping in Series Process Descriptions}
\label{s:Scoping}
In a process description, the series that are parameters to the process, as well as the new series bound by each operator (to the left of the @<-@ marks) can only be used as series arguments to other operators. These new series are \emph{not in scope} in the worker expressions ($exp_{work}$) and cannot be used in the expression given to @yields@. The results produced by series \emph{sinks} (to the left of the @<=@ marks) are also not in scope of the workers, but can be used in the expression given to @yields@. To put this another way, the workers may refer to environment variables defined outside the process they are contained in, but not variables bound internally in the process. These rules are needed to reject processes like the following:

\begin{code}
  process badNorm (k1 : &) (s1 : Series k1 Float)
   with { total <= fold k1 Float s1 with (+) and 0
        ; s2    <- map  k1 Int  (/ total) s1
        ; vec   <= create k1 Int s2 }
   yields vec
\end{code}

Here, because the worker function for @map@ refers to the result @total@, to evaluate this code we would need to finish computation of the @fold@ operator before embarking on the @map@. However, as we will see in \S\ref{s:SchedulingLoops}, we intend to compile each process description into a single fused loop, and this is not possible for @badNorm@. It would be possible to automatically split such \emph{compound} processes into individual parts before passing them to the scheduler defined in \S\ref{s:SchedulingLoops}, but we leave this to future work.


% -----------------------------------------------------------------------------
\subsubsection{Normalizing the Input Core Program }
\label{s:Normalizing}
The Core version of a series process is easy to slurp to the @Process@ language of Figure~\ref{f:Process}, provided we make use of the explicit type annotations in Core, and perform some appropriate normalizations beforehand.

Before conversion, we a-normalize and eta-expand the definitions in the module so that every intermediate series has an explicit name. These names are identified with the edges of the extracted data-flow graph. We also use a preparation transform to force worker functions to be floated into their use sites --- so that combinators like @mkSel@, @map@ and @fold@ are directly applied to workers. For example, as part of the @filterMax@ example we get the following Core snippet:
%
\begin{alltt}
  mkSel @k1 @(Vector Int, Int) s1
    (\(\Lambda\)(k2 : &). \(\lambda\)(sel : Sel k1 k2). 
     let s4 : Series k2 Int
            = pack @k1 @k2 @Int sel s2
     in ...)
\end{alltt}

With the above code we already have all the information we need to produce the equivalent snippet of the @Process@ language:
%
\begin{alltt}
  mkSel (k2 : &) (sel : Sel k1 k2) 
       from k1 s1 in 
    \{ s4  <= pack k1 k2 Int sel s2
      ... \}
\end{alltt}

Although the example above is really just a change of syntax, as mentioned in \S\ref{s:Slurping} the real point is that the @Process@ language is smaller than the input Core language. We use the intermediate @Process@ language primarily as way to reject features of the input language that cannot be expressed as static, first order, non recursive data flow graphs. For example, the following function cannot be converted to a @Process@ because we have no way to represent the inner @if@ construct. 

\begin{alltt}
 badSwitchy :: \(\forall\)(k : &). Bool 
            -> Series k Int -> Series k Int -> Int
 badSwitchy flag s1 s2
  = fold (+) 0 (if flag then (map (* 2) s1) 
                        else (map (* 4) s2)
\end{alltt}

The above function does not express a \emph{static} data flow graph, because we do not statically know which input series to use for the @fold@ operator. We have no way to compile this function into a single loop that fuses the contained @fold@ and @map@ operators, even though they operate on series all at the same rate. 


% -----------------------------------------------------------------------------
\subsubsection{Processes are Hyper-strict}
A process description is naturally hyper-strict, meaning every value that is mentioned will be computed. If we are not careful then this can lead to unused values being computed. For example, consider the following function that choses between two fold results:

\begin{alltt}
strictSwitchy :: \(\forall\)(k : &). Bool 
              -> Series k Int -> Series k Int -> Int
strictSwitchy flag s1 s2
 = choose flag (fold (+) 0 (map (* 2) s1))
               (fold (+) 0 (map (* 4) s2))
\end{alltt}

The above function is similar to @badSwitchy@ from the previous section, except that we have used the @if@-like function @choose@ to select between the two folded results. When evaluated using Haskell semantics, the fact that @choose@ is non-strict in both arguments will mean that only one of the @fold@ results will be computed. However, a-normalizing and then slurping a @Process@ from this code produces:

\begin{alltt}
 process strictSwitchy (k : &) (flag : Bool) 
          (s1 : Series k Int) (s2 : Series k Int)
  with \{ s3 <- map  k Int (* 2) s1
       ; s4 <- map  k Int (* 4) s2
       ; x1 <= fold k Int s3 with (+) and 0
       ; x2 <= fold k Int s4 with (+) and 0 \}
  yields (choose flag x1 x2)
\end{alltt}

As mentioned earlier, we intend to compile this whole process description into a single loop. If we use the code above then both reductions will be computed at the same time, before choosing the desired result after the loop completes. 

To avoid this problem automatically we could check whether values produced with @fold@ or @create@ are used strictly in the expression given to @yields@, and if not, emit a warning to the user. It may also be possible to automatically massage the source program or @Process@ descriptions to ensure that unused results are not computed, but we have not investigated this further. As mentioned in \S\ref{s:Introduction}, a primary client of our new fusion system is Data Parallel Haskell (DPH), and the DPH vectorizer eliminates conditional operators like @choose@ as part of its existing vectorization process. 

%!TEX root = ../Main.tex

% -----------------------------------------------------------------------------
\begin{figure}
\begin{tabbing}
MMMMx           \= MM \= \kill
$name$  \> $\to$ \> (procedure name)            \\
$x,~ s$ \> $\to$ \> (value variable)            \\
$a,~ k$ \> $\to$ \> (type variable)             
\\[1ex]

$vec$   \> $\to$ \> (vector variable)           \\
$acc$   \> $\to$ \> (accumulator)
\\[2ex]

$kind$          \> \tt{::=}
                \>      $\tt{*} ~~|~~ \tt{\&}$

\\[1ex]
$type$          \> \tt{::=}
                \>      $a ~~|~~ \tt{Int} ~~|~~ \tt{Float} ~~|~~ ...$ 
\end{tabbing}

\begin{tabbing}
MMMMx           \= MM \= M \= MMMM \= \kill
$procedure$ 
 \> \tt{::=}    
        \> $\tt{procedure}~ name~ (k_{in} : \&)~ \ov{(a : kind)}$ \\
 \>     \> ~~~~~~~~~~~~~~~~~~~~~~~~~~~~ $\ov{(x : type)} 
                ~~ \ov{(s : \tt{Series}~ k_{in}~ type)}$     \\
 \>     \> ~~~$\tt{with}~~ \ov{nest}
                ~~ \tt{yields}~~ exp$
\end{tabbing}

\begin{tabbing}
MMMMx            \= MM \= x \= MMMM \= \kill
$nest$ 
 \> \tt{::=}       \> $\tt{loop}~~ k$            \\
 \>             \> \> \tt{start}  \> $\ov{stmtstart}$   \\
 \>             \> \> \tt{body}   \> $\ov{stmtbody}$    \\
 \>             \> \> \tt{inner}  \> $\ov{nest}$        \\
 \>             \> \> \tt{end}    \> $\ov{stmtend}$     
 \\[1ex]

 \> $~~|~~$     \> $\tt{guard}~~ (k_{inner} : \&)~~   
                        \tt{with}~~ k_{outer}~~ x_{flag}$ \\
 \>             \> \> \tt{body}   \> $\ov{stmtbody}$      \\
 \>             \> \> \tt{inner}  \> $\ov{stmtend}$
\end{tabbing}

\begin{tabbing}
MMMMx            \= Mx \= MMMx \= MMMM \= \kill
$stmtstart$     
 \> \tt{::=}    \> ~~ $vec$      \> $\tt{ =}~~~ \tt{newVec}~~ k$         \\[0.5ex]
 \> $~~~|$      \> ~~ $acc$      \> $\tt{ =}~~~ \tt{newAcc}~~ exp_{zero}$
\\[1.5ex]

$stmtbody$      
 \> \tt{::=}    \> ~~ $x_{elem}$ \> $\tt{ =}~~~ x'_{elem}$               \\[0.5ex]
 \> $~~~|$      \> ~~ $x_{elem}$ \> $\tt{ =}~~~ \tt{next}~~ k~~ s_{in}$  \\[0.5ex]
 \> $~~~|$      \> ~~ $x_{elem}$ \> $\tt{ =}~~~ exp_{worker}~~ \ov{x_{elem}} $  \\[0.5ex]
 \> $~~~|$      \> ~~ $acc$      \> $\tt{:=}~~~ exp_{worker}~~ acc~~ x_{elem}$  \\[0.5ex]
 \> $~~~|$      \> ~~ $\tt{writeVec}~~ k~~ vec~~ x_{elem}$
\\[1.5ex]

$stmtend$
 \> \tt{::=}    \> ~~ $x_{result}$ \> $\tt{ =}~~~ \tt{read}~~ acc$      \\[0.5ex]
 \> $~~~|$      \> ~~ $\tt{sliceVec}~~ k~~ vec$
\end{tabbing}

\caption{Series Procedures as Abstract Loop Nests}
\label{f:Procedure}
\end{figure}


%!TEX root = ../Main.tex

% -----------------------------------------------------------------------------
\begin{figure*}[ht]
\fbox{$process \Rightarrow procedure$}
$$
\infer  { \begin{aligned}
           & \tt{process}~~~~
                ~~ name
                ~~      (k_{in} : \tt{\&})
                ~~      \ov{(a : kind)}
                ~~      \ov{(x : type)}
                ~~      \ov{(s_{n} : \tt{Series}~ k_{in}~ type_n)}^{n}
                ~~ \tt{with}
                ~~      \ov{operator}
                ~~ \tt{yields}
                ~~      exp
          \\ 
          \Rightarrow~~
          & \tt{procedure}
                ~~ name
                ~~      (k_{in} : \tt{\&})
                ~~      \ov{(a : kind)}
                ~~      \ov{(x : type)}
                ~~      \ov{(s_{n} : \tt{Series}~ k_{in}~ type_n)}^{n}
                ~~ \tt{with}
                ~~      nest
                ~~~~~~~~~~ \tt{yields}
                ~~      exp
          \end{aligned}
        }
        { \tt{loop}~~ k_{in}~~
                \tt{body}~~ \{ \ov{s^{elem}_n ~\tt{=}~ \tt{next}~~ k_{in}~~ s_{n}}^{n} \}
        ~~ \vdash
        ~~      \ov{operator}
        ~~ \Rightarrow
        ~~      nest
        }
$$

\medskip
\fbox{$nest \vdash \ov{operator} \Rightarrow nest$}
$$
\infer  {  nest  
        ~~ \vdash
                ~~ \tt{mkSel}
                ~~      (k_{inner} : \tt{\&})
                ~~      (x_{sel} : \tt{Sel}~~ k_{outer}~~ k_{inner})
                ~~ \tt{from}
                ~~      k_{outer}
                ~~      s_{flags}
                ~~ \tt{in}
                ~~ \ov{operator}
        ~~ \Rightarrow
                ~~ nest'
        }
        { nest 
        ~~ \rhd
        ~~ (k_{outer}~ \times
                ~~ \tt{inner}
                ~~ \{   ~~ \tt{guard}~~ (k_{inner}~ : \tt{\&})
                        ~~ \tt{with}~~ k_{outer}~~ s^{elem}_{flags} 
                ~~ \})
        ~~ \vdash
        ~~      \ov{operator}
        ~~ \Rightarrow
        ~~      nest'
        }
$$


\begin{tabbing}
MMMM \= MMMM \= MMMMMMMMMMMMMMMMMMMM \= Mx \= MM \= MMMx \= \kill

% -- map ----------------
 \> $nest~ \vdash$    
        \> $s_{out} ~~~~~\tt{<-}
                ~~~ \tt{map}^{\;n}
                        ~~ k_{in}
                        ~~ \ov{type}^{\;n}
                        ~~ exp_{work}
                        ~~ \ov{s_{in}}^{\;n}$

\> $\Rightarrow$ 
        \> $nest$
        \> $\rhd~~ (k_{in}$
        \> $\times~~ \tt{body}~~~~ 
                \{~~ s^{elem}_{out} ~~~~~~\tt{=}~~ exp_{work}~~ \ov{s^{elem}_{in}}^{\;n} \} )$ 
\\[1em]


% -- pack --------------
 \> $nest~ \vdash$
        \> $s_{out} ~~~~~\tt{<-}
               ~~~ \tt{pack} 
                       ~~ k_{out}
                       ~~ k_{in}
                       ~~ type_{in}
                       ~~ x_{sel}
                       ~~ s_{in}$
\> $\Rightarrow$
        \> $nest$
        \> $\rhd~~ (k_{out}$
        \> $\times~~ \tt{body}~~~~ 
                \{~~ s^{elem}_{out} ~~~~~~\tt{=}~~ s^{elem}_{in} ~~\})$
\\[1em]


% -- fold --------------
 \> $nest~ \vdash$
        \> $x_{result}~~  
                        \tt{<= fold}~~ k_{in}~~ type_{in}~~ type_{result}~~ s_{in}$ 
 \> $\Rightarrow$
        \> $nest$
        \> $\rhd~~ (\top$
        \> $\times~~ \tt{start}~~   
                \{~~ x^{acc}_{result} ~~~~\tt{=}~~ \tt{newAcc}~~ exp_{zero} ~~\})$
 \\[0.5ex]     


 \>     \> \hspace{5em} $\tt{with}   ~~ exp_{work}
                      ~~ \tt{and}    ~~ exp_{zero}$
 \>     \> 
        \> $\rhd~~ (k_{in}$
        \> $\times~~ \tt{body}~~~~    
                \{~~ x^{acc}_{result}~~ \tt{:=}~~ exp_{work}~~ x^{acc}_{result}~~ s^{elem}_{in} ~~\})$
 \\[0.5ex]


 \>     \>
 \>     \> 
        \> $\rhd~~ (\top$
        \> $\times~~ \tt{end}~~~~~~ 
                \{~~ x_{result}~~~~~ \tt{=}~~ \tt{read}~~ x^{acc}_{result} ~~\})$
 \\[1em]


% -- create ------------
 \> $nest~ \vdash$    
        \> $x_{vec}~~~~~
                        \tt{<= create}~~ k_{in}~~ type_{in}~~ s_{in}$
 \> $\Rightarrow$
        \> $nest$
        \> $\rhd~~ (\top$
        \> $\times~~ \tt{start}~~
                \{~~ x_{vec}~~~~~~~~ \tt{=}~~ \tt{newVec}~~ k_{in} ~~\})$
 \\[0.5ex]

 \>     \>
 \>     \> 
        \> $\rhd~~ (k_{in}$
        \> $\times~~ \tt{body}~~~~
                \{~~ \tt{writeVec}~~ k_{in}~~ x_{vec}~~ s^{elem}_{in}~~ \})$
 \\[0.5ex]

 \>     \>
 \>     \>
        \> $\rhd~~ (\top$
        \> $\times~~ \tt{end}~~~~~~
                \{~~ \tt{sliceVec}~~ k_{in}~~ x_{vec}~~ \})$
\end{tabbing}
\caption{Scheduling Series Processes into Procedures}
\label{f:Scheduling}
\end{figure*}



% -----------------------------------------------------------------------------
\subsection{Scheduling Processes into Procedures}
\label{s:SchedulingLoops}
The Schedule phase takes a list of processes and converts it to a list of procedures.
%
\begin{code}
  schedule    :: [Process] -> [Procedure]
\end{code}

The definition of @Procedure@ is given in Figure~\ref{f:Procedure}. A @Procedure@ expresses the same computation as a @Process@, except that it is defined by an abstract, imperative loop nest instead of an operator graph. In Figure~\ref{f:Procedure} the fields of the @loop@ construct represent the \emph{anatomy} of the loop, similarly to \cite{Shivers:anatomy-of-a-loop}. The idea of representing a loop in this ``broken up'' format also appears in work by Waters on series expressions \cite{Waters:series-expressions, Waters:series-expressions-interpretation}, which was inspired by the @loop@ macro package of Common LISP~\cite{Steele:lisp}. 

For example, here is a simple @Process@ that sums the elements of an input series @s1@:
%
\begin{code}
 process sum_s (k : &) (s1 : Series k Int)
  with { x1 <= fold k Int s1 with (+) and 0 }
  yields x1
\end{code}


\eject
% ----------------------------------------------------------------------------
\noindent
The scheduled procedure is then:
%
\begin{code}
 procedure sum_d (k : &) (s1 : Series k Int) 
  with loop k
       start { x1_acc   = newAcc 0 }
       body  { s1_elem  = next k s1
             ; x1_acc  := (+) x1_acc s1_elem }
       inner {}
       end   { x1       = read x1_acc }
  yields x1
\end{code}

The @start@ field of a @loop@ holds setup statements to execute before entering the loop proper; @body@ contains the statements to execute for each iteration; @inner@ holds some inner nests to run for every iteration, and @end@ contains cleanup code that runs after the loop has completed. In future we elide empty fields, instead of writing @inner {}@ and so on.

In the @sum_d@ example, @x1_acc = newAcc 0@ creates a new accumulator @x1_acc@. The statement @s1_elem = next k s1@ takes the next element from @s1@. The statement @x1 = read x1_acc@ reads the accumulator. Note that the @next@, update @(:=)@ and @read@ statements are side effecting, imperative operations. 

Now, suppose we wanted a procedure like @sum_d@ that also computed the product of @s1@ concurrently with its sum, then added both results together. We would do this by appending an extra statement to each of the fields of the nest, and changing the yielded expression.

\begin{code}
procedure sumProd (k : &) (s1 : Series k Int)
 with loop k
      start { acc1      = newAcc 0
            ; acc2      = newAcc 1 }          *NEW
      body  { s1_elem   = next k s1
            ; acc1     := (+) acc1 s1_elem
            ; acc2     := (*) acc2 s1_elem }  *NEW
      end   { x1        = read acc1
            ; x2        = read acc2 }         *NEW
 yields (x1 + x2)                         *CHANGED
\end{code}

In general terms, to convert a process into a procedure we consider each operator from the process in turn, and insert statements into the procedure that implement that operator. The fact that our procedure is expressed as an \emph{anatomy} of separate @start@, @body@ and @end@ fields means that we can produce code that interleaves the computation of each operator. This interleaving of code is the primary mechanism which lets us deal with branching data flows, which we will return to in \S\ref{s:Related}. 

The scheduling process is formalized in Figure~\ref{f:Scheduling}, and we will give more examples in coming sections. The top level judgment $process \Rightarrow procedure$ converts a process into a similarly named procedure. Note that the resulting procedure has the same parameters and yielded expression, but its implementation has changed from an operator list into an abstract loop nest.

The $\rhd$ operator used in Figure~\ref{f:Scheduling} has the following type:
%
\[ \rhd : nest \to ((rate + \top) \times field) \to nest \]
%
where $field$ is one of the fields attached to the @loop@ or @guard@ constructs of Figure~\ref{f:Procedure} (@start@, @body@ and so on). The application ($n \rhd k \times f$) recursively descends into the nest $n$ until it finds the @loop@ or @guard@ construct that matches rate $k$, and then appends the statements in field $f$ to the similar field of that construct. When searching for matching rates we use the $k$ in (@loop@ $k$) and the $k_{inner}$ in ($\tt{guard}~ (k_{inner} : \&)~~ \tt{with}~~ ...$ ). The @guard@ construct is an abstract @if@ expression, and $k_{inner}$ corresponds to the number of times the body is entered. In some rules we use $\top$ in place of a real rate variable, which matches the rate of the outer-most loop construct in the nest.


% -----------------------------------------------------------------------------
\subsubsection{Scheduling Maps}
\label{s:SchedulingMaps}
As an example of how to schedule the @map@ operator, suppose we have a process like @sumProd@ from \S\ref{s:SchedulingLoops} that also performs a @map fn@ on the elements before taking the sum and product, for some arbitrary @fn@. Here is the full process description:

\begin{code}
 process sumProdFn (k : &) (s1 : Series k Int)
  with { s2  <- map  k Int fn s1
       ; xs  <= fold k Int s2 with (+) and 0
       ; xp  <= fold k Int s2 with (*) and 1 }
  yields (xs + xp)
\end{code}
%
Using just the first rule in Figure~\ref{f:Scheduling} and the one for @map@ would produce the following procedure, where we still need to schedule the two @fold@ operators:
%
\begin{code}
 procedure sumProdFn (k : &) (s1 : Series k Int)
  with loop k
       body  { s1_elem   = next k s1
             ; s2_elem   = fn s1_elem }
  yields (xs + xp)                  ** NOT FINISHED
\end{code}
%
Importantly, note that the intermediate variables @s1_elem@ and @s2_elem@ are named after the corresponding series @s1@ and @s2@. In Figure~\ref{f:Scheduling} this naming convention is indicated by the superscript on variable names, so $s^{elem}$ is a concrete variable name related to the name $s$. We use this trick to avoid maintaining an environment that maps series names ($s$) to the variable names that bind their individual elements ($s_{elem}$). We similarly relate the names of non-series variables ($x$) to their corresponding accumulators ($x^{acc}$). 

Returning to Figure~\ref{f:Scheduling}, the rule for @fold@ adds statements to the nest that first initialize an accumulator, update it in the body of the loop, and then read back the final value. The accumulator lives for the entirety of the loop, so we insert the statements to initialize and read its value in the outer most context, indicated by $\top$.

\eject
Scheduling the two fold operations of @sumProdFn@ produces the following.

\begin{code}
 procedure sumProdFn (k : &) (s1 : Series k Int)
 with loop k
      start { xs_acc    = newAcc 0             *NEW
            ; xp_acc    = newAcc 1 }           *NEW
      body  { s1_elem   = next k s1
            ; s2_elem   = fn s1_elem   
            ; xs_acc   := (+) xs_acc s2_elem   *NEW
            ; xp_acc   := (*) xp_acc s2_elem } *NEW
      end   { xs        = read xs_acc          *NEW
            ; xp        = read xp_acc }        *NEW
 yields (xs + xp)
\end{code}

When we convert this back to concrete Core code in the next section, we will generate the outer structure of the loop that causes the statements in the body to be evaluated the correct number of times This is governed by the associated rate variable @k@. 


% -----------------------------------------------------------------------------
\subsubsection{Scheduling Pack and Create}
\label{s:SchedulingPacks}
Operators from @mkSel@ contexts in the process description are scheduled into the body of an abstract @if@ statement represented by the @guard@ construct of Figure~\ref{f:Procedure}. A @guard@ binds the rate variable $k_{inner}$ for the inner context, and is also tagged with the rate $k_{outer}$ of the outer context. With @guard@, the @body@ and @inner@ fields contain more statements to run when the corresponding flag $x_{flag}$ is true. The @inner@ nest is needed when a packed series is packed again using a different selector, as this creates a more deeply nested parallel context. The NestedFilter example described in \S\ref{s:Benchmarks} does this.

Figure~\ref{f:filterMax-nest} shows the procedure generated for the explicitly typed version of @filterMax@ back in Figure~\ref{f:new-filterMax}, whose process description is in \S\ref{s:Slurping}. On the right of each statement we show the series operator that statement is associated with. 

Scheduling the use of @create@ has added three separate statements to the nest: one to allocate the new vector buffer; one to write elements into it, and one to destructively slice it down to the final size by overwriting its length field. Note that the use of the $\rhd$ operator in the corresponding rule from Figure~\ref{f:Scheduling} ensures that the result vector is only written to inside the body of the @guard@. Also, note that @newVec@ and @sliceVec@ statements are tagged with @k2@, which is not in scope in the outer context. This reflects that fact that the length of the result vector cannot be known until the loop completes, as only then will we know how many times the body of the @guard@ was entered. We will fix this in the \emph{concretization} phase described in the next section, by rewriting the code to keep track of the length of the result vector.

Interestingly, the @pack@ operation becomes a trivial renaming, as shown in the first statement of the inner @body@ field of Figure~\ref{f:filterMax-nest}. Recall that in the version of @filterMax@ from Figure~\ref{f:new-filterMax}, the pack operation was written @s4 = pack k1 k2 Int sel s2@, and here we have scheduled it as just @s4_elem = s2_elem@. It is the @mkSel@ function that actually creates the selector context, and that context is expressed here as a @guard@ construct. When looking at the source code of @filterMax@, we might imagine that pack would perform some sort of indexing operation, after the discussion in \S\ref{s:SelectorsAndPacking}. However, when the code is expressed in the form of Figure~\ref{f:filterMax-nest} the current index into each series is implicit. 

From a logical / type-theory perspective we view @pack@ as a \emph{coercion} from a series of the outer rate (@k1@) to a series of the inner rate (@k2@). This coercion is justified by \emph{evidence} that these rates are related, which is expressed as our selector @sel@. However, our compilation process eliminates the selector completely, so the physical packing operation as described in \S\ref{s:SelectorsAndPacking} is never performed.


\begin{figure}
\begin{code}
procedure filterMax_d (k1 : &) (s1 : Series k1 Int)
with loop k1
 start { vec'     = newVec k2             *create
       ; mx_acc   = newAcc 0       }      *fold max
 body  { s1_elem  = next k1 s1
       ; s2_elem  = (+ 1) s1_elem         *map (+ 1)
       ; s3_elem  = (> 0) s2_elem  }      *map (> 0)
 inner guard (k2 : &) with k1 s3_elem
       body { s4_elem = s2_elem           *pack
            ; writeVec k2 vec' s4_elem    *create
            ; mx_acc := max mx_acc s4_elem} *fold max
 end   { sliceVec k2 vec'                 *create
       ; mx       = read mx_acc }         *fold max
yields (vec', mx)
\end{code}
\caption{Procedure for \texttt{filterMax}}
\label{f:filterMax-nest}
\end{figure}


% -----------------------------------------------------------------------------
\subsection{Concretization}
\label{s:Concretization}
The concretization phase rewrites constructs that use type level rate variables into ones that use real indices and loop counters. 
\begin{code}
  concretize :: [Procedure] -> [ProcedureI]
\end{code}

The @ProcedureI@ language is very similar to @Procedure@ in Figure~\ref{f:Procedure} except that every appearance of a rate variable in @Procedure@ is replaced by loop counter or known length in @ProcedureI@.\footnote{In our real implementation we use the same data type to represent both, and simply fill-in accumulation variables during concretization.} The concrete version of @filterMax@ is shown in Figure~\ref{f:filterMax-concrete}, which is the transformed version of Figure~\ref{f:filterMax-nest}. 

Deciding whether to change a rate variable to a loop counter (like @k1@ to @k1_ix@), accumulator (@k2@ to @k2_acc@) or a known length (@k1@ to @length s1@) is based on the form of the construct being rewritten. For example, for @sliceVec@ we always rewrite its rate variable to a similarly named accumulator.

The operator @guard (k2 : &) with k1 s3_elem@ in Figure~\ref{f:filterMax-nest} changes to @guardI (k2_ix: Int) k2_acc with s3_elem@ in Figure~\ref{f:filterMax-concrete}. For each @guard@ we also insert an accumulator (@k2_acc@) to keep track of how many times the guard is entered. The new @guardI@ construct executes by first checking the flag @s3_elem@, and if it is true, reads @k2_acc@ to get the current entry counter and binds it to @k2_ix@ before evaluating the body. In the application of @writeVecI@ this index @k2_ix@ is then used when constructing the filtered result vector @vec'@. The final @sliceVec@ statement is also rewritten to use @k2_acc@, so it knows the final length.

The last task is to rewrite rate variables on @loop@ constructs and @newVec@ statements to use the lengths of known series. For this we simply look in the environment for a series whose type contains the same rate variable and use its length. If there are multiple series with the same rate variable then we just choose the first one --- as all series tagged with the same rate are guaranteed to have the same length. 


% -----------------------------------------------------------------------------
\begin{figure}
\begin{code}
procedure filterMax_c (k1 : &) (s1 : Series k1 Int)
with loopI (k1_ix : Int) (length s1)
 start { vec'     = newVecI (length s1)    *CHANGED
       ; acc      = newAcc  0 
       ; k2_acc   = newAcc  0 }            *NEW
 body  { s1_elem  = nextI k1_ix s1         *CHANGED
       ; s2_elem  = (+ 1) s1_elem
       ; s3_elem  = (> 0) s2_elem  }
 inner guardI (k2_ix : Int) k2_acc with s3_elem            
                                           *CHANGED
       body { s4_elem = s2_elem
            ; writeVecI k2_ix vec' s4_elem *CHANGED
            ; acc    := max acc s4_elem }
 end   { sliceVecI k2_acc vec'             *CHANGED
       ; mx       = read acc }            
yields (vec', mx)
\end{code}
\caption{Concrete Procedure for \texttt{filterMax}}
\label{f:filterMax-concrete}
\end{figure}


% -----------------------------------------------------------------------------
\subsection{Extracting Implementation Code}
\label{s:Extract}
The Extract phase takes our concretized list of procedures and converts them back to a module of imperative flavoured Core code. We end up with a top-level binding for each @Procedure@, which mirrors the @slurp@ phase from \S\ref{s:Slurping}.
%
\begin{code}
  extract     :: [Procedure] -> Module
\end{code}
%
The extracted code for @filterMax@ is shown in Figure~\ref{f:filterMax-extracted}. As we can see, this final phase is again mostly a change of syntax. The real work of fusion has been performed by the scheduling phase, and the concretization pass in the previous section has already reduced the abstraction level of our program to something that looks like real loop code. 

In the extracted code the @loopI@ and @guardI@ constructs have changed to calls to similarly named functions. As we will see in the next section, these can be implemented as Haskell library functions and then inlined, or transformed to into tail recursive loops as discussed in \S\ref{s:LoopWinding}.


% -----------------------------------------------------------------------------
\begin{figure}
\begin{alltt}
filterMax_x :: \(\forall\)(k1 : &)
            .  Series k1 Int -> (Vector Int, Int)
 = \(\Lambda\)(k1 : &).
   \(\lambda\)(s1 : Series k1 Int).
   let vec'   : Vector Int = newVec @Int (length s1)
   let acc    : Ref Int    = newRef @Int 0    
   let k2_acc : Ref Int    = newRef @Int 0    
   let _ : Unit
       = loopI (length s1)
         (\(\lambda\)(k1_ix : Int).
          let s1_elem = next @k1 @Int s1 k1_ix 
          let s2_elem = add  @Int 1    s1_elem 
          let s3_elem = gt   @Int 0    s2_elem 
          let _  : Unit
              = guardI k2_acc s3_elem
                 (\(\lambda\)(k2_ix : Int).
                  let s4_elem = s2_elem
                  let _ = writeVec @Int
                            vec k2_ix s4_elem 
                  let _ = writeRef @Int acc
                            (max (readRef @Int acc) 
                                 s4_elem) 
                  in ()) 
          in ()) 
   let k2_ix : Int        = readRef  @Int k2_acc
   let vec'' : Vector Int = sliceVec @Int k2_ix vec'
   let mx    : Int        = readRef  @Int acc
   in  (vec'', mx)
\end{alltt}
\caption{Extracted Imperative Core Code for \texttt{filterMax}}
\label{f:filterMax-extracted}
\end{figure}


%!TEX root = ../Main.tex

\section{Details of the Conversion}
\label{s:Conversion}

The previous section contains the main details of the fusion transformation, starting with a high level description of the computation to be performed, and ending in imperative loop code.

As GHC Core is a pure functional language, the imperative code in Figure~\ref{f:filterMax-extracted} is not the end of the story. In our implementation we express the code in Figure~\ref{f:filterMax-extracted} in an imperative version of GHC Core named Core Flow. This language is essentially the same as GHC Core, being a version of System-F, except that it is strict, has untracked side effects and includes imperative functions like @newVec@ and @readRef@ as primitive operators.


% -----------------------------------------------------------------------------
\subsection{State Threading}
As GHC uses monadic state threading to sequence effectful statements, we must thread GHC's primitive world token though the extracted code before converting it back to real GHC Core. We have implemented this state threading transform generically, so it is parameterized by two sets of type signatures: one that assumes a language with untracked side effects, and one that uses state threading. For example, the two versions of the signature for @writeVec#@ are as follows, where @writeVecE#@ has untracked effects and @writeVecW#@ uses a world token of type @W@.

\begin{alltt}
writeVecE# :: \(\forall\)(a:*). Vector a -> Int -> a -> ()
writeVecW# :: \(\forall\)(a:*). Vector a -> Int -> a -> W -> W
\end{alltt}


% -----------------------------------------------------------------------------
\subsection{Loop Winding}
\label{s:LoopWinding}
After the extracted code of Figure~\ref{f:filterMax-extracted} has had the world token threaded through it, it can converted back to real GHC Core and type checked. 

In our implementation we originally wrote @newVec@, @newRef@, @loopI@, @guardI@, @next@ and so on as standard Haskell library functions. Although this allows the program to run, the fact that GHC does not track pointer aliasing between heap objects results in inefficient object code when using mutable references.

To avoid this problem, we instead perform a \emph{loop winding} transformation on the code that converts uses of @loopI@ and @guardI@ into real tail recursive loops, and mutable references into accumulating parameters. This transform is ad-hoc because it assumes that mutable references do not escape the extracted function, and that there is no additional aliasing between reference variables like @acc@ and @k2_acc@. However, because we generated the code ourselves we know these properties are true. 


% -----------------------------------------------------------------------------
\subsection{Primitive Arithmetic and Unboxed Types}
Unlike GHC Core, the Core Flow language does not make a distinction between boxed and unboxed types \cite{PeytonJones:unboxed}. When we @slurp@ a @Process@ description from the original GHC Core program we require that program to use boxed numeric values and operators. However, when we convert extracted code \emph{back} to GHC Core, we use the unboxed versions. Unboxed primitive operators typically compile down to single machine instructions. To handle the impedance mismatch we then generate a wrapper function that marshals between the signature of the original source function and the extracted version. For example, the wrapper for @filterMax@ function would be:

\begin{alltt}
 filterMax = \(\Lambda\)(k : &). \(\lambda\)(s : Series k Int).
             case filterMax_x s of
              (# vec, n #) -> (vec, I# n)
\end{alltt}

Standard unboxing techniques guided by strictness information \emph{usually} work, but as strictness analysis is conservative the unboxing is not guaranteed. When the rest of the loop body has been fused well enough to execute in only a handful of cycles, the cost of a single unboxing operation in an inner loop can easily dominate program runtime. Brutally converting arithmetic operations from their boxed to unboxed versions during flow fusion ensures that we never pay the price of thunk entry in fused code.
 


%!TEX root = ../Main.tex


% -----------------------------------------------------------------------------
\begin{figure*}[t]
\begin{center}
\begin{tabular}{lccccrccc}
Benchmark 
        & Input size 
        & Stream (ms) 
        & \multicolumn{2}{c}{Flow (ms)} 
        & \multicolumn{2}{c}{Unfused Flow (ms)}  
        & \multicolumn{2}{c}{Hand-fused C (ms)}    \\
\hline
Dot Product &  $10^8$    & 655    & 489    &  (75\%) & 1,096  & (167\%)  & 474      & (72\%) \\
MapMap      &  $10^8$    & 842    & 636    &  (75\%) & 842    & (100\%)  & 615      & (73\%) \\
FilterSum   &  $10^8$    & 505    & 430    &  (85\%) & 1,132  & (224\%)  & 344      & (68\%) \\
FilterMax   &  $10^8$    & 567    & 521    &  (91\%) & 1,496  & (263\%)  & 360      & (63\%) \\
NestedFilter&  $10^8$    & 485    & 420    &  (86\%) & 1,202  & (247\%)  & 376      & (77\%) \\
QuickHull   &  $10^7$    & 419    & 208    &  (49\%) & 857    & (204\%)  & 183      & (43\%) \\
\end{tabular}
\caption{Benchmark Results for Flow Fusion}
\label{f:benchmark-table}
\end{center}
\end{figure*}


% -----------------------------------------------------------------------------
\section{Benchmarks}
\label{s:Benchmarks}
Benchmarks were conducted on a MacBook Pro with 2.8GHz Intel Core i7 with 8GB of RAM. Source code is available from the @repa-plugin@ darcs repository. 

We use micro-benchmarks because our fusion system addresses these specific programming patterns, rather than being an improvement on the ambient performance of the program --- as with optimizations like pointer tagging~\cite{Marlow:pointer-tagging}.

For each benchmark we compare four implementations:
\begin{itemize}
\item \emph{Stream}: using stream fusion~\cite{Coutts:stream-fusion} and unboxed vectors;\footnote{from the @vector@ library on Hackage}

\item \emph{Flow}: using our new Flow fusion framework;

\item \emph{Unfused Flow}: using our Flow API but without the plugin;

\item \emph{Hand-fused}: hand written and fused C code.
\end{itemize}

The Unfused Flow versions use the exact same code as the Flow versions, except they are compiled without the plugin that actually performs the fusion transformation. In this case the benchmarks are compiled via a fallback implementation of the user-facing API of Figure~\ref{f:SeriesOperators}, implemented in terms of standard stream fusion~\cite{Coutts:stream-fusion}. The fallback implementation provides a quick compilation path for people that do not want to install the plugin or care about the last ounce of 
performance, as well as a convenient way of testing the plugin itself.


% -----------------------------------------------------------------------------
\subsection{Dot Product}
A pair of two-dimensional vectors are multiplied element-wise and the results summed. Each two dimensional vector is stored as two arrays of integers, giving four arrays in total. As discussed in \S\ref{s:streams-zipWith}, code compiled with stream fusion produces a loop counter for each vector. With flow fusion the concretization phase (\S\ref{s:Concretization}) naturally causes loop counters to be shared. This provides a 25\% speedup over stream fusion and puts us on par with the reference C implementation. 


% -----------------------------------------------------------------------------
\subsection{MapMap}
The elements of a vector of integers are doubled, and the resulting vector has a constant added in one operation and subtracted in another. With stream fusion the first result is materialized in memory, and then read back by each of the subsequent vector operations. With flow fusion the first result is not materialized, which also puts us on par with the reference C implementation.


% -----------------------------------------------------------------------------
\subsection{FilterSum}
A vector of integers is filtered and the elements of the original and result vectors are separately summed. The result vector and both sums are returned in a 3-tuple. With stream fusion the code uses three separate loops, one for each operator. With flow fusion the code uses a single loop that sums both the original and result vectors while constructing the result. 

Although the Core code produced by flow fusion is optimal, the low-level object code suffers from pointer aliasing problems. The back-end code generator does not know that the input vector does not alias with the result vector, nor that writes to the element data of the result do not affect its starting offset field. Ultimately, the length field of the first vector, and starting offsets for both vectors are repeatedly read in the inner loop. The C compiler can infer correct aliasing information, and thus saves three memory reads per loop iteration.


% -----------------------------------------------------------------------------
\subsection{FilterMax}
This is the @filterMax@ example described earlier. With stream fusion the filtered result vector must be read back from memory to sum its elements. With flow fusion the sum is performed in the same loop as the filter. Similarly to the FilterSum benchmark, although the Core code is optimal, low level pointer aliasing problems in the object code prevent us from matching the performance of the C version.


% -----------------------------------------------------------------------------
\subsection{NestedFilter}
An input vector is filtered, this result filtered again by another predicate, and both results are returned in a tuple. With stream fusion the first result is constructed in memory and then read back to perform the second filter. With flow fusion both results are constructed in the same loop and the first is not read back. As mentioned in \S\ref{s:SchedulingPacks} this benchmark uses two nested applications of @mkSel@, thus the Core code contains two nested guards. 


% -----------------------------------------------------------------------------
\subsection{QuickHull}
QuickHull finds the smallest convex hull of a set of points in the 2-d plane. The algorithm operates in two phases. The first is an initialization phase where we determine the left-most and right-most points in the input set. If these results are computed in two separate fold operations then stream fusion cannot fuse this code. In the second phase, we need to determine the set of points above a cutting line and also the point furthest from it. This is a @filterMax@-like operation that stream fusion can also not fuse.



%!TEX root = ../Main.tex
\section{Related Work}

\begin{itemize}
\item Iteratee Enumerator
\item Conduit, we don't have await and yield. Flows are not monad transformers.
\item Pipes
\item Machines support multiple inputs and fanout.
\item Monad par, is event flow network using IVars.
\item Impala.
\item Hive.
\item Spark.
\item Google Tensorflow
\item Lustre, Lucid sync data flow, Kahn networks.
\item StreamIt, Brook.
\item Scala streams library.
\item FRP libraries, eg reflex.
\end{itemize}


% Repa flow fills a sweet spot between the roles functional array library and analytic database. In terms of the programming model, a key feature of Repa Flow compared with Scoobi and Scalding is that the API carefully distinguishes between operators that run in constant space and those which do not. Systems based on map-reduce make implicit use of a \emph{shuffle} operator that distributes data between the compute nodes. The \emph{shuffle} operator sends data between the nodes in a data-dependent way, which can result in a skewed workload where most data is sent to a subset of nodes while the others are starved. When all source-level queries are converted into map-reduce jobs then there is no systematic way in which skew can be avoided. Taking inspiration from the work on synchronous data flow and Khan networks, we have arranged our API so that most operators execute without buffering. With Repa Flow it is easy to write programs where both buffering and data skew are avoided by construction, or admitted only in a controlled way.

%!TEX root = ../Main.tex

\vspace{-1ex}
\section{Future Work}
\label{s:FutureWork}

This paper only discusses a few array combinators: @map@, @fold@, @pack@ and so on, but others such as @append@ and @scan@ can be implemented in a similar way. For example, @append@ is an instance of the more general @combine@ combinator, that takes a series of flags, two series of elements, and then chooses which element to return based on the flag:
\begin{code}
  combine [T F T T F] [1 2] [3 4 5] = [3 1 4 5 2]
\end{code}

After \S\ref{s:SelectorsAndPacking}, the vector of flags would be implemented as an extended selector @Sel2 k1 k2 k3@ that relates three separate rates: the rate of the flags, the rates of the two input vectors of elements. In the @Procedure@ language, this new selector context would be compiled as a real @if@ construct with both a @then@ and @else@ branches, unlike our @guard@ construct that only has the @then@ branch.

In future work we will perform further transformations at the @Procedure@ level to introduce SIMD instructions and multicore evaluation. Unlike the loop parallelization systems in imperative language compilers, we do not need an iteration dependency analysis. Our our loops lack such dependencies by construction.







% -----------------------------------------------------------------------------
\vspace{-1ex}
\paragraph{Acknowledgements}
Thanks to Barry Jay for the suggestion to phrase the problem with short cut fusion in terms of what information is available to the program transformations; Peter Gammie for the connection with John Hughes's early work; Simon Peyton Jones and Geoff Mainland for comments on a draft version of this paper. This work was supported in part by the Australian Research Council under grant number LP0989507. 

\bibliographystyle{plain}
\vspace{-1ex}
\bibliography{Main}

\end{document}

