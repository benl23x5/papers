%!TEX root = ../Main.tex
\section{Icicle Source}
\label{s:Source}

The two main types in Icicle are @Stream@ and @Fold@.
Streams represent data values as they flow through the program.
Streams do not always have data flowing through them, but have an associated clock which describes when data occurs.
Two streams with the same clock both have data at the same time.
Folds, on the other hand, are results of computations over the stream data that has been seen.
Folds always have a well-defined value.
Folds can be easily converted to streams, by sampling their current value whenever the stream clock is true.
It is harder to convert a stream to a fold, as the stream has gaps where it is undefined, and one must specify how to fill the gaps (hold last, fill with zero, etc).
Finally, folds can be computed recursively based on their previous values.

%!TEX root = ../Main.tex

\begin{figure}

\begin{tabbing}
MMMM \= MM \= MMMMMMMMMMM \= MM \= \kill
$\mi{e}$
\GrammarDef $x$
\> $|$ \> $v$
\GrammarAlt $e~e$
\> $|$ \> $\lambda{}x~:~\tau.~e$

\\
\GrammarAlt $@when@~e~e$
\> $|$ \> $@sample@~e~e$
\GrammarAlt $@mapS@~e~e$
\> $|$ \> $@zipS@~e~e$
\GrammarAlt $@mapF@~e~e$
\> $|$ \> $@zipF@~e~e$
\\
\GrammarAlt $@let@~\mi{Let}~@in@~e$
\\
\GrammarAlt $\langle e,~ e \rangle ~|~ \langle\rangle$
\\
\\

$\mi{Let}$
\GrammarDef $x~=~e$
\GrammarAlt $@folds@~\{~x_i~=~e_i,~e_i~\}_i$
\\
\\

$\mi{v}$
\GrammarDef $\NN ~|~ \BB ~|~ [v]$
\GrammarAlt $(v,~v) ~|~ \bot$
\\
\\


$\mi{Program}$
\GrammarDef @inputs@~$\{ x_i~:~@Stream@~c_i~\tau_i \}_i~@in@~\mi{e}$ \\
\end{tabbing}

\caption{Grammar for Icicle Source}
\label{fig:source:grammar}
\end{figure}


The grammar for Icicle is given in figure~\ref{fig:source:grammar}.
The first four rules are rather standard lambda calculus.

Next are the primitives.
@when@ takes a stream and a fold, and returns a new fold whose value is the stream \emph{when}ever the stream is defined, and the fold otherwise.
This is used for defining folds that depend on stream values.
@sample@ takes a stream of any type and a fold, creates a new stream using the values of the fold.
@mapS@ and @zipS@ perform map and zip operations on streams of the same rate.
@mapF@ and @zipF@ perform map and zip operations on folds.

Let expressions such as $@let@~x~=e~@in@~e$ are as usual. 
Folds are defined using the syntax @let folds@ syntax. 
Multiple folds can be defined together, and each fold is defined by the initial value, and the ``kons'' part, defining the next value.

For example, a simple fold comprised of nothing but $0$
\begin{tabbing}
MM \= \kill
$@let folds@~\mi{zeros}~=~0,~\mi{zeros}$ \\
$@in@~\mi{zeros}$
\\
\\
$\mi{zeros}$ \> $=~\langle 0~0~0~0~\cdots~\rangle$ \\
\end{tabbing}

When computing the sum over another fold, the newly-defined fold and the input fold are zipped and then added together.
\begin{tabbing}
$\mi{sum}~=~\lambda{}(\mi{values}~:~@Fold@~\NN).$ \\
$@let folds@~\mi{s}~=~0,~@mapF@~(+)~(@zipF@~\mi{s}~\mi{values})$ \\
$@in@~\mi{s}$
\\
\\
$\mi{sum}~\langle 0~1~2~3~\cdots~\rangle$ \\
$~=~\langle 0~1~3~6~\cdots~\rangle$ \\
\end{tabbing}

This $\mi{sum}$ function requires a fold argument, but it can be applied to a stream $s$ by creating a fold that is the stream when it is defined, and $0$ otherwise: $@when@~s~zeros$.
A more general way is to define a new $\mi{sum'}$ function that works directly over streams:
\begin{tabbing}
MMMMMM \= MM \= MM \kill
$\mi{sum'}~=~\lambda{}(\mi{values}~:~@Stream@~c~\NN).$ \\
$@let folds@~\mi{s}~=~0,$ \\
\> $@when@$ \> $(@mapS@~(+)~(@zipS@~(@sample@~\mi{values}~\mi{s})~\mi{values}))$ \\
\> \> $~\mi{s}$ \\
$@in@~\mi{s}$
\\
\\
$\mi{sum'}~\langle \bot~1~\bot~3~\cdots~\rangle$ \\
$~=~\langle 0~1~1~4~\cdots~\rangle$ \\
\end{tabbing}

Here, the fold $s$ is sampled to a stream with the same clock as the input stream.
Whenever the input stream is defined, the previous value of $s$ and the current value of the $\mi{values}$ will be added together, producing the new sum.
When the input stream is not defined, the previous value of $s$ is used unchanged.

When referring to the newly defined folds in the right-hand-side of the fold (the kons), all values refer to the immediate previous value.
The order of the fold bindings has no effect on the program.
\begin{tabbing}
MM \= \kill
$@let folds@$ \\
\> $\mi{one}~=~1,~\mi{zero}$ \\
\> $\mi{zero}~=~0,~\mi{one}$ \\
$@in@~\mi{one}$
\\
\\
$\mi{one}$  \> $=~\langle 1~0~1~0~\cdots~\rangle$ \\
$\mi{zero}$ \> $=~\langle 0~1~0~1~\cdots~\rangle$ \\
\end{tabbing}

The grammar also has rules for stream literals: $\langle e_1,~e_2 \rangle$ and $\langle\rangle$. The stream can contain bottoms, $\bot$, which means that the stream is undefined at that point.
Stream literals have concrete clock types, for example $\langle T,~F \rangle$ for $\langle 1,~\bot \rangle$, while input streams have existential clocks: their actual clock is unknown until runtime.
These stream literals are not expected to occur in an actual program until evaluation.


Yes, I realise the grammar doesn't have @+@ or @/@, and that it doesn't have polymorphism.
For the examples, assume that a specialisation pass occurs, and that there is a ``sensible set of primitives on natural numbers and lists''.

Define group here.
\begin{code}
data Unpack r
 = forall a.
    Unpack a (Fold (a -> a)) (a -> r)

sum' vs = Unpack 0 (+vs) id

group (key : Stream c k)
      (val : Unpack v)
           : [(k,v)]
group key (Unpack z k r)
 = trivial
\end{code}



%!TEX root = ../Main.tex

\begin{figure*}

$$
\boxed{\SourceStepZ{e}{\mi{acc}}{\mi{res}}}
$$

$$
\ruleAx
{
    \SourceStepZ{x}{()}{x}
}{ZVar}
\ruleAx
{
    \SourceStepZ{v}{()}{v}
}{ZValue}
\ruleIN
{
    \SourceStepZ{e}{a}{r}
}
{
    \SourceStepZ{\lambda{}x~:~\tau.~e}{a}{\lambda{}x~:~\tau.~r}
}{ZLam}
$$

$$
\ruleAx
{
    \SourceStepZ{\langle e_1,~e_2 \rangle}{\langle e_1,~e_2 \rangle}{\bot}
}{ZStream1}
\ruleAx
{
    \SourceStepZ{\langle \rangle}{\langle \rangle}{\bot}
}{ZStream2}
\ruleIN
{
    \SourceStepZ{e_1}{a_1}{r_1}
    \quad
    \SourceStepZ{e_2}{a_2}{r_2}
}
{
    \SourceStepZ{@when@~e_1~e_2}{a_1,a_2}{r_2}
}{ZWhen}
\ruleIN
{
    \SourceStepZ{e_1}{a_1}{r_1}
    \quad
    \SourceStepZ{e_2}{a_2}{r_2}
}
{
    \SourceStepZ{@sample@~e_1~e_2}{a_1,a_2}{\bot}
}{ZSample}
$$

$$
\ruleIN
{
    \SourceStepZ{e}{a}{r}
}
{
    \SourceStepZ{@mapS@~f~e}{a}{\bot}
}{ZMapS}
\ruleIN
{
    \SourceStepZ{e_1}{a_1}{r_1}
    \quad
    \SourceStepZ{e_2}{a_2}{r_2}
}
{
    \SourceStepZ{@zipS@~e_1~e_2}{a_1,a_2}{\bot}
}{ZZipS}
$$

$$
\ruleIN
{
    \SourceStepZ{e}{a}{r}
}
{
    \SourceStepZ{@mapF@~f~e}{a}{f~r}
}{ZMapF}
\ruleIN
{
    \SourceStepZ{e_1}{a_1}{r_1}
    \quad
    \SourceStepZ{e_2}{a_2}{r_2}
}
{
    \SourceStepZ{@zipF@~e_1~e_2}{a_1,a_2}{r_1,r_2}
}{ZZipF}
$$

$$
\ruleIN
{
    \SourceStepZ{e_1}{a_1}{r_1}
    \quad
    \SourceStepZ{e_2}{a_2}{r_2}
}
{
    \SourceStepZ{@let@~x~=~e_1~@in@~e_2}{a_1,a_2}{r_2[x=r_1]}
}{ZLet}
\ruleIN
{
    \{ \SourceStepZ{e_i}{a_i}{r_i} \}_i
    \quad
    \SourceStepZ{e}{a}{r}
}
{
    \SourceStepZ{@let folds@~\{x_i~=~e_i,~\_\}_i~@in@~e}{\{a_i\}_i,a}{r[\{x_i=r_i\}_i]}
}{ZFolds}
$$

$$
\boxed{\SourceStepK{\mi{acc}}{e}{\mi{acc}}{\mi{res}}}
$$


$$
\ruleAx
{
    \SourceStepK{()}{x}{()}{x}
}{KVar}
\ruleAx
{
    \SourceStepK{()}{v}{()}{v}
}{KValue}
\ruleIN
{
    \SourceStepK{a}{e}{a'}{r'}
}
{
    \SourceStepK{a}{\lambda{}x~:~\tau.~e}{a'}{\lambda{}x~:~\tau.~r'}
}{KLam}
$$

$$
\ruleAx
{
    \SourceStepK{\langle e_1,~e_2 \rangle}{\langle \cdots \rangle}{e_2}{e_1}
}{KStream1}
\ruleAx
{
    \SourceStepK{\langle \rangle}{\langle \cdots \rangle}{\langle\rangle}{\bot}
}{KStream2}
\ruleIN
{
    \SourceStepK{a_1}{e_1}{a_1'}{r_1'}
    \quad
    \SourceStepK{a_1}{e_2}{a_2'}{r_2'}
}
{
    \SourceStepK{a_1,a_2}{@when@~e_1~e_2}{a_1',a_2'}{\Ifnbot{r_1'}{r_1'}{r_2'}}
}{KWhen}
$$

$$
\ruleIN
{
    \SourceStepK{a_1}{e_1}{a_1'}{r_1'}
    \quad
    \SourceStepK{a_2}{e_2}{a_2'}{r_2'}
}
{
    \SourceStepK{a_1,a_2}{@sample@~e_1~e_2}{a_1',a_2'}{\Ifnbot{r_1'}{r_2'}{\bot}}
}{KSample}
$$

$$
\ruleIN
{
    \SourceStepK{a}{e}{a'}{r'}
}
{
    \SourceStepK{a}{@mapS@~f~e}{a'}{\Ifnbot{r'}{f~r'}{\bot}}
}{KMapS}
\ruleIN
{
    \SourceStepK{a_1}{e_1}{a_1'}{r_1'}
    \quad
    \SourceStepK{a_2}{e_2}{a_2'}{r_2'}
}
{
    \SourceStepK{a_1,a_2}{@zipS@~e_1~e_2}{a_1',a_2'}{r_1',r_2'}
}{KZipS}
$$

$$
\ruleIN
{
    \SourceStepK{a}{e}{a'}{r'}
}
{
    \SourceStepK{a}{@mapF@~f~e}{a'}{f~r}
}{KMapF}
\ruleIN
{
    \SourceStepK{a_1}{e_1}{a_1'}{r_1'}
    \quad
    \SourceStepK{a_2}{e_2}{a_2'}{r_2'}
}
{
    \SourceStepK{a_1,a_2}{@zipF@~e_1~e_2}{a_1',a_2'}{r_1',r_2'}
}{KZipF}
$$

$$
\ruleIN
{
    \SourceStepK{a_1}{e_1}{a_1'}{r_1'}
    \quad
    \SourceStepK{a_2}{e_2}{a_2'}{r_2'}
}
{
    \SourceStepK{a_1,a_2}{@let@~x~=~e_1~@in@~e_2}{a_1',a_2'}{r_2[x=r_1]}
}{KLet}
\ruleIN
{
    \{ \SourceStepK{a_i}{e_i}{a_i'}{r_i'} \}_i
    \quad
    \SourceStepK{a}{e}{a'}{r'}
}
{
    \SourceStepK{\{a_i\}_i,a}{@let folds@~\{x_i~=~\_,~e_i\}_i~@in@~e}{\{a_i'\}_i,a'}{r'[\{x_i=r_i'\}_i]}
}{KFolds}
$$



$$
\boxed{\SourceStepX{e}{e}}
$$

$$
\ruleIN
{
    \SourceStepX{e_1}{e_1'}
}
{
    \SourceStepX{e_1~e_2}{e_1'~e_2}
}{XApp1}
\ruleIN
{
    \SourceStepX{e_2}{e_2'}
}
{
    \SourceStepX{v_1~e_2}{v_1~e_2'}
}{XApp2}
\ruleAx
{
    \SourceStepX{(\lambda{}x.~e_1)~e_2}{e_1[x:=e_2]}
}{XApp3}
$$
$$
\ruleIN
{
    \SourceStepX{e_1}{e_1'}
}
{
    \SourceStepX{@let@~x~=~e_1~@in@~e_2}{@let@~x~=~e_1'~@in@~e_2}
}{XLet1}
\ruleAx
{
    \SourceStepX{@let@~x~=~v_1~@in@~e_2}{e_2[x:=v_1]}
}{XLet2}
$$

\caption{Evaluation rules}
\label{fig:source:eval}
\end{figure*}




\subsection{Obvious extensions}
Obvious extensions and things I removed for simplification.

\subsubsection{Fixpoint}
$@fix@~(x~:~\tau)~@in@~e$.
By requiring $\tau$ to have kind @Data@ rather than @Flow@, we can allow recursion at runtime, while still ensuring a static dataflow graph.
I am tempted to kill the kinds for the simple version because with no fixpoint they do not serve much purpose (the dataflow graph is static regardless).

\subsubsection{ Maps and zips }
@mapS/mapF@, @zipS/zipF@.
For the simple version these could be merged into a single map/zip operating over just @Fold@s.
You can convert from a @Stream@ to a @Fold@ and back easily enough - you need a zero element for the @Fold@ but whatever.

\subsubsection{ Subrates }
$@let-rate@~(c'~:_k~@Clock@)~(w~:~c'~\le~c)~=~(e~:~@Stream@~c~\BB)$ and $@subrate@~(w~:~c'~\le~c)~(e~:~@Stream@~c~\tau)~:~@Stream@~c'~\tau$.
You can fake this by using Folds anyway.

\subsubsection{ Unpack }
$@let-unpack@~(x_\tau~:_k~@Data@)~(x_z~:~x_\tau)~(x_k~:~x_\tau~\to~@Fold@~x_\tau)~(x_r~:~x_\tau~\to~\tau)~=~(e~:~@Fold@~\tau)$.
Pulling apart folds to make new folds.
Useful for groups, segmented operations, and so on where the fold can be restarted or actually be many folds, as well as for other higher order folds.

