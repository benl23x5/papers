% !!!!!!!!!!!!!!!!!!!!!!!
% I commented the \input out in section/04-ILP.tex

\subsection{Proof}
To prove correctness of our linear program formulation, we need to prove two different things.
Firstly, the formulation's constraints must always be satisfiable; that is, there must exist a variable assignment that satisfies all constraints.
This is rather simple to show, but guarantees that the linear program will always give an answer.
The next thing to show is that any produced clustering is legal: if a variable assignment satisfies the constraints, then it is a valid and legal clustering.
This means that, not only do we get \emph{an} answer, we also get the \emph{right} answer.

%!TEX root = ../Main.tex
\subsubsection{Satisfiability}
For any program $p$, there exists a trivial clustering with no fusion at all.
We can use this as the variable assignment of $ilp(p)$.
For each pair of nodes $m,n \in p$, $x_{mn} = 1$ --- no fusion is possible.
For the $\pi$ variables, we must find a topographical ordering of the nodes in $p$, which is simple since we are assured it is a dag.

\TODO{Now, prove that this assignment actually satisfies the constraints.}


\subsubsection{Soundness}
For any program $p$ and variable asignment $v$, if $v$ satisfies the constraints for $ilp(p)$, the clustering denoted by $x_{ij}$ in $v$ is legal.

For a clustering to be legal, it must satisfy three constraints:
\begin{description}
\item[Acyclic]
after merging nodes of same cluster together, the resulting graph must be a dag
\item[Precedence preserving]
if there is an edge between two nodes $i$ and $j$, and they are not merged together, then we require $\pi_j > \pi_i$
\item[Fusion preventing]
likewise, if there is a fusion-preventing edge between two nodes $i$ and $j$, then we require $\pi_j > \pi_i$, which implies that they are not merged together
\item[Type constraint]
if two nodes $i$ and $j$ are in the same cluster, then $\tau_i = \tau_j$, or if $\tau_i$ is a subtype of $\tau_j$ (or $\tau_j$ is a subtype of $\tau_i$), then the \emph{generator} for $\tau_i$ (or $\tau_j$) must also be in the same cluster as $i$ and $j$.
    \\
    \TODO Actually, let us say $x_{ij} = 0 \implies check_{ij}$

\end{description}
where
\begin{tabbing}
MMMMM      \= M \= MMMMMMM \= MM \= \kill
$check$ \> @::@  \> $array \times array$ \> $\to$ \> $\mathbb{B}$ \\
MMMMM      \= M \= MMMMMM \= MM \= \kill
$check(i, j)$     \> $|$ \> $tau_i = tau_j$ \> $=$ \> $x_{i,j} = 0$                        \\
$check(i, j)$     \> $|$ \> $i' \in gen(i) $ \> $=$ \> $x_{i',j} = 0 \wedge x_{i,i'} = 0 \wedge check(i', j)$                        \\
$check(i, j)$     \> $|$ \> $j' \in gen(j) $ \> $=$ \> $x_{i,j'} = 0 \wedge x_{j,j'} = 0 \wedge check(i, j')$                        \\
$check(i, j)$     \> $|$ \> $tau_i \not= tau_j$\> $=$ \> $\bot$
\end{tabbing}







