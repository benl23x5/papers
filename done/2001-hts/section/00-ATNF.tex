
\section{The ATNF Vacation Scholarship Program}

The Australia Telescope National Facility (ATNF) is a government funded body that undertakes research in radio astronomy. It is currently operating as a national facility under the guidelines originally established by ASTEC in January 1984 [1]. The ATNF provides operational, research and engineering support for the Australia Telescope which includes the Narrabri Compact Array, the Parks Observatory and the Mopra Observatory. 

The vacation scholarship program is managed by the ATNF Scientific and Community Liaison group. Each year around 20 third and fourth year students studying engineering, physics or science are placed across the ATNF and the CSIRO Telecommunications and Industrial Physics organisation (CTIP). Students are given a chance to work on short 10 to 12 week projects in an area related to their organisations. They work as a member of their organisation for the duration of the scholarship program.

The ATNF receivers group where I was placed is responsible for producing and maintaining cooled radio receiver systems for the Australia Telescope. They are also involved in contract work for radio astronomy organisations in other countries. At group level each person has their own area of responsibility be it feed systems, cryogenics, low noise amplifiers, or mechanical systems. The group also interacts with CTIP who share their site at Marsfield, NSW, and have facilities at the Linfield National Measurement Laboratory (NML).
