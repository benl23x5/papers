%!TEX root = ../Main.tex
\section{Conclusion and Future Work}

Future work would be to repeat the filter sign process, but using a center frequency intentionally 35MHz too low, and have it fabricated on the other half of the wafer used for the first filter. The properties of the substrate should remain relatively constant over the entire wafer and if all goes to plan, the new filter would come out right on frequency.

This trick may or may not work if another wafer is used. There are a number of things to consider, one of them being whether the frequency shift really was due to inaccuracies in the dielectric constant, the simulation model being used, or a combination of both. There is also the question of how other properties of the material vary between wafers.

A more promising and reproducible avenue is to design a filter grossly too low, say 50MHz and have it trimmed down until it is at the right frequency. The Marsfield CTIP site has a focused ion beam machine which should be capable of trimming filters to the correct lengths. From simulation, it was found that the center frequency of the filter can be shifted by this amount without adversely effecting the overall filter response. 


