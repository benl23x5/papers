
\clearpage{}
\subsection{Effect joining in value types}

Consider the following source program:

\code{
	$\ifive$	& $= 5$ \\
	$f$		& \mc{3}{$= \kif \ ... \ $} \\
			& & $\kthen$ & $(\lambda (). \ \isucc \ \ifive)$  \\
			& & $\kelse$ & $(\lambda (). \ \kdo \ \{ \ \iputStr ``\texttt{hello}"; \ \isucc \ \ifive \} )$ 
}

\quad with

\code{
	$\iputStr$	
	& $::$		& $\forall r_1. \iString \ r_1 \lfuna{e_1} ()$	\\
	& $\rhd$	& $e_1 = \iRead \ r_1 \lor \iConsole$
}

Note that in the definition of $f$, the two functions in the right of the if-expression have different effects. The first reads the integer bound to $\ifive$, but the second also prints to the console. If we set $\ifive$ to have type $\iInt \ r_5$, then these two expressions have the following types:

\code{
	\mc{3}{$(\lambda x. \ \isucc \ \ifive)$}  \\
		& :: 		& $() \lfuna{e_1} \iInt \ r_1$ \\
		& $\rhd$	& $e_1 = \iRead \ r_5$ \\
}

\code{
	\mc{3}{$(\lambda (). \ \kdo \ \{ \ \iputStr ``\texttt{hello}"; \ \isucc \ \ifive \} )$}  \\
		& :: 		& $() \lfuna{e_2} \iInt \ r_1$ \\
		& $\rhd$	& $e_2 = \iRead \ r_5 \lor \iConsole$ \\
}

As it stands, we cannot translate this program directly to our core language, because the typing rule TyIf of \S\ref{Core:Simplified:TypesOfTerms} requires both alternatives to have similar types, inclusive of effect information. We support such programs by extending the definition of $\lor$ to join the effects contained within value types, and use this operator to compute the resulting type of if-expressions. This mirrors what happens during type inference.

\quad
\begin{tabular}{rll}
	$(\tau_{1} \lfuna{e_1} \tau_{2}) \lor (\tau_{1} \lfuna{e_2} \tau_{3})$
		& $\equiv$	& $\tau_{1} \funa{e_1 \lor e_2} (\tau_{2} \lor \tau_{3})$		
\end{tabular}

\ruleBox{
	\begin{gather}
	\ruleI	{TyIfJoin}
		{ \begin{aligned}
			\\
			\tyJudge	{\Gamma} {\Sigma} {t_1} {\iBool \ \varphi} {\sigma_1}
		  \end{aligned}
		  \tspace
		  \begin{aligned}
		  	\tyJudge	{\Gamma} {\Sigma&} {t_2} {\tau_2} {\sigma_2} \\
			\tyJudge	{\Gamma} {\Sigma&} {t_3} {\tau_3} {\sigma_3} 
		  \end{aligned}
		  \tspace 
		  \begin{aligned}
		  	\\
			\tau = \tau_2 \lor \tau_3
		  \end{aligned}
		}
		{ \tyJudgeGS	{\teIf {t_1} {t_2} {t_3}} {\tau} 
				{\sigma_1 \lor \sigma_2 \lor \sigma_3 \lor \iRead \ \varphi}
		}
	\end{gather}
}

\bigskip
Note that as our type inference algorithm only performs generalisation at let-bindings, the types of alternatives will never contain quantifiers. For this reason we don't need to define $\lor$ over quantified types. Also, as we only strengthen the manifest effects of a function type, the effect annotations on parameters will always be variables. Similarly to \S\ref{Core:Extensions:Bounded}, this guarantees that the effect annotations in function parameters are variables, so we don't have to join them.

An alternate method would be to provide an explicit type annotation for the result of the if-expression, and use the subsumption judgement to check that the types of both alternatives are less than the annotation. This approach was taken in \cite{nielson:from-cml-to-its-process-algebra}, but we avoid it because it increases the volume of annotation.
