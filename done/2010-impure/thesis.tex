\documentclass[a4paper,11pt]{book}

% -- setup
\usepackage{style/setup}

\begin{document}

% -- Title --------------------------------------
\begin{titlepage}
\begin{center}
\vspace*{\fill}
\Huge Type Inference and Optimisation 
	\\ for an Impure World.
\\
\vfill\vfill
\Large Ben Lippmeier
\\
\vfill\vfill
	DRAFT
\\
	June, 2009
\\
\vfill\vfill 
\normalsize
      A thesis submitted for the degree of Doctor of Philosophy\\
      of the Australian National University
\vfill
      \includegraphics{ANU.eps}
\end{center}

\end{titlepage}


% \title		
% {	Type Inference and Optimisation 
% 	\\ for an Impure World.}
% \author	{Ben Lippmeier}
% \date	{April, 2009}
%
% \maketitle


% -- Declaration ------------------------------------
\chapter*{Declaration}\label{declaration}
\thispagestyle{empty}
The work in this thesis is my own except where otherwise stated.

\vspace{1in}

\hfill\hfill\hfill
%
Ben Lippmeier
%
\hspace*{\fill}

% -- Acknowledgements --------------------------------
\chapter*{Thanks}\label{acknowledgements}
\addcontentsline{toc}{chapter}{Thanks}

Thanks to my family and friends for putting up with me and my occasional disappearances.

Sincere thanks to my supervisor Clem Baker-Finch, to Simon Marlow for an internship at GHC HQ, and to Bernie Pope for reading draft versions of this thesis.

Shouts out to the fp-syd crew, for pleasant monthly diversions when I probably should have been writing up instead.

Thanks to the General, the Particular and FRATER PERDURABO, who taught me that great truths are also great lies.

Thanks to the people who receive no thanks.

All Hail Discordia.


% -- Abstract --------------------------------------
\chapter*{Abstract}\label{abstract}
\addcontentsline{toc}{chapter}{Abstract}

We address the problem of reasoning about the behaviour of functional programs that use destructive update and other side effecting actions. All general purpose languages must support such actions, but adding them in an undisciplined manner destroys the nice algebraic properties that compiler writers depend on to transform code.

We present a type based mutability, effect and data sharing analysis for reasoning about such programs. Our analysis is based on a combination of Talpin and Jouvelot's type and effect discipline, and Leroy's closure typing system. The extra information is added to our System-F style core language and we use it to guide code transformation optimisations similar to those implemented in GHC. We use a type classing mechanism to express constraints on regions, effects and closures and show how these constraints can be represented in the same way as the type equality constraints of System-Fc. 

We use type directed projections to express records and to eliminate need for ML style mutable references. Our type inference algorithm extracts type constraints from the desugared source program, then solves them by building a type graph. This multi-staged approach is similar to that used by the Helium Haskell compiler, and the type graph helps to manage the effect and closure constraints of recursive functions. 

Our language uses call-by-value evaluation by default, but allows the seamless integration of call-by-need. We use our analysis to detect when the combination of side effects and call-by-need evaluation would yield a result different from the call-by-value case. We contrast our approach with several other systems, and argue that it is more important for a compiler to be able to reason about the behaviour of a program, than for the language to be purely functional in a formal sense.

As opposed to using source level state monads, effect typing allows the programmer to offload the task of maintaining the intended sequence of effects onto the compiler. This helps to separate the conceptually orthogonal notions of value and effect, and reduces the need to refactor existing code when developing programs. We discuss the Disciplined Disciple Compiler (DDC), which implements our system, along with example programs and opportunities for future work.

% -- Layout -----------------------------------------
\chapter*{Layout}
\addcontentsline{toc}{chapter}{Layout}

The layout and formatting of this thesis was influenced by the work of Edward R. Tufte, especially his book ``Visual Explanations''. To reduce the need for the reader to flip pages back and forth when progressing through the material, I have avoided using floating figures, and have tried to keep related diagrams and discussion on the same two page spread. For this reason some diagrams have been repeated, and some sections contain extra white space.


% -- Contents ---------------------------------------
\tableofcontents


% -- Chapters --------------------------------------------
\documentclass[preprint]{sigplanconf}
\usepackage{amssymb}
\usepackage{amsthm}
\usepackage{graphicx}
\usepackage{amsmath}
\usepackage{mathptmx}
\usepackage{mathtools}
\usepackage{stmaryrd}
\usepackage{hyperref}
\usepackage{alltt}
\usepackage{url}
\usepackage{float}
\usepackage{style/code}
\usepackage{style/proof}
\usepackage{style/utils}
\usepackage{style/judgements}

% -----------------------------------------------------------------------------
\begin{document}

% \exclusivelicense
% \conferenceinfo{}{}
% \copyrightyear{2015}
% \copyrightdata{}
\doi{}
% \pagenumbering{gobble} 

\title{Icicle: fuse your queries}

\authorinfo{
  Amos Robinson$^\dagger$$^\ddagger$
  \and Ben Lippmeier$^\dagger$
}{
  \vspace{5pt}
  \shortstack{
    $^\dagger$Computer Science and Engineering \\
    University of New South Wales, Australia \\[2pt]
    \textsf{amosr,benl@cse.unsw.edu.au}
  }
  \shortstack{
    $^\ddagger$Ambiata      \\
    Big data and shit       \\[2pt]
    \textsf{amos.robinson@ambiata.com}
  }
}

\maketitle
\makeatactive

\begin{abstract}
When streaming a large amount of data, simply iterating over the data may take hours.
If multiple queries are to be performed, it is important that work is not duplicated.
Queries that can be performed together must be performed in the same iteration.
We introduce a simple streaming language for computing queries in a single-pass over the data.
By using an appropriate intermediate language we guarantee fusion between queries on the same input streams, before extracting efficient C code.
\end{abstract}


\category
	{D.3.4}
	{Programming Languages}
	{Processors---Compilers; Optimization}

\terms
	Languages, Performance

\keywords
	Arrays; Fusion


\section{Introdution}

This is some stuff.

%!TEX root = ../Main.tex
\section{Icicle Source}
\label{s:Source}

The two main types in Icicle are @Stream@ and @Fold@.
Streams represent data values as they flow through the program.
Streams do not always have data flowing through them, but have an associated clock which describes when data occurs.
Two streams with the same clock both have data at the same time.
Folds, on the other hand, are results of computations over the stream data that has been seen.
Folds always have a well-defined value.
Folds can be easily converted to streams, by sampling their current value whenever the stream clock is true.
It is harder to convert a stream to a fold, as the stream has gaps where it is undefined, and one must specify how to fill the gaps (hold last, fill with zero, etc).
Finally, folds can be computed recursively based on their previous values.

%!TEX root = ../Main.tex

\begin{figure}

\begin{tabbing}
MMMM \= MM \= MMMMMMMMMMM \= MM \= \kill
$\mi{e}$
\GrammarDef $x$
\> $|$ \> $v$
\GrammarAlt $e~e$
\> $|$ \> $\lambda{}x~:~\tau.~e$

\\
\GrammarAlt $@when@~e~e$
\> $|$ \> $@sample@~e~e$
\GrammarAlt $@mapS@~e~e$
\> $|$ \> $@zipS@~e~e$
\GrammarAlt $@mapF@~e~e$
\> $|$ \> $@zipF@~e~e$
\\
\GrammarAlt $@let@~\mi{Let}~@in@~e$
\\
\GrammarAlt $\langle e,~ e \rangle ~|~ \langle\rangle$
\\
\\

$\mi{Let}$
\GrammarDef $x~=~e$
\GrammarAlt $@folds@~\{~x_i~=~e_i,~e_i~\}_i$
\\
\\

$\mi{v}$
\GrammarDef $\NN ~|~ \BB ~|~ [v]$
\GrammarAlt $(v,~v) ~|~ \bot$
\\
\\


$\mi{Program}$
\GrammarDef @inputs@~$\{ x_i~:~@Stream@~c_i~\tau_i \}_i~@in@~\mi{e}$ \\
\end{tabbing}

\caption{Grammar for Icicle Source}
\label{fig:source:grammar}
\end{figure}


The grammar for Icicle is given in figure~\ref{fig:source:grammar}.
The first four rules are rather standard lambda calculus.

Next are the primitives.
@when@ takes a stream and a fold, and returns a new fold whose value is the stream \emph{when}ever the stream is defined, and the fold otherwise.
This is used for defining folds that depend on stream values.
@sample@ takes a stream of any type and a fold, creates a new stream using the values of the fold.
@mapS@ and @zipS@ perform map and zip operations on streams of the same rate.
@mapF@ and @zipF@ perform map and zip operations on folds.

Let expressions such as $@let@~x~=e~@in@~e$ are as usual. 
Folds are defined using the syntax @let folds@ syntax. 
Multiple folds can be defined together, and each fold is defined by the initial value, and the ``kons'' part, defining the next value.

For example, a simple fold comprised of nothing but $0$
\begin{tabbing}
MM \= \kill
$@let folds@~\mi{zeros}~=~0,~\mi{zeros}$ \\
$@in@~\mi{zeros}$
\\
\\
$\mi{zeros}$ \> $=~\langle 0~0~0~0~\cdots~\rangle$ \\
\end{tabbing}

When computing the sum over another fold, the newly-defined fold and the input fold are zipped and then added together.
\begin{tabbing}
$\mi{sum}~=~\lambda{}(\mi{values}~:~@Fold@~\NN).$ \\
$@let folds@~\mi{s}~=~0,~@mapF@~(+)~(@zipF@~\mi{s}~\mi{values})$ \\
$@in@~\mi{s}$
\\
\\
$\mi{sum}~\langle 0~1~2~3~\cdots~\rangle$ \\
$~=~\langle 0~1~3~6~\cdots~\rangle$ \\
\end{tabbing}

This $\mi{sum}$ function requires a fold argument, but it can be applied to a stream $s$ by creating a fold that is the stream when it is defined, and $0$ otherwise: $@when@~s~zeros$.
A more general way is to define a new $\mi{sum'}$ function that works directly over streams:
\begin{tabbing}
MMMMMM \= MM \= MM \kill
$\mi{sum'}~=~\lambda{}(\mi{values}~:~@Stream@~c~\NN).$ \\
$@let folds@~\mi{s}~=~0,$ \\
\> $@when@$ \> $(@mapS@~(+)~(@zipS@~(@sample@~\mi{values}~\mi{s})~\mi{values}))$ \\
\> \> $~\mi{s}$ \\
$@in@~\mi{s}$
\\
\\
$\mi{sum'}~\langle \bot~1~\bot~3~\cdots~\rangle$ \\
$~=~\langle 0~1~1~4~\cdots~\rangle$ \\
\end{tabbing}

Here, the fold $s$ is sampled to a stream with the same clock as the input stream.
Whenever the input stream is defined, the previous value of $s$ and the current value of the $\mi{values}$ will be added together, producing the new sum.
When the input stream is not defined, the previous value of $s$ is used unchanged.

When referring to the newly defined folds in the right-hand-side of the fold (the kons), all values refer to the immediate previous value.
The order of the fold bindings has no effect on the program.
\begin{tabbing}
MM \= \kill
$@let folds@$ \\
\> $\mi{one}~=~1,~\mi{zero}$ \\
\> $\mi{zero}~=~0,~\mi{one}$ \\
$@in@~\mi{one}$
\\
\\
$\mi{one}$  \> $=~\langle 1~0~1~0~\cdots~\rangle$ \\
$\mi{zero}$ \> $=~\langle 0~1~0~1~\cdots~\rangle$ \\
\end{tabbing}

The grammar also has rules for stream literals: $\langle e_1,~e_2 \rangle$ and $\langle\rangle$. The stream can contain bottoms, $\bot$, which means that the stream is undefined at that point.
Stream literals have concrete clock types, for example $\langle T,~F \rangle$ for $\langle 1,~\bot \rangle$, while input streams have existential clocks: their actual clock is unknown until runtime.
These stream literals are not expected to occur in an actual program until evaluation.


Yes, I realise the grammar doesn't have @+@ or @/@, and that it doesn't have polymorphism.
For the examples, assume that a specialisation pass occurs, and that there is a ``sensible set of primitives on natural numbers and lists''.

Define group here.
\begin{code}
data Unpack r
 = forall a.
    Unpack a (Fold (a -> a)) (a -> r)

sum' vs = Unpack 0 (+vs) id

group (key : Stream c k)
      (val : Unpack v)
           : [(k,v)]
group key (Unpack z k r)
 = trivial
\end{code}



%!TEX root = ../Main.tex

\begin{figure*}

$$
\boxed{\SourceStepZ{e}{\mi{acc}}{\mi{res}}}
$$

$$
\ruleAx
{
    \SourceStepZ{x}{()}{x}
}{ZVar}
\ruleAx
{
    \SourceStepZ{v}{()}{v}
}{ZValue}
\ruleIN
{
    \SourceStepZ{e}{a}{r}
}
{
    \SourceStepZ{\lambda{}x~:~\tau.~e}{a}{\lambda{}x~:~\tau.~r}
}{ZLam}
$$

$$
\ruleAx
{
    \SourceStepZ{\langle e_1,~e_2 \rangle}{\langle e_1,~e_2 \rangle}{\bot}
}{ZStream1}
\ruleAx
{
    \SourceStepZ{\langle \rangle}{\langle \rangle}{\bot}
}{ZStream2}
\ruleIN
{
    \SourceStepZ{e_1}{a_1}{r_1}
    \quad
    \SourceStepZ{e_2}{a_2}{r_2}
}
{
    \SourceStepZ{@when@~e_1~e_2}{a_1,a_2}{r_2}
}{ZWhen}
\ruleIN
{
    \SourceStepZ{e_1}{a_1}{r_1}
    \quad
    \SourceStepZ{e_2}{a_2}{r_2}
}
{
    \SourceStepZ{@sample@~e_1~e_2}{a_1,a_2}{\bot}
}{ZSample}
$$

$$
\ruleIN
{
    \SourceStepZ{e}{a}{r}
}
{
    \SourceStepZ{@mapS@~f~e}{a}{\bot}
}{ZMapS}
\ruleIN
{
    \SourceStepZ{e_1}{a_1}{r_1}
    \quad
    \SourceStepZ{e_2}{a_2}{r_2}
}
{
    \SourceStepZ{@zipS@~e_1~e_2}{a_1,a_2}{\bot}
}{ZZipS}
$$

$$
\ruleIN
{
    \SourceStepZ{e}{a}{r}
}
{
    \SourceStepZ{@mapF@~f~e}{a}{f~r}
}{ZMapF}
\ruleIN
{
    \SourceStepZ{e_1}{a_1}{r_1}
    \quad
    \SourceStepZ{e_2}{a_2}{r_2}
}
{
    \SourceStepZ{@zipF@~e_1~e_2}{a_1,a_2}{r_1,r_2}
}{ZZipF}
$$

$$
\ruleIN
{
    \SourceStepZ{e_1}{a_1}{r_1}
    \quad
    \SourceStepZ{e_2}{a_2}{r_2}
}
{
    \SourceStepZ{@let@~x~=~e_1~@in@~e_2}{a_1,a_2}{r_2[x=r_1]}
}{ZLet}
\ruleIN
{
    \{ \SourceStepZ{e_i}{a_i}{r_i} \}_i
    \quad
    \SourceStepZ{e}{a}{r}
}
{
    \SourceStepZ{@let folds@~\{x_i~=~e_i,~\_\}_i~@in@~e}{\{a_i\}_i,a}{r[\{x_i=r_i\}_i]}
}{ZFolds}
$$

$$
\boxed{\SourceStepK{\mi{acc}}{e}{\mi{acc}}{\mi{res}}}
$$


$$
\ruleAx
{
    \SourceStepK{()}{x}{()}{x}
}{KVar}
\ruleAx
{
    \SourceStepK{()}{v}{()}{v}
}{KValue}
\ruleIN
{
    \SourceStepK{a}{e}{a'}{r'}
}
{
    \SourceStepK{a}{\lambda{}x~:~\tau.~e}{a'}{\lambda{}x~:~\tau.~r'}
}{KLam}
$$

$$
\ruleAx
{
    \SourceStepK{\langle e_1,~e_2 \rangle}{\langle \cdots \rangle}{e_2}{e_1}
}{KStream1}
\ruleAx
{
    \SourceStepK{\langle \rangle}{\langle \cdots \rangle}{\langle\rangle}{\bot}
}{KStream2}
\ruleIN
{
    \SourceStepK{a_1}{e_1}{a_1'}{r_1'}
    \quad
    \SourceStepK{a_1}{e_2}{a_2'}{r_2'}
}
{
    \SourceStepK{a_1,a_2}{@when@~e_1~e_2}{a_1',a_2'}{\Ifnbot{r_1'}{r_1'}{r_2'}}
}{KWhen}
$$

$$
\ruleIN
{
    \SourceStepK{a_1}{e_1}{a_1'}{r_1'}
    \quad
    \SourceStepK{a_2}{e_2}{a_2'}{r_2'}
}
{
    \SourceStepK{a_1,a_2}{@sample@~e_1~e_2}{a_1',a_2'}{\Ifnbot{r_1'}{r_2'}{\bot}}
}{KSample}
$$

$$
\ruleIN
{
    \SourceStepK{a}{e}{a'}{r'}
}
{
    \SourceStepK{a}{@mapS@~f~e}{a'}{\Ifnbot{r'}{f~r'}{\bot}}
}{KMapS}
\ruleIN
{
    \SourceStepK{a_1}{e_1}{a_1'}{r_1'}
    \quad
    \SourceStepK{a_2}{e_2}{a_2'}{r_2'}
}
{
    \SourceStepK{a_1,a_2}{@zipS@~e_1~e_2}{a_1',a_2'}{r_1',r_2'}
}{KZipS}
$$

$$
\ruleIN
{
    \SourceStepK{a}{e}{a'}{r'}
}
{
    \SourceStepK{a}{@mapF@~f~e}{a'}{f~r}
}{KMapF}
\ruleIN
{
    \SourceStepK{a_1}{e_1}{a_1'}{r_1'}
    \quad
    \SourceStepK{a_2}{e_2}{a_2'}{r_2'}
}
{
    \SourceStepK{a_1,a_2}{@zipF@~e_1~e_2}{a_1',a_2'}{r_1',r_2'}
}{KZipF}
$$

$$
\ruleIN
{
    \SourceStepK{a_1}{e_1}{a_1'}{r_1'}
    \quad
    \SourceStepK{a_2}{e_2}{a_2'}{r_2'}
}
{
    \SourceStepK{a_1,a_2}{@let@~x~=~e_1~@in@~e_2}{a_1',a_2'}{r_2[x=r_1]}
}{KLet}
\ruleIN
{
    \{ \SourceStepK{a_i}{e_i}{a_i'}{r_i'} \}_i
    \quad
    \SourceStepK{a}{e}{a'}{r'}
}
{
    \SourceStepK{\{a_i\}_i,a}{@let folds@~\{x_i~=~\_,~e_i\}_i~@in@~e}{\{a_i'\}_i,a'}{r'[\{x_i=r_i'\}_i]}
}{KFolds}
$$



$$
\boxed{\SourceStepX{e}{e}}
$$

$$
\ruleIN
{
    \SourceStepX{e_1}{e_1'}
}
{
    \SourceStepX{e_1~e_2}{e_1'~e_2}
}{XApp1}
\ruleIN
{
    \SourceStepX{e_2}{e_2'}
}
{
    \SourceStepX{v_1~e_2}{v_1~e_2'}
}{XApp2}
\ruleAx
{
    \SourceStepX{(\lambda{}x.~e_1)~e_2}{e_1[x:=e_2]}
}{XApp3}
$$
$$
\ruleIN
{
    \SourceStepX{e_1}{e_1'}
}
{
    \SourceStepX{@let@~x~=~e_1~@in@~e_2}{@let@~x~=~e_1'~@in@~e_2}
}{XLet1}
\ruleAx
{
    \SourceStepX{@let@~x~=~v_1~@in@~e_2}{e_2[x:=v_1]}
}{XLet2}
$$

\caption{Evaluation rules}
\label{fig:source:eval}
\end{figure*}




\subsection{Obvious extensions}
Obvious extensions and things I removed for simplification.

\subsubsection{Fixpoint}
$@fix@~(x~:~\tau)~@in@~e$.
By requiring $\tau$ to have kind @Data@ rather than @Flow@, we can allow recursion at runtime, while still ensuring a static dataflow graph.
I am tempted to kill the kinds for the simple version because with no fixpoint they do not serve much purpose (the dataflow graph is static regardless).

\subsubsection{ Maps and zips }
@mapS/mapF@, @zipS/zipF@.
For the simple version these could be merged into a single map/zip operating over just @Fold@s.
You can convert from a @Stream@ to a @Fold@ and back easily enough - you need a zero element for the @Fold@ but whatever.

\subsubsection{ Subrates }
$@let-rate@~(c'~:_k~@Clock@)~(w~:~c'~\le~c)~=~(e~:~@Stream@~c~\BB)$ and $@subrate@~(w~:~c'~\le~c)~(e~:~@Stream@~c~\tau)~:~@Stream@~c'~\tau$.
You can fake this by using Folds anyway.

\subsubsection{ Unpack }
$@let-unpack@~(x_\tau~:_k~@Data@)~(x_z~:~x_\tau)~(x_k~:~x_\tau~\to~@Fold@~x_\tau)~(x_r~:~x_\tau~\to~\tau)~=~(e~:~@Fold@~\tau)$.
Pulling apart folds to make new folds.
Useful for groups, segmented operations, and so on where the fold can be restarted or actually be many folds, as well as for other higher order folds.


%!TEX root = ../Main.tex
\section{Types}
\label{s:Types}

%!TEX root = ../Main.tex

\begin{figure*}

\begin{tabbing}
MM \= MM \= \kill
$\mi{Kind}$
\GrammarDef $@Data@~|~@Clock@~|~@Flow@$ \\

T
\GrammarDef $x~|~\NN~|~\BB~|~@()@
    ~|~ @List@~T
    ~|~ T~\to~T
    ~|~ (T,T)$
\GrammarAlt $@Stream@~T~T
    ~|~ @Fold@~T$
    \\
\end{tabbing}

\caption{Types and kinds}
\label{fig:source:type:types}
\end{figure*}


\begin{figure*}

$$
\boxed{\TypeWf{\Delta}{\tau}{\mi{Kind}}}
$$

$$
\ruleIN
{
    x~:_k~k~\in~\Delta
}
{
    \TypeWf{\Delta}{x}{k}
}{TVar}
\ruleAx
{
    \TypeWf{\Delta}{\NN}{@Data@}
}{TNat}
\ruleAx
{
    \TypeWf{\Delta}{\BB}{@Data@}
}{TBool}
\ruleAx
{
    \TypeWf{\Delta}{@()@}{@Data@}
}{TUnit}
$$

$$
\ruleIN
{
    \TypeWf{\Delta}{\tau}{@Data@}
}
{
    \TypeWf{\Delta}{@List@~\tau}{@Data@}
}{TList}
\ruleIN
{
    \TypeWf{\Delta}{\tau_1}{k_1}
    \quad
    \TypeWf{\Delta}{\tau_2}{k_2}
}
{
    \TypeWf{\Delta}{\tau_1~\to~\tau_2}{k_2}
}{TFun}
$$

$$
\ruleIN
{
    \TypeWf{\Delta}{c}{@Clock@}
    \quad
    \TypeWf{\Delta}{\tau}{@Data@}
}
{
    \TypeWf{\Delta}{@Stream@~c~\tau}{@Flow@}
}{TStream}
\ruleIN
{
    \TypeWf{\Delta}{\tau}{@Data@}
}
{
    \TypeWf{\Delta}{@Fold@~\tau}{@Flow@}
}{TFold}
$$



\caption{Kinds of types}
\label{fig:source:type:kinds}
\end{figure*}


\begin{figure*}

$$
\boxed{\Typecheck{\Delta}{\Gamma}{e}{T}}
$$


$$
\ruleIN
{
    (x~:~\tau)~\in~\Gamma
}
{ 
    \Typecheck{\Delta}{\Gamma}{x}{\tau}
}{TcVar}
\ruleIN
{
    v~:~\tau
}
{ 
    \Typecheck{\Delta}{\Gamma}{v}{\tau}
}{TcValue}
\ruleIN
{
    \Typecheck{\Delta}{\Gamma}{e_1}{\tau_1~\to~\tau_2}
    \quad
    \Typecheck{\Delta}{\Gamma}{e_2}{\tau_1}
}
{ 
    \Typecheck{\Delta}{\Gamma}{e_1~e_2}{\tau_2}
}{TcApp}
$$

$$
\ruleIN
{
    \Typecheck{\Delta}{\Gamma,~x~:~\tau}{e}{\tau'}
}
{
    \Typecheck{\Delta}{\Gamma}{\lambda{}x~:~\tau.~e}{\tau~\to~\tau'}
}{TcLam}
$$

$$
\ruleAx
{
    \Typecheck{\Delta}{\Gamma}{\langle \rangle}{@Stream@~\langle\rangle~\tau}
}{TcStrm1}
\ruleIN
{
    \Typecheck{\Delta}{\Gamma}{e_1}{\tau}
    \quad
    \Typecheck{\Delta}{\Gamma}{e_2}{@Stream@~c~\tau}
}
{
    \Typecheck{\Delta}{\Gamma}{\langle e_1,~e_2 \rangle}{@Stream@~\langle @T@,~c \rangle~\tau}
}{TcStrm2}
\ruleIN
{
    \Typecheck{\Delta}{\Gamma}{e_2}{@Stream@~c~\tau}
}
{
    \Typecheck{\Delta}{\Gamma}{\langle \bot,~e_2 \rangle}{@Stream@~\langle @F@,~c \rangle~\tau}
}{TcStrm3}
$$

$$
\ruleIN
{
    \Typecheck{\Delta}{\Gamma}{e_1}{@Stream@~c~\tau}
    \quad
    \Typecheck{\Delta}{\Gamma}{e_2}{@Fold@~\tau}
}
{
    \Typecheck{\Delta}{\Gamma}{@when@~e_1~e_2}{@Fold@~\tau}
}{TcWhen}
\ruleIN
{
    \Typecheck{\Delta}{\Gamma}{e_1}{@Stream@~c~\tau'}
    \quad
    \Typecheck{\Delta}{\Gamma}{e_2}{@Fold@~\tau}
}
{
    \Typecheck{\Delta}{\Gamma}{@sample@~e_1~e_2}{@Stream@~c~\tau}
}{TcSample}
$$

$$
\ruleIN
{
    \Typecheck{\Delta}{\Gamma}{e_1}{\tau_1~\to~\tau_2}
    \quad
    \Typecheck{\Delta}{\Gamma}{e_2}{@Stream@~c~\tau_1}
}
{
    \Typecheck{\Delta}{\Gamma}{@mapS@~e_1~e_2}{@Stream@~c~\tau_2}
}{TcMapS}
\ruleIN
{
    \Typecheck{\Delta}{\Gamma}{e_1}{@Stream@~c~\tau_1}
    \quad
    \Typecheck{\Delta}{\Gamma}{e_2}{@Stream@~c~\tau_2}
}
{
    \Typecheck{\Delta}{\Gamma}{@zipS@~e_1~e_2}{@Stream@~c~(\tau_1,\tau_2)}
}{TcZipS}
$$

$$
\ruleIN
{
    \Typecheck{\Delta}{\Gamma}{e_1}{\tau_1~\to~\tau_2}
    \quad
    \Typecheck{\Delta}{\Gamma}{e_2}{@Fold@~\tau_1}
}
{
    \Typecheck{\Delta}{\Gamma}{@mapF@~e_1~e_2}{@Fold@~\tau_2}
}{TcMapF}
\ruleIN
{
    \Typecheck{\Delta}{\Gamma}{e_1}{@Fold@~\tau_1}
    \quad
    \Typecheck{\Delta}{\Gamma}{e_2}{@Fold@~\tau_2}
}
{
    \Typecheck{\Delta}{\Gamma}{@zipF@~e_1~e_2}{@Fold@~(\tau_1,\tau_2)}
}{TcZipF}
$$

$$
\ruleIN
{
    \Typecheck{\Delta}{\Gamma}{e_1}{\tau_1}
    \quad
    \Typecheck{\Delta}{\Gamma,~x~:~\tau_1}{e_2}{\tau_2}
}
{
    \Typecheck{\Delta}{\Gamma}{@let@~x~=~e_1~@in@~e_2}{\tau_2}
}{TcLet}
$$

$$
\ruleIN
{
    \Gamma'~=~\Gamma,~\{~x_i~:~@Fold@~\tau_i~\}_i
    \quad
    \{
    \Typecheck{\Delta}{\Gamma}{e_{z_i}}{\tau_i}
    \}_i
    \quad
    \{
    \Typecheck{\Delta}{\Gamma'}{e_{k_i}}{@Fold@~\tau_i}
    \}_i
    \quad
    \Typecheck{\Delta}{\Gamma'}{e}{\tau}
}
{
    \Typecheck{\Delta}{\Gamma}
        {@let@~@folds@~\{~x_i~=~e_{z_i}~@then@~e_{k_i}~\}_i~@in@~e}
        {\tau}
}{TcLetFolds}
$$


\caption{Types of expressions}
\label{fig:source:type:exp}
\end{figure*}




%!TEX root = ../Main.tex
\section{Core language}
\label{s:Core}

This section presents the Core language, which we use as an intermediate language before converting to imperative code.


\subsection{Grammar}

%!TEX root = ../Main.tex

\begin{figure}

\begin{tabbing}
MMMM \= MM \= MM \= MM \= \kill
$\mi{e_c}$
\GrammarDef $x$ \> $|$ \> $v$
\GrammarAlt $e_c~e_c$ \> $|$ \> $p$
\\
\\
$c$
\GrammarDef $x$ \> $|$ \> $\mi{Always}$
\\
\\
$\mi{FoldNest}$
\GrammarDef $@nest@~\mi{Op}*$
\\
\\
$\mi{Op}$
\GrammarDef $@let@~x~=~e_c~@when@~c$
\GrammarAlt $@fold@~x~=~e_c,~(e_c~@when@~c~@else@~e_c)$
\\
\\
$\mi{Bind}$
\GrammarDef $x~=~\lambda{}x.~e_c$
\\
\\
$\mi{Program}$
\GrammarDef @inputs@ \\
    \>      \> \> $(x~:~@Stream@~c~\tau)*$ \\
    \>      \> @binds@ \\
    \>      \> \> $\mi{Bind}*$ \\
    \>      \> @with@ \\
    \>      \> \> $\mi{FoldNest}*$ \\
    \>      \> @in@ \\
    \>      \> \> $x*$ \\
\end{tabbing}


\caption{Grammar for Icicle Core}
\label{fig:core:grammar}
\end{figure}



%!TEX root = ../Main.tex

\begin{figure*}

$$
\boxed{\CoreComp{e}{[Bind]}{[Op]}{x}}
$$

$$
\ruleAx
{
    \CoreComp{v}{\emptyset}{@let@~x'~=~v}{x'}
}{CVal}
\ruleAx
{
    \CoreComp{x}{\emptyset}{\emptyset}{x}
}{CVar}
\ruleIN
{
    \CoreComp{e_1}{b_1}{o_1}{x_1}
    \quad
    \CoreComp{e_2}{b_2}{o_2}{x_2}
}{
    \CoreComp{e_1~e_2}{b_1;b_2}{o_1;o_2;~@let@~x'~=~x_1~x_2}{x'}
}{CApp}
$$


\caption{Conversion to Core}
\label{fig:core:compile}
\end{figure*}





%!TEX root = ../Main.tex
\section{Code generation}
\label{s:Generation}

Code generation should be pretty easy I guess cite flow fusion paper.

Because we don't have any zips or other that takes multiple streams, we don't have to deal with hard problems.

While not all streams have the same rate, there is no way to express an operation that takes two different rates.


%!TEX root = ../Main.tex
\section{Conclusion}
\label{s:Conclusion}





%!TEX root = ../Main.tex
\section{Related work}
\label{s:Related}
\subsection{Data flow languages}

The closest related work are synchronous data flow languages such as {\sc Lustre}, Icicle programs are restricted to those that can be computed in bounded memory.
{\sc Lustre}\cite{halbwachs1991synchronous} achieves this in three main ways:

\begin{enumerate}
\item They allow only restricted set of primitives such as @when@ for filtering, @pre@ for a single element buffer, and so on.
\item Cycles in the graph must contain at least one @pre@, to break the dependency loop.
\item Operators such as addition can only be applied to input streams with the same clock or rate; an expression like @(X when C)@ @+@ @(Y when (not C))@ would require an unbounded buffer\CITE{Lustre clock stuff}.
\end{enumerate}

In summary, while our language is quite similar to existing synchronous data flow languages, the extra restrictions we impose allow us to use a simpler type system.


% %!TEX root = ../Main.tex

\newcommand\JudgeK[2]
{       #1 :: #2
}

\newcommand\JudgeT[3]
{       #1 \vdash #2 :: #3
}

\newcommand\JudgeTS[5]
{       #1 \vdash #2 :: #3 ~;~ #4 ~;~ #5
}

\newcommand\kbox        {\textrm{\textbf{box}}}
\newcommand\krun        {\textrm{\textbf{run}}}
\newcommand\kthen       {\textrm{\textbf{then}}}
\newcommand\kpure       {\textrm{\textbf{pure}}}

\newcommand\rData       {\textrm{Data}}
\newcommand\rClock      {\textrm{Clock}}
\newcommand\rDefn       {\textrm{Defn}}
\newcommand\rComp       {\textrm{Comp}}

\newcommand\rStream     {\textrm{Stream}}
\newcommand\rArray      {\textrm{Array}}
\newcommand\rNat        {\textrm{Nat}}
\newcommand\rBool       {\textrm{Bool}}

\newcommand\rdrain      {\textrm{drain}}
\newcommand\rstream     {\textrm{stream}}
\newcommand\rsum        {\textrm{sum}}
\newcommand\rsmap       {\textrm{smap}}
\newcommand\rsfold      {\textrm{sfold}}
\newcommand\rsscan      {\textrm{sscan}}
\newcommand\rsfilter    {\textrm{sfilter}}


\begin{figure*}
$$
\boxed{\JudgeT{\Gamma}{e}{\tau}}
$$

$$
\ruleI
{       x : \tau \in \Gamma
}
{       \JudgeT{\Gamma}{x}{\tau}
}
\textrm{(TyVar)}
\quad\quad
\ruleI
{       \JudgeT{\Gamma}{e_1}{\tau_1 \to \tau_2}
        \quad
        \JudgeT{\Gamma}{e_2}{\tau_1}
}
{       \JudgeT{\Gamma}{e_1~e_2}{\tau_2}       
}
\textrm{(TyApp)}
\quad\quad
\ruleI
{       \JudgeT{\Gamma,x:\tau_1}{e_2}{\tau_2}
}
{       \JudgeT{\Gamma}{\lambda x : \tau_1}{\tau_1 \to \tau_2}
}
\textrm{(TyAbs)}
$$



% -- Effectful --------------------------------------------
$$
\boxed{\JudgeTS{\Gamma}{e}{\tau}{n}{k}}
$$

$$
\ruleI
{       \JudgeTS{\Gamma}{e}{\tau}{n}{k}
}
{       \JudgeT{\Gamma}{\kbox~ e}{\Box~ n~ k~ \tau}
}
\textrm{(TyBox)}
\quad\quad
\ruleI
{       \JudgeT{\Gamma}{e}{\tau}
}
{       \JudgeTS{\Gamma}{\kpure~e}{\tau}{n}{k}
}
\textrm{(TyPure)}
$$

$$
\ruleI
{       \{ \JudgeT
                {\Gamma}{e_i}{\Box~ n_i~ k~ \tau_i} \}^i 
        \quad
        \JudgeTS
                {\Gamma,~ \{x_i : \tau_i \}^i}
                {e'}{\tau'}{n'}{k'}
}
{       \JudgeTS
                {\Gamma}
                {\krun~ \{ x_i : \tau_i = e_i \}^i ~\kthen~ e'}
                {\tau'}
                {max~ \{ n_i \}^i + n'}
                {k'}
}
\textrm{(TyRun)}
$$

% -- Prims ------------------------------------------------
\vspace{2em}
$$
\begin{array}{ll}
\rNat,\rBool      & :: \rData
\\[1ex]
%
\rArray           & :: \rClock \to \rData \to \rData
\\[1ex]
%
\rStream          & :: \rClock \to \rData \to \rDefn
\\[1ex]
%
\Box              & :: \rNat   \to \rClock \to \rData \to \rComp
\\[1ex]
%
\rdrain_{k,\tau}  & :: \rStream~k~\tau \to \Box^1_k~ (\rArray~ k~ \tau)
\\[1ex]
%
\rstream_{k,\tau} & :: \rArray~k~\tau  \to \rStream~k~\tau
\\[1ex]
%
\rsum_k           & :: \rStream~k~\rNat \to \Box^1_k~ \rNat
\\[1ex]
%
\rsmap_{k,a,b}    & :: (a \to b) \to \rStream~k~a \to \rStream~k~b
\\[1ex]
%
\rsfold_{k,a,b}   & :: (a \to b \to b) \to b \to \rStream~k~a \to \Box^1_k~ b
\\[1ex]
%
\rsscan_{k,a,b}   & :: (a \to b \to b) \to b \to \rStream~k~a \to \rStream~k~b
\end{array}
$$

\caption{Typing}
\label{fig:source:type:modal}
\end{figure*}


\section*{Acknowledgements}

\bibliographystyle{plain}
\bibliography{Main}

\end{document}



\documentclass[preprint]{sigplanconf}
\usepackage{amssymb}
\usepackage{amsthm}
\usepackage{graphicx}
\usepackage{amsmath}
\usepackage{mathptmx}
\usepackage{mathtools}
\usepackage{stmaryrd}
\usepackage{hyperref}
\usepackage{alltt}
\usepackage{url}
\usepackage{float}
\usepackage{style/code}
\usepackage{style/proof}
\usepackage{style/utils}
\usepackage{style/judgements}

% -----------------------------------------------------------------------------
\begin{document}

% \exclusivelicense
% \conferenceinfo{}{}
% \copyrightyear{2015}
% \copyrightdata{}
\doi{}
% \pagenumbering{gobble} 

\title{Icicle: fuse your queries}

\authorinfo{
  Amos Robinson$^\dagger$$^\ddagger$
  \and Ben Lippmeier$^\dagger$
}{
  \vspace{5pt}
  \shortstack{
    $^\dagger$Computer Science and Engineering \\
    University of New South Wales, Australia \\[2pt]
    \textsf{amosr,benl@cse.unsw.edu.au}
  }
  \shortstack{
    $^\ddagger$Ambiata      \\
    Big data and shit       \\[2pt]
    \textsf{amos.robinson@ambiata.com}
  }
}

\maketitle
\makeatactive

\begin{abstract}
When streaming a large amount of data, simply iterating over the data may take hours.
If multiple queries are to be performed, it is important that work is not duplicated.
Queries that can be performed together must be performed in the same iteration.
We introduce a simple streaming language for computing queries in a single-pass over the data.
By using an appropriate intermediate language we guarantee fusion between queries on the same input streams, before extracting efficient C code.
\end{abstract}


\category
	{D.3.4}
	{Programming Languages}
	{Processors---Compilers; Optimization}

\terms
	Languages, Performance

\keywords
	Arrays; Fusion


\section{Introdution}

This is some stuff.

%!TEX root = ../Main.tex
\section{Icicle Source}
\label{s:Source}

The two main types in Icicle are @Stream@ and @Fold@.
Streams represent data values as they flow through the program.
Streams do not always have data flowing through them, but have an associated clock which describes when data occurs.
Two streams with the same clock both have data at the same time.
Folds, on the other hand, are results of computations over the stream data that has been seen.
Folds always have a well-defined value.
Folds can be easily converted to streams, by sampling their current value whenever the stream clock is true.
It is harder to convert a stream to a fold, as the stream has gaps where it is undefined, and one must specify how to fill the gaps (hold last, fill with zero, etc).
Finally, folds can be computed recursively based on their previous values.

%!TEX root = ../Main.tex

\begin{figure}

\begin{tabbing}
MMMM \= MM \= MMMMMMMMMMM \= MM \= \kill
$\mi{e}$
\GrammarDef $x$
\> $|$ \> $v$
\GrammarAlt $e~e$
\> $|$ \> $\lambda{}x~:~\tau.~e$

\\
\GrammarAlt $@when@~e~e$
\> $|$ \> $@sample@~e~e$
\GrammarAlt $@mapS@~e~e$
\> $|$ \> $@zipS@~e~e$
\GrammarAlt $@mapF@~e~e$
\> $|$ \> $@zipF@~e~e$
\\
\GrammarAlt $@let@~\mi{Let}~@in@~e$
\\
\GrammarAlt $\langle e,~ e \rangle ~|~ \langle\rangle$
\\
\\

$\mi{Let}$
\GrammarDef $x~=~e$
\GrammarAlt $@folds@~\{~x_i~=~e_i,~e_i~\}_i$
\\
\\

$\mi{v}$
\GrammarDef $\NN ~|~ \BB ~|~ [v]$
\GrammarAlt $(v,~v) ~|~ \bot$
\\
\\


$\mi{Program}$
\GrammarDef @inputs@~$\{ x_i~:~@Stream@~c_i~\tau_i \}_i~@in@~\mi{e}$ \\
\end{tabbing}

\caption{Grammar for Icicle Source}
\label{fig:source:grammar}
\end{figure}


The grammar for Icicle is given in figure~\ref{fig:source:grammar}.
The first four rules are rather standard lambda calculus.

Next are the primitives.
@when@ takes a stream and a fold, and returns a new fold whose value is the stream \emph{when}ever the stream is defined, and the fold otherwise.
This is used for defining folds that depend on stream values.
@sample@ takes a stream of any type and a fold, creates a new stream using the values of the fold.
@mapS@ and @zipS@ perform map and zip operations on streams of the same rate.
@mapF@ and @zipF@ perform map and zip operations on folds.

Let expressions such as $@let@~x~=e~@in@~e$ are as usual. 
Folds are defined using the syntax @let folds@ syntax. 
Multiple folds can be defined together, and each fold is defined by the initial value, and the ``kons'' part, defining the next value.

For example, a simple fold comprised of nothing but $0$
\begin{tabbing}
MM \= \kill
$@let folds@~\mi{zeros}~=~0,~\mi{zeros}$ \\
$@in@~\mi{zeros}$
\\
\\
$\mi{zeros}$ \> $=~\langle 0~0~0~0~\cdots~\rangle$ \\
\end{tabbing}

When computing the sum over another fold, the newly-defined fold and the input fold are zipped and then added together.
\begin{tabbing}
$\mi{sum}~=~\lambda{}(\mi{values}~:~@Fold@~\NN).$ \\
$@let folds@~\mi{s}~=~0,~@mapF@~(+)~(@zipF@~\mi{s}~\mi{values})$ \\
$@in@~\mi{s}$
\\
\\
$\mi{sum}~\langle 0~1~2~3~\cdots~\rangle$ \\
$~=~\langle 0~1~3~6~\cdots~\rangle$ \\
\end{tabbing}

This $\mi{sum}$ function requires a fold argument, but it can be applied to a stream $s$ by creating a fold that is the stream when it is defined, and $0$ otherwise: $@when@~s~zeros$.
A more general way is to define a new $\mi{sum'}$ function that works directly over streams:
\begin{tabbing}
MMMMMM \= MM \= MM \kill
$\mi{sum'}~=~\lambda{}(\mi{values}~:~@Stream@~c~\NN).$ \\
$@let folds@~\mi{s}~=~0,$ \\
\> $@when@$ \> $(@mapS@~(+)~(@zipS@~(@sample@~\mi{values}~\mi{s})~\mi{values}))$ \\
\> \> $~\mi{s}$ \\
$@in@~\mi{s}$
\\
\\
$\mi{sum'}~\langle \bot~1~\bot~3~\cdots~\rangle$ \\
$~=~\langle 0~1~1~4~\cdots~\rangle$ \\
\end{tabbing}

Here, the fold $s$ is sampled to a stream with the same clock as the input stream.
Whenever the input stream is defined, the previous value of $s$ and the current value of the $\mi{values}$ will be added together, producing the new sum.
When the input stream is not defined, the previous value of $s$ is used unchanged.

When referring to the newly defined folds in the right-hand-side of the fold (the kons), all values refer to the immediate previous value.
The order of the fold bindings has no effect on the program.
\begin{tabbing}
MM \= \kill
$@let folds@$ \\
\> $\mi{one}~=~1,~\mi{zero}$ \\
\> $\mi{zero}~=~0,~\mi{one}$ \\
$@in@~\mi{one}$
\\
\\
$\mi{one}$  \> $=~\langle 1~0~1~0~\cdots~\rangle$ \\
$\mi{zero}$ \> $=~\langle 0~1~0~1~\cdots~\rangle$ \\
\end{tabbing}

The grammar also has rules for stream literals: $\langle e_1,~e_2 \rangle$ and $\langle\rangle$. The stream can contain bottoms, $\bot$, which means that the stream is undefined at that point.
Stream literals have concrete clock types, for example $\langle T,~F \rangle$ for $\langle 1,~\bot \rangle$, while input streams have existential clocks: their actual clock is unknown until runtime.
These stream literals are not expected to occur in an actual program until evaluation.


Yes, I realise the grammar doesn't have @+@ or @/@, and that it doesn't have polymorphism.
For the examples, assume that a specialisation pass occurs, and that there is a ``sensible set of primitives on natural numbers and lists''.

Define group here.
\begin{code}
data Unpack r
 = forall a.
    Unpack a (Fold (a -> a)) (a -> r)

sum' vs = Unpack 0 (+vs) id

group (key : Stream c k)
      (val : Unpack v)
           : [(k,v)]
group key (Unpack z k r)
 = trivial
\end{code}



%!TEX root = ../Main.tex

\begin{figure*}

$$
\boxed{\SourceStepZ{e}{\mi{acc}}{\mi{res}}}
$$

$$
\ruleAx
{
    \SourceStepZ{x}{()}{x}
}{ZVar}
\ruleAx
{
    \SourceStepZ{v}{()}{v}
}{ZValue}
\ruleIN
{
    \SourceStepZ{e}{a}{r}
}
{
    \SourceStepZ{\lambda{}x~:~\tau.~e}{a}{\lambda{}x~:~\tau.~r}
}{ZLam}
$$

$$
\ruleAx
{
    \SourceStepZ{\langle e_1,~e_2 \rangle}{\langle e_1,~e_2 \rangle}{\bot}
}{ZStream1}
\ruleAx
{
    \SourceStepZ{\langle \rangle}{\langle \rangle}{\bot}
}{ZStream2}
\ruleIN
{
    \SourceStepZ{e_1}{a_1}{r_1}
    \quad
    \SourceStepZ{e_2}{a_2}{r_2}
}
{
    \SourceStepZ{@when@~e_1~e_2}{a_1,a_2}{r_2}
}{ZWhen}
\ruleIN
{
    \SourceStepZ{e_1}{a_1}{r_1}
    \quad
    \SourceStepZ{e_2}{a_2}{r_2}
}
{
    \SourceStepZ{@sample@~e_1~e_2}{a_1,a_2}{\bot}
}{ZSample}
$$

$$
\ruleIN
{
    \SourceStepZ{e}{a}{r}
}
{
    \SourceStepZ{@mapS@~f~e}{a}{\bot}
}{ZMapS}
\ruleIN
{
    \SourceStepZ{e_1}{a_1}{r_1}
    \quad
    \SourceStepZ{e_2}{a_2}{r_2}
}
{
    \SourceStepZ{@zipS@~e_1~e_2}{a_1,a_2}{\bot}
}{ZZipS}
$$

$$
\ruleIN
{
    \SourceStepZ{e}{a}{r}
}
{
    \SourceStepZ{@mapF@~f~e}{a}{f~r}
}{ZMapF}
\ruleIN
{
    \SourceStepZ{e_1}{a_1}{r_1}
    \quad
    \SourceStepZ{e_2}{a_2}{r_2}
}
{
    \SourceStepZ{@zipF@~e_1~e_2}{a_1,a_2}{r_1,r_2}
}{ZZipF}
$$

$$
\ruleIN
{
    \SourceStepZ{e_1}{a_1}{r_1}
    \quad
    \SourceStepZ{e_2}{a_2}{r_2}
}
{
    \SourceStepZ{@let@~x~=~e_1~@in@~e_2}{a_1,a_2}{r_2[x=r_1]}
}{ZLet}
\ruleIN
{
    \{ \SourceStepZ{e_i}{a_i}{r_i} \}_i
    \quad
    \SourceStepZ{e}{a}{r}
}
{
    \SourceStepZ{@let folds@~\{x_i~=~e_i,~\_\}_i~@in@~e}{\{a_i\}_i,a}{r[\{x_i=r_i\}_i]}
}{ZFolds}
$$

$$
\boxed{\SourceStepK{\mi{acc}}{e}{\mi{acc}}{\mi{res}}}
$$


$$
\ruleAx
{
    \SourceStepK{()}{x}{()}{x}
}{KVar}
\ruleAx
{
    \SourceStepK{()}{v}{()}{v}
}{KValue}
\ruleIN
{
    \SourceStepK{a}{e}{a'}{r'}
}
{
    \SourceStepK{a}{\lambda{}x~:~\tau.~e}{a'}{\lambda{}x~:~\tau.~r'}
}{KLam}
$$

$$
\ruleAx
{
    \SourceStepK{\langle e_1,~e_2 \rangle}{\langle \cdots \rangle}{e_2}{e_1}
}{KStream1}
\ruleAx
{
    \SourceStepK{\langle \rangle}{\langle \cdots \rangle}{\langle\rangle}{\bot}
}{KStream2}
\ruleIN
{
    \SourceStepK{a_1}{e_1}{a_1'}{r_1'}
    \quad
    \SourceStepK{a_1}{e_2}{a_2'}{r_2'}
}
{
    \SourceStepK{a_1,a_2}{@when@~e_1~e_2}{a_1',a_2'}{\Ifnbot{r_1'}{r_1'}{r_2'}}
}{KWhen}
$$

$$
\ruleIN
{
    \SourceStepK{a_1}{e_1}{a_1'}{r_1'}
    \quad
    \SourceStepK{a_2}{e_2}{a_2'}{r_2'}
}
{
    \SourceStepK{a_1,a_2}{@sample@~e_1~e_2}{a_1',a_2'}{\Ifnbot{r_1'}{r_2'}{\bot}}
}{KSample}
$$

$$
\ruleIN
{
    \SourceStepK{a}{e}{a'}{r'}
}
{
    \SourceStepK{a}{@mapS@~f~e}{a'}{\Ifnbot{r'}{f~r'}{\bot}}
}{KMapS}
\ruleIN
{
    \SourceStepK{a_1}{e_1}{a_1'}{r_1'}
    \quad
    \SourceStepK{a_2}{e_2}{a_2'}{r_2'}
}
{
    \SourceStepK{a_1,a_2}{@zipS@~e_1~e_2}{a_1',a_2'}{r_1',r_2'}
}{KZipS}
$$

$$
\ruleIN
{
    \SourceStepK{a}{e}{a'}{r'}
}
{
    \SourceStepK{a}{@mapF@~f~e}{a'}{f~r}
}{KMapF}
\ruleIN
{
    \SourceStepK{a_1}{e_1}{a_1'}{r_1'}
    \quad
    \SourceStepK{a_2}{e_2}{a_2'}{r_2'}
}
{
    \SourceStepK{a_1,a_2}{@zipF@~e_1~e_2}{a_1',a_2'}{r_1',r_2'}
}{KZipF}
$$

$$
\ruleIN
{
    \SourceStepK{a_1}{e_1}{a_1'}{r_1'}
    \quad
    \SourceStepK{a_2}{e_2}{a_2'}{r_2'}
}
{
    \SourceStepK{a_1,a_2}{@let@~x~=~e_1~@in@~e_2}{a_1',a_2'}{r_2[x=r_1]}
}{KLet}
\ruleIN
{
    \{ \SourceStepK{a_i}{e_i}{a_i'}{r_i'} \}_i
    \quad
    \SourceStepK{a}{e}{a'}{r'}
}
{
    \SourceStepK{\{a_i\}_i,a}{@let folds@~\{x_i~=~\_,~e_i\}_i~@in@~e}{\{a_i'\}_i,a'}{r'[\{x_i=r_i'\}_i]}
}{KFolds}
$$



$$
\boxed{\SourceStepX{e}{e}}
$$

$$
\ruleIN
{
    \SourceStepX{e_1}{e_1'}
}
{
    \SourceStepX{e_1~e_2}{e_1'~e_2}
}{XApp1}
\ruleIN
{
    \SourceStepX{e_2}{e_2'}
}
{
    \SourceStepX{v_1~e_2}{v_1~e_2'}
}{XApp2}
\ruleAx
{
    \SourceStepX{(\lambda{}x.~e_1)~e_2}{e_1[x:=e_2]}
}{XApp3}
$$
$$
\ruleIN
{
    \SourceStepX{e_1}{e_1'}
}
{
    \SourceStepX{@let@~x~=~e_1~@in@~e_2}{@let@~x~=~e_1'~@in@~e_2}
}{XLet1}
\ruleAx
{
    \SourceStepX{@let@~x~=~v_1~@in@~e_2}{e_2[x:=v_1]}
}{XLet2}
$$

\caption{Evaluation rules}
\label{fig:source:eval}
\end{figure*}




\subsection{Obvious extensions}
Obvious extensions and things I removed for simplification.

\subsubsection{Fixpoint}
$@fix@~(x~:~\tau)~@in@~e$.
By requiring $\tau$ to have kind @Data@ rather than @Flow@, we can allow recursion at runtime, while still ensuring a static dataflow graph.
I am tempted to kill the kinds for the simple version because with no fixpoint they do not serve much purpose (the dataflow graph is static regardless).

\subsubsection{ Maps and zips }
@mapS/mapF@, @zipS/zipF@.
For the simple version these could be merged into a single map/zip operating over just @Fold@s.
You can convert from a @Stream@ to a @Fold@ and back easily enough - you need a zero element for the @Fold@ but whatever.

\subsubsection{ Subrates }
$@let-rate@~(c'~:_k~@Clock@)~(w~:~c'~\le~c)~=~(e~:~@Stream@~c~\BB)$ and $@subrate@~(w~:~c'~\le~c)~(e~:~@Stream@~c~\tau)~:~@Stream@~c'~\tau$.
You can fake this by using Folds anyway.

\subsubsection{ Unpack }
$@let-unpack@~(x_\tau~:_k~@Data@)~(x_z~:~x_\tau)~(x_k~:~x_\tau~\to~@Fold@~x_\tau)~(x_r~:~x_\tau~\to~\tau)~=~(e~:~@Fold@~\tau)$.
Pulling apart folds to make new folds.
Useful for groups, segmented operations, and so on where the fold can be restarted or actually be many folds, as well as for other higher order folds.


%!TEX root = ../Main.tex
\section{Types}
\label{s:Types}

%!TEX root = ../Main.tex

\begin{figure*}

\begin{tabbing}
MM \= MM \= \kill
$\mi{Kind}$
\GrammarDef $@Data@~|~@Clock@~|~@Flow@$ \\

T
\GrammarDef $x~|~\NN~|~\BB~|~@()@
    ~|~ @List@~T
    ~|~ T~\to~T
    ~|~ (T,T)$
\GrammarAlt $@Stream@~T~T
    ~|~ @Fold@~T$
    \\
\end{tabbing}

\caption{Types and kinds}
\label{fig:source:type:types}
\end{figure*}


\begin{figure*}

$$
\boxed{\TypeWf{\Delta}{\tau}{\mi{Kind}}}
$$

$$
\ruleIN
{
    x~:_k~k~\in~\Delta
}
{
    \TypeWf{\Delta}{x}{k}
}{TVar}
\ruleAx
{
    \TypeWf{\Delta}{\NN}{@Data@}
}{TNat}
\ruleAx
{
    \TypeWf{\Delta}{\BB}{@Data@}
}{TBool}
\ruleAx
{
    \TypeWf{\Delta}{@()@}{@Data@}
}{TUnit}
$$

$$
\ruleIN
{
    \TypeWf{\Delta}{\tau}{@Data@}
}
{
    \TypeWf{\Delta}{@List@~\tau}{@Data@}
}{TList}
\ruleIN
{
    \TypeWf{\Delta}{\tau_1}{k_1}
    \quad
    \TypeWf{\Delta}{\tau_2}{k_2}
}
{
    \TypeWf{\Delta}{\tau_1~\to~\tau_2}{k_2}
}{TFun}
$$

$$
\ruleIN
{
    \TypeWf{\Delta}{c}{@Clock@}
    \quad
    \TypeWf{\Delta}{\tau}{@Data@}
}
{
    \TypeWf{\Delta}{@Stream@~c~\tau}{@Flow@}
}{TStream}
\ruleIN
{
    \TypeWf{\Delta}{\tau}{@Data@}
}
{
    \TypeWf{\Delta}{@Fold@~\tau}{@Flow@}
}{TFold}
$$



\caption{Kinds of types}
\label{fig:source:type:kinds}
\end{figure*}


\begin{figure*}

$$
\boxed{\Typecheck{\Delta}{\Gamma}{e}{T}}
$$


$$
\ruleIN
{
    (x~:~\tau)~\in~\Gamma
}
{ 
    \Typecheck{\Delta}{\Gamma}{x}{\tau}
}{TcVar}
\ruleIN
{
    v~:~\tau
}
{ 
    \Typecheck{\Delta}{\Gamma}{v}{\tau}
}{TcValue}
\ruleIN
{
    \Typecheck{\Delta}{\Gamma}{e_1}{\tau_1~\to~\tau_2}
    \quad
    \Typecheck{\Delta}{\Gamma}{e_2}{\tau_1}
}
{ 
    \Typecheck{\Delta}{\Gamma}{e_1~e_2}{\tau_2}
}{TcApp}
$$

$$
\ruleIN
{
    \Typecheck{\Delta}{\Gamma,~x~:~\tau}{e}{\tau'}
}
{
    \Typecheck{\Delta}{\Gamma}{\lambda{}x~:~\tau.~e}{\tau~\to~\tau'}
}{TcLam}
$$

$$
\ruleAx
{
    \Typecheck{\Delta}{\Gamma}{\langle \rangle}{@Stream@~\langle\rangle~\tau}
}{TcStrm1}
\ruleIN
{
    \Typecheck{\Delta}{\Gamma}{e_1}{\tau}
    \quad
    \Typecheck{\Delta}{\Gamma}{e_2}{@Stream@~c~\tau}
}
{
    \Typecheck{\Delta}{\Gamma}{\langle e_1,~e_2 \rangle}{@Stream@~\langle @T@,~c \rangle~\tau}
}{TcStrm2}
\ruleIN
{
    \Typecheck{\Delta}{\Gamma}{e_2}{@Stream@~c~\tau}
}
{
    \Typecheck{\Delta}{\Gamma}{\langle \bot,~e_2 \rangle}{@Stream@~\langle @F@,~c \rangle~\tau}
}{TcStrm3}
$$

$$
\ruleIN
{
    \Typecheck{\Delta}{\Gamma}{e_1}{@Stream@~c~\tau}
    \quad
    \Typecheck{\Delta}{\Gamma}{e_2}{@Fold@~\tau}
}
{
    \Typecheck{\Delta}{\Gamma}{@when@~e_1~e_2}{@Fold@~\tau}
}{TcWhen}
\ruleIN
{
    \Typecheck{\Delta}{\Gamma}{e_1}{@Stream@~c~\tau'}
    \quad
    \Typecheck{\Delta}{\Gamma}{e_2}{@Fold@~\tau}
}
{
    \Typecheck{\Delta}{\Gamma}{@sample@~e_1~e_2}{@Stream@~c~\tau}
}{TcSample}
$$

$$
\ruleIN
{
    \Typecheck{\Delta}{\Gamma}{e_1}{\tau_1~\to~\tau_2}
    \quad
    \Typecheck{\Delta}{\Gamma}{e_2}{@Stream@~c~\tau_1}
}
{
    \Typecheck{\Delta}{\Gamma}{@mapS@~e_1~e_2}{@Stream@~c~\tau_2}
}{TcMapS}
\ruleIN
{
    \Typecheck{\Delta}{\Gamma}{e_1}{@Stream@~c~\tau_1}
    \quad
    \Typecheck{\Delta}{\Gamma}{e_2}{@Stream@~c~\tau_2}
}
{
    \Typecheck{\Delta}{\Gamma}{@zipS@~e_1~e_2}{@Stream@~c~(\tau_1,\tau_2)}
}{TcZipS}
$$

$$
\ruleIN
{
    \Typecheck{\Delta}{\Gamma}{e_1}{\tau_1~\to~\tau_2}
    \quad
    \Typecheck{\Delta}{\Gamma}{e_2}{@Fold@~\tau_1}
}
{
    \Typecheck{\Delta}{\Gamma}{@mapF@~e_1~e_2}{@Fold@~\tau_2}
}{TcMapF}
\ruleIN
{
    \Typecheck{\Delta}{\Gamma}{e_1}{@Fold@~\tau_1}
    \quad
    \Typecheck{\Delta}{\Gamma}{e_2}{@Fold@~\tau_2}
}
{
    \Typecheck{\Delta}{\Gamma}{@zipF@~e_1~e_2}{@Fold@~(\tau_1,\tau_2)}
}{TcZipF}
$$

$$
\ruleIN
{
    \Typecheck{\Delta}{\Gamma}{e_1}{\tau_1}
    \quad
    \Typecheck{\Delta}{\Gamma,~x~:~\tau_1}{e_2}{\tau_2}
}
{
    \Typecheck{\Delta}{\Gamma}{@let@~x~=~e_1~@in@~e_2}{\tau_2}
}{TcLet}
$$

$$
\ruleIN
{
    \Gamma'~=~\Gamma,~\{~x_i~:~@Fold@~\tau_i~\}_i
    \quad
    \{
    \Typecheck{\Delta}{\Gamma}{e_{z_i}}{\tau_i}
    \}_i
    \quad
    \{
    \Typecheck{\Delta}{\Gamma'}{e_{k_i}}{@Fold@~\tau_i}
    \}_i
    \quad
    \Typecheck{\Delta}{\Gamma'}{e}{\tau}
}
{
    \Typecheck{\Delta}{\Gamma}
        {@let@~@folds@~\{~x_i~=~e_{z_i}~@then@~e_{k_i}~\}_i~@in@~e}
        {\tau}
}{TcLetFolds}
$$


\caption{Types of expressions}
\label{fig:source:type:exp}
\end{figure*}




%!TEX root = ../Main.tex
\section{Core language}
\label{s:Core}

This section presents the Core language, which we use as an intermediate language before converting to imperative code.


\subsection{Grammar}

%!TEX root = ../Main.tex

\begin{figure}

\begin{tabbing}
MMMM \= MM \= MM \= MM \= \kill
$\mi{e_c}$
\GrammarDef $x$ \> $|$ \> $v$
\GrammarAlt $e_c~e_c$ \> $|$ \> $p$
\\
\\
$c$
\GrammarDef $x$ \> $|$ \> $\mi{Always}$
\\
\\
$\mi{FoldNest}$
\GrammarDef $@nest@~\mi{Op}*$
\\
\\
$\mi{Op}$
\GrammarDef $@let@~x~=~e_c~@when@~c$
\GrammarAlt $@fold@~x~=~e_c,~(e_c~@when@~c~@else@~e_c)$
\\
\\
$\mi{Bind}$
\GrammarDef $x~=~\lambda{}x.~e_c$
\\
\\
$\mi{Program}$
\GrammarDef @inputs@ \\
    \>      \> \> $(x~:~@Stream@~c~\tau)*$ \\
    \>      \> @binds@ \\
    \>      \> \> $\mi{Bind}*$ \\
    \>      \> @with@ \\
    \>      \> \> $\mi{FoldNest}*$ \\
    \>      \> @in@ \\
    \>      \> \> $x*$ \\
\end{tabbing}


\caption{Grammar for Icicle Core}
\label{fig:core:grammar}
\end{figure}



%!TEX root = ../Main.tex

\begin{figure*}

$$
\boxed{\CoreComp{e}{[Bind]}{[Op]}{x}}
$$

$$
\ruleAx
{
    \CoreComp{v}{\emptyset}{@let@~x'~=~v}{x'}
}{CVal}
\ruleAx
{
    \CoreComp{x}{\emptyset}{\emptyset}{x}
}{CVar}
\ruleIN
{
    \CoreComp{e_1}{b_1}{o_1}{x_1}
    \quad
    \CoreComp{e_2}{b_2}{o_2}{x_2}
}{
    \CoreComp{e_1~e_2}{b_1;b_2}{o_1;o_2;~@let@~x'~=~x_1~x_2}{x'}
}{CApp}
$$


\caption{Conversion to Core}
\label{fig:core:compile}
\end{figure*}





%!TEX root = ../Main.tex
\section{Code generation}
\label{s:Generation}

Code generation should be pretty easy I guess cite flow fusion paper.

Because we don't have any zips or other that takes multiple streams, we don't have to deal with hard problems.

While not all streams have the same rate, there is no way to express an operation that takes two different rates.


%!TEX root = ../Main.tex
\section{Conclusion}
\label{s:Conclusion}





%!TEX root = ../Main.tex
\section{Related work}
\label{s:Related}
\subsection{Data flow languages}

The closest related work are synchronous data flow languages such as {\sc Lustre}, Icicle programs are restricted to those that can be computed in bounded memory.
{\sc Lustre}\cite{halbwachs1991synchronous} achieves this in three main ways:

\begin{enumerate}
\item They allow only restricted set of primitives such as @when@ for filtering, @pre@ for a single element buffer, and so on.
\item Cycles in the graph must contain at least one @pre@, to break the dependency loop.
\item Operators such as addition can only be applied to input streams with the same clock or rate; an expression like @(X when C)@ @+@ @(Y when (not C))@ would require an unbounded buffer\CITE{Lustre clock stuff}.
\end{enumerate}

In summary, while our language is quite similar to existing synchronous data flow languages, the extra restrictions we impose allow us to use a simpler type system.


% %!TEX root = ../Main.tex

\newcommand\JudgeK[2]
{       #1 :: #2
}

\newcommand\JudgeT[3]
{       #1 \vdash #2 :: #3
}

\newcommand\JudgeTS[5]
{       #1 \vdash #2 :: #3 ~;~ #4 ~;~ #5
}

\newcommand\kbox        {\textrm{\textbf{box}}}
\newcommand\krun        {\textrm{\textbf{run}}}
\newcommand\kthen       {\textrm{\textbf{then}}}
\newcommand\kpure       {\textrm{\textbf{pure}}}

\newcommand\rData       {\textrm{Data}}
\newcommand\rClock      {\textrm{Clock}}
\newcommand\rDefn       {\textrm{Defn}}
\newcommand\rComp       {\textrm{Comp}}

\newcommand\rStream     {\textrm{Stream}}
\newcommand\rArray      {\textrm{Array}}
\newcommand\rNat        {\textrm{Nat}}
\newcommand\rBool       {\textrm{Bool}}

\newcommand\rdrain      {\textrm{drain}}
\newcommand\rstream     {\textrm{stream}}
\newcommand\rsum        {\textrm{sum}}
\newcommand\rsmap       {\textrm{smap}}
\newcommand\rsfold      {\textrm{sfold}}
\newcommand\rsscan      {\textrm{sscan}}
\newcommand\rsfilter    {\textrm{sfilter}}


\begin{figure*}
$$
\boxed{\JudgeT{\Gamma}{e}{\tau}}
$$

$$
\ruleI
{       x : \tau \in \Gamma
}
{       \JudgeT{\Gamma}{x}{\tau}
}
\textrm{(TyVar)}
\quad\quad
\ruleI
{       \JudgeT{\Gamma}{e_1}{\tau_1 \to \tau_2}
        \quad
        \JudgeT{\Gamma}{e_2}{\tau_1}
}
{       \JudgeT{\Gamma}{e_1~e_2}{\tau_2}       
}
\textrm{(TyApp)}
\quad\quad
\ruleI
{       \JudgeT{\Gamma,x:\tau_1}{e_2}{\tau_2}
}
{       \JudgeT{\Gamma}{\lambda x : \tau_1}{\tau_1 \to \tau_2}
}
\textrm{(TyAbs)}
$$



% -- Effectful --------------------------------------------
$$
\boxed{\JudgeTS{\Gamma}{e}{\tau}{n}{k}}
$$

$$
\ruleI
{       \JudgeTS{\Gamma}{e}{\tau}{n}{k}
}
{       \JudgeT{\Gamma}{\kbox~ e}{\Box~ n~ k~ \tau}
}
\textrm{(TyBox)}
\quad\quad
\ruleI
{       \JudgeT{\Gamma}{e}{\tau}
}
{       \JudgeTS{\Gamma}{\kpure~e}{\tau}{n}{k}
}
\textrm{(TyPure)}
$$

$$
\ruleI
{       \{ \JudgeT
                {\Gamma}{e_i}{\Box~ n_i~ k~ \tau_i} \}^i 
        \quad
        \JudgeTS
                {\Gamma,~ \{x_i : \tau_i \}^i}
                {e'}{\tau'}{n'}{k'}
}
{       \JudgeTS
                {\Gamma}
                {\krun~ \{ x_i : \tau_i = e_i \}^i ~\kthen~ e'}
                {\tau'}
                {max~ \{ n_i \}^i + n'}
                {k'}
}
\textrm{(TyRun)}
$$

% -- Prims ------------------------------------------------
\vspace{2em}
$$
\begin{array}{ll}
\rNat,\rBool      & :: \rData
\\[1ex]
%
\rArray           & :: \rClock \to \rData \to \rData
\\[1ex]
%
\rStream          & :: \rClock \to \rData \to \rDefn
\\[1ex]
%
\Box              & :: \rNat   \to \rClock \to \rData \to \rComp
\\[1ex]
%
\rdrain_{k,\tau}  & :: \rStream~k~\tau \to \Box^1_k~ (\rArray~ k~ \tau)
\\[1ex]
%
\rstream_{k,\tau} & :: \rArray~k~\tau  \to \rStream~k~\tau
\\[1ex]
%
\rsum_k           & :: \rStream~k~\rNat \to \Box^1_k~ \rNat
\\[1ex]
%
\rsmap_{k,a,b}    & :: (a \to b) \to \rStream~k~a \to \rStream~k~b
\\[1ex]
%
\rsfold_{k,a,b}   & :: (a \to b \to b) \to b \to \rStream~k~a \to \Box^1_k~ b
\\[1ex]
%
\rsscan_{k,a,b}   & :: (a \to b \to b) \to b \to \rStream~k~a \to \rStream~k~b
\end{array}
$$

\caption{Typing}
\label{fig:source:type:modal}
\end{figure*}


\section*{Acknowledgements}

\bibliographystyle{plain}
\bibliography{Main}

\end{document}



\documentclass[preprint]{sigplanconf}
\usepackage{amssymb}
\usepackage{amsthm}
\usepackage{graphicx}
\usepackage{amsmath}
\usepackage{mathptmx}
\usepackage{mathtools}
\usepackage{stmaryrd}
\usepackage{hyperref}
\usepackage{alltt}
\usepackage{url}
\usepackage{float}
\usepackage{style/code}
\usepackage{style/proof}
\usepackage{style/utils}
\usepackage{style/judgements}

% -----------------------------------------------------------------------------
\begin{document}

% \exclusivelicense
% \conferenceinfo{}{}
% \copyrightyear{2015}
% \copyrightdata{}
\doi{}
% \pagenumbering{gobble} 

\title{Icicle: fuse your queries}

\authorinfo{
  Amos Robinson$^\dagger$$^\ddagger$
  \and Ben Lippmeier$^\dagger$
}{
  \vspace{5pt}
  \shortstack{
    $^\dagger$Computer Science and Engineering \\
    University of New South Wales, Australia \\[2pt]
    \textsf{amosr,benl@cse.unsw.edu.au}
  }
  \shortstack{
    $^\ddagger$Ambiata      \\
    Big data and shit       \\[2pt]
    \textsf{amos.robinson@ambiata.com}
  }
}

\maketitle
\makeatactive

\begin{abstract}
When streaming a large amount of data, simply iterating over the data may take hours.
If multiple queries are to be performed, it is important that work is not duplicated.
Queries that can be performed together must be performed in the same iteration.
We introduce a simple streaming language for computing queries in a single-pass over the data.
By using an appropriate intermediate language we guarantee fusion between queries on the same input streams, before extracting efficient C code.
\end{abstract}


\category
	{D.3.4}
	{Programming Languages}
	{Processors---Compilers; Optimization}

\terms
	Languages, Performance

\keywords
	Arrays; Fusion


\section{Introdution}

This is some stuff.

%!TEX root = ../Main.tex
\section{Icicle Source}
\label{s:Source}

The two main types in Icicle are @Stream@ and @Fold@.
Streams represent data values as they flow through the program.
Streams do not always have data flowing through them, but have an associated clock which describes when data occurs.
Two streams with the same clock both have data at the same time.
Folds, on the other hand, are results of computations over the stream data that has been seen.
Folds always have a well-defined value.
Folds can be easily converted to streams, by sampling their current value whenever the stream clock is true.
It is harder to convert a stream to a fold, as the stream has gaps where it is undefined, and one must specify how to fill the gaps (hold last, fill with zero, etc).
Finally, folds can be computed recursively based on their previous values.

%!TEX root = ../Main.tex

\begin{figure}

\begin{tabbing}
MMMM \= MM \= MMMMMMMMMMM \= MM \= \kill
$\mi{e}$
\GrammarDef $x$
\> $|$ \> $v$
\GrammarAlt $e~e$
\> $|$ \> $\lambda{}x~:~\tau.~e$

\\
\GrammarAlt $@when@~e~e$
\> $|$ \> $@sample@~e~e$
\GrammarAlt $@mapS@~e~e$
\> $|$ \> $@zipS@~e~e$
\GrammarAlt $@mapF@~e~e$
\> $|$ \> $@zipF@~e~e$
\\
\GrammarAlt $@let@~\mi{Let}~@in@~e$
\\
\GrammarAlt $\langle e,~ e \rangle ~|~ \langle\rangle$
\\
\\

$\mi{Let}$
\GrammarDef $x~=~e$
\GrammarAlt $@folds@~\{~x_i~=~e_i,~e_i~\}_i$
\\
\\

$\mi{v}$
\GrammarDef $\NN ~|~ \BB ~|~ [v]$
\GrammarAlt $(v,~v) ~|~ \bot$
\\
\\


$\mi{Program}$
\GrammarDef @inputs@~$\{ x_i~:~@Stream@~c_i~\tau_i \}_i~@in@~\mi{e}$ \\
\end{tabbing}

\caption{Grammar for Icicle Source}
\label{fig:source:grammar}
\end{figure}


The grammar for Icicle is given in figure~\ref{fig:source:grammar}.
The first four rules are rather standard lambda calculus.

Next are the primitives.
@when@ takes a stream and a fold, and returns a new fold whose value is the stream \emph{when}ever the stream is defined, and the fold otherwise.
This is used for defining folds that depend on stream values.
@sample@ takes a stream of any type and a fold, creates a new stream using the values of the fold.
@mapS@ and @zipS@ perform map and zip operations on streams of the same rate.
@mapF@ and @zipF@ perform map and zip operations on folds.

Let expressions such as $@let@~x~=e~@in@~e$ are as usual. 
Folds are defined using the syntax @let folds@ syntax. 
Multiple folds can be defined together, and each fold is defined by the initial value, and the ``kons'' part, defining the next value.

For example, a simple fold comprised of nothing but $0$
\begin{tabbing}
MM \= \kill
$@let folds@~\mi{zeros}~=~0,~\mi{zeros}$ \\
$@in@~\mi{zeros}$
\\
\\
$\mi{zeros}$ \> $=~\langle 0~0~0~0~\cdots~\rangle$ \\
\end{tabbing}

When computing the sum over another fold, the newly-defined fold and the input fold are zipped and then added together.
\begin{tabbing}
$\mi{sum}~=~\lambda{}(\mi{values}~:~@Fold@~\NN).$ \\
$@let folds@~\mi{s}~=~0,~@mapF@~(+)~(@zipF@~\mi{s}~\mi{values})$ \\
$@in@~\mi{s}$
\\
\\
$\mi{sum}~\langle 0~1~2~3~\cdots~\rangle$ \\
$~=~\langle 0~1~3~6~\cdots~\rangle$ \\
\end{tabbing}

This $\mi{sum}$ function requires a fold argument, but it can be applied to a stream $s$ by creating a fold that is the stream when it is defined, and $0$ otherwise: $@when@~s~zeros$.
A more general way is to define a new $\mi{sum'}$ function that works directly over streams:
\begin{tabbing}
MMMMMM \= MM \= MM \kill
$\mi{sum'}~=~\lambda{}(\mi{values}~:~@Stream@~c~\NN).$ \\
$@let folds@~\mi{s}~=~0,$ \\
\> $@when@$ \> $(@mapS@~(+)~(@zipS@~(@sample@~\mi{values}~\mi{s})~\mi{values}))$ \\
\> \> $~\mi{s}$ \\
$@in@~\mi{s}$
\\
\\
$\mi{sum'}~\langle \bot~1~\bot~3~\cdots~\rangle$ \\
$~=~\langle 0~1~1~4~\cdots~\rangle$ \\
\end{tabbing}

Here, the fold $s$ is sampled to a stream with the same clock as the input stream.
Whenever the input stream is defined, the previous value of $s$ and the current value of the $\mi{values}$ will be added together, producing the new sum.
When the input stream is not defined, the previous value of $s$ is used unchanged.

When referring to the newly defined folds in the right-hand-side of the fold (the kons), all values refer to the immediate previous value.
The order of the fold bindings has no effect on the program.
\begin{tabbing}
MM \= \kill
$@let folds@$ \\
\> $\mi{one}~=~1,~\mi{zero}$ \\
\> $\mi{zero}~=~0,~\mi{one}$ \\
$@in@~\mi{one}$
\\
\\
$\mi{one}$  \> $=~\langle 1~0~1~0~\cdots~\rangle$ \\
$\mi{zero}$ \> $=~\langle 0~1~0~1~\cdots~\rangle$ \\
\end{tabbing}

The grammar also has rules for stream literals: $\langle e_1,~e_2 \rangle$ and $\langle\rangle$. The stream can contain bottoms, $\bot$, which means that the stream is undefined at that point.
Stream literals have concrete clock types, for example $\langle T,~F \rangle$ for $\langle 1,~\bot \rangle$, while input streams have existential clocks: their actual clock is unknown until runtime.
These stream literals are not expected to occur in an actual program until evaluation.


Yes, I realise the grammar doesn't have @+@ or @/@, and that it doesn't have polymorphism.
For the examples, assume that a specialisation pass occurs, and that there is a ``sensible set of primitives on natural numbers and lists''.

Define group here.
\begin{code}
data Unpack r
 = forall a.
    Unpack a (Fold (a -> a)) (a -> r)

sum' vs = Unpack 0 (+vs) id

group (key : Stream c k)
      (val : Unpack v)
           : [(k,v)]
group key (Unpack z k r)
 = trivial
\end{code}



%!TEX root = ../Main.tex

\begin{figure*}

$$
\boxed{\SourceStepZ{e}{\mi{acc}}{\mi{res}}}
$$

$$
\ruleAx
{
    \SourceStepZ{x}{()}{x}
}{ZVar}
\ruleAx
{
    \SourceStepZ{v}{()}{v}
}{ZValue}
\ruleIN
{
    \SourceStepZ{e}{a}{r}
}
{
    \SourceStepZ{\lambda{}x~:~\tau.~e}{a}{\lambda{}x~:~\tau.~r}
}{ZLam}
$$

$$
\ruleAx
{
    \SourceStepZ{\langle e_1,~e_2 \rangle}{\langle e_1,~e_2 \rangle}{\bot}
}{ZStream1}
\ruleAx
{
    \SourceStepZ{\langle \rangle}{\langle \rangle}{\bot}
}{ZStream2}
\ruleIN
{
    \SourceStepZ{e_1}{a_1}{r_1}
    \quad
    \SourceStepZ{e_2}{a_2}{r_2}
}
{
    \SourceStepZ{@when@~e_1~e_2}{a_1,a_2}{r_2}
}{ZWhen}
\ruleIN
{
    \SourceStepZ{e_1}{a_1}{r_1}
    \quad
    \SourceStepZ{e_2}{a_2}{r_2}
}
{
    \SourceStepZ{@sample@~e_1~e_2}{a_1,a_2}{\bot}
}{ZSample}
$$

$$
\ruleIN
{
    \SourceStepZ{e}{a}{r}
}
{
    \SourceStepZ{@mapS@~f~e}{a}{\bot}
}{ZMapS}
\ruleIN
{
    \SourceStepZ{e_1}{a_1}{r_1}
    \quad
    \SourceStepZ{e_2}{a_2}{r_2}
}
{
    \SourceStepZ{@zipS@~e_1~e_2}{a_1,a_2}{\bot}
}{ZZipS}
$$

$$
\ruleIN
{
    \SourceStepZ{e}{a}{r}
}
{
    \SourceStepZ{@mapF@~f~e}{a}{f~r}
}{ZMapF}
\ruleIN
{
    \SourceStepZ{e_1}{a_1}{r_1}
    \quad
    \SourceStepZ{e_2}{a_2}{r_2}
}
{
    \SourceStepZ{@zipF@~e_1~e_2}{a_1,a_2}{r_1,r_2}
}{ZZipF}
$$

$$
\ruleIN
{
    \SourceStepZ{e_1}{a_1}{r_1}
    \quad
    \SourceStepZ{e_2}{a_2}{r_2}
}
{
    \SourceStepZ{@let@~x~=~e_1~@in@~e_2}{a_1,a_2}{r_2[x=r_1]}
}{ZLet}
\ruleIN
{
    \{ \SourceStepZ{e_i}{a_i}{r_i} \}_i
    \quad
    \SourceStepZ{e}{a}{r}
}
{
    \SourceStepZ{@let folds@~\{x_i~=~e_i,~\_\}_i~@in@~e}{\{a_i\}_i,a}{r[\{x_i=r_i\}_i]}
}{ZFolds}
$$

$$
\boxed{\SourceStepK{\mi{acc}}{e}{\mi{acc}}{\mi{res}}}
$$


$$
\ruleAx
{
    \SourceStepK{()}{x}{()}{x}
}{KVar}
\ruleAx
{
    \SourceStepK{()}{v}{()}{v}
}{KValue}
\ruleIN
{
    \SourceStepK{a}{e}{a'}{r'}
}
{
    \SourceStepK{a}{\lambda{}x~:~\tau.~e}{a'}{\lambda{}x~:~\tau.~r'}
}{KLam}
$$

$$
\ruleAx
{
    \SourceStepK{\langle e_1,~e_2 \rangle}{\langle \cdots \rangle}{e_2}{e_1}
}{KStream1}
\ruleAx
{
    \SourceStepK{\langle \rangle}{\langle \cdots \rangle}{\langle\rangle}{\bot}
}{KStream2}
\ruleIN
{
    \SourceStepK{a_1}{e_1}{a_1'}{r_1'}
    \quad
    \SourceStepK{a_1}{e_2}{a_2'}{r_2'}
}
{
    \SourceStepK{a_1,a_2}{@when@~e_1~e_2}{a_1',a_2'}{\Ifnbot{r_1'}{r_1'}{r_2'}}
}{KWhen}
$$

$$
\ruleIN
{
    \SourceStepK{a_1}{e_1}{a_1'}{r_1'}
    \quad
    \SourceStepK{a_2}{e_2}{a_2'}{r_2'}
}
{
    \SourceStepK{a_1,a_2}{@sample@~e_1~e_2}{a_1',a_2'}{\Ifnbot{r_1'}{r_2'}{\bot}}
}{KSample}
$$

$$
\ruleIN
{
    \SourceStepK{a}{e}{a'}{r'}
}
{
    \SourceStepK{a}{@mapS@~f~e}{a'}{\Ifnbot{r'}{f~r'}{\bot}}
}{KMapS}
\ruleIN
{
    \SourceStepK{a_1}{e_1}{a_1'}{r_1'}
    \quad
    \SourceStepK{a_2}{e_2}{a_2'}{r_2'}
}
{
    \SourceStepK{a_1,a_2}{@zipS@~e_1~e_2}{a_1',a_2'}{r_1',r_2'}
}{KZipS}
$$

$$
\ruleIN
{
    \SourceStepK{a}{e}{a'}{r'}
}
{
    \SourceStepK{a}{@mapF@~f~e}{a'}{f~r}
}{KMapF}
\ruleIN
{
    \SourceStepK{a_1}{e_1}{a_1'}{r_1'}
    \quad
    \SourceStepK{a_2}{e_2}{a_2'}{r_2'}
}
{
    \SourceStepK{a_1,a_2}{@zipF@~e_1~e_2}{a_1',a_2'}{r_1',r_2'}
}{KZipF}
$$

$$
\ruleIN
{
    \SourceStepK{a_1}{e_1}{a_1'}{r_1'}
    \quad
    \SourceStepK{a_2}{e_2}{a_2'}{r_2'}
}
{
    \SourceStepK{a_1,a_2}{@let@~x~=~e_1~@in@~e_2}{a_1',a_2'}{r_2[x=r_1]}
}{KLet}
\ruleIN
{
    \{ \SourceStepK{a_i}{e_i}{a_i'}{r_i'} \}_i
    \quad
    \SourceStepK{a}{e}{a'}{r'}
}
{
    \SourceStepK{\{a_i\}_i,a}{@let folds@~\{x_i~=~\_,~e_i\}_i~@in@~e}{\{a_i'\}_i,a'}{r'[\{x_i=r_i'\}_i]}
}{KFolds}
$$



$$
\boxed{\SourceStepX{e}{e}}
$$

$$
\ruleIN
{
    \SourceStepX{e_1}{e_1'}
}
{
    \SourceStepX{e_1~e_2}{e_1'~e_2}
}{XApp1}
\ruleIN
{
    \SourceStepX{e_2}{e_2'}
}
{
    \SourceStepX{v_1~e_2}{v_1~e_2'}
}{XApp2}
\ruleAx
{
    \SourceStepX{(\lambda{}x.~e_1)~e_2}{e_1[x:=e_2]}
}{XApp3}
$$
$$
\ruleIN
{
    \SourceStepX{e_1}{e_1'}
}
{
    \SourceStepX{@let@~x~=~e_1~@in@~e_2}{@let@~x~=~e_1'~@in@~e_2}
}{XLet1}
\ruleAx
{
    \SourceStepX{@let@~x~=~v_1~@in@~e_2}{e_2[x:=v_1]}
}{XLet2}
$$

\caption{Evaluation rules}
\label{fig:source:eval}
\end{figure*}




\subsection{Obvious extensions}
Obvious extensions and things I removed for simplification.

\subsubsection{Fixpoint}
$@fix@~(x~:~\tau)~@in@~e$.
By requiring $\tau$ to have kind @Data@ rather than @Flow@, we can allow recursion at runtime, while still ensuring a static dataflow graph.
I am tempted to kill the kinds for the simple version because with no fixpoint they do not serve much purpose (the dataflow graph is static regardless).

\subsubsection{ Maps and zips }
@mapS/mapF@, @zipS/zipF@.
For the simple version these could be merged into a single map/zip operating over just @Fold@s.
You can convert from a @Stream@ to a @Fold@ and back easily enough - you need a zero element for the @Fold@ but whatever.

\subsubsection{ Subrates }
$@let-rate@~(c'~:_k~@Clock@)~(w~:~c'~\le~c)~=~(e~:~@Stream@~c~\BB)$ and $@subrate@~(w~:~c'~\le~c)~(e~:~@Stream@~c~\tau)~:~@Stream@~c'~\tau$.
You can fake this by using Folds anyway.

\subsubsection{ Unpack }
$@let-unpack@~(x_\tau~:_k~@Data@)~(x_z~:~x_\tau)~(x_k~:~x_\tau~\to~@Fold@~x_\tau)~(x_r~:~x_\tau~\to~\tau)~=~(e~:~@Fold@~\tau)$.
Pulling apart folds to make new folds.
Useful for groups, segmented operations, and so on where the fold can be restarted or actually be many folds, as well as for other higher order folds.


%!TEX root = ../Main.tex
\section{Types}
\label{s:Types}

%!TEX root = ../Main.tex

\begin{figure*}

\begin{tabbing}
MM \= MM \= \kill
$\mi{Kind}$
\GrammarDef $@Data@~|~@Clock@~|~@Flow@$ \\

T
\GrammarDef $x~|~\NN~|~\BB~|~@()@
    ~|~ @List@~T
    ~|~ T~\to~T
    ~|~ (T,T)$
\GrammarAlt $@Stream@~T~T
    ~|~ @Fold@~T$
    \\
\end{tabbing}

\caption{Types and kinds}
\label{fig:source:type:types}
\end{figure*}


\begin{figure*}

$$
\boxed{\TypeWf{\Delta}{\tau}{\mi{Kind}}}
$$

$$
\ruleIN
{
    x~:_k~k~\in~\Delta
}
{
    \TypeWf{\Delta}{x}{k}
}{TVar}
\ruleAx
{
    \TypeWf{\Delta}{\NN}{@Data@}
}{TNat}
\ruleAx
{
    \TypeWf{\Delta}{\BB}{@Data@}
}{TBool}
\ruleAx
{
    \TypeWf{\Delta}{@()@}{@Data@}
}{TUnit}
$$

$$
\ruleIN
{
    \TypeWf{\Delta}{\tau}{@Data@}
}
{
    \TypeWf{\Delta}{@List@~\tau}{@Data@}
}{TList}
\ruleIN
{
    \TypeWf{\Delta}{\tau_1}{k_1}
    \quad
    \TypeWf{\Delta}{\tau_2}{k_2}
}
{
    \TypeWf{\Delta}{\tau_1~\to~\tau_2}{k_2}
}{TFun}
$$

$$
\ruleIN
{
    \TypeWf{\Delta}{c}{@Clock@}
    \quad
    \TypeWf{\Delta}{\tau}{@Data@}
}
{
    \TypeWf{\Delta}{@Stream@~c~\tau}{@Flow@}
}{TStream}
\ruleIN
{
    \TypeWf{\Delta}{\tau}{@Data@}
}
{
    \TypeWf{\Delta}{@Fold@~\tau}{@Flow@}
}{TFold}
$$



\caption{Kinds of types}
\label{fig:source:type:kinds}
\end{figure*}


\begin{figure*}

$$
\boxed{\Typecheck{\Delta}{\Gamma}{e}{T}}
$$


$$
\ruleIN
{
    (x~:~\tau)~\in~\Gamma
}
{ 
    \Typecheck{\Delta}{\Gamma}{x}{\tau}
}{TcVar}
\ruleIN
{
    v~:~\tau
}
{ 
    \Typecheck{\Delta}{\Gamma}{v}{\tau}
}{TcValue}
\ruleIN
{
    \Typecheck{\Delta}{\Gamma}{e_1}{\tau_1~\to~\tau_2}
    \quad
    \Typecheck{\Delta}{\Gamma}{e_2}{\tau_1}
}
{ 
    \Typecheck{\Delta}{\Gamma}{e_1~e_2}{\tau_2}
}{TcApp}
$$

$$
\ruleIN
{
    \Typecheck{\Delta}{\Gamma,~x~:~\tau}{e}{\tau'}
}
{
    \Typecheck{\Delta}{\Gamma}{\lambda{}x~:~\tau.~e}{\tau~\to~\tau'}
}{TcLam}
$$

$$
\ruleAx
{
    \Typecheck{\Delta}{\Gamma}{\langle \rangle}{@Stream@~\langle\rangle~\tau}
}{TcStrm1}
\ruleIN
{
    \Typecheck{\Delta}{\Gamma}{e_1}{\tau}
    \quad
    \Typecheck{\Delta}{\Gamma}{e_2}{@Stream@~c~\tau}
}
{
    \Typecheck{\Delta}{\Gamma}{\langle e_1,~e_2 \rangle}{@Stream@~\langle @T@,~c \rangle~\tau}
}{TcStrm2}
\ruleIN
{
    \Typecheck{\Delta}{\Gamma}{e_2}{@Stream@~c~\tau}
}
{
    \Typecheck{\Delta}{\Gamma}{\langle \bot,~e_2 \rangle}{@Stream@~\langle @F@,~c \rangle~\tau}
}{TcStrm3}
$$

$$
\ruleIN
{
    \Typecheck{\Delta}{\Gamma}{e_1}{@Stream@~c~\tau}
    \quad
    \Typecheck{\Delta}{\Gamma}{e_2}{@Fold@~\tau}
}
{
    \Typecheck{\Delta}{\Gamma}{@when@~e_1~e_2}{@Fold@~\tau}
}{TcWhen}
\ruleIN
{
    \Typecheck{\Delta}{\Gamma}{e_1}{@Stream@~c~\tau'}
    \quad
    \Typecheck{\Delta}{\Gamma}{e_2}{@Fold@~\tau}
}
{
    \Typecheck{\Delta}{\Gamma}{@sample@~e_1~e_2}{@Stream@~c~\tau}
}{TcSample}
$$

$$
\ruleIN
{
    \Typecheck{\Delta}{\Gamma}{e_1}{\tau_1~\to~\tau_2}
    \quad
    \Typecheck{\Delta}{\Gamma}{e_2}{@Stream@~c~\tau_1}
}
{
    \Typecheck{\Delta}{\Gamma}{@mapS@~e_1~e_2}{@Stream@~c~\tau_2}
}{TcMapS}
\ruleIN
{
    \Typecheck{\Delta}{\Gamma}{e_1}{@Stream@~c~\tau_1}
    \quad
    \Typecheck{\Delta}{\Gamma}{e_2}{@Stream@~c~\tau_2}
}
{
    \Typecheck{\Delta}{\Gamma}{@zipS@~e_1~e_2}{@Stream@~c~(\tau_1,\tau_2)}
}{TcZipS}
$$

$$
\ruleIN
{
    \Typecheck{\Delta}{\Gamma}{e_1}{\tau_1~\to~\tau_2}
    \quad
    \Typecheck{\Delta}{\Gamma}{e_2}{@Fold@~\tau_1}
}
{
    \Typecheck{\Delta}{\Gamma}{@mapF@~e_1~e_2}{@Fold@~\tau_2}
}{TcMapF}
\ruleIN
{
    \Typecheck{\Delta}{\Gamma}{e_1}{@Fold@~\tau_1}
    \quad
    \Typecheck{\Delta}{\Gamma}{e_2}{@Fold@~\tau_2}
}
{
    \Typecheck{\Delta}{\Gamma}{@zipF@~e_1~e_2}{@Fold@~(\tau_1,\tau_2)}
}{TcZipF}
$$

$$
\ruleIN
{
    \Typecheck{\Delta}{\Gamma}{e_1}{\tau_1}
    \quad
    \Typecheck{\Delta}{\Gamma,~x~:~\tau_1}{e_2}{\tau_2}
}
{
    \Typecheck{\Delta}{\Gamma}{@let@~x~=~e_1~@in@~e_2}{\tau_2}
}{TcLet}
$$

$$
\ruleIN
{
    \Gamma'~=~\Gamma,~\{~x_i~:~@Fold@~\tau_i~\}_i
    \quad
    \{
    \Typecheck{\Delta}{\Gamma}{e_{z_i}}{\tau_i}
    \}_i
    \quad
    \{
    \Typecheck{\Delta}{\Gamma'}{e_{k_i}}{@Fold@~\tau_i}
    \}_i
    \quad
    \Typecheck{\Delta}{\Gamma'}{e}{\tau}
}
{
    \Typecheck{\Delta}{\Gamma}
        {@let@~@folds@~\{~x_i~=~e_{z_i}~@then@~e_{k_i}~\}_i~@in@~e}
        {\tau}
}{TcLetFolds}
$$


\caption{Types of expressions}
\label{fig:source:type:exp}
\end{figure*}




%!TEX root = ../Main.tex
\section{Core language}
\label{s:Core}

This section presents the Core language, which we use as an intermediate language before converting to imperative code.


\subsection{Grammar}

%!TEX root = ../Main.tex

\begin{figure}

\begin{tabbing}
MMMM \= MM \= MM \= MM \= \kill
$\mi{e_c}$
\GrammarDef $x$ \> $|$ \> $v$
\GrammarAlt $e_c~e_c$ \> $|$ \> $p$
\\
\\
$c$
\GrammarDef $x$ \> $|$ \> $\mi{Always}$
\\
\\
$\mi{FoldNest}$
\GrammarDef $@nest@~\mi{Op}*$
\\
\\
$\mi{Op}$
\GrammarDef $@let@~x~=~e_c~@when@~c$
\GrammarAlt $@fold@~x~=~e_c,~(e_c~@when@~c~@else@~e_c)$
\\
\\
$\mi{Bind}$
\GrammarDef $x~=~\lambda{}x.~e_c$
\\
\\
$\mi{Program}$
\GrammarDef @inputs@ \\
    \>      \> \> $(x~:~@Stream@~c~\tau)*$ \\
    \>      \> @binds@ \\
    \>      \> \> $\mi{Bind}*$ \\
    \>      \> @with@ \\
    \>      \> \> $\mi{FoldNest}*$ \\
    \>      \> @in@ \\
    \>      \> \> $x*$ \\
\end{tabbing}


\caption{Grammar for Icicle Core}
\label{fig:core:grammar}
\end{figure}



%!TEX root = ../Main.tex

\begin{figure*}

$$
\boxed{\CoreComp{e}{[Bind]}{[Op]}{x}}
$$

$$
\ruleAx
{
    \CoreComp{v}{\emptyset}{@let@~x'~=~v}{x'}
}{CVal}
\ruleAx
{
    \CoreComp{x}{\emptyset}{\emptyset}{x}
}{CVar}
\ruleIN
{
    \CoreComp{e_1}{b_1}{o_1}{x_1}
    \quad
    \CoreComp{e_2}{b_2}{o_2}{x_2}
}{
    \CoreComp{e_1~e_2}{b_1;b_2}{o_1;o_2;~@let@~x'~=~x_1~x_2}{x'}
}{CApp}
$$


\caption{Conversion to Core}
\label{fig:core:compile}
\end{figure*}





%!TEX root = ../Main.tex
\section{Code generation}
\label{s:Generation}

Code generation should be pretty easy I guess cite flow fusion paper.

Because we don't have any zips or other that takes multiple streams, we don't have to deal with hard problems.

While not all streams have the same rate, there is no way to express an operation that takes two different rates.


%!TEX root = ../Main.tex
\section{Conclusion}
\label{s:Conclusion}





%!TEX root = ../Main.tex
\section{Related work}
\label{s:Related}
\subsection{Data flow languages}

The closest related work are synchronous data flow languages such as {\sc Lustre}, Icicle programs are restricted to those that can be computed in bounded memory.
{\sc Lustre}\cite{halbwachs1991synchronous} achieves this in three main ways:

\begin{enumerate}
\item They allow only restricted set of primitives such as @when@ for filtering, @pre@ for a single element buffer, and so on.
\item Cycles in the graph must contain at least one @pre@, to break the dependency loop.
\item Operators such as addition can only be applied to input streams with the same clock or rate; an expression like @(X when C)@ @+@ @(Y when (not C))@ would require an unbounded buffer\CITE{Lustre clock stuff}.
\end{enumerate}

In summary, while our language is quite similar to existing synchronous data flow languages, the extra restrictions we impose allow us to use a simpler type system.


% %!TEX root = ../Main.tex

\newcommand\JudgeK[2]
{       #1 :: #2
}

\newcommand\JudgeT[3]
{       #1 \vdash #2 :: #3
}

\newcommand\JudgeTS[5]
{       #1 \vdash #2 :: #3 ~;~ #4 ~;~ #5
}

\newcommand\kbox        {\textrm{\textbf{box}}}
\newcommand\krun        {\textrm{\textbf{run}}}
\newcommand\kthen       {\textrm{\textbf{then}}}
\newcommand\kpure       {\textrm{\textbf{pure}}}

\newcommand\rData       {\textrm{Data}}
\newcommand\rClock      {\textrm{Clock}}
\newcommand\rDefn       {\textrm{Defn}}
\newcommand\rComp       {\textrm{Comp}}

\newcommand\rStream     {\textrm{Stream}}
\newcommand\rArray      {\textrm{Array}}
\newcommand\rNat        {\textrm{Nat}}
\newcommand\rBool       {\textrm{Bool}}

\newcommand\rdrain      {\textrm{drain}}
\newcommand\rstream     {\textrm{stream}}
\newcommand\rsum        {\textrm{sum}}
\newcommand\rsmap       {\textrm{smap}}
\newcommand\rsfold      {\textrm{sfold}}
\newcommand\rsscan      {\textrm{sscan}}
\newcommand\rsfilter    {\textrm{sfilter}}


\begin{figure*}
$$
\boxed{\JudgeT{\Gamma}{e}{\tau}}
$$

$$
\ruleI
{       x : \tau \in \Gamma
}
{       \JudgeT{\Gamma}{x}{\tau}
}
\textrm{(TyVar)}
\quad\quad
\ruleI
{       \JudgeT{\Gamma}{e_1}{\tau_1 \to \tau_2}
        \quad
        \JudgeT{\Gamma}{e_2}{\tau_1}
}
{       \JudgeT{\Gamma}{e_1~e_2}{\tau_2}       
}
\textrm{(TyApp)}
\quad\quad
\ruleI
{       \JudgeT{\Gamma,x:\tau_1}{e_2}{\tau_2}
}
{       \JudgeT{\Gamma}{\lambda x : \tau_1}{\tau_1 \to \tau_2}
}
\textrm{(TyAbs)}
$$



% -- Effectful --------------------------------------------
$$
\boxed{\JudgeTS{\Gamma}{e}{\tau}{n}{k}}
$$

$$
\ruleI
{       \JudgeTS{\Gamma}{e}{\tau}{n}{k}
}
{       \JudgeT{\Gamma}{\kbox~ e}{\Box~ n~ k~ \tau}
}
\textrm{(TyBox)}
\quad\quad
\ruleI
{       \JudgeT{\Gamma}{e}{\tau}
}
{       \JudgeTS{\Gamma}{\kpure~e}{\tau}{n}{k}
}
\textrm{(TyPure)}
$$

$$
\ruleI
{       \{ \JudgeT
                {\Gamma}{e_i}{\Box~ n_i~ k~ \tau_i} \}^i 
        \quad
        \JudgeTS
                {\Gamma,~ \{x_i : \tau_i \}^i}
                {e'}{\tau'}{n'}{k'}
}
{       \JudgeTS
                {\Gamma}
                {\krun~ \{ x_i : \tau_i = e_i \}^i ~\kthen~ e'}
                {\tau'}
                {max~ \{ n_i \}^i + n'}
                {k'}
}
\textrm{(TyRun)}
$$

% -- Prims ------------------------------------------------
\vspace{2em}
$$
\begin{array}{ll}
\rNat,\rBool      & :: \rData
\\[1ex]
%
\rArray           & :: \rClock \to \rData \to \rData
\\[1ex]
%
\rStream          & :: \rClock \to \rData \to \rDefn
\\[1ex]
%
\Box              & :: \rNat   \to \rClock \to \rData \to \rComp
\\[1ex]
%
\rdrain_{k,\tau}  & :: \rStream~k~\tau \to \Box^1_k~ (\rArray~ k~ \tau)
\\[1ex]
%
\rstream_{k,\tau} & :: \rArray~k~\tau  \to \rStream~k~\tau
\\[1ex]
%
\rsum_k           & :: \rStream~k~\rNat \to \Box^1_k~ \rNat
\\[1ex]
%
\rsmap_{k,a,b}    & :: (a \to b) \to \rStream~k~a \to \rStream~k~b
\\[1ex]
%
\rsfold_{k,a,b}   & :: (a \to b \to b) \to b \to \rStream~k~a \to \Box^1_k~ b
\\[1ex]
%
\rsscan_{k,a,b}   & :: (a \to b \to b) \to b \to \rStream~k~a \to \rStream~k~b
\end{array}
$$

\caption{Typing}
\label{fig:source:type:modal}
\end{figure*}


\section*{Acknowledgements}

\bibliographystyle{plain}
\bibliography{Main}

\end{document}



\documentclass[preprint]{sigplanconf}
\usepackage{amssymb}
\usepackage{amsthm}
\usepackage{graphicx}
\usepackage{amsmath}
\usepackage{mathptmx}
\usepackage{mathtools}
\usepackage{stmaryrd}
\usepackage{hyperref}
\usepackage{alltt}
\usepackage{url}
\usepackage{float}
\usepackage{style/code}
\usepackage{style/proof}
\usepackage{style/utils}
\usepackage{style/judgements}

% -----------------------------------------------------------------------------
\begin{document}

% \exclusivelicense
% \conferenceinfo{}{}
% \copyrightyear{2015}
% \copyrightdata{}
\doi{}
% \pagenumbering{gobble} 

\title{Icicle: fuse your queries}

\authorinfo{
  Amos Robinson$^\dagger$$^\ddagger$
  \and Ben Lippmeier$^\dagger$
}{
  \vspace{5pt}
  \shortstack{
    $^\dagger$Computer Science and Engineering \\
    University of New South Wales, Australia \\[2pt]
    \textsf{amosr,benl@cse.unsw.edu.au}
  }
  \shortstack{
    $^\ddagger$Ambiata      \\
    Big data and shit       \\[2pt]
    \textsf{amos.robinson@ambiata.com}
  }
}

\maketitle
\makeatactive

\begin{abstract}
When streaming a large amount of data, simply iterating over the data may take hours.
If multiple queries are to be performed, it is important that work is not duplicated.
Queries that can be performed together must be performed in the same iteration.
We introduce a simple streaming language for computing queries in a single-pass over the data.
By using an appropriate intermediate language we guarantee fusion between queries on the same input streams, before extracting efficient C code.
\end{abstract}


\category
	{D.3.4}
	{Programming Languages}
	{Processors---Compilers; Optimization}

\terms
	Languages, Performance

\keywords
	Arrays; Fusion


\section{Introdution}

This is some stuff.

%!TEX root = ../Main.tex
\section{Icicle Source}
\label{s:Source}

The two main types in Icicle are @Stream@ and @Fold@.
Streams represent data values as they flow through the program.
Streams do not always have data flowing through them, but have an associated clock which describes when data occurs.
Two streams with the same clock both have data at the same time.
Folds, on the other hand, are results of computations over the stream data that has been seen.
Folds always have a well-defined value.
Folds can be easily converted to streams, by sampling their current value whenever the stream clock is true.
It is harder to convert a stream to a fold, as the stream has gaps where it is undefined, and one must specify how to fill the gaps (hold last, fill with zero, etc).
Finally, folds can be computed recursively based on their previous values.

%!TEX root = ../Main.tex

\begin{figure}

\begin{tabbing}
MMMM \= MM \= MMMMMMMMMMM \= MM \= \kill
$\mi{e}$
\GrammarDef $x$
\> $|$ \> $v$
\GrammarAlt $e~e$
\> $|$ \> $\lambda{}x~:~\tau.~e$

\\
\GrammarAlt $@when@~e~e$
\> $|$ \> $@sample@~e~e$
\GrammarAlt $@mapS@~e~e$
\> $|$ \> $@zipS@~e~e$
\GrammarAlt $@mapF@~e~e$
\> $|$ \> $@zipF@~e~e$
\\
\GrammarAlt $@let@~\mi{Let}~@in@~e$
\\
\GrammarAlt $\langle e,~ e \rangle ~|~ \langle\rangle$
\\
\\

$\mi{Let}$
\GrammarDef $x~=~e$
\GrammarAlt $@folds@~\{~x_i~=~e_i,~e_i~\}_i$
\\
\\

$\mi{v}$
\GrammarDef $\NN ~|~ \BB ~|~ [v]$
\GrammarAlt $(v,~v) ~|~ \bot$
\\
\\


$\mi{Program}$
\GrammarDef @inputs@~$\{ x_i~:~@Stream@~c_i~\tau_i \}_i~@in@~\mi{e}$ \\
\end{tabbing}

\caption{Grammar for Icicle Source}
\label{fig:source:grammar}
\end{figure}


The grammar for Icicle is given in figure~\ref{fig:source:grammar}.
The first four rules are rather standard lambda calculus.

Next are the primitives.
@when@ takes a stream and a fold, and returns a new fold whose value is the stream \emph{when}ever the stream is defined, and the fold otherwise.
This is used for defining folds that depend on stream values.
@sample@ takes a stream of any type and a fold, creates a new stream using the values of the fold.
@mapS@ and @zipS@ perform map and zip operations on streams of the same rate.
@mapF@ and @zipF@ perform map and zip operations on folds.

Let expressions such as $@let@~x~=e~@in@~e$ are as usual. 
Folds are defined using the syntax @let folds@ syntax. 
Multiple folds can be defined together, and each fold is defined by the initial value, and the ``kons'' part, defining the next value.

For example, a simple fold comprised of nothing but $0$
\begin{tabbing}
MM \= \kill
$@let folds@~\mi{zeros}~=~0,~\mi{zeros}$ \\
$@in@~\mi{zeros}$
\\
\\
$\mi{zeros}$ \> $=~\langle 0~0~0~0~\cdots~\rangle$ \\
\end{tabbing}

When computing the sum over another fold, the newly-defined fold and the input fold are zipped and then added together.
\begin{tabbing}
$\mi{sum}~=~\lambda{}(\mi{values}~:~@Fold@~\NN).$ \\
$@let folds@~\mi{s}~=~0,~@mapF@~(+)~(@zipF@~\mi{s}~\mi{values})$ \\
$@in@~\mi{s}$
\\
\\
$\mi{sum}~\langle 0~1~2~3~\cdots~\rangle$ \\
$~=~\langle 0~1~3~6~\cdots~\rangle$ \\
\end{tabbing}

This $\mi{sum}$ function requires a fold argument, but it can be applied to a stream $s$ by creating a fold that is the stream when it is defined, and $0$ otherwise: $@when@~s~zeros$.
A more general way is to define a new $\mi{sum'}$ function that works directly over streams:
\begin{tabbing}
MMMMMM \= MM \= MM \kill
$\mi{sum'}~=~\lambda{}(\mi{values}~:~@Stream@~c~\NN).$ \\
$@let folds@~\mi{s}~=~0,$ \\
\> $@when@$ \> $(@mapS@~(+)~(@zipS@~(@sample@~\mi{values}~\mi{s})~\mi{values}))$ \\
\> \> $~\mi{s}$ \\
$@in@~\mi{s}$
\\
\\
$\mi{sum'}~\langle \bot~1~\bot~3~\cdots~\rangle$ \\
$~=~\langle 0~1~1~4~\cdots~\rangle$ \\
\end{tabbing}

Here, the fold $s$ is sampled to a stream with the same clock as the input stream.
Whenever the input stream is defined, the previous value of $s$ and the current value of the $\mi{values}$ will be added together, producing the new sum.
When the input stream is not defined, the previous value of $s$ is used unchanged.

When referring to the newly defined folds in the right-hand-side of the fold (the kons), all values refer to the immediate previous value.
The order of the fold bindings has no effect on the program.
\begin{tabbing}
MM \= \kill
$@let folds@$ \\
\> $\mi{one}~=~1,~\mi{zero}$ \\
\> $\mi{zero}~=~0,~\mi{one}$ \\
$@in@~\mi{one}$
\\
\\
$\mi{one}$  \> $=~\langle 1~0~1~0~\cdots~\rangle$ \\
$\mi{zero}$ \> $=~\langle 0~1~0~1~\cdots~\rangle$ \\
\end{tabbing}

The grammar also has rules for stream literals: $\langle e_1,~e_2 \rangle$ and $\langle\rangle$. The stream can contain bottoms, $\bot$, which means that the stream is undefined at that point.
Stream literals have concrete clock types, for example $\langle T,~F \rangle$ for $\langle 1,~\bot \rangle$, while input streams have existential clocks: their actual clock is unknown until runtime.
These stream literals are not expected to occur in an actual program until evaluation.


Yes, I realise the grammar doesn't have @+@ or @/@, and that it doesn't have polymorphism.
For the examples, assume that a specialisation pass occurs, and that there is a ``sensible set of primitives on natural numbers and lists''.

Define group here.
\begin{code}
data Unpack r
 = forall a.
    Unpack a (Fold (a -> a)) (a -> r)

sum' vs = Unpack 0 (+vs) id

group (key : Stream c k)
      (val : Unpack v)
           : [(k,v)]
group key (Unpack z k r)
 = trivial
\end{code}



%!TEX root = ../Main.tex

\begin{figure*}

$$
\boxed{\SourceStepZ{e}{\mi{acc}}{\mi{res}}}
$$

$$
\ruleAx
{
    \SourceStepZ{x}{()}{x}
}{ZVar}
\ruleAx
{
    \SourceStepZ{v}{()}{v}
}{ZValue}
\ruleIN
{
    \SourceStepZ{e}{a}{r}
}
{
    \SourceStepZ{\lambda{}x~:~\tau.~e}{a}{\lambda{}x~:~\tau.~r}
}{ZLam}
$$

$$
\ruleAx
{
    \SourceStepZ{\langle e_1,~e_2 \rangle}{\langle e_1,~e_2 \rangle}{\bot}
}{ZStream1}
\ruleAx
{
    \SourceStepZ{\langle \rangle}{\langle \rangle}{\bot}
}{ZStream2}
\ruleIN
{
    \SourceStepZ{e_1}{a_1}{r_1}
    \quad
    \SourceStepZ{e_2}{a_2}{r_2}
}
{
    \SourceStepZ{@when@~e_1~e_2}{a_1,a_2}{r_2}
}{ZWhen}
\ruleIN
{
    \SourceStepZ{e_1}{a_1}{r_1}
    \quad
    \SourceStepZ{e_2}{a_2}{r_2}
}
{
    \SourceStepZ{@sample@~e_1~e_2}{a_1,a_2}{\bot}
}{ZSample}
$$

$$
\ruleIN
{
    \SourceStepZ{e}{a}{r}
}
{
    \SourceStepZ{@mapS@~f~e}{a}{\bot}
}{ZMapS}
\ruleIN
{
    \SourceStepZ{e_1}{a_1}{r_1}
    \quad
    \SourceStepZ{e_2}{a_2}{r_2}
}
{
    \SourceStepZ{@zipS@~e_1~e_2}{a_1,a_2}{\bot}
}{ZZipS}
$$

$$
\ruleIN
{
    \SourceStepZ{e}{a}{r}
}
{
    \SourceStepZ{@mapF@~f~e}{a}{f~r}
}{ZMapF}
\ruleIN
{
    \SourceStepZ{e_1}{a_1}{r_1}
    \quad
    \SourceStepZ{e_2}{a_2}{r_2}
}
{
    \SourceStepZ{@zipF@~e_1~e_2}{a_1,a_2}{r_1,r_2}
}{ZZipF}
$$

$$
\ruleIN
{
    \SourceStepZ{e_1}{a_1}{r_1}
    \quad
    \SourceStepZ{e_2}{a_2}{r_2}
}
{
    \SourceStepZ{@let@~x~=~e_1~@in@~e_2}{a_1,a_2}{r_2[x=r_1]}
}{ZLet}
\ruleIN
{
    \{ \SourceStepZ{e_i}{a_i}{r_i} \}_i
    \quad
    \SourceStepZ{e}{a}{r}
}
{
    \SourceStepZ{@let folds@~\{x_i~=~e_i,~\_\}_i~@in@~e}{\{a_i\}_i,a}{r[\{x_i=r_i\}_i]}
}{ZFolds}
$$

$$
\boxed{\SourceStepK{\mi{acc}}{e}{\mi{acc}}{\mi{res}}}
$$


$$
\ruleAx
{
    \SourceStepK{()}{x}{()}{x}
}{KVar}
\ruleAx
{
    \SourceStepK{()}{v}{()}{v}
}{KValue}
\ruleIN
{
    \SourceStepK{a}{e}{a'}{r'}
}
{
    \SourceStepK{a}{\lambda{}x~:~\tau.~e}{a'}{\lambda{}x~:~\tau.~r'}
}{KLam}
$$

$$
\ruleAx
{
    \SourceStepK{\langle e_1,~e_2 \rangle}{\langle \cdots \rangle}{e_2}{e_1}
}{KStream1}
\ruleAx
{
    \SourceStepK{\langle \rangle}{\langle \cdots \rangle}{\langle\rangle}{\bot}
}{KStream2}
\ruleIN
{
    \SourceStepK{a_1}{e_1}{a_1'}{r_1'}
    \quad
    \SourceStepK{a_1}{e_2}{a_2'}{r_2'}
}
{
    \SourceStepK{a_1,a_2}{@when@~e_1~e_2}{a_1',a_2'}{\Ifnbot{r_1'}{r_1'}{r_2'}}
}{KWhen}
$$

$$
\ruleIN
{
    \SourceStepK{a_1}{e_1}{a_1'}{r_1'}
    \quad
    \SourceStepK{a_2}{e_2}{a_2'}{r_2'}
}
{
    \SourceStepK{a_1,a_2}{@sample@~e_1~e_2}{a_1',a_2'}{\Ifnbot{r_1'}{r_2'}{\bot}}
}{KSample}
$$

$$
\ruleIN
{
    \SourceStepK{a}{e}{a'}{r'}
}
{
    \SourceStepK{a}{@mapS@~f~e}{a'}{\Ifnbot{r'}{f~r'}{\bot}}
}{KMapS}
\ruleIN
{
    \SourceStepK{a_1}{e_1}{a_1'}{r_1'}
    \quad
    \SourceStepK{a_2}{e_2}{a_2'}{r_2'}
}
{
    \SourceStepK{a_1,a_2}{@zipS@~e_1~e_2}{a_1',a_2'}{r_1',r_2'}
}{KZipS}
$$

$$
\ruleIN
{
    \SourceStepK{a}{e}{a'}{r'}
}
{
    \SourceStepK{a}{@mapF@~f~e}{a'}{f~r}
}{KMapF}
\ruleIN
{
    \SourceStepK{a_1}{e_1}{a_1'}{r_1'}
    \quad
    \SourceStepK{a_2}{e_2}{a_2'}{r_2'}
}
{
    \SourceStepK{a_1,a_2}{@zipF@~e_1~e_2}{a_1',a_2'}{r_1',r_2'}
}{KZipF}
$$

$$
\ruleIN
{
    \SourceStepK{a_1}{e_1}{a_1'}{r_1'}
    \quad
    \SourceStepK{a_2}{e_2}{a_2'}{r_2'}
}
{
    \SourceStepK{a_1,a_2}{@let@~x~=~e_1~@in@~e_2}{a_1',a_2'}{r_2[x=r_1]}
}{KLet}
\ruleIN
{
    \{ \SourceStepK{a_i}{e_i}{a_i'}{r_i'} \}_i
    \quad
    \SourceStepK{a}{e}{a'}{r'}
}
{
    \SourceStepK{\{a_i\}_i,a}{@let folds@~\{x_i~=~\_,~e_i\}_i~@in@~e}{\{a_i'\}_i,a'}{r'[\{x_i=r_i'\}_i]}
}{KFolds}
$$



$$
\boxed{\SourceStepX{e}{e}}
$$

$$
\ruleIN
{
    \SourceStepX{e_1}{e_1'}
}
{
    \SourceStepX{e_1~e_2}{e_1'~e_2}
}{XApp1}
\ruleIN
{
    \SourceStepX{e_2}{e_2'}
}
{
    \SourceStepX{v_1~e_2}{v_1~e_2'}
}{XApp2}
\ruleAx
{
    \SourceStepX{(\lambda{}x.~e_1)~e_2}{e_1[x:=e_2]}
}{XApp3}
$$
$$
\ruleIN
{
    \SourceStepX{e_1}{e_1'}
}
{
    \SourceStepX{@let@~x~=~e_1~@in@~e_2}{@let@~x~=~e_1'~@in@~e_2}
}{XLet1}
\ruleAx
{
    \SourceStepX{@let@~x~=~v_1~@in@~e_2}{e_2[x:=v_1]}
}{XLet2}
$$

\caption{Evaluation rules}
\label{fig:source:eval}
\end{figure*}




\subsection{Obvious extensions}
Obvious extensions and things I removed for simplification.

\subsubsection{Fixpoint}
$@fix@~(x~:~\tau)~@in@~e$.
By requiring $\tau$ to have kind @Data@ rather than @Flow@, we can allow recursion at runtime, while still ensuring a static dataflow graph.
I am tempted to kill the kinds for the simple version because with no fixpoint they do not serve much purpose (the dataflow graph is static regardless).

\subsubsection{ Maps and zips }
@mapS/mapF@, @zipS/zipF@.
For the simple version these could be merged into a single map/zip operating over just @Fold@s.
You can convert from a @Stream@ to a @Fold@ and back easily enough - you need a zero element for the @Fold@ but whatever.

\subsubsection{ Subrates }
$@let-rate@~(c'~:_k~@Clock@)~(w~:~c'~\le~c)~=~(e~:~@Stream@~c~\BB)$ and $@subrate@~(w~:~c'~\le~c)~(e~:~@Stream@~c~\tau)~:~@Stream@~c'~\tau$.
You can fake this by using Folds anyway.

\subsubsection{ Unpack }
$@let-unpack@~(x_\tau~:_k~@Data@)~(x_z~:~x_\tau)~(x_k~:~x_\tau~\to~@Fold@~x_\tau)~(x_r~:~x_\tau~\to~\tau)~=~(e~:~@Fold@~\tau)$.
Pulling apart folds to make new folds.
Useful for groups, segmented operations, and so on where the fold can be restarted or actually be many folds, as well as for other higher order folds.


%!TEX root = ../Main.tex
\section{Types}
\label{s:Types}

%!TEX root = ../Main.tex

\begin{figure*}

\begin{tabbing}
MM \= MM \= \kill
$\mi{Kind}$
\GrammarDef $@Data@~|~@Clock@~|~@Flow@$ \\

T
\GrammarDef $x~|~\NN~|~\BB~|~@()@
    ~|~ @List@~T
    ~|~ T~\to~T
    ~|~ (T,T)$
\GrammarAlt $@Stream@~T~T
    ~|~ @Fold@~T$
    \\
\end{tabbing}

\caption{Types and kinds}
\label{fig:source:type:types}
\end{figure*}


\begin{figure*}

$$
\boxed{\TypeWf{\Delta}{\tau}{\mi{Kind}}}
$$

$$
\ruleIN
{
    x~:_k~k~\in~\Delta
}
{
    \TypeWf{\Delta}{x}{k}
}{TVar}
\ruleAx
{
    \TypeWf{\Delta}{\NN}{@Data@}
}{TNat}
\ruleAx
{
    \TypeWf{\Delta}{\BB}{@Data@}
}{TBool}
\ruleAx
{
    \TypeWf{\Delta}{@()@}{@Data@}
}{TUnit}
$$

$$
\ruleIN
{
    \TypeWf{\Delta}{\tau}{@Data@}
}
{
    \TypeWf{\Delta}{@List@~\tau}{@Data@}
}{TList}
\ruleIN
{
    \TypeWf{\Delta}{\tau_1}{k_1}
    \quad
    \TypeWf{\Delta}{\tau_2}{k_2}
}
{
    \TypeWf{\Delta}{\tau_1~\to~\tau_2}{k_2}
}{TFun}
$$

$$
\ruleIN
{
    \TypeWf{\Delta}{c}{@Clock@}
    \quad
    \TypeWf{\Delta}{\tau}{@Data@}
}
{
    \TypeWf{\Delta}{@Stream@~c~\tau}{@Flow@}
}{TStream}
\ruleIN
{
    \TypeWf{\Delta}{\tau}{@Data@}
}
{
    \TypeWf{\Delta}{@Fold@~\tau}{@Flow@}
}{TFold}
$$



\caption{Kinds of types}
\label{fig:source:type:kinds}
\end{figure*}


\begin{figure*}

$$
\boxed{\Typecheck{\Delta}{\Gamma}{e}{T}}
$$


$$
\ruleIN
{
    (x~:~\tau)~\in~\Gamma
}
{ 
    \Typecheck{\Delta}{\Gamma}{x}{\tau}
}{TcVar}
\ruleIN
{
    v~:~\tau
}
{ 
    \Typecheck{\Delta}{\Gamma}{v}{\tau}
}{TcValue}
\ruleIN
{
    \Typecheck{\Delta}{\Gamma}{e_1}{\tau_1~\to~\tau_2}
    \quad
    \Typecheck{\Delta}{\Gamma}{e_2}{\tau_1}
}
{ 
    \Typecheck{\Delta}{\Gamma}{e_1~e_2}{\tau_2}
}{TcApp}
$$

$$
\ruleIN
{
    \Typecheck{\Delta}{\Gamma,~x~:~\tau}{e}{\tau'}
}
{
    \Typecheck{\Delta}{\Gamma}{\lambda{}x~:~\tau.~e}{\tau~\to~\tau'}
}{TcLam}
$$

$$
\ruleAx
{
    \Typecheck{\Delta}{\Gamma}{\langle \rangle}{@Stream@~\langle\rangle~\tau}
}{TcStrm1}
\ruleIN
{
    \Typecheck{\Delta}{\Gamma}{e_1}{\tau}
    \quad
    \Typecheck{\Delta}{\Gamma}{e_2}{@Stream@~c~\tau}
}
{
    \Typecheck{\Delta}{\Gamma}{\langle e_1,~e_2 \rangle}{@Stream@~\langle @T@,~c \rangle~\tau}
}{TcStrm2}
\ruleIN
{
    \Typecheck{\Delta}{\Gamma}{e_2}{@Stream@~c~\tau}
}
{
    \Typecheck{\Delta}{\Gamma}{\langle \bot,~e_2 \rangle}{@Stream@~\langle @F@,~c \rangle~\tau}
}{TcStrm3}
$$

$$
\ruleIN
{
    \Typecheck{\Delta}{\Gamma}{e_1}{@Stream@~c~\tau}
    \quad
    \Typecheck{\Delta}{\Gamma}{e_2}{@Fold@~\tau}
}
{
    \Typecheck{\Delta}{\Gamma}{@when@~e_1~e_2}{@Fold@~\tau}
}{TcWhen}
\ruleIN
{
    \Typecheck{\Delta}{\Gamma}{e_1}{@Stream@~c~\tau'}
    \quad
    \Typecheck{\Delta}{\Gamma}{e_2}{@Fold@~\tau}
}
{
    \Typecheck{\Delta}{\Gamma}{@sample@~e_1~e_2}{@Stream@~c~\tau}
}{TcSample}
$$

$$
\ruleIN
{
    \Typecheck{\Delta}{\Gamma}{e_1}{\tau_1~\to~\tau_2}
    \quad
    \Typecheck{\Delta}{\Gamma}{e_2}{@Stream@~c~\tau_1}
}
{
    \Typecheck{\Delta}{\Gamma}{@mapS@~e_1~e_2}{@Stream@~c~\tau_2}
}{TcMapS}
\ruleIN
{
    \Typecheck{\Delta}{\Gamma}{e_1}{@Stream@~c~\tau_1}
    \quad
    \Typecheck{\Delta}{\Gamma}{e_2}{@Stream@~c~\tau_2}
}
{
    \Typecheck{\Delta}{\Gamma}{@zipS@~e_1~e_2}{@Stream@~c~(\tau_1,\tau_2)}
}{TcZipS}
$$

$$
\ruleIN
{
    \Typecheck{\Delta}{\Gamma}{e_1}{\tau_1~\to~\tau_2}
    \quad
    \Typecheck{\Delta}{\Gamma}{e_2}{@Fold@~\tau_1}
}
{
    \Typecheck{\Delta}{\Gamma}{@mapF@~e_1~e_2}{@Fold@~\tau_2}
}{TcMapF}
\ruleIN
{
    \Typecheck{\Delta}{\Gamma}{e_1}{@Fold@~\tau_1}
    \quad
    \Typecheck{\Delta}{\Gamma}{e_2}{@Fold@~\tau_2}
}
{
    \Typecheck{\Delta}{\Gamma}{@zipF@~e_1~e_2}{@Fold@~(\tau_1,\tau_2)}
}{TcZipF}
$$

$$
\ruleIN
{
    \Typecheck{\Delta}{\Gamma}{e_1}{\tau_1}
    \quad
    \Typecheck{\Delta}{\Gamma,~x~:~\tau_1}{e_2}{\tau_2}
}
{
    \Typecheck{\Delta}{\Gamma}{@let@~x~=~e_1~@in@~e_2}{\tau_2}
}{TcLet}
$$

$$
\ruleIN
{
    \Gamma'~=~\Gamma,~\{~x_i~:~@Fold@~\tau_i~\}_i
    \quad
    \{
    \Typecheck{\Delta}{\Gamma}{e_{z_i}}{\tau_i}
    \}_i
    \quad
    \{
    \Typecheck{\Delta}{\Gamma'}{e_{k_i}}{@Fold@~\tau_i}
    \}_i
    \quad
    \Typecheck{\Delta}{\Gamma'}{e}{\tau}
}
{
    \Typecheck{\Delta}{\Gamma}
        {@let@~@folds@~\{~x_i~=~e_{z_i}~@then@~e_{k_i}~\}_i~@in@~e}
        {\tau}
}{TcLetFolds}
$$


\caption{Types of expressions}
\label{fig:source:type:exp}
\end{figure*}




%!TEX root = ../Main.tex
\section{Core language}
\label{s:Core}

This section presents the Core language, which we use as an intermediate language before converting to imperative code.


\subsection{Grammar}

%!TEX root = ../Main.tex

\begin{figure}

\begin{tabbing}
MMMM \= MM \= MM \= MM \= \kill
$\mi{e_c}$
\GrammarDef $x$ \> $|$ \> $v$
\GrammarAlt $e_c~e_c$ \> $|$ \> $p$
\\
\\
$c$
\GrammarDef $x$ \> $|$ \> $\mi{Always}$
\\
\\
$\mi{FoldNest}$
\GrammarDef $@nest@~\mi{Op}*$
\\
\\
$\mi{Op}$
\GrammarDef $@let@~x~=~e_c~@when@~c$
\GrammarAlt $@fold@~x~=~e_c,~(e_c~@when@~c~@else@~e_c)$
\\
\\
$\mi{Bind}$
\GrammarDef $x~=~\lambda{}x.~e_c$
\\
\\
$\mi{Program}$
\GrammarDef @inputs@ \\
    \>      \> \> $(x~:~@Stream@~c~\tau)*$ \\
    \>      \> @binds@ \\
    \>      \> \> $\mi{Bind}*$ \\
    \>      \> @with@ \\
    \>      \> \> $\mi{FoldNest}*$ \\
    \>      \> @in@ \\
    \>      \> \> $x*$ \\
\end{tabbing}


\caption{Grammar for Icicle Core}
\label{fig:core:grammar}
\end{figure}



%!TEX root = ../Main.tex

\begin{figure*}

$$
\boxed{\CoreComp{e}{[Bind]}{[Op]}{x}}
$$

$$
\ruleAx
{
    \CoreComp{v}{\emptyset}{@let@~x'~=~v}{x'}
}{CVal}
\ruleAx
{
    \CoreComp{x}{\emptyset}{\emptyset}{x}
}{CVar}
\ruleIN
{
    \CoreComp{e_1}{b_1}{o_1}{x_1}
    \quad
    \CoreComp{e_2}{b_2}{o_2}{x_2}
}{
    \CoreComp{e_1~e_2}{b_1;b_2}{o_1;o_2;~@let@~x'~=~x_1~x_2}{x'}
}{CApp}
$$


\caption{Conversion to Core}
\label{fig:core:compile}
\end{figure*}





%!TEX root = ../Main.tex
\section{Code generation}
\label{s:Generation}

Code generation should be pretty easy I guess cite flow fusion paper.

Because we don't have any zips or other that takes multiple streams, we don't have to deal with hard problems.

While not all streams have the same rate, there is no way to express an operation that takes two different rates.


%!TEX root = ../Main.tex
\section{Conclusion}
\label{s:Conclusion}





%!TEX root = ../Main.tex
\section{Related work}
\label{s:Related}
\subsection{Data flow languages}

The closest related work are synchronous data flow languages such as {\sc Lustre}, Icicle programs are restricted to those that can be computed in bounded memory.
{\sc Lustre}\cite{halbwachs1991synchronous} achieves this in three main ways:

\begin{enumerate}
\item They allow only restricted set of primitives such as @when@ for filtering, @pre@ for a single element buffer, and so on.
\item Cycles in the graph must contain at least one @pre@, to break the dependency loop.
\item Operators such as addition can only be applied to input streams with the same clock or rate; an expression like @(X when C)@ @+@ @(Y when (not C))@ would require an unbounded buffer\CITE{Lustre clock stuff}.
\end{enumerate}

In summary, while our language is quite similar to existing synchronous data flow languages, the extra restrictions we impose allow us to use a simpler type system.


% %!TEX root = ../Main.tex

\newcommand\JudgeK[2]
{       #1 :: #2
}

\newcommand\JudgeT[3]
{       #1 \vdash #2 :: #3
}

\newcommand\JudgeTS[5]
{       #1 \vdash #2 :: #3 ~;~ #4 ~;~ #5
}

\newcommand\kbox        {\textrm{\textbf{box}}}
\newcommand\krun        {\textrm{\textbf{run}}}
\newcommand\kthen       {\textrm{\textbf{then}}}
\newcommand\kpure       {\textrm{\textbf{pure}}}

\newcommand\rData       {\textrm{Data}}
\newcommand\rClock      {\textrm{Clock}}
\newcommand\rDefn       {\textrm{Defn}}
\newcommand\rComp       {\textrm{Comp}}

\newcommand\rStream     {\textrm{Stream}}
\newcommand\rArray      {\textrm{Array}}
\newcommand\rNat        {\textrm{Nat}}
\newcommand\rBool       {\textrm{Bool}}

\newcommand\rdrain      {\textrm{drain}}
\newcommand\rstream     {\textrm{stream}}
\newcommand\rsum        {\textrm{sum}}
\newcommand\rsmap       {\textrm{smap}}
\newcommand\rsfold      {\textrm{sfold}}
\newcommand\rsscan      {\textrm{sscan}}
\newcommand\rsfilter    {\textrm{sfilter}}


\begin{figure*}
$$
\boxed{\JudgeT{\Gamma}{e}{\tau}}
$$

$$
\ruleI
{       x : \tau \in \Gamma
}
{       \JudgeT{\Gamma}{x}{\tau}
}
\textrm{(TyVar)}
\quad\quad
\ruleI
{       \JudgeT{\Gamma}{e_1}{\tau_1 \to \tau_2}
        \quad
        \JudgeT{\Gamma}{e_2}{\tau_1}
}
{       \JudgeT{\Gamma}{e_1~e_2}{\tau_2}       
}
\textrm{(TyApp)}
\quad\quad
\ruleI
{       \JudgeT{\Gamma,x:\tau_1}{e_2}{\tau_2}
}
{       \JudgeT{\Gamma}{\lambda x : \tau_1}{\tau_1 \to \tau_2}
}
\textrm{(TyAbs)}
$$



% -- Effectful --------------------------------------------
$$
\boxed{\JudgeTS{\Gamma}{e}{\tau}{n}{k}}
$$

$$
\ruleI
{       \JudgeTS{\Gamma}{e}{\tau}{n}{k}
}
{       \JudgeT{\Gamma}{\kbox~ e}{\Box~ n~ k~ \tau}
}
\textrm{(TyBox)}
\quad\quad
\ruleI
{       \JudgeT{\Gamma}{e}{\tau}
}
{       \JudgeTS{\Gamma}{\kpure~e}{\tau}{n}{k}
}
\textrm{(TyPure)}
$$

$$
\ruleI
{       \{ \JudgeT
                {\Gamma}{e_i}{\Box~ n_i~ k~ \tau_i} \}^i 
        \quad
        \JudgeTS
                {\Gamma,~ \{x_i : \tau_i \}^i}
                {e'}{\tau'}{n'}{k'}
}
{       \JudgeTS
                {\Gamma}
                {\krun~ \{ x_i : \tau_i = e_i \}^i ~\kthen~ e'}
                {\tau'}
                {max~ \{ n_i \}^i + n'}
                {k'}
}
\textrm{(TyRun)}
$$

% -- Prims ------------------------------------------------
\vspace{2em}
$$
\begin{array}{ll}
\rNat,\rBool      & :: \rData
\\[1ex]
%
\rArray           & :: \rClock \to \rData \to \rData
\\[1ex]
%
\rStream          & :: \rClock \to \rData \to \rDefn
\\[1ex]
%
\Box              & :: \rNat   \to \rClock \to \rData \to \rComp
\\[1ex]
%
\rdrain_{k,\tau}  & :: \rStream~k~\tau \to \Box^1_k~ (\rArray~ k~ \tau)
\\[1ex]
%
\rstream_{k,\tau} & :: \rArray~k~\tau  \to \rStream~k~\tau
\\[1ex]
%
\rsum_k           & :: \rStream~k~\rNat \to \Box^1_k~ \rNat
\\[1ex]
%
\rsmap_{k,a,b}    & :: (a \to b) \to \rStream~k~a \to \rStream~k~b
\\[1ex]
%
\rsfold_{k,a,b}   & :: (a \to b \to b) \to b \to \rStream~k~a \to \Box^1_k~ b
\\[1ex]
%
\rsscan_{k,a,b}   & :: (a \to b \to b) \to b \to \rStream~k~a \to \rStream~k~b
\end{array}
$$

\caption{Typing}
\label{fig:source:type:modal}
\end{figure*}


\section*{Acknowledgements}

\bibliographystyle{plain}
\bibliography{Main}

\end{document}



\documentclass[preprint]{sigplanconf}
\usepackage{amssymb}
\usepackage{amsthm}
\usepackage{graphicx}
\usepackage{amsmath}
\usepackage{mathptmx}
\usepackage{mathtools}
\usepackage{stmaryrd}
\usepackage{hyperref}
\usepackage{alltt}
\usepackage{url}
\usepackage{float}
\usepackage{style/code}
\usepackage{style/proof}
\usepackage{style/utils}
\usepackage{style/judgements}

% -----------------------------------------------------------------------------
\begin{document}

% \exclusivelicense
% \conferenceinfo{}{}
% \copyrightyear{2015}
% \copyrightdata{}
\doi{}
% \pagenumbering{gobble} 

\title{Icicle: fuse your queries}

\authorinfo{
  Amos Robinson$^\dagger$$^\ddagger$
  \and Ben Lippmeier$^\dagger$
}{
  \vspace{5pt}
  \shortstack{
    $^\dagger$Computer Science and Engineering \\
    University of New South Wales, Australia \\[2pt]
    \textsf{amosr,benl@cse.unsw.edu.au}
  }
  \shortstack{
    $^\ddagger$Ambiata      \\
    Big data and shit       \\[2pt]
    \textsf{amos.robinson@ambiata.com}
  }
}

\maketitle
\makeatactive

\begin{abstract}
When streaming a large amount of data, simply iterating over the data may take hours.
If multiple queries are to be performed, it is important that work is not duplicated.
Queries that can be performed together must be performed in the same iteration.
We introduce a simple streaming language for computing queries in a single-pass over the data.
By using an appropriate intermediate language we guarantee fusion between queries on the same input streams, before extracting efficient C code.
\end{abstract}


\category
	{D.3.4}
	{Programming Languages}
	{Processors---Compilers; Optimization}

\terms
	Languages, Performance

\keywords
	Arrays; Fusion


\section{Introdution}

This is some stuff.

%!TEX root = ../Main.tex
\section{Icicle Source}
\label{s:Source}

The two main types in Icicle are @Stream@ and @Fold@.
Streams represent data values as they flow through the program.
Streams do not always have data flowing through them, but have an associated clock which describes when data occurs.
Two streams with the same clock both have data at the same time.
Folds, on the other hand, are results of computations over the stream data that has been seen.
Folds always have a well-defined value.
Folds can be easily converted to streams, by sampling their current value whenever the stream clock is true.
It is harder to convert a stream to a fold, as the stream has gaps where it is undefined, and one must specify how to fill the gaps (hold last, fill with zero, etc).
Finally, folds can be computed recursively based on their previous values.

%!TEX root = ../Main.tex

\begin{figure}

\begin{tabbing}
MMMM \= MM \= MMMMMMMMMMM \= MM \= \kill
$\mi{e}$
\GrammarDef $x$
\> $|$ \> $v$
\GrammarAlt $e~e$
\> $|$ \> $\lambda{}x~:~\tau.~e$

\\
\GrammarAlt $@when@~e~e$
\> $|$ \> $@sample@~e~e$
\GrammarAlt $@mapS@~e~e$
\> $|$ \> $@zipS@~e~e$
\GrammarAlt $@mapF@~e~e$
\> $|$ \> $@zipF@~e~e$
\\
\GrammarAlt $@let@~\mi{Let}~@in@~e$
\\
\GrammarAlt $\langle e,~ e \rangle ~|~ \langle\rangle$
\\
\\

$\mi{Let}$
\GrammarDef $x~=~e$
\GrammarAlt $@folds@~\{~x_i~=~e_i,~e_i~\}_i$
\\
\\

$\mi{v}$
\GrammarDef $\NN ~|~ \BB ~|~ [v]$
\GrammarAlt $(v,~v) ~|~ \bot$
\\
\\


$\mi{Program}$
\GrammarDef @inputs@~$\{ x_i~:~@Stream@~c_i~\tau_i \}_i~@in@~\mi{e}$ \\
\end{tabbing}

\caption{Grammar for Icicle Source}
\label{fig:source:grammar}
\end{figure}


The grammar for Icicle is given in figure~\ref{fig:source:grammar}.
The first four rules are rather standard lambda calculus.

Next are the primitives.
@when@ takes a stream and a fold, and returns a new fold whose value is the stream \emph{when}ever the stream is defined, and the fold otherwise.
This is used for defining folds that depend on stream values.
@sample@ takes a stream of any type and a fold, creates a new stream using the values of the fold.
@mapS@ and @zipS@ perform map and zip operations on streams of the same rate.
@mapF@ and @zipF@ perform map and zip operations on folds.

Let expressions such as $@let@~x~=e~@in@~e$ are as usual. 
Folds are defined using the syntax @let folds@ syntax. 
Multiple folds can be defined together, and each fold is defined by the initial value, and the ``kons'' part, defining the next value.

For example, a simple fold comprised of nothing but $0$
\begin{tabbing}
MM \= \kill
$@let folds@~\mi{zeros}~=~0,~\mi{zeros}$ \\
$@in@~\mi{zeros}$
\\
\\
$\mi{zeros}$ \> $=~\langle 0~0~0~0~\cdots~\rangle$ \\
\end{tabbing}

When computing the sum over another fold, the newly-defined fold and the input fold are zipped and then added together.
\begin{tabbing}
$\mi{sum}~=~\lambda{}(\mi{values}~:~@Fold@~\NN).$ \\
$@let folds@~\mi{s}~=~0,~@mapF@~(+)~(@zipF@~\mi{s}~\mi{values})$ \\
$@in@~\mi{s}$
\\
\\
$\mi{sum}~\langle 0~1~2~3~\cdots~\rangle$ \\
$~=~\langle 0~1~3~6~\cdots~\rangle$ \\
\end{tabbing}

This $\mi{sum}$ function requires a fold argument, but it can be applied to a stream $s$ by creating a fold that is the stream when it is defined, and $0$ otherwise: $@when@~s~zeros$.
A more general way is to define a new $\mi{sum'}$ function that works directly over streams:
\begin{tabbing}
MMMMMM \= MM \= MM \kill
$\mi{sum'}~=~\lambda{}(\mi{values}~:~@Stream@~c~\NN).$ \\
$@let folds@~\mi{s}~=~0,$ \\
\> $@when@$ \> $(@mapS@~(+)~(@zipS@~(@sample@~\mi{values}~\mi{s})~\mi{values}))$ \\
\> \> $~\mi{s}$ \\
$@in@~\mi{s}$
\\
\\
$\mi{sum'}~\langle \bot~1~\bot~3~\cdots~\rangle$ \\
$~=~\langle 0~1~1~4~\cdots~\rangle$ \\
\end{tabbing}

Here, the fold $s$ is sampled to a stream with the same clock as the input stream.
Whenever the input stream is defined, the previous value of $s$ and the current value of the $\mi{values}$ will be added together, producing the new sum.
When the input stream is not defined, the previous value of $s$ is used unchanged.

When referring to the newly defined folds in the right-hand-side of the fold (the kons), all values refer to the immediate previous value.
The order of the fold bindings has no effect on the program.
\begin{tabbing}
MM \= \kill
$@let folds@$ \\
\> $\mi{one}~=~1,~\mi{zero}$ \\
\> $\mi{zero}~=~0,~\mi{one}$ \\
$@in@~\mi{one}$
\\
\\
$\mi{one}$  \> $=~\langle 1~0~1~0~\cdots~\rangle$ \\
$\mi{zero}$ \> $=~\langle 0~1~0~1~\cdots~\rangle$ \\
\end{tabbing}

The grammar also has rules for stream literals: $\langle e_1,~e_2 \rangle$ and $\langle\rangle$. The stream can contain bottoms, $\bot$, which means that the stream is undefined at that point.
Stream literals have concrete clock types, for example $\langle T,~F \rangle$ for $\langle 1,~\bot \rangle$, while input streams have existential clocks: their actual clock is unknown until runtime.
These stream literals are not expected to occur in an actual program until evaluation.


Yes, I realise the grammar doesn't have @+@ or @/@, and that it doesn't have polymorphism.
For the examples, assume that a specialisation pass occurs, and that there is a ``sensible set of primitives on natural numbers and lists''.

Define group here.
\begin{code}
data Unpack r
 = forall a.
    Unpack a (Fold (a -> a)) (a -> r)

sum' vs = Unpack 0 (+vs) id

group (key : Stream c k)
      (val : Unpack v)
           : [(k,v)]
group key (Unpack z k r)
 = trivial
\end{code}



%!TEX root = ../Main.tex

\begin{figure*}

$$
\boxed{\SourceStepZ{e}{\mi{acc}}{\mi{res}}}
$$

$$
\ruleAx
{
    \SourceStepZ{x}{()}{x}
}{ZVar}
\ruleAx
{
    \SourceStepZ{v}{()}{v}
}{ZValue}
\ruleIN
{
    \SourceStepZ{e}{a}{r}
}
{
    \SourceStepZ{\lambda{}x~:~\tau.~e}{a}{\lambda{}x~:~\tau.~r}
}{ZLam}
$$

$$
\ruleAx
{
    \SourceStepZ{\langle e_1,~e_2 \rangle}{\langle e_1,~e_2 \rangle}{\bot}
}{ZStream1}
\ruleAx
{
    \SourceStepZ{\langle \rangle}{\langle \rangle}{\bot}
}{ZStream2}
\ruleIN
{
    \SourceStepZ{e_1}{a_1}{r_1}
    \quad
    \SourceStepZ{e_2}{a_2}{r_2}
}
{
    \SourceStepZ{@when@~e_1~e_2}{a_1,a_2}{r_2}
}{ZWhen}
\ruleIN
{
    \SourceStepZ{e_1}{a_1}{r_1}
    \quad
    \SourceStepZ{e_2}{a_2}{r_2}
}
{
    \SourceStepZ{@sample@~e_1~e_2}{a_1,a_2}{\bot}
}{ZSample}
$$

$$
\ruleIN
{
    \SourceStepZ{e}{a}{r}
}
{
    \SourceStepZ{@mapS@~f~e}{a}{\bot}
}{ZMapS}
\ruleIN
{
    \SourceStepZ{e_1}{a_1}{r_1}
    \quad
    \SourceStepZ{e_2}{a_2}{r_2}
}
{
    \SourceStepZ{@zipS@~e_1~e_2}{a_1,a_2}{\bot}
}{ZZipS}
$$

$$
\ruleIN
{
    \SourceStepZ{e}{a}{r}
}
{
    \SourceStepZ{@mapF@~f~e}{a}{f~r}
}{ZMapF}
\ruleIN
{
    \SourceStepZ{e_1}{a_1}{r_1}
    \quad
    \SourceStepZ{e_2}{a_2}{r_2}
}
{
    \SourceStepZ{@zipF@~e_1~e_2}{a_1,a_2}{r_1,r_2}
}{ZZipF}
$$

$$
\ruleIN
{
    \SourceStepZ{e_1}{a_1}{r_1}
    \quad
    \SourceStepZ{e_2}{a_2}{r_2}
}
{
    \SourceStepZ{@let@~x~=~e_1~@in@~e_2}{a_1,a_2}{r_2[x=r_1]}
}{ZLet}
\ruleIN
{
    \{ \SourceStepZ{e_i}{a_i}{r_i} \}_i
    \quad
    \SourceStepZ{e}{a}{r}
}
{
    \SourceStepZ{@let folds@~\{x_i~=~e_i,~\_\}_i~@in@~e}{\{a_i\}_i,a}{r[\{x_i=r_i\}_i]}
}{ZFolds}
$$

$$
\boxed{\SourceStepK{\mi{acc}}{e}{\mi{acc}}{\mi{res}}}
$$


$$
\ruleAx
{
    \SourceStepK{()}{x}{()}{x}
}{KVar}
\ruleAx
{
    \SourceStepK{()}{v}{()}{v}
}{KValue}
\ruleIN
{
    \SourceStepK{a}{e}{a'}{r'}
}
{
    \SourceStepK{a}{\lambda{}x~:~\tau.~e}{a'}{\lambda{}x~:~\tau.~r'}
}{KLam}
$$

$$
\ruleAx
{
    \SourceStepK{\langle e_1,~e_2 \rangle}{\langle \cdots \rangle}{e_2}{e_1}
}{KStream1}
\ruleAx
{
    \SourceStepK{\langle \rangle}{\langle \cdots \rangle}{\langle\rangle}{\bot}
}{KStream2}
\ruleIN
{
    \SourceStepK{a_1}{e_1}{a_1'}{r_1'}
    \quad
    \SourceStepK{a_1}{e_2}{a_2'}{r_2'}
}
{
    \SourceStepK{a_1,a_2}{@when@~e_1~e_2}{a_1',a_2'}{\Ifnbot{r_1'}{r_1'}{r_2'}}
}{KWhen}
$$

$$
\ruleIN
{
    \SourceStepK{a_1}{e_1}{a_1'}{r_1'}
    \quad
    \SourceStepK{a_2}{e_2}{a_2'}{r_2'}
}
{
    \SourceStepK{a_1,a_2}{@sample@~e_1~e_2}{a_1',a_2'}{\Ifnbot{r_1'}{r_2'}{\bot}}
}{KSample}
$$

$$
\ruleIN
{
    \SourceStepK{a}{e}{a'}{r'}
}
{
    \SourceStepK{a}{@mapS@~f~e}{a'}{\Ifnbot{r'}{f~r'}{\bot}}
}{KMapS}
\ruleIN
{
    \SourceStepK{a_1}{e_1}{a_1'}{r_1'}
    \quad
    \SourceStepK{a_2}{e_2}{a_2'}{r_2'}
}
{
    \SourceStepK{a_1,a_2}{@zipS@~e_1~e_2}{a_1',a_2'}{r_1',r_2'}
}{KZipS}
$$

$$
\ruleIN
{
    \SourceStepK{a}{e}{a'}{r'}
}
{
    \SourceStepK{a}{@mapF@~f~e}{a'}{f~r}
}{KMapF}
\ruleIN
{
    \SourceStepK{a_1}{e_1}{a_1'}{r_1'}
    \quad
    \SourceStepK{a_2}{e_2}{a_2'}{r_2'}
}
{
    \SourceStepK{a_1,a_2}{@zipF@~e_1~e_2}{a_1',a_2'}{r_1',r_2'}
}{KZipF}
$$

$$
\ruleIN
{
    \SourceStepK{a_1}{e_1}{a_1'}{r_1'}
    \quad
    \SourceStepK{a_2}{e_2}{a_2'}{r_2'}
}
{
    \SourceStepK{a_1,a_2}{@let@~x~=~e_1~@in@~e_2}{a_1',a_2'}{r_2[x=r_1]}
}{KLet}
\ruleIN
{
    \{ \SourceStepK{a_i}{e_i}{a_i'}{r_i'} \}_i
    \quad
    \SourceStepK{a}{e}{a'}{r'}
}
{
    \SourceStepK{\{a_i\}_i,a}{@let folds@~\{x_i~=~\_,~e_i\}_i~@in@~e}{\{a_i'\}_i,a'}{r'[\{x_i=r_i'\}_i]}
}{KFolds}
$$



$$
\boxed{\SourceStepX{e}{e}}
$$

$$
\ruleIN
{
    \SourceStepX{e_1}{e_1'}
}
{
    \SourceStepX{e_1~e_2}{e_1'~e_2}
}{XApp1}
\ruleIN
{
    \SourceStepX{e_2}{e_2'}
}
{
    \SourceStepX{v_1~e_2}{v_1~e_2'}
}{XApp2}
\ruleAx
{
    \SourceStepX{(\lambda{}x.~e_1)~e_2}{e_1[x:=e_2]}
}{XApp3}
$$
$$
\ruleIN
{
    \SourceStepX{e_1}{e_1'}
}
{
    \SourceStepX{@let@~x~=~e_1~@in@~e_2}{@let@~x~=~e_1'~@in@~e_2}
}{XLet1}
\ruleAx
{
    \SourceStepX{@let@~x~=~v_1~@in@~e_2}{e_2[x:=v_1]}
}{XLet2}
$$

\caption{Evaluation rules}
\label{fig:source:eval}
\end{figure*}




\subsection{Obvious extensions}
Obvious extensions and things I removed for simplification.

\subsubsection{Fixpoint}
$@fix@~(x~:~\tau)~@in@~e$.
By requiring $\tau$ to have kind @Data@ rather than @Flow@, we can allow recursion at runtime, while still ensuring a static dataflow graph.
I am tempted to kill the kinds for the simple version because with no fixpoint they do not serve much purpose (the dataflow graph is static regardless).

\subsubsection{ Maps and zips }
@mapS/mapF@, @zipS/zipF@.
For the simple version these could be merged into a single map/zip operating over just @Fold@s.
You can convert from a @Stream@ to a @Fold@ and back easily enough - you need a zero element for the @Fold@ but whatever.

\subsubsection{ Subrates }
$@let-rate@~(c'~:_k~@Clock@)~(w~:~c'~\le~c)~=~(e~:~@Stream@~c~\BB)$ and $@subrate@~(w~:~c'~\le~c)~(e~:~@Stream@~c~\tau)~:~@Stream@~c'~\tau$.
You can fake this by using Folds anyway.

\subsubsection{ Unpack }
$@let-unpack@~(x_\tau~:_k~@Data@)~(x_z~:~x_\tau)~(x_k~:~x_\tau~\to~@Fold@~x_\tau)~(x_r~:~x_\tau~\to~\tau)~=~(e~:~@Fold@~\tau)$.
Pulling apart folds to make new folds.
Useful for groups, segmented operations, and so on where the fold can be restarted or actually be many folds, as well as for other higher order folds.


%!TEX root = ../Main.tex
\section{Types}
\label{s:Types}

%!TEX root = ../Main.tex

\begin{figure*}

\begin{tabbing}
MM \= MM \= \kill
$\mi{Kind}$
\GrammarDef $@Data@~|~@Clock@~|~@Flow@$ \\

T
\GrammarDef $x~|~\NN~|~\BB~|~@()@
    ~|~ @List@~T
    ~|~ T~\to~T
    ~|~ (T,T)$
\GrammarAlt $@Stream@~T~T
    ~|~ @Fold@~T$
    \\
\end{tabbing}

\caption{Types and kinds}
\label{fig:source:type:types}
\end{figure*}


\begin{figure*}

$$
\boxed{\TypeWf{\Delta}{\tau}{\mi{Kind}}}
$$

$$
\ruleIN
{
    x~:_k~k~\in~\Delta
}
{
    \TypeWf{\Delta}{x}{k}
}{TVar}
\ruleAx
{
    \TypeWf{\Delta}{\NN}{@Data@}
}{TNat}
\ruleAx
{
    \TypeWf{\Delta}{\BB}{@Data@}
}{TBool}
\ruleAx
{
    \TypeWf{\Delta}{@()@}{@Data@}
}{TUnit}
$$

$$
\ruleIN
{
    \TypeWf{\Delta}{\tau}{@Data@}
}
{
    \TypeWf{\Delta}{@List@~\tau}{@Data@}
}{TList}
\ruleIN
{
    \TypeWf{\Delta}{\tau_1}{k_1}
    \quad
    \TypeWf{\Delta}{\tau_2}{k_2}
}
{
    \TypeWf{\Delta}{\tau_1~\to~\tau_2}{k_2}
}{TFun}
$$

$$
\ruleIN
{
    \TypeWf{\Delta}{c}{@Clock@}
    \quad
    \TypeWf{\Delta}{\tau}{@Data@}
}
{
    \TypeWf{\Delta}{@Stream@~c~\tau}{@Flow@}
}{TStream}
\ruleIN
{
    \TypeWf{\Delta}{\tau}{@Data@}
}
{
    \TypeWf{\Delta}{@Fold@~\tau}{@Flow@}
}{TFold}
$$



\caption{Kinds of types}
\label{fig:source:type:kinds}
\end{figure*}


\begin{figure*}

$$
\boxed{\Typecheck{\Delta}{\Gamma}{e}{T}}
$$


$$
\ruleIN
{
    (x~:~\tau)~\in~\Gamma
}
{ 
    \Typecheck{\Delta}{\Gamma}{x}{\tau}
}{TcVar}
\ruleIN
{
    v~:~\tau
}
{ 
    \Typecheck{\Delta}{\Gamma}{v}{\tau}
}{TcValue}
\ruleIN
{
    \Typecheck{\Delta}{\Gamma}{e_1}{\tau_1~\to~\tau_2}
    \quad
    \Typecheck{\Delta}{\Gamma}{e_2}{\tau_1}
}
{ 
    \Typecheck{\Delta}{\Gamma}{e_1~e_2}{\tau_2}
}{TcApp}
$$

$$
\ruleIN
{
    \Typecheck{\Delta}{\Gamma,~x~:~\tau}{e}{\tau'}
}
{
    \Typecheck{\Delta}{\Gamma}{\lambda{}x~:~\tau.~e}{\tau~\to~\tau'}
}{TcLam}
$$

$$
\ruleAx
{
    \Typecheck{\Delta}{\Gamma}{\langle \rangle}{@Stream@~\langle\rangle~\tau}
}{TcStrm1}
\ruleIN
{
    \Typecheck{\Delta}{\Gamma}{e_1}{\tau}
    \quad
    \Typecheck{\Delta}{\Gamma}{e_2}{@Stream@~c~\tau}
}
{
    \Typecheck{\Delta}{\Gamma}{\langle e_1,~e_2 \rangle}{@Stream@~\langle @T@,~c \rangle~\tau}
}{TcStrm2}
\ruleIN
{
    \Typecheck{\Delta}{\Gamma}{e_2}{@Stream@~c~\tau}
}
{
    \Typecheck{\Delta}{\Gamma}{\langle \bot,~e_2 \rangle}{@Stream@~\langle @F@,~c \rangle~\tau}
}{TcStrm3}
$$

$$
\ruleIN
{
    \Typecheck{\Delta}{\Gamma}{e_1}{@Stream@~c~\tau}
    \quad
    \Typecheck{\Delta}{\Gamma}{e_2}{@Fold@~\tau}
}
{
    \Typecheck{\Delta}{\Gamma}{@when@~e_1~e_2}{@Fold@~\tau}
}{TcWhen}
\ruleIN
{
    \Typecheck{\Delta}{\Gamma}{e_1}{@Stream@~c~\tau'}
    \quad
    \Typecheck{\Delta}{\Gamma}{e_2}{@Fold@~\tau}
}
{
    \Typecheck{\Delta}{\Gamma}{@sample@~e_1~e_2}{@Stream@~c~\tau}
}{TcSample}
$$

$$
\ruleIN
{
    \Typecheck{\Delta}{\Gamma}{e_1}{\tau_1~\to~\tau_2}
    \quad
    \Typecheck{\Delta}{\Gamma}{e_2}{@Stream@~c~\tau_1}
}
{
    \Typecheck{\Delta}{\Gamma}{@mapS@~e_1~e_2}{@Stream@~c~\tau_2}
}{TcMapS}
\ruleIN
{
    \Typecheck{\Delta}{\Gamma}{e_1}{@Stream@~c~\tau_1}
    \quad
    \Typecheck{\Delta}{\Gamma}{e_2}{@Stream@~c~\tau_2}
}
{
    \Typecheck{\Delta}{\Gamma}{@zipS@~e_1~e_2}{@Stream@~c~(\tau_1,\tau_2)}
}{TcZipS}
$$

$$
\ruleIN
{
    \Typecheck{\Delta}{\Gamma}{e_1}{\tau_1~\to~\tau_2}
    \quad
    \Typecheck{\Delta}{\Gamma}{e_2}{@Fold@~\tau_1}
}
{
    \Typecheck{\Delta}{\Gamma}{@mapF@~e_1~e_2}{@Fold@~\tau_2}
}{TcMapF}
\ruleIN
{
    \Typecheck{\Delta}{\Gamma}{e_1}{@Fold@~\tau_1}
    \quad
    \Typecheck{\Delta}{\Gamma}{e_2}{@Fold@~\tau_2}
}
{
    \Typecheck{\Delta}{\Gamma}{@zipF@~e_1~e_2}{@Fold@~(\tau_1,\tau_2)}
}{TcZipF}
$$

$$
\ruleIN
{
    \Typecheck{\Delta}{\Gamma}{e_1}{\tau_1}
    \quad
    \Typecheck{\Delta}{\Gamma,~x~:~\tau_1}{e_2}{\tau_2}
}
{
    \Typecheck{\Delta}{\Gamma}{@let@~x~=~e_1~@in@~e_2}{\tau_2}
}{TcLet}
$$

$$
\ruleIN
{
    \Gamma'~=~\Gamma,~\{~x_i~:~@Fold@~\tau_i~\}_i
    \quad
    \{
    \Typecheck{\Delta}{\Gamma}{e_{z_i}}{\tau_i}
    \}_i
    \quad
    \{
    \Typecheck{\Delta}{\Gamma'}{e_{k_i}}{@Fold@~\tau_i}
    \}_i
    \quad
    \Typecheck{\Delta}{\Gamma'}{e}{\tau}
}
{
    \Typecheck{\Delta}{\Gamma}
        {@let@~@folds@~\{~x_i~=~e_{z_i}~@then@~e_{k_i}~\}_i~@in@~e}
        {\tau}
}{TcLetFolds}
$$


\caption{Types of expressions}
\label{fig:source:type:exp}
\end{figure*}




%!TEX root = ../Main.tex
\section{Core language}
\label{s:Core}

This section presents the Core language, which we use as an intermediate language before converting to imperative code.


\subsection{Grammar}

%!TEX root = ../Main.tex

\begin{figure}

\begin{tabbing}
MMMM \= MM \= MM \= MM \= \kill
$\mi{e_c}$
\GrammarDef $x$ \> $|$ \> $v$
\GrammarAlt $e_c~e_c$ \> $|$ \> $p$
\\
\\
$c$
\GrammarDef $x$ \> $|$ \> $\mi{Always}$
\\
\\
$\mi{FoldNest}$
\GrammarDef $@nest@~\mi{Op}*$
\\
\\
$\mi{Op}$
\GrammarDef $@let@~x~=~e_c~@when@~c$
\GrammarAlt $@fold@~x~=~e_c,~(e_c~@when@~c~@else@~e_c)$
\\
\\
$\mi{Bind}$
\GrammarDef $x~=~\lambda{}x.~e_c$
\\
\\
$\mi{Program}$
\GrammarDef @inputs@ \\
    \>      \> \> $(x~:~@Stream@~c~\tau)*$ \\
    \>      \> @binds@ \\
    \>      \> \> $\mi{Bind}*$ \\
    \>      \> @with@ \\
    \>      \> \> $\mi{FoldNest}*$ \\
    \>      \> @in@ \\
    \>      \> \> $x*$ \\
\end{tabbing}


\caption{Grammar for Icicle Core}
\label{fig:core:grammar}
\end{figure}



%!TEX root = ../Main.tex

\begin{figure*}

$$
\boxed{\CoreComp{e}{[Bind]}{[Op]}{x}}
$$

$$
\ruleAx
{
    \CoreComp{v}{\emptyset}{@let@~x'~=~v}{x'}
}{CVal}
\ruleAx
{
    \CoreComp{x}{\emptyset}{\emptyset}{x}
}{CVar}
\ruleIN
{
    \CoreComp{e_1}{b_1}{o_1}{x_1}
    \quad
    \CoreComp{e_2}{b_2}{o_2}{x_2}
}{
    \CoreComp{e_1~e_2}{b_1;b_2}{o_1;o_2;~@let@~x'~=~x_1~x_2}{x'}
}{CApp}
$$


\caption{Conversion to Core}
\label{fig:core:compile}
\end{figure*}





%!TEX root = ../Main.tex
\section{Code generation}
\label{s:Generation}

Code generation should be pretty easy I guess cite flow fusion paper.

Because we don't have any zips or other that takes multiple streams, we don't have to deal with hard problems.

While not all streams have the same rate, there is no way to express an operation that takes two different rates.


%!TEX root = ../Main.tex
\section{Conclusion}
\label{s:Conclusion}





%!TEX root = ../Main.tex
\section{Related work}
\label{s:Related}
\subsection{Data flow languages}

The closest related work are synchronous data flow languages such as {\sc Lustre}, Icicle programs are restricted to those that can be computed in bounded memory.
{\sc Lustre}\cite{halbwachs1991synchronous} achieves this in three main ways:

\begin{enumerate}
\item They allow only restricted set of primitives such as @when@ for filtering, @pre@ for a single element buffer, and so on.
\item Cycles in the graph must contain at least one @pre@, to break the dependency loop.
\item Operators such as addition can only be applied to input streams with the same clock or rate; an expression like @(X when C)@ @+@ @(Y when (not C))@ would require an unbounded buffer\CITE{Lustre clock stuff}.
\end{enumerate}

In summary, while our language is quite similar to existing synchronous data flow languages, the extra restrictions we impose allow us to use a simpler type system.


% %!TEX root = ../Main.tex

\newcommand\JudgeK[2]
{       #1 :: #2
}

\newcommand\JudgeT[3]
{       #1 \vdash #2 :: #3
}

\newcommand\JudgeTS[5]
{       #1 \vdash #2 :: #3 ~;~ #4 ~;~ #5
}

\newcommand\kbox        {\textrm{\textbf{box}}}
\newcommand\krun        {\textrm{\textbf{run}}}
\newcommand\kthen       {\textrm{\textbf{then}}}
\newcommand\kpure       {\textrm{\textbf{pure}}}

\newcommand\rData       {\textrm{Data}}
\newcommand\rClock      {\textrm{Clock}}
\newcommand\rDefn       {\textrm{Defn}}
\newcommand\rComp       {\textrm{Comp}}

\newcommand\rStream     {\textrm{Stream}}
\newcommand\rArray      {\textrm{Array}}
\newcommand\rNat        {\textrm{Nat}}
\newcommand\rBool       {\textrm{Bool}}

\newcommand\rdrain      {\textrm{drain}}
\newcommand\rstream     {\textrm{stream}}
\newcommand\rsum        {\textrm{sum}}
\newcommand\rsmap       {\textrm{smap}}
\newcommand\rsfold      {\textrm{sfold}}
\newcommand\rsscan      {\textrm{sscan}}
\newcommand\rsfilter    {\textrm{sfilter}}


\begin{figure*}
$$
\boxed{\JudgeT{\Gamma}{e}{\tau}}
$$

$$
\ruleI
{       x : \tau \in \Gamma
}
{       \JudgeT{\Gamma}{x}{\tau}
}
\textrm{(TyVar)}
\quad\quad
\ruleI
{       \JudgeT{\Gamma}{e_1}{\tau_1 \to \tau_2}
        \quad
        \JudgeT{\Gamma}{e_2}{\tau_1}
}
{       \JudgeT{\Gamma}{e_1~e_2}{\tau_2}       
}
\textrm{(TyApp)}
\quad\quad
\ruleI
{       \JudgeT{\Gamma,x:\tau_1}{e_2}{\tau_2}
}
{       \JudgeT{\Gamma}{\lambda x : \tau_1}{\tau_1 \to \tau_2}
}
\textrm{(TyAbs)}
$$



% -- Effectful --------------------------------------------
$$
\boxed{\JudgeTS{\Gamma}{e}{\tau}{n}{k}}
$$

$$
\ruleI
{       \JudgeTS{\Gamma}{e}{\tau}{n}{k}
}
{       \JudgeT{\Gamma}{\kbox~ e}{\Box~ n~ k~ \tau}
}
\textrm{(TyBox)}
\quad\quad
\ruleI
{       \JudgeT{\Gamma}{e}{\tau}
}
{       \JudgeTS{\Gamma}{\kpure~e}{\tau}{n}{k}
}
\textrm{(TyPure)}
$$

$$
\ruleI
{       \{ \JudgeT
                {\Gamma}{e_i}{\Box~ n_i~ k~ \tau_i} \}^i 
        \quad
        \JudgeTS
                {\Gamma,~ \{x_i : \tau_i \}^i}
                {e'}{\tau'}{n'}{k'}
}
{       \JudgeTS
                {\Gamma}
                {\krun~ \{ x_i : \tau_i = e_i \}^i ~\kthen~ e'}
                {\tau'}
                {max~ \{ n_i \}^i + n'}
                {k'}
}
\textrm{(TyRun)}
$$

% -- Prims ------------------------------------------------
\vspace{2em}
$$
\begin{array}{ll}
\rNat,\rBool      & :: \rData
\\[1ex]
%
\rArray           & :: \rClock \to \rData \to \rData
\\[1ex]
%
\rStream          & :: \rClock \to \rData \to \rDefn
\\[1ex]
%
\Box              & :: \rNat   \to \rClock \to \rData \to \rComp
\\[1ex]
%
\rdrain_{k,\tau}  & :: \rStream~k~\tau \to \Box^1_k~ (\rArray~ k~ \tau)
\\[1ex]
%
\rstream_{k,\tau} & :: \rArray~k~\tau  \to \rStream~k~\tau
\\[1ex]
%
\rsum_k           & :: \rStream~k~\rNat \to \Box^1_k~ \rNat
\\[1ex]
%
\rsmap_{k,a,b}    & :: (a \to b) \to \rStream~k~a \to \rStream~k~b
\\[1ex]
%
\rsfold_{k,a,b}   & :: (a \to b \to b) \to b \to \rStream~k~a \to \Box^1_k~ b
\\[1ex]
%
\rsscan_{k,a,b}   & :: (a \to b \to b) \to b \to \rStream~k~a \to \rStream~k~b
\end{array}
$$

\caption{Typing}
\label{fig:source:type:modal}
\end{figure*}


\section*{Acknowledgements}

\bibliographystyle{plain}
\bibliography{Main}

\end{document}




% -- bibliography
\bibliography{bib/a,bib/b,bib/c,bib/d,bib/e,bib/f,bib/g,bib/h,bib/i,bib/j,bib/k,bib/l,bib/m,bib/n,bib/o,bib/p,bib/q,bib/r,bib/s,bib/t,bib/u,bib/v,bib/w,bib/x,bib/y,bib/z}{}
\bibliographystyle{amsalpha}

\documentclass[preprint]{sigplanconf}
\usepackage{amssymb}
\usepackage{amsthm}
\usepackage{graphicx}
\usepackage{amsmath}
\usepackage{mathptmx}
\usepackage{mathtools}
\usepackage{stmaryrd}
\usepackage{hyperref}
\usepackage{alltt}
\usepackage{url}
\usepackage{float}
\usepackage{style/code}
\usepackage{style/proof}
\usepackage{style/utils}
\usepackage{style/judgements}

% -----------------------------------------------------------------------------
\begin{document}

% \exclusivelicense
% \conferenceinfo{}{}
% \copyrightyear{2015}
% \copyrightdata{}
\doi{}
% \pagenumbering{gobble} 

\title{Icicle: fuse your queries}

\authorinfo{
  Amos Robinson$^\dagger$$^\ddagger$
  \and Ben Lippmeier$^\dagger$
}{
  \vspace{5pt}
  \shortstack{
    $^\dagger$Computer Science and Engineering \\
    University of New South Wales, Australia \\[2pt]
    \textsf{amosr,benl@cse.unsw.edu.au}
  }
  \shortstack{
    $^\ddagger$Ambiata      \\
    Big data and shit       \\[2pt]
    \textsf{amos.robinson@ambiata.com}
  }
}

\maketitle
\makeatactive

\begin{abstract}
When streaming a large amount of data, simply iterating over the data may take hours.
If multiple queries are to be performed, it is important that work is not duplicated.
Queries that can be performed together must be performed in the same iteration.
We introduce a simple streaming language for computing queries in a single-pass over the data.
By using an appropriate intermediate language we guarantee fusion between queries on the same input streams, before extracting efficient C code.
\end{abstract}


\category
	{D.3.4}
	{Programming Languages}
	{Processors---Compilers; Optimization}

\terms
	Languages, Performance

\keywords
	Arrays; Fusion


\section{Introdution}

This is some stuff.

%!TEX root = ../Main.tex
\section{Icicle Source}
\label{s:Source}

The two main types in Icicle are @Stream@ and @Fold@.
Streams represent data values as they flow through the program.
Streams do not always have data flowing through them, but have an associated clock which describes when data occurs.
Two streams with the same clock both have data at the same time.
Folds, on the other hand, are results of computations over the stream data that has been seen.
Folds always have a well-defined value.
Folds can be easily converted to streams, by sampling their current value whenever the stream clock is true.
It is harder to convert a stream to a fold, as the stream has gaps where it is undefined, and one must specify how to fill the gaps (hold last, fill with zero, etc).
Finally, folds can be computed recursively based on their previous values.

%!TEX root = ../Main.tex

\begin{figure}

\begin{tabbing}
MMMM \= MM \= MMMMMMMMMMM \= MM \= \kill
$\mi{e}$
\GrammarDef $x$
\> $|$ \> $v$
\GrammarAlt $e~e$
\> $|$ \> $\lambda{}x~:~\tau.~e$

\\
\GrammarAlt $@when@~e~e$
\> $|$ \> $@sample@~e~e$
\GrammarAlt $@mapS@~e~e$
\> $|$ \> $@zipS@~e~e$
\GrammarAlt $@mapF@~e~e$
\> $|$ \> $@zipF@~e~e$
\\
\GrammarAlt $@let@~\mi{Let}~@in@~e$
\\
\GrammarAlt $\langle e,~ e \rangle ~|~ \langle\rangle$
\\
\\

$\mi{Let}$
\GrammarDef $x~=~e$
\GrammarAlt $@folds@~\{~x_i~=~e_i,~e_i~\}_i$
\\
\\

$\mi{v}$
\GrammarDef $\NN ~|~ \BB ~|~ [v]$
\GrammarAlt $(v,~v) ~|~ \bot$
\\
\\


$\mi{Program}$
\GrammarDef @inputs@~$\{ x_i~:~@Stream@~c_i~\tau_i \}_i~@in@~\mi{e}$ \\
\end{tabbing}

\caption{Grammar for Icicle Source}
\label{fig:source:grammar}
\end{figure}


The grammar for Icicle is given in figure~\ref{fig:source:grammar}.
The first four rules are rather standard lambda calculus.

Next are the primitives.
@when@ takes a stream and a fold, and returns a new fold whose value is the stream \emph{when}ever the stream is defined, and the fold otherwise.
This is used for defining folds that depend on stream values.
@sample@ takes a stream of any type and a fold, creates a new stream using the values of the fold.
@mapS@ and @zipS@ perform map and zip operations on streams of the same rate.
@mapF@ and @zipF@ perform map and zip operations on folds.

Let expressions such as $@let@~x~=e~@in@~e$ are as usual. 
Folds are defined using the syntax @let folds@ syntax. 
Multiple folds can be defined together, and each fold is defined by the initial value, and the ``kons'' part, defining the next value.

For example, a simple fold comprised of nothing but $0$
\begin{tabbing}
MM \= \kill
$@let folds@~\mi{zeros}~=~0,~\mi{zeros}$ \\
$@in@~\mi{zeros}$
\\
\\
$\mi{zeros}$ \> $=~\langle 0~0~0~0~\cdots~\rangle$ \\
\end{tabbing}

When computing the sum over another fold, the newly-defined fold and the input fold are zipped and then added together.
\begin{tabbing}
$\mi{sum}~=~\lambda{}(\mi{values}~:~@Fold@~\NN).$ \\
$@let folds@~\mi{s}~=~0,~@mapF@~(+)~(@zipF@~\mi{s}~\mi{values})$ \\
$@in@~\mi{s}$
\\
\\
$\mi{sum}~\langle 0~1~2~3~\cdots~\rangle$ \\
$~=~\langle 0~1~3~6~\cdots~\rangle$ \\
\end{tabbing}

This $\mi{sum}$ function requires a fold argument, but it can be applied to a stream $s$ by creating a fold that is the stream when it is defined, and $0$ otherwise: $@when@~s~zeros$.
A more general way is to define a new $\mi{sum'}$ function that works directly over streams:
\begin{tabbing}
MMMMMM \= MM \= MM \kill
$\mi{sum'}~=~\lambda{}(\mi{values}~:~@Stream@~c~\NN).$ \\
$@let folds@~\mi{s}~=~0,$ \\
\> $@when@$ \> $(@mapS@~(+)~(@zipS@~(@sample@~\mi{values}~\mi{s})~\mi{values}))$ \\
\> \> $~\mi{s}$ \\
$@in@~\mi{s}$
\\
\\
$\mi{sum'}~\langle \bot~1~\bot~3~\cdots~\rangle$ \\
$~=~\langle 0~1~1~4~\cdots~\rangle$ \\
\end{tabbing}

Here, the fold $s$ is sampled to a stream with the same clock as the input stream.
Whenever the input stream is defined, the previous value of $s$ and the current value of the $\mi{values}$ will be added together, producing the new sum.
When the input stream is not defined, the previous value of $s$ is used unchanged.

When referring to the newly defined folds in the right-hand-side of the fold (the kons), all values refer to the immediate previous value.
The order of the fold bindings has no effect on the program.
\begin{tabbing}
MM \= \kill
$@let folds@$ \\
\> $\mi{one}~=~1,~\mi{zero}$ \\
\> $\mi{zero}~=~0,~\mi{one}$ \\
$@in@~\mi{one}$
\\
\\
$\mi{one}$  \> $=~\langle 1~0~1~0~\cdots~\rangle$ \\
$\mi{zero}$ \> $=~\langle 0~1~0~1~\cdots~\rangle$ \\
\end{tabbing}

The grammar also has rules for stream literals: $\langle e_1,~e_2 \rangle$ and $\langle\rangle$. The stream can contain bottoms, $\bot$, which means that the stream is undefined at that point.
Stream literals have concrete clock types, for example $\langle T,~F \rangle$ for $\langle 1,~\bot \rangle$, while input streams have existential clocks: their actual clock is unknown until runtime.
These stream literals are not expected to occur in an actual program until evaluation.


Yes, I realise the grammar doesn't have @+@ or @/@, and that it doesn't have polymorphism.
For the examples, assume that a specialisation pass occurs, and that there is a ``sensible set of primitives on natural numbers and lists''.

Define group here.
\begin{code}
data Unpack r
 = forall a.
    Unpack a (Fold (a -> a)) (a -> r)

sum' vs = Unpack 0 (+vs) id

group (key : Stream c k)
      (val : Unpack v)
           : [(k,v)]
group key (Unpack z k r)
 = trivial
\end{code}



%!TEX root = ../Main.tex

\begin{figure*}

$$
\boxed{\SourceStepZ{e}{\mi{acc}}{\mi{res}}}
$$

$$
\ruleAx
{
    \SourceStepZ{x}{()}{x}
}{ZVar}
\ruleAx
{
    \SourceStepZ{v}{()}{v}
}{ZValue}
\ruleIN
{
    \SourceStepZ{e}{a}{r}
}
{
    \SourceStepZ{\lambda{}x~:~\tau.~e}{a}{\lambda{}x~:~\tau.~r}
}{ZLam}
$$

$$
\ruleAx
{
    \SourceStepZ{\langle e_1,~e_2 \rangle}{\langle e_1,~e_2 \rangle}{\bot}
}{ZStream1}
\ruleAx
{
    \SourceStepZ{\langle \rangle}{\langle \rangle}{\bot}
}{ZStream2}
\ruleIN
{
    \SourceStepZ{e_1}{a_1}{r_1}
    \quad
    \SourceStepZ{e_2}{a_2}{r_2}
}
{
    \SourceStepZ{@when@~e_1~e_2}{a_1,a_2}{r_2}
}{ZWhen}
\ruleIN
{
    \SourceStepZ{e_1}{a_1}{r_1}
    \quad
    \SourceStepZ{e_2}{a_2}{r_2}
}
{
    \SourceStepZ{@sample@~e_1~e_2}{a_1,a_2}{\bot}
}{ZSample}
$$

$$
\ruleIN
{
    \SourceStepZ{e}{a}{r}
}
{
    \SourceStepZ{@mapS@~f~e}{a}{\bot}
}{ZMapS}
\ruleIN
{
    \SourceStepZ{e_1}{a_1}{r_1}
    \quad
    \SourceStepZ{e_2}{a_2}{r_2}
}
{
    \SourceStepZ{@zipS@~e_1~e_2}{a_1,a_2}{\bot}
}{ZZipS}
$$

$$
\ruleIN
{
    \SourceStepZ{e}{a}{r}
}
{
    \SourceStepZ{@mapF@~f~e}{a}{f~r}
}{ZMapF}
\ruleIN
{
    \SourceStepZ{e_1}{a_1}{r_1}
    \quad
    \SourceStepZ{e_2}{a_2}{r_2}
}
{
    \SourceStepZ{@zipF@~e_1~e_2}{a_1,a_2}{r_1,r_2}
}{ZZipF}
$$

$$
\ruleIN
{
    \SourceStepZ{e_1}{a_1}{r_1}
    \quad
    \SourceStepZ{e_2}{a_2}{r_2}
}
{
    \SourceStepZ{@let@~x~=~e_1~@in@~e_2}{a_1,a_2}{r_2[x=r_1]}
}{ZLet}
\ruleIN
{
    \{ \SourceStepZ{e_i}{a_i}{r_i} \}_i
    \quad
    \SourceStepZ{e}{a}{r}
}
{
    \SourceStepZ{@let folds@~\{x_i~=~e_i,~\_\}_i~@in@~e}{\{a_i\}_i,a}{r[\{x_i=r_i\}_i]}
}{ZFolds}
$$

$$
\boxed{\SourceStepK{\mi{acc}}{e}{\mi{acc}}{\mi{res}}}
$$


$$
\ruleAx
{
    \SourceStepK{()}{x}{()}{x}
}{KVar}
\ruleAx
{
    \SourceStepK{()}{v}{()}{v}
}{KValue}
\ruleIN
{
    \SourceStepK{a}{e}{a'}{r'}
}
{
    \SourceStepK{a}{\lambda{}x~:~\tau.~e}{a'}{\lambda{}x~:~\tau.~r'}
}{KLam}
$$

$$
\ruleAx
{
    \SourceStepK{\langle e_1,~e_2 \rangle}{\langle \cdots \rangle}{e_2}{e_1}
}{KStream1}
\ruleAx
{
    \SourceStepK{\langle \rangle}{\langle \cdots \rangle}{\langle\rangle}{\bot}
}{KStream2}
\ruleIN
{
    \SourceStepK{a_1}{e_1}{a_1'}{r_1'}
    \quad
    \SourceStepK{a_1}{e_2}{a_2'}{r_2'}
}
{
    \SourceStepK{a_1,a_2}{@when@~e_1~e_2}{a_1',a_2'}{\Ifnbot{r_1'}{r_1'}{r_2'}}
}{KWhen}
$$

$$
\ruleIN
{
    \SourceStepK{a_1}{e_1}{a_1'}{r_1'}
    \quad
    \SourceStepK{a_2}{e_2}{a_2'}{r_2'}
}
{
    \SourceStepK{a_1,a_2}{@sample@~e_1~e_2}{a_1',a_2'}{\Ifnbot{r_1'}{r_2'}{\bot}}
}{KSample}
$$

$$
\ruleIN
{
    \SourceStepK{a}{e}{a'}{r'}
}
{
    \SourceStepK{a}{@mapS@~f~e}{a'}{\Ifnbot{r'}{f~r'}{\bot}}
}{KMapS}
\ruleIN
{
    \SourceStepK{a_1}{e_1}{a_1'}{r_1'}
    \quad
    \SourceStepK{a_2}{e_2}{a_2'}{r_2'}
}
{
    \SourceStepK{a_1,a_2}{@zipS@~e_1~e_2}{a_1',a_2'}{r_1',r_2'}
}{KZipS}
$$

$$
\ruleIN
{
    \SourceStepK{a}{e}{a'}{r'}
}
{
    \SourceStepK{a}{@mapF@~f~e}{a'}{f~r}
}{KMapF}
\ruleIN
{
    \SourceStepK{a_1}{e_1}{a_1'}{r_1'}
    \quad
    \SourceStepK{a_2}{e_2}{a_2'}{r_2'}
}
{
    \SourceStepK{a_1,a_2}{@zipF@~e_1~e_2}{a_1',a_2'}{r_1',r_2'}
}{KZipF}
$$

$$
\ruleIN
{
    \SourceStepK{a_1}{e_1}{a_1'}{r_1'}
    \quad
    \SourceStepK{a_2}{e_2}{a_2'}{r_2'}
}
{
    \SourceStepK{a_1,a_2}{@let@~x~=~e_1~@in@~e_2}{a_1',a_2'}{r_2[x=r_1]}
}{KLet}
\ruleIN
{
    \{ \SourceStepK{a_i}{e_i}{a_i'}{r_i'} \}_i
    \quad
    \SourceStepK{a}{e}{a'}{r'}
}
{
    \SourceStepK{\{a_i\}_i,a}{@let folds@~\{x_i~=~\_,~e_i\}_i~@in@~e}{\{a_i'\}_i,a'}{r'[\{x_i=r_i'\}_i]}
}{KFolds}
$$



$$
\boxed{\SourceStepX{e}{e}}
$$

$$
\ruleIN
{
    \SourceStepX{e_1}{e_1'}
}
{
    \SourceStepX{e_1~e_2}{e_1'~e_2}
}{XApp1}
\ruleIN
{
    \SourceStepX{e_2}{e_2'}
}
{
    \SourceStepX{v_1~e_2}{v_1~e_2'}
}{XApp2}
\ruleAx
{
    \SourceStepX{(\lambda{}x.~e_1)~e_2}{e_1[x:=e_2]}
}{XApp3}
$$
$$
\ruleIN
{
    \SourceStepX{e_1}{e_1'}
}
{
    \SourceStepX{@let@~x~=~e_1~@in@~e_2}{@let@~x~=~e_1'~@in@~e_2}
}{XLet1}
\ruleAx
{
    \SourceStepX{@let@~x~=~v_1~@in@~e_2}{e_2[x:=v_1]}
}{XLet2}
$$

\caption{Evaluation rules}
\label{fig:source:eval}
\end{figure*}




\subsection{Obvious extensions}
Obvious extensions and things I removed for simplification.

\subsubsection{Fixpoint}
$@fix@~(x~:~\tau)~@in@~e$.
By requiring $\tau$ to have kind @Data@ rather than @Flow@, we can allow recursion at runtime, while still ensuring a static dataflow graph.
I am tempted to kill the kinds for the simple version because with no fixpoint they do not serve much purpose (the dataflow graph is static regardless).

\subsubsection{ Maps and zips }
@mapS/mapF@, @zipS/zipF@.
For the simple version these could be merged into a single map/zip operating over just @Fold@s.
You can convert from a @Stream@ to a @Fold@ and back easily enough - you need a zero element for the @Fold@ but whatever.

\subsubsection{ Subrates }
$@let-rate@~(c'~:_k~@Clock@)~(w~:~c'~\le~c)~=~(e~:~@Stream@~c~\BB)$ and $@subrate@~(w~:~c'~\le~c)~(e~:~@Stream@~c~\tau)~:~@Stream@~c'~\tau$.
You can fake this by using Folds anyway.

\subsubsection{ Unpack }
$@let-unpack@~(x_\tau~:_k~@Data@)~(x_z~:~x_\tau)~(x_k~:~x_\tau~\to~@Fold@~x_\tau)~(x_r~:~x_\tau~\to~\tau)~=~(e~:~@Fold@~\tau)$.
Pulling apart folds to make new folds.
Useful for groups, segmented operations, and so on where the fold can be restarted or actually be many folds, as well as for other higher order folds.


%!TEX root = ../Main.tex
\section{Types}
\label{s:Types}

%!TEX root = ../Main.tex

\begin{figure*}

\begin{tabbing}
MM \= MM \= \kill
$\mi{Kind}$
\GrammarDef $@Data@~|~@Clock@~|~@Flow@$ \\

T
\GrammarDef $x~|~\NN~|~\BB~|~@()@
    ~|~ @List@~T
    ~|~ T~\to~T
    ~|~ (T,T)$
\GrammarAlt $@Stream@~T~T
    ~|~ @Fold@~T$
    \\
\end{tabbing}

\caption{Types and kinds}
\label{fig:source:type:types}
\end{figure*}


\begin{figure*}

$$
\boxed{\TypeWf{\Delta}{\tau}{\mi{Kind}}}
$$

$$
\ruleIN
{
    x~:_k~k~\in~\Delta
}
{
    \TypeWf{\Delta}{x}{k}
}{TVar}
\ruleAx
{
    \TypeWf{\Delta}{\NN}{@Data@}
}{TNat}
\ruleAx
{
    \TypeWf{\Delta}{\BB}{@Data@}
}{TBool}
\ruleAx
{
    \TypeWf{\Delta}{@()@}{@Data@}
}{TUnit}
$$

$$
\ruleIN
{
    \TypeWf{\Delta}{\tau}{@Data@}
}
{
    \TypeWf{\Delta}{@List@~\tau}{@Data@}
}{TList}
\ruleIN
{
    \TypeWf{\Delta}{\tau_1}{k_1}
    \quad
    \TypeWf{\Delta}{\tau_2}{k_2}
}
{
    \TypeWf{\Delta}{\tau_1~\to~\tau_2}{k_2}
}{TFun}
$$

$$
\ruleIN
{
    \TypeWf{\Delta}{c}{@Clock@}
    \quad
    \TypeWf{\Delta}{\tau}{@Data@}
}
{
    \TypeWf{\Delta}{@Stream@~c~\tau}{@Flow@}
}{TStream}
\ruleIN
{
    \TypeWf{\Delta}{\tau}{@Data@}
}
{
    \TypeWf{\Delta}{@Fold@~\tau}{@Flow@}
}{TFold}
$$



\caption{Kinds of types}
\label{fig:source:type:kinds}
\end{figure*}


\begin{figure*}

$$
\boxed{\Typecheck{\Delta}{\Gamma}{e}{T}}
$$


$$
\ruleIN
{
    (x~:~\tau)~\in~\Gamma
}
{ 
    \Typecheck{\Delta}{\Gamma}{x}{\tau}
}{TcVar}
\ruleIN
{
    v~:~\tau
}
{ 
    \Typecheck{\Delta}{\Gamma}{v}{\tau}
}{TcValue}
\ruleIN
{
    \Typecheck{\Delta}{\Gamma}{e_1}{\tau_1~\to~\tau_2}
    \quad
    \Typecheck{\Delta}{\Gamma}{e_2}{\tau_1}
}
{ 
    \Typecheck{\Delta}{\Gamma}{e_1~e_2}{\tau_2}
}{TcApp}
$$

$$
\ruleIN
{
    \Typecheck{\Delta}{\Gamma,~x~:~\tau}{e}{\tau'}
}
{
    \Typecheck{\Delta}{\Gamma}{\lambda{}x~:~\tau.~e}{\tau~\to~\tau'}
}{TcLam}
$$

$$
\ruleAx
{
    \Typecheck{\Delta}{\Gamma}{\langle \rangle}{@Stream@~\langle\rangle~\tau}
}{TcStrm1}
\ruleIN
{
    \Typecheck{\Delta}{\Gamma}{e_1}{\tau}
    \quad
    \Typecheck{\Delta}{\Gamma}{e_2}{@Stream@~c~\tau}
}
{
    \Typecheck{\Delta}{\Gamma}{\langle e_1,~e_2 \rangle}{@Stream@~\langle @T@,~c \rangle~\tau}
}{TcStrm2}
\ruleIN
{
    \Typecheck{\Delta}{\Gamma}{e_2}{@Stream@~c~\tau}
}
{
    \Typecheck{\Delta}{\Gamma}{\langle \bot,~e_2 \rangle}{@Stream@~\langle @F@,~c \rangle~\tau}
}{TcStrm3}
$$

$$
\ruleIN
{
    \Typecheck{\Delta}{\Gamma}{e_1}{@Stream@~c~\tau}
    \quad
    \Typecheck{\Delta}{\Gamma}{e_2}{@Fold@~\tau}
}
{
    \Typecheck{\Delta}{\Gamma}{@when@~e_1~e_2}{@Fold@~\tau}
}{TcWhen}
\ruleIN
{
    \Typecheck{\Delta}{\Gamma}{e_1}{@Stream@~c~\tau'}
    \quad
    \Typecheck{\Delta}{\Gamma}{e_2}{@Fold@~\tau}
}
{
    \Typecheck{\Delta}{\Gamma}{@sample@~e_1~e_2}{@Stream@~c~\tau}
}{TcSample}
$$

$$
\ruleIN
{
    \Typecheck{\Delta}{\Gamma}{e_1}{\tau_1~\to~\tau_2}
    \quad
    \Typecheck{\Delta}{\Gamma}{e_2}{@Stream@~c~\tau_1}
}
{
    \Typecheck{\Delta}{\Gamma}{@mapS@~e_1~e_2}{@Stream@~c~\tau_2}
}{TcMapS}
\ruleIN
{
    \Typecheck{\Delta}{\Gamma}{e_1}{@Stream@~c~\tau_1}
    \quad
    \Typecheck{\Delta}{\Gamma}{e_2}{@Stream@~c~\tau_2}
}
{
    \Typecheck{\Delta}{\Gamma}{@zipS@~e_1~e_2}{@Stream@~c~(\tau_1,\tau_2)}
}{TcZipS}
$$

$$
\ruleIN
{
    \Typecheck{\Delta}{\Gamma}{e_1}{\tau_1~\to~\tau_2}
    \quad
    \Typecheck{\Delta}{\Gamma}{e_2}{@Fold@~\tau_1}
}
{
    \Typecheck{\Delta}{\Gamma}{@mapF@~e_1~e_2}{@Fold@~\tau_2}
}{TcMapF}
\ruleIN
{
    \Typecheck{\Delta}{\Gamma}{e_1}{@Fold@~\tau_1}
    \quad
    \Typecheck{\Delta}{\Gamma}{e_2}{@Fold@~\tau_2}
}
{
    \Typecheck{\Delta}{\Gamma}{@zipF@~e_1~e_2}{@Fold@~(\tau_1,\tau_2)}
}{TcZipF}
$$

$$
\ruleIN
{
    \Typecheck{\Delta}{\Gamma}{e_1}{\tau_1}
    \quad
    \Typecheck{\Delta}{\Gamma,~x~:~\tau_1}{e_2}{\tau_2}
}
{
    \Typecheck{\Delta}{\Gamma}{@let@~x~=~e_1~@in@~e_2}{\tau_2}
}{TcLet}
$$

$$
\ruleIN
{
    \Gamma'~=~\Gamma,~\{~x_i~:~@Fold@~\tau_i~\}_i
    \quad
    \{
    \Typecheck{\Delta}{\Gamma}{e_{z_i}}{\tau_i}
    \}_i
    \quad
    \{
    \Typecheck{\Delta}{\Gamma'}{e_{k_i}}{@Fold@~\tau_i}
    \}_i
    \quad
    \Typecheck{\Delta}{\Gamma'}{e}{\tau}
}
{
    \Typecheck{\Delta}{\Gamma}
        {@let@~@folds@~\{~x_i~=~e_{z_i}~@then@~e_{k_i}~\}_i~@in@~e}
        {\tau}
}{TcLetFolds}
$$


\caption{Types of expressions}
\label{fig:source:type:exp}
\end{figure*}




%!TEX root = ../Main.tex
\section{Core language}
\label{s:Core}

This section presents the Core language, which we use as an intermediate language before converting to imperative code.


\subsection{Grammar}

%!TEX root = ../Main.tex

\begin{figure}

\begin{tabbing}
MMMM \= MM \= MM \= MM \= \kill
$\mi{e_c}$
\GrammarDef $x$ \> $|$ \> $v$
\GrammarAlt $e_c~e_c$ \> $|$ \> $p$
\\
\\
$c$
\GrammarDef $x$ \> $|$ \> $\mi{Always}$
\\
\\
$\mi{FoldNest}$
\GrammarDef $@nest@~\mi{Op}*$
\\
\\
$\mi{Op}$
\GrammarDef $@let@~x~=~e_c~@when@~c$
\GrammarAlt $@fold@~x~=~e_c,~(e_c~@when@~c~@else@~e_c)$
\\
\\
$\mi{Bind}$
\GrammarDef $x~=~\lambda{}x.~e_c$
\\
\\
$\mi{Program}$
\GrammarDef @inputs@ \\
    \>      \> \> $(x~:~@Stream@~c~\tau)*$ \\
    \>      \> @binds@ \\
    \>      \> \> $\mi{Bind}*$ \\
    \>      \> @with@ \\
    \>      \> \> $\mi{FoldNest}*$ \\
    \>      \> @in@ \\
    \>      \> \> $x*$ \\
\end{tabbing}


\caption{Grammar for Icicle Core}
\label{fig:core:grammar}
\end{figure}



%!TEX root = ../Main.tex

\begin{figure*}

$$
\boxed{\CoreComp{e}{[Bind]}{[Op]}{x}}
$$

$$
\ruleAx
{
    \CoreComp{v}{\emptyset}{@let@~x'~=~v}{x'}
}{CVal}
\ruleAx
{
    \CoreComp{x}{\emptyset}{\emptyset}{x}
}{CVar}
\ruleIN
{
    \CoreComp{e_1}{b_1}{o_1}{x_1}
    \quad
    \CoreComp{e_2}{b_2}{o_2}{x_2}
}{
    \CoreComp{e_1~e_2}{b_1;b_2}{o_1;o_2;~@let@~x'~=~x_1~x_2}{x'}
}{CApp}
$$


\caption{Conversion to Core}
\label{fig:core:compile}
\end{figure*}





%!TEX root = ../Main.tex
\section{Code generation}
\label{s:Generation}

Code generation should be pretty easy I guess cite flow fusion paper.

Because we don't have any zips or other that takes multiple streams, we don't have to deal with hard problems.

While not all streams have the same rate, there is no way to express an operation that takes two different rates.


%!TEX root = ../Main.tex
\section{Conclusion}
\label{s:Conclusion}





%!TEX root = ../Main.tex
\section{Related work}
\label{s:Related}
\subsection{Data flow languages}

The closest related work are synchronous data flow languages such as {\sc Lustre}, Icicle programs are restricted to those that can be computed in bounded memory.
{\sc Lustre}\cite{halbwachs1991synchronous} achieves this in three main ways:

\begin{enumerate}
\item They allow only restricted set of primitives such as @when@ for filtering, @pre@ for a single element buffer, and so on.
\item Cycles in the graph must contain at least one @pre@, to break the dependency loop.
\item Operators such as addition can only be applied to input streams with the same clock or rate; an expression like @(X when C)@ @+@ @(Y when (not C))@ would require an unbounded buffer\CITE{Lustre clock stuff}.
\end{enumerate}

In summary, while our language is quite similar to existing synchronous data flow languages, the extra restrictions we impose allow us to use a simpler type system.


% %!TEX root = ../Main.tex

\newcommand\JudgeK[2]
{       #1 :: #2
}

\newcommand\JudgeT[3]
{       #1 \vdash #2 :: #3
}

\newcommand\JudgeTS[5]
{       #1 \vdash #2 :: #3 ~;~ #4 ~;~ #5
}

\newcommand\kbox        {\textrm{\textbf{box}}}
\newcommand\krun        {\textrm{\textbf{run}}}
\newcommand\kthen       {\textrm{\textbf{then}}}
\newcommand\kpure       {\textrm{\textbf{pure}}}

\newcommand\rData       {\textrm{Data}}
\newcommand\rClock      {\textrm{Clock}}
\newcommand\rDefn       {\textrm{Defn}}
\newcommand\rComp       {\textrm{Comp}}

\newcommand\rStream     {\textrm{Stream}}
\newcommand\rArray      {\textrm{Array}}
\newcommand\rNat        {\textrm{Nat}}
\newcommand\rBool       {\textrm{Bool}}

\newcommand\rdrain      {\textrm{drain}}
\newcommand\rstream     {\textrm{stream}}
\newcommand\rsum        {\textrm{sum}}
\newcommand\rsmap       {\textrm{smap}}
\newcommand\rsfold      {\textrm{sfold}}
\newcommand\rsscan      {\textrm{sscan}}
\newcommand\rsfilter    {\textrm{sfilter}}


\begin{figure*}
$$
\boxed{\JudgeT{\Gamma}{e}{\tau}}
$$

$$
\ruleI
{       x : \tau \in \Gamma
}
{       \JudgeT{\Gamma}{x}{\tau}
}
\textrm{(TyVar)}
\quad\quad
\ruleI
{       \JudgeT{\Gamma}{e_1}{\tau_1 \to \tau_2}
        \quad
        \JudgeT{\Gamma}{e_2}{\tau_1}
}
{       \JudgeT{\Gamma}{e_1~e_2}{\tau_2}       
}
\textrm{(TyApp)}
\quad\quad
\ruleI
{       \JudgeT{\Gamma,x:\tau_1}{e_2}{\tau_2}
}
{       \JudgeT{\Gamma}{\lambda x : \tau_1}{\tau_1 \to \tau_2}
}
\textrm{(TyAbs)}
$$



% -- Effectful --------------------------------------------
$$
\boxed{\JudgeTS{\Gamma}{e}{\tau}{n}{k}}
$$

$$
\ruleI
{       \JudgeTS{\Gamma}{e}{\tau}{n}{k}
}
{       \JudgeT{\Gamma}{\kbox~ e}{\Box~ n~ k~ \tau}
}
\textrm{(TyBox)}
\quad\quad
\ruleI
{       \JudgeT{\Gamma}{e}{\tau}
}
{       \JudgeTS{\Gamma}{\kpure~e}{\tau}{n}{k}
}
\textrm{(TyPure)}
$$

$$
\ruleI
{       \{ \JudgeT
                {\Gamma}{e_i}{\Box~ n_i~ k~ \tau_i} \}^i 
        \quad
        \JudgeTS
                {\Gamma,~ \{x_i : \tau_i \}^i}
                {e'}{\tau'}{n'}{k'}
}
{       \JudgeTS
                {\Gamma}
                {\krun~ \{ x_i : \tau_i = e_i \}^i ~\kthen~ e'}
                {\tau'}
                {max~ \{ n_i \}^i + n'}
                {k'}
}
\textrm{(TyRun)}
$$

% -- Prims ------------------------------------------------
\vspace{2em}
$$
\begin{array}{ll}
\rNat,\rBool      & :: \rData
\\[1ex]
%
\rArray           & :: \rClock \to \rData \to \rData
\\[1ex]
%
\rStream          & :: \rClock \to \rData \to \rDefn
\\[1ex]
%
\Box              & :: \rNat   \to \rClock \to \rData \to \rComp
\\[1ex]
%
\rdrain_{k,\tau}  & :: \rStream~k~\tau \to \Box^1_k~ (\rArray~ k~ \tau)
\\[1ex]
%
\rstream_{k,\tau} & :: \rArray~k~\tau  \to \rStream~k~\tau
\\[1ex]
%
\rsum_k           & :: \rStream~k~\rNat \to \Box^1_k~ \rNat
\\[1ex]
%
\rsmap_{k,a,b}    & :: (a \to b) \to \rStream~k~a \to \rStream~k~b
\\[1ex]
%
\rsfold_{k,a,b}   & :: (a \to b \to b) \to b \to \rStream~k~a \to \Box^1_k~ b
\\[1ex]
%
\rsscan_{k,a,b}   & :: (a \to b \to b) \to b \to \rStream~k~a \to \rStream~k~b
\end{array}
$$

\caption{Typing}
\label{fig:source:type:modal}
\end{figure*}


\section*{Acknowledgements}

\bibliographystyle{plain}
\bibliography{Main}

\end{document}




\end{document}
