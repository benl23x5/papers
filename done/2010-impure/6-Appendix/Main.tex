\appendix

\chapter{Proofs of Language Properties}
\label{proofs}

In this appendix we present the formal proofs of language properties for the system in \S\ref{Core:Simplified}, culminating in a proof of soundness. Each of these proofs is by induction over the derivation of a typing judgement. When presenting each case, we will assume the statement being considered and then invoke the standard inversion lemmas \cite{pierce:tapl} to fill in the appropriate premises.

For example, the proof of Substitution of Values in Values starts with the case $t = x$. We assume the statement 
\ $\tyJudge{\Gamma, \ x : \tau_2}{\Sigma}{x}{\tau_1}{\bot}$, and use the inversion lemma to give \ 
$x : \tau_1 \in \Gamma, \ x : \tau_2$.

$$
	\frac
	{ (2) \ x : \tau_1 \in \Gamma, \ x : \tau_2 }
	{ (\un{1}) \ 
	  \tyJudge
		{\Gamma, \ x : \tau_2}{\Sigma}
		{x}
		{\tau_1}
		{\bot}
	}
$$	

Each statement is numbered for identification purposes, and we underline the numbers of statements which are assumptions. In some
cases, not all statements obtained by the inversion lemmas will be used, but will include them as premises so that the typing rules maintain their familiar shapes.

We will omit the quantifiers ``for all'' and ``for some'' when they are obvious from the context, as they clutter the proof without providing much additional information.

Firstly, some standard lemmas:



% ---------------------
\lemma{Forms of Terms and Types}

When a term is in normal form we can determine its shape by inspecting its type \cite{pierce:tapl}. Similarly, when a type is in normal form we can determine its shape by inspecting its kind.

For example:

\begin{tabular}{ll}
	If	& $t$ is a value \\
	 \ and	& $\tyJudge{\emptyset}{\Sigma}{t}{\tau_1 \to \tau_2}{\sigma}$ \\
	then	& $t = \teLam{x}{\tau_1}{t'}$
\end{tabular}

By inspection of the typing rules. The only values which can have function types are lambda abstractions and variables, but if the type environment is empty the value cannot be a variable. 

\clearpage{}
% ---------------------
\lemma{No free witness variables in effects}

\begin{tabular}{ll}
	If 	& $\tyJudge{\Gamma}{\Sigma}{t}{\varphi}{\sigma}$ \\
	\ and	& $\kiJudge{\Gamma}{\Sigma}{w}{\kappa}$ and $\ksJudgeGS{\kappa}{\Diamond}$ \\[1ex]
	then	& $\sigma[\delta/w] \equiv \sigma$
\end{tabular}

By inspection of the kinding rules for effect constructors.


% ---------------------
\lemma{No free witness variables in term types}

\begin{tabular}{ll}
	If 	& $\tyJudge{\Gamma}{\Sigma}{t}{\varphi}{\sigma}$ \\
	\ and	& $\kiJudge{\Gamma}{\Sigma}{w}{\kappa}$ and $\ksJudgeGS{\kappa}{\Diamond}$ \\[1ex]
	then	& $\varphi[\delta/w] \equiv \varphi$
\end{tabular}

By inspection of the kinding rules for value type constructors.


% ---------------------
\lemma{Weaken Store Typing}

If we can assign a term $t$ some type and effect, then we can also assign $t$ the same type
and effect under a larger store typing. This property is also true for kind and similarity judgements.

\medskip
\begin{tabular}{ll}
	If 	& $\tyJudge{\Gamma}{\Sigma}{t}{\tau}{\sigma}$ 	\\
	\ and	& $\Sigma' \supseteq \Sigma$			\\[1ex]
	then	& $\tyJudge{\Gamma}{\Sigma'}{t}{\tau}{\sigma}$
\end{tabular}
\medskip

By induction over the derivation of \ $\tyJudge{\Gamma}{\Sigma}{t}{\tau}{\sigma}$. At the top of the
derivation tree we will have uses of (TyLoc) which include statements such as $l : \tau \in \Sigma$. 
These statements remain true when $\Sigma$ is extended.


% ---------------------
\lemma{Strengthen Type Environment}

\medskip
\begin{tabular}{ll}
	If	& $\kiJudge{\Gamma, \ x : \tau}{\Sigma}{\varphi}{\kappa}$ \\
	 \ then	& $\kiJudge{\Gamma}{\Sigma}{\varphi}{\kappa}$
\end{tabular}
\medskip

By inspection of the forms of types. Types to not contain value variables.


% ---------------------
\lemma{Similarity under Substitution}

\begin{tabular}{ll}
	If	& $\varphi_1 \sims \varphi_1'$ \\
	 \ and	& $\varphi_2 \sims \varphi_2'$ 
	\\[1ex]
	then	& $\varphi_1[\varphi_2/a] \sims \varphi_1'[\varphi_2'/a]$
\end{tabular}

Easy induction.


% ---------------------
\lemma{Region Witness Assertion}

If we add a property to the heap, then we can always evaluate the witness constructor that
tests for it. 

The statement \ $\heap, \ \ov{\trm{propOf}(\Delta)} \ ; \ \ov{\delta \leadsto \Delta}$ 
	and $\delta \in \{ \iMkConst \ r, \iMkMutable \ r \}$ \ for some $r$, $\Delta$ is true.

By inspection of the transition rules (EwConst) and (EwMutable).

\clearpage{}
% ---------------------
\lemma{Progress of Purity}

\begin{tabular}{ll}
	If 	& $\kiJudge{\emptyset}{\Sigma}{\delta}{\iPure \ \sigma}$ \\
	 \ and 	& $\text{nofab}(\delta)$ \\
	then 	& $\delta = \un{\rpure \ \sigma}$ \\
 	 \ or	& $\heap \with \delta \leadsto \delta'$ for some $\heap, \ \delta'$.
\end{tabular}

\bigskip
\rbProof:
\begin{tabbing}
MM \= MM \= MMMMMMMMMMMMMMMMMMMMMMMMM \= MMMMM  \kill
\>	$(\un{1})$
		\> $\kiJudge{\emptyset}{\Gamma}{\delta}{\iPure \ \sigma}$
		\> \pby{assume}
\\[1ex]	
\>	(\un{2}) 
		\> $\text{nofab}(\delta)$
		\> \pby{assume}
\\[1ex]	
\>	(3) 	\> $\delta \in 
			\{ \iMkPure \ \bot
			 , \ \iMkPurify \ \un{\rho} \ \delta_1,$ 
\\[1ex]	
\>		\> $\qq  \iMkPureJoin \ \sigma_2 \ \sigma_3 \ \delta_2 \ \delta_3 \}$
		\> \pby{Forms of Types 1}
\\
\\
\>	$\emph{Case:}  \ \ \delta = \iMkPure \ \bot$
\\[1ex]	
\>	(5) 	\> \quad $\heap \ ; \ \iMkPure \ \bot \leadsto \un{\rpure \ \bot}$
		\> \pby{EwPure}
\\
\\
\>	$\emph{Case:}  \ \ \delta = \iMkPurify \ \un{\rho} \ \delta_1$
\\[1ex]	
\>	(6) 	\> \quad $\delta_1 = \un{\rconst \ \rho}$
		\> \pby{Kind of $\iMkPurify$ \ 2}
\\[1ex]	
\>	(7) 	\> \quad $\heap \ ; \ \iMkPurify \ (\un{\rconst \ \rho}) \leadsto \un{\rpure \ (\iRead \ \rho)}$
		\> \pby{EwPurify 6 7}
\\
\\
\>	$\emph{Case:}  \ \ \delta = \iMkPureJoin \ \sigma_2 \ \sigma_3 \ \delta_2 \ \delta_3$
\\[1ex]	
\>	(8) 	\> \quad $\kiJudge{\emptyset}{\Sigma}{\delta_2}{\iPure \ \sigma_2}$
		\> \pby{Kind of $\iMkPureJoin$}
\\[1ex]	
\>	(9) 	\> \quad $\text{nofab}(\delta_2)$
		\> \pby{Def. nofab 2}
\\[1ex]	
\>	(10)	\> \quad $\delta_2 = \un{\rpure \ \sigma_2} \ \ \text{or} \ \
		   \heap \with \delta_2 \leadsto \delta_2'$
		\> \pby{IH 8 9}
\\[1ex]	
\>	(11)	\> \quad $\delta_3 = \un{\rpure \ \sigma_3} \ \ \text{or} \ \
		   \heap \with \delta_3 \leadsto \delta_3'$
		\> \pby{Similarly}
\\[1ex]	
\>	(12)	\> \quad Either (EwPureJoin1), (EwPureJoin2)
\\[1ex]	
\> 		\> \quad \qq  or (EwPureJoin3) applies
\end{tabbing}


% -- Lemma: Preservation of types under value substitution -----------------------------------------
\clearpage{}
\begin{flushleft}
\textbf{Lemma: (Substitution of Values in Values)}

\begin{tabular}{ll}
	If  	&  $\tyJudge{\Gamma, \ x : \tau_2}{\Sigma}{t}{\tau_1}{\sigma}$ \\
	\ and  	&  $\tyJudge{\Gamma}{\Sigma}{v^\circ}{\tau_2'}{\bot}$ \\
	\ and 	&  $\tau_2 \sim_\Sigma \tau_2'$ \\[1ex]
	then 	&  $\tyJudge{\Gamma}{\Sigma}{t[v^\circ/x]}{\tau_1'}{\sigma'}$ \\
	\ and  	&  $\tau_1 \sim_\Sigma \tau_1'$ \\
	\ and  	&  $\sigma \: \sim_\Sigma \sigma'$ \\
\end{tabular}

\medskip
\trb{Proof:} By induction over the derivation of \ $\tyJudge{\Gamma, \ x : \tau_2}{\Sigma}{t}{\tau_1}{\sigma}$

\medskip
\begin{tabbing}
MM \= MMMM \= MMMMMMMMMMMMMMMMMMMMMMMMM \= MMMMM  \kill
\>	(IH) 	\> Subst. Values/Values holds for all subterms of $t$.
		\> (assume)
\end{tabbing}

\medskip

% ---------------------
$\rbCase: \ t = y$ / TyVar

Trivial. $x \neq y$ so $t$ is unaffected.

\bigskip


% ---------------------
$\rbCase: \ t = x$ / TyVar

$$
\qq \qq
	\frac
	{ (4) \ x : \tau_1 \in \Gamma, \ x : \tau_2 }
	{ (\un{1}) \ 
	  \tyJudge
		{\Gamma, \ x : \tau_2}{\Sigma}
		{x}
		{\tau_1}
		{\bot}
	}
$$	

\begin{tabbing}
MM \= MMMM \= MMMMMMMMMMMMMMMMMMMMMMMMM \= MMMMM  \kill
\> 	$(\un{2})$	\> $\tyJudge{\Gamma}{\Sigma}{v^\circ}{\tau_2'}{\bot}$ 
			\> \pby{assume} 
\\[1ex]
\>	$(\un{3})$	\> $\tau_2 \sim_\Sigma \tau_2'$
			\> \pby{assume}
\\[1ex]
\>	(5)		\> $\tau_1 \equiv \tau_2$
			\> \pby{Def. Type Env 4}
\\[1ex]
\>	(6)		\> $\tyJudge{\Gamma}{\Sigma}{x[v^\circ/x]}{\tau_2'}{\bot}$
			\> \pby{Def. Sub. 2}
\\[1ex]
\>	(7)		\> $\tau_1 \sim_\Sigma \tau_2'$
			\> \pby{3 5}
\end{tabbing}
\bigskip


% ---------------------
$\rbCase: \ t = \teLAM{x}{\kappa}{t_1}$ / TyAbsT

$$
\qq \qq
	\frac
	{ (4) \ 
	  \tyJudge
		{\Gamma, \ x : \tau_2, \ a : \kappa}{\Sigma}
		{t_1}
		{\tau_1}
		{\sigma}
	}
	{ (\un{1}) \ 
	  \tyJudge
		{\Gamma, \ x : \tau_2}{\Sigma}
		{\teLAM{a}{\kappa}{t_1}}
		{\tyForall{a}{\kappa}{\tau_1}}
		{\sigma}
	}
$$
\begin{tabbing}
MM \= MMMM \= MMMMMMMMMMMMMMMMMMMMMMMMM \= MMMMM  \kill
\>	(\un{2}) 	\> $\tyJudge{\Gamma} {\Sigma} {v^\circ} {\tau_2'} {\bot}$
			\> \pby{assume}
\\[1ex]
\>	(\un{3})	\> $\tau_2 \sim_\Sigma \tau_2'$
			\> \pby{assume} 
\\[1ex]
\>	(5)		\> $\tyJudge{\Gamma, \ a : \kappa}{\Sigma}
				{v^\circ}
				{\tau_2'}
				{\bot}$
			\> \pby{Weak. Type Env 2} 
\\[1ex]
\>	(6..8)		\> $\tyJudge
				{\Gamma, \ a : \kappa}{\Sigma}
				{t_1[v^\circ/x]}
				{\tau_1'}
				{\sigma'},$ 
\\[0.2ex]
\>			\> $\qq	\tau_1   \sim_\Sigma \tau_1', \
			  	\sigma \sim_\Sigma \sigma'$ 
			\> \pby{IH 4 5 3} 
\\[1ex]
\>	(9)		\> $\tyJudge
				{\Gamma}{\Sigma}
				{\teLAM{a}{\kappa}{(t_1[v^\circ/x]})}
				{\forall(a : \kappa). \tau_1'}
				{\sigma'}$
			\> \pby{TyAbsT 6} 
\\[1ex]
\>	(10) 		\> $\tyJudge
				{\Gamma}{\Sigma}
				{(\teLAM{a}{\kappa}{t_1})[v^\circ/x]}
				{\forall(a : \kappa). \tau_1'}
				{\sigma'}$
			\> \pby{Def. Sub. 9}
\end{tabbing}

\bigskip

\clearpage{}
% ------------------
\pCase{$t = t_1 \ \varphi_2$ / TyAppT}
$$
\frac
	{ \begin{aligned}
	  \\
	  (4) \ \tyJudge{\Gamma, \ x : \tau_2} 
			{\Sigma} 
			{t_1} 
			{\tyForall{a}{\kappa_{11}}{\varphi_{12}}} 
			{\sigma} 
	  \end{aligned}
	  \tspace
	  \begin{aligned}
	  (6) \ \kappa_{11} \sim_\Sigma \kappa_2
	  \\
	  (5) \ \kiJudge{\Gamma, \ x : \tau_2} 
	  		{\Sigma} 
			{\varphi_2} 
			{\kappa_2}
	  \end{aligned}
	}
	{ (\un{1}) \ \tyJudge
			{\Gamma, \ x : \tau_2} 
			{\Sigma} 
			{t_1 \ \varphi_2} {\varphi_{12}[\varphi_2/a]} 
			{\sigma[\varphi_2/a]} 
	}
$$
\begin{tabbing}
MM \= MMMM \= MMMMMMMMMMMMMMMMMMMMMMM \= MMMMM  \kill
\>	(\un{2})	\> $\tyJudge{\Gamma} {\Sigma} {v^\circ} {\tau_2'} {\bot}$
			\> \pby{assume} 
\\[1ex]
\>	(\un{3})	\> $\tau_2 \sim_\Sigma \tau_2'$
			\> \pby{assume}
\\[1ex]
\>	(7..9)		\> $\tyJudge
				{\Gamma} {\Sigma} 
				{t_1[v^\circ/x]}
				{\tyForall{a}{\kappa_{11}'}{\varphi_{12}'}}
				{\sigma'}$
\\[0.2ex]
\>			\> \qq 	$\tyForall{a}{\kappa_{11}}{\varphi_{12}}
				\sim_\Sigma
				\tyForall{a}{\kappa_{11}'}{\varphi_{12}'}$
\\[0.2ex]
\>			\> \qq	$\sigma[\varphi_2/a] \sim_\Sigma \sigma'$
			\> \pby{IH 4 2 3}
\\[1ex]
\>	(10)		\> $\kappa_{11}  \sims \kappa_{11}'$
			\> \pby{SimAll 8}
\\[1ex]
\>	(11) 		\> $\kappa_{11}' \sim_\Sigma \kappa_2$
			\> \pby{SimTrans 6 10}
\\[1ex]
\>	(12) 		\> $\kiJudge
				{\Gamma} 
		  		{\Sigma} 
				{\varphi_2} 
				{\kappa_2} $
			\> \pby{Str. Type Env 5}
\\[1ex]
\>	(13) 		\> $\tyJudge
				{\Gamma} {\Sigma} 
				{t_1[v^\circ/x] \ \varphi_2}
				{\varphi_{12}'[\varphi_2/a]}
				{\sigma'[\varphi_2/a]}$
			\> \pby{TyAppT 7 12 11}
\\[1ex]
\>	(14) 		\> $\tyJudge
				{\Gamma} {\Sigma}
				{(t_1 \ \varphi_2)[v^\circ/x]}
				{\varphi_{12}'[\varphi_2/a]}
				{\sigma'[\varphi_2/a]}$
			\> \pby{Def. Sub. 13}
\end{tabbing}
\bigskip

% ---------------------
\pCase{$t = \teLam{x}{\tau}{t_1}$ / TyAbs}

Similarly to TyAbsT case.
\bigskip


% ---------------------
\pCase{$t = t_1 \ t_2$ / TyApp}
$$
	\frac
	{ \begin{aligned}
	  \\
	  (4) \ 
	  \tyJudge
		{\Gamma, \ x : \tau_3}
		{\Sigma}
		{t_1}
		{\tau_{11} \funa{\sigma} \tau_{12}}
		{\sigma_1}
	  \end{aligned}
	  \tspace
	  \begin{aligned}
	  (6) \
   	  \tau_{11} \sims \tau_2
	  \\
	  (5) \ 
	  \tyJudge
	  	{\Gamma, \ x : \tau_3} 
	  	{\Sigma}
		{t_2}
		{\tau_2}
		{\sigma_2}
	  &
	  \end{aligned}
	}
	{ (\un{1}) \ \tyJudge 
			{\Gamma, \ x : \tau_3}
			{\Sigma} {t_1 \ t_2} {\tau_{12}}
			{\sigma_1 \lor \sigma_2 \lor \sigma} 
	}
$$	
\begin{tabbing}
MM \= MMMM \= MMMMMMMMMMMMMMMMMMMMMMM \= MMMMM  \kill
\>	(\un{2}) 	\> $\tyJudge{\Gamma} {\Sigma} {v^\circ} {\tau_3'} {\bot}$
			\> \pby{assume}
\\[1ex]
\>	(\un{3}) 	\> $\tau_3 \sims \tau_3'$
			\> \pby{assume}
\\[1ex]
\>	(7..9) 		\> $\tyJudgeGS
				{t_1[v^\circ/x]}
				{\tau_{11}' \funa{\sigma'} \tau_{12}'} {\sigma_1'}$
\\[0.2ex]
\>			\> \qq	${\tau_{11}' \funa{\sigma'} \tau_{12}'}
				\sims
				{\tau_{11} \funa{\sigma} \tau_{12}}$
\\[0.2ex]
\>			\> \qq	$\sigma_1 \sims \sigma_1'$
			\> \pby{IH 4 2 3}
\\[1ex]
\>	(10..12) 	\> $\tyJudge{\Gamma} {\Sigma} {t_2[v^\circ/x]} {\tau_2'} {\sigma_2'}$
\\[0.2ex]
\>			\> \qq	$\tau_2 \sims \tau_2'$
\\[0.2ex]
\>			\> \qq	$\sigma_2 \sims \sigma_2'$
			\> \pby{IH 5 2 3}
\\[1ex]
\>	(13)		\> $\tau_{11}' \sims \tau_2'$
			\> \pby{SimApp, SimTrans 8 6 11} 
\\[1ex]
\>	(14) 		\> $\tyJudgeGS 
				{t_1[v^\circ/x] \ t_2[v^\circ/x]} {\tau_{12}'} 
				{ \sigma_1' \lor \sigma_2' \lor \sigma' }$
			\> \pby{TyApp 7 10 13}
\\[1ex]
\>	(15) 		\> $\tyJudgeGS
				{(t_1 \ t_2)[v^\circ/x]} {\tau_{12}'}
				{ \sigma_1' \lor \sigma_2' \lor \sigma' }$
			\> \pby{Def. Sub. 14}
\end{tabbing}
\bigskip


% ---------------------
\pCase{$t = (\teLet{x}{t_1}{t_2})$ / TyLet} \\
\pCase{$t = (\teLetR{r}{\ov{w_i :: \delta_i}}{t_1})$ / TyLetRegion} \\
\pCase{$t = (\teIf{t_1}{t_2}{t_3})$ / TyIf}

Similarly to TyApp case.
\bigskip


% ---------------------
\pCase{$t = \iTrue  \ \varphi$ / TyTrue} \\
\pCase{$t = \iFalse \ \varphi$ / TyFalse} 

Trivial. Types contain do not contain value variables.
\bigskip


% ---------------------
\pCase{$t = \iupdate \ \delta \ t_1 \ t_2$ / TyUpdate} \\
\pCase{$t = \isuspend \ \delta \ t_1 \ t_2$ / TySuspend} 

Similarly to TyApp case.
\bigskip

% ---------------------
\pCase{$t = ()$ / TyUnit} 

Trivial. Unit does not contain value variables. 

\bigskip

% ---------------------
\pCase{$t = \un{l}$ / TyLoc} 

Trivial. Locations do not contain value variables.


\end{flushleft}


\clearpage{}
\begin{flushleft}
\textbf{Lemma: (Substitution of Types in Values)}

\begin{tabular}{ll}
 	If  	& $\tyJudge{\Gamma, \ a : \kappa_3}{\Sigma}{t}{\tau_1}{\sigma}$ 
	\\
 	\ and  	& $\kiJudge{\Gamma}{\Sigma}{\varphi_2}{\kappa_2}$ 
	\\
	\ and	& $\kappa_3 \sims \kappa_2$ 
	\\[1ex]
	then	& $\tyJudge{\Gamma[\varphi_2/a]}{\Sigma}
				{t[\varphi_2/a]}{\tau_1[\varphi_2/a]}{\sigma[\varphi_2/a]}$ \\
\end{tabular}

\medskip
\trb{Proof:} by induction over the derivation of $\tyJudge{\Gamma, \ a : \kappa_3}{\Sigma}{t}{\tau_1}{\sigma}$

\medskip
\begin{tabbing}
M \= MMx \= MMMMMMMMMMMMMMMMMMMMMMMMMMMMMMM \= MMMMM  \kill
\>	 (IH) 
	\> Substitution holds for all subterms of $t$.
	\> (assume) 
\end{tabbing}

% ---------------------
\pCase{$t = x$ / TyVar}

Trival. No type vars in value vars.


\bigskip
% ---------------------
\pCase{$t = \teLAM{a}{\kappa}{t}$ / TyAbsT}
$$
\qq \qq	\infer
	{ (\un{1}) \
	  \tyJudge
		{\Gamma, \ a : \kappa_3}
		{\Sigma}
		{\teLAM{a_{11}}{\kappa_{11}}{t_{12}}}
		{\tyForall{a_{11}}{\kappa_{11}}{\tau_{12}}}
		{\sigma_1}
	}
	{ (4) \
	  \tyJudge
	  	{\Gamma, \ a : \kappa_3, \ a_{11} : \kappa_{11}}
		{\Sigma}
		{t_{12}}
		{\tau_{12}}
		{\sigma_1}
	}
$$
\begin{tabbing}
M \= MMx \= MMMMMMMMMMMMMMMMMMMMMMMMMM \= MMMMM  \kill
\>	(\un{2}) 
		\> $\kiJudgeGS{\varphi_2}{\kappa_2}$
		\> \pby{assume}
\\[1ex]
\>	(\un{3}) 
		\> $\kappa_3 \sims \kappa_2$
		\> \pby{assume}
\\[1ex]
\>	(5) 	\> $\kiJudge
			{\Gamma, \ a_{11} : \kappa_{11}}
			{\Sigma}
			{\varphi_2}
			{\kappa_2}$
		\> \pby{Weak. Type Env 2}
\\[1ex]
\>	(6) 	\> $\ \ \ \ \ \: 
			{(\Gamma, \  a_{11} : \kappa_{11})[\varphi_2/a]} 
			\ \vert \ 	{\Sigma} 
			\judge 		{t_{12}[\varphi_2/a]}$
\\[0.2ex]
\>		\> \hspace{12em}	
			$:: \	{\tau_{12}[\varphi_2/a]}
			 \ ; \  {\sigma_1[\varphi_2/a]}$
		\> \pby{IH 4 5 3}
\\[1ex]
\>	(7)	\> $	{\Gamma[\varphi_2/a], \ a_{11} : \kappa_{11}[\varphi_2/a]} 
			\ \vert \ 	{\Sigma}
			\judge 		{t_{12}[\varphi_2/a]}$
\\[0.2ex]
\>		\> \hspace{12em}
			$:: \	{\tau_{12}[\varphi_2/a]}
			 \ ; \	{\sigma_1[\varphi_2/a]}$
		\> \pby{Def. Sub. 6}
\\[1ex]
\>	(8)	\> ${\Gamma[\varphi_2/a]} \ \arrowvert \ {\Sigma} \ \vdash
			{(\teLAM
				{a_{11}}
				{\kappa_{11}}
				{t_{12}})
				[\varphi_2/a]
			}$
\\[1ex]
\>		\> $\hspace{5em}
			:: \
			{(\tyForall 
				{a_{11}}
				{\kappa_{11}}
				{\tau_{12}})
				[\varphi_2/a]
			}
			\ ; \ 
			{\sigma_1[\varphi_2/a]}$
		\> \pby{Def. Sub, TyAbsT 7}
\end{tabbing}


\bigskip
% ---------------------
\pCase{$t = t_{11} \ \varphi_{12}$ / TyAppT}
$$
	\infer
	{ (\un{1}) \
	  \tyJudge
		{\Gamma, \ a : \kappa_4}
		{\Sigma}
		{t_1 \ \varphi_2}
		{\varphi_{12}[\varphi_2/a_1]}
		{\sigma_1[\varphi_2/a_1]}
	}
	{ (4) \
	  \tyJudge
	  	{\Gamma, \ a : \kappa_4}
		{\Sigma}
		{t_1}
		{\tyForall{a_1}{\kappa_{11}}{\varphi_{12}}}
		{\sigma_1}	
	  \quad
	  (5) \ 
	  \kiJudge
	  	{\Gamma, \ a : \kappa_4}
		{\Sigma}
		{\varphi_2}
		{\kappa_2}
	  \quad
	  (6) \ 
	  \kappa_{11} \sims \kappa_{2}
	}
$$
\begin{tabbing}
M \= MMx \= MMMMMMMMMMMMMMMMMMMMMMMMMM \= MMMMM  \kill
\>	(\un{2})
		\> $\kiJudgeGS{\varphi_3}{\kappa_3}$
		\> \pby{assume}
\\[1ex]
\>	(\un{3})
		\> $\kappa_4 \sims \kappa_3$
		\> \pby{assume}
\\[1ex]
\>	(7)	\> ${\Gamma[\varphi_3/a]} 
			\ \vert \ 	{\Sigma}
			\ \vdash \	{t_1[\varphi_3/a]}$
\\[0.2ex]
\>		\> $\hspace{5.6em} 
			  :: \ { (\tyForall{a_1}{\kappa_{11}}{\varphi_{12}})[\varphi_3/a] }
			\ ; \ { \sigma_1[\varphi_3/a] }$
		\> \pby{IH 4 2 3}
\\[1ex]
\>	(8)	\> ${\Gamma[\varphi_3/a]}
			\ \vert \	{\Sigma}
			\ \vdash \ 	{t_1[\varphi_3/a]}$
\\[0.2ex]
\>		\> $\hspace{5.6em} ::
			{ \tyForall
				{a_1}
				{\kappa_{11}[\varphi_3/a]}
				{\varphi_{12}[\varphi_3/a] }
			}
			\ ; \ { \sigma_1[\varphi_3/a] }$
		\> \pby{Def. Sub. 7}
\\[1ex]
\>	(9)	\> $\kiJudge
			{\Gamma[\varphi_3/a]}
			{\Sigma}
			{\varphi_2[\varphi_3/a]}
			{\kappa_2[\varphi_3/a]}$
		\> \pby{Sub. Type/Type 5 2 3}
\\[1ex]
\>	(10)	\> $\kappa_{11}[\varphi_3/a] \sims 
		  \kappa_{2}[\varphi_3/a]$
		\> \pby{Def. Sub, Def. ($\sim$), 6}
\\[1ex]
\>	(11)	\> ${\Gamma[\varphi_3/a]} \ \arrowvert \ {\Sigma} \ \vdash \
			{t_1[\varphi_3/a] \ \ \varphi_2[\varphi_3/a]}$
\\[0.2ex]
\>		\> $\hspace{5.6em} :: \ {(\varphi_{12}[\varphi_3/a])[\varphi_2[\varphi_3/a]/a_1]}$
\\[0.2ex]
\>		\> $\hspace{5.6em} ; \ \  {(\sigma_1    [\varphi_3/a])[\varphi_2[\varphi_3/a]/a_1]}$
		\> \pby{TyAppT 8 9 10}
\\[1ex]
\>	(12)	\> $a \ne a_1$
		\> \pby{No Var Capture 4}
\\[1ex]
\>	(13)	\> ${\Gamma[\varphi_3/a]} \ \arrowvert \ {\Sigma} \ \vdash \
			{(t_1 \ \varphi_2)[\varphi_3/a]}$
\\[0.2ex]
\>		\> $\hspace{5.4em} :: \ {(\varphi_{12}[\varphi_2/a_1])[\varphi_3/a]}
			\ \ ; \ {(\sigma_1[\varphi_2/a_1])[\varphi_3/a]}$
		\> \pby{Def. Sub. 11 12}
\end{tabbing}

\clearpage{}
% ----------------------
\pCase{$t = \teLam{x}{\tau_{11}}{t_{12}}$ / TyAbs} \\
\pCase{$t = (t_1 \ t_2)$ / TyApp} \\
\pCase{$t = (\teLet{x}{t_1}{t_2})$ / TyLet}

Similarly to TyAbsT Case

\bigskip
% ----------------------
\pCase{$t = \teLetR{r}{\ov{w_i = \delta_i}}{t_1}$ / TyLetRegion}

$$
\qq	\infer
	{ (\un{1}) \
	  \tyJudge
		{\Gamma, \ a : \kappa_3}
		{\Sigma}
		{\teLetR{r}{\ov{w_i = \delta_i}}{t_1}}
		{\sigma}
	}
	{
	  \begin{aligned}
		(6) \
		\ov{\delta_i} \ \text{well formed}
		\\
	  	(4) \
	  	\tyJudge
			{\Gamma, \ a : \kappa_3, \ r : \%, \ \ov{w_i = \kappa_i}}
			{\Sigma}
			{t_1}
			{\tau}
			{\sigma}
	  \end{aligned}
	  \quad
	  \begin{aligned}
		(7) \
	        \ksJudge
			{\Gamma}
			{\Sigma&}
			{\kappa_i}
			{\Diamond}
		\\
	        (5) \
	  	\kiJudge
			{\Gamma, \ a : \kappa_3}
			{\Sigma&}
			{\delta_i}
			{\kappa_i}
	  \end{aligned}
	}
$$

\begin{tabbing}
M \= MMx \= MMMMMMMMMMMMMMMMMMMMMMMMMMMMM \= MMMMM  \kill
\>	(\un{2}) 
		\> $\kiJudgeGS{\varphi_2}{\kappa_2}$
		\> \pby{assume}
\\[1ex]
\>	(\un{3}) 
		\> $\kappa_3 \sims \kappa_2$
		\> \pby{assume}
\\[1ex]
\>	(8) 	\> $\kiJudge
			{\Gamma, \ r : \%, \ \ov{w_i : \kappa_i}}
			{\Sigma}
			{\varphi_2}
			{\kappa_2}$
		\> \pby{Weak. Type Env 2}
\\[1ex]
\>	(9)	\> $\tyJudge
			{(\Gamma, \ r : \%, \ \ov{w_i : \kappa_i})[\varphi_2/a]}
			{\Sigma}
			{t[\varphi_2/a]}
			{\tau[\varphi_2/a]}
			{\sigma[\varphi_2/a]}$
		\> \pby{IH 4 8 3}
\\[1ex]
\>	(10)	\> ${\Gamma[\varphi_2/a], \ r : \%, \ (\ov{w_i : \kappa_i[\varphi_2/a]})} \arrowvert {\Sigma} 
				\judge 	{t[\varphi_2/a]}$
\\[0.2ex]
\>		\> $\hspace{14.5em} :: {\tau[\varphi_2/a]} \ ; \ 
		    {\sigma[\varphi_2/a]}$
		\> \pby{Def. Sub. 9}
\\[1ex]
\>	(11)	\> $\kiJudge
			{\Gamma[\varphi_2/a]}
			{\Sigma}
			{\delta_i[\varphi_2/a]}
			{\kappa_i[\varphi_2/a]}$
		\> \pby{Sub. Type/Type 5 2 3}
\\[1ex]
\>	(12)	\> $\ksJudge
			{\Gamma}
			{\kappa_i[\varphi_2/a]}
			{\Diamond}$
		\> \pby{Insp. Kinding Rules}
\\[1ex]
\>	(13)	\> $\ov{\delta_i}[\varphi_2/a] \ \text{well formed}$
		\> \pby{Def. Well Formed 6}
\\[1ex]
\>	(14)	\> ${\Gamma[\varphi_2/a]} \ 
			\arrowvert \ {\Sigma}
			\vdash \ {\teLetR{r}{\ov{w_i = \delta_i[\varphi_2/a]}}{t[\varphi_2/a]}}$
\\[0.2ex]
\>		\> $\hspace{5em}
			:: \ {\tau[\varphi_2/a]} \ ; {\sigma[\varphi_2/a]}$
		\> \pby{TyLetRegion 10..13}
\\[1ex]
\>	(15)	\> ${\Gamma[\varphi_2/a]} \
			\arrowvert \ {\Sigma}
			\vdash \ {(\teLetR{r}{\ov{w_i = \delta_i}}{t})[\varphi_2/a]}$
\\[0.2ex]
\>		\> $\hspace{5em}
			:: \ {\tau[\varphi_2/a]} \ ; {\sigma[\varphi_2/a]}$
		\> \pby{Def. Sub. 14}
\end{tabbing}

\bigskip
% --------------------
\pCase{$t = \teIf{t_1}{t_2}{t_3}$ / TyIf}

Similarly to TyAbsT case.

\bigskip
% -------------------
\pCase{$t = \iTrue \ \varphi$ / TyTrue}
$$
\qq	\infer
	{ (\un{1}) \
	  \tyJudge
		{\Gamma, \ a : \kappa_3}
		{\Sigma}
		{\iTrue \ \varphi}
		{\iBool \ \varphi}
		{\bot}
	}
	{ (4) \
	  \kiJudge
		{\Gamma, \ a : \kappa_3}
		{\Sigma}
		{\varphi}
		{\%}
	}	
$$
\begin{tabbing}
M \= MMx \= MMMMMMMMMMMMMMMMMMMMMMMMMMMMM \= MMMMM  \kill
\>	(\un{2}) 
		\> $\kiJudgeGS{\varphi_2}{\kappa_2}$
		\> \pby{assume}
\\[1ex]
\>	(\un{3}) 
		\> $\kappa_3 \sims \kappa_2$
		\> \pby{assume}
\\[1ex]
\>	(5) 	\> $\kiJudge
			{\Gamma[\varphi_2/a]}
			{\Sigma}
			{\varphi[\varphi_2/a]}
			{\%[\varphi_2/a]}$
		\> \pby{Sub Type/Type 4 2 3}
\\[1ex]
\>	(6) 	\> $\kiJudge
			{\Gamma[\varphi_2/a]}
			{\Sigma}
			{\varphi[\varphi_2/a]}
			{\%}$
		\> \pby{Def. Sub. 5}
\\[1ex]
\>	(7) 	\> $\tyJudge
			{\Gamma[\varphi_2/a]}
			{\Sigma}
			{\iTrue \ (\varphi[\varphi_2/a])}
			{\iBool \ (\varphi[\varphi_2/a])}
			{\bot}$
		\> \pby{TyTrue 6}
\\[1ex]
\>	(8) 	\> $\tyJudge
			{\Gamma[\varphi_2/a]}
			{\Sigma}
			{(\iTrue \ \varphi)[\varphi_2/a]}
			{(\iBool \ \varphi)[\varphi_2/a]}
			{\bot}$
		\> \pby{Def. Sub. 7}
\end{tabbing}

\clearpage{}
% --------------------
\pCase{$t = \iFalse \ \varphi$}

Similarly to TyTrue Case.

\bigskip
% --------------------
\pCase{$t = \iupdate \ \delta \ t_1 \ t_2$ / TyUpdate}
$$
\qq	\infer
	{ (\un{1}) \
	  \tyJudge
		{\Gamma, \ a : \kappa_4}
		{\Sigma}
		{\iupdate \ \delta \ t_1 \ t_2}
		{()}
		{\sigma_1 \lor \sigma_2 \lor \iRead \ \varphi_2 \lor \iWrite \ \varphi_1}
	}
	{
	  \begin{aligned}		
	 	\\
		(4) \
	 	\kiJudge
			{\Gamma, \ a : \kappa_4}
			{\Sigma}
			{\delta}
			{\iMutable \ \sigma}
	  \end{aligned}
	  \quad
	  \begin{aligned}
	  	(5) \
	 	\tyJudge
			{\Gamma, \ a : \kappa_4}
			{\Sigma}
			{t_1}
			{\iBool \ \varphi_1}
			{\sigma_1}
		\\
		(6) \
		\tyJudge
			{\Gamma, \ a : \kappa_4}
			{\Sigma}
			{t_2}
			{\iBool \ \varphi_2}
			{\sigma_2}
	  \end{aligned}
	}
$$
\begin{tabbing}
M \= MMx \= MMMMMMMMMMMMMMMMMMMMMMMMMM \= MMMMM  \kill
\>	(\un{2}) 
		\> $\kiJudgeGS{\varphi_3}{\kappa_3}$
		\> \pby{assume}
\\[1ex]
\>	(\un{3}) \
		\> $\kappa_4 \sims \kappa_3$
		\> \pby{assume}
\\[1ex]
\>	(7) 	\> $\kiJudge
			{\Gamma, \ a : \kappa_4}
			{\Sigma}
			{\varphi_3}
			{\kappa_3}$
		\> \pby{Weak. Type Env 2}
\\[1ex]
\>	(8) 	\> $\tyJudge
			{\Gamma[\varphi_3/a]}
			{\Sigma}
			{t_1[\varphi_3/a]}
			{\iBool \ (\varphi_1[\varphi_3/a])}
			{\sigma_1[\varphi_3/a]}$
		\> \pby{IH, Def. Sub. 5 7 3}
\\[1ex]
\>	(9) 	\> $\tyJudge
			{\Gamma[\varphi_3/a]}
			{\Sigma}
			{t_2[\varphi_3/a]}
			{\iBool \ (\varphi_2[\varphi_3/a])}
			{\sigma_2[\varphi_3/a]}$
		\> \pby{IH, Def. Sub. 6 7 3}
\\[1ex]
\>	(10) 	\> $\kiJudge
			{\Gamma[\varphi_3/a]}
			{\Sigma}
			{\delta[\varphi_3/a]}
			{(\iMutable \ \delta)[\varphi_3/a]}$
		\> \pby{Sub. Type/Type 4 2 3}
\\[1ex]
\>	(11) 	\> $\kiJudge
			{\Gamma[\varphi_3/a]}
			{\Sigma}
			{\delta[\varphi_3/a]}
			{\iMutable \ (\delta[\varphi_3/a])}$
		\> \pby{Def. Sub. 10}
\\[1ex]
\>	(12) 	\> $\Gamma[\varphi_3/a] \: \arrowvert \: \Sigma
			\ \vdash \ 
			{\iupdate 
				\ (\delta[\varphi_3/a])
				\ (t_1[\varphi_3/a])
				\ (t_2[\varphi_3/a])}$
\\[0.2ex]
\>		\> $\hspace{5em} \ :: \ ()$
\\[0.2ex]
\>		\> $\hspace{5em} \ ; \ \
			\sigma_1[\varphi_3/a] \lor \sigma_2[\varphi_3/a]
			\lor \iRead  \ (\varphi_2[\varphi_3/a])$
\\[0.2ex]
\>		\> $\hspace{14.8em} \lor \iWrite \ (\varphi_1[\varphi_3/a])$
		\> \pby{TyUpdate 8 9 11}
\\[1ex]
\>	(13) 	\>$ \Gamma[\varphi_3/a] \: \arrowvert \: \Sigma
			\ \vdash \ 
			{(\iupdate \ \delta \ t_1 \ t_2)[\varphi_3/a]}$
\\[0.2ex]
\>		\> $\hspace{5em} \ :: \ ()$
\\[0.2ex]
\>		\> $\hspace{5em} \ ; \ \
			{(\sigma_1 \lor \sigma_2 \lor \iRead \ \varphi_2 \lor \iWrite \ \varphi_1)
				[\varphi_3/a]}$
		\> \pby{Def. Sub. 12}
\end{tabbing}

\bigskip
% --------------------
\pCase{$t = \isuspend \ \delta \ t_1 \ t_2$ / TySuspend}

Similarly to TyApp / TyUpdate case.

\bigskip
% --------------------
\pCase{$t = ()$ / TyUnit} \\
\pCase{$t = \un{l}$ / TyLoc}

Trivial. No free type vars.


\end{flushleft}






\clearpage{}
\begin{flushleft}
\textbf{Lemma: (Substitution of Types in Types)}

\begin{tabular}{ll}
 	If  	& $\kiJudge
			{\Gamma, \ a : \kappa_3}
			{\Sigma}
			{\varphi_1}
			{\kappa_1}$ 
	\\
 	\ and  	& $\kiJudge
			{\Gamma}
			{\Sigma}
			{\varphi_2}
			{\kappa_2}$ 
	\\
	\ and	& $\kappa_3 \sims \kappa_2$ 
	\\[1ex]
	then	& $\kiJudge
			{\Gamma[\varphi_2/a]}
			{\Sigma}
			{\varphi_1[\varphi_2/a]}{\kappa_1[\varphi_2/a]}$ 
	\\
\end{tabular}

\medskip
\trb{Proof:} by induction over the derivation of
		$\kiJudge
			{\Gamma, \ a : \kappa_3}
			{\Sigma}
			{\varphi_1}
			{\kappa_1}$ 

\medskip
\begin{tabbing}
MM \= MMx \= MMMMMMMMMMMMMMMMMMMMMMMMMMMMMx \= MMMMM  \kill
\> (IH) 
	\> Substitution holds for all subterms of $\varphi_1$.
	\> (assume) 
\end{tabbing}

\medskip
% --------------------
\pCase{$\varphi = a$ / KiVar}

Similarly to Subst Var/Var TyVar case.

\bigskip
% --------------------
\pCase{$\varphi = \tyForall{b}{\kappa_1}{\tau}$ / KiAll}
$$
\qq	\infer
	{ (\un{1}) \
	  \kiJudge
		{\Gamma, \ a : \kappa_4}
		{\varphi_1}
		{\tyForall{b}{\kappa_1}{\tau_1}}
		{\kappa_2}
	}
	{ (4) \ 
	  \kiJudge
		{\Gamma, \ a : \kappa_4}
		{\Sigma}
		{\sigma}
		{\kappa_1}
	  \quad
	  (5) \
	  \kiJudge
		{\Gamma, \ a : \kappa_4, \ b : \kappa_1}
		{\Sigma}
		{\tau_1}
		{\kappa_2}
	}
$$	
\begin{tabbing}
MM \= MMx \= MMMMMMMMMMMMMMMMMMMMMMMMMMM \= MMMMM  \kill
\>	(\un{2}) 
		\> $\kiJudgeGS{\varphi_3}{\kappa_3}$
		\> \pby{assume}
\\[1ex]
\>	(\un{3}) 
		\> $\kappa_4 \sims \kappa_3$
		\> \pby{assume}
\\[1ex]
\>	(6) 	\> $\kiJudge
			{\Gamma[\varphi_3/a]}
			{\Sigma}
			{\varphi_1[\varphi_3/a]}
			{\kappa_1[\varphi_3/a]}$
		\> \pby{IH 4 2 3}
\\[1ex]
\>	(7) 	\> $\kiJudge
			{\Gamma, \ b : \kappa_1}
			{\Sigma}
			{\varphi_3}
			{\kappa_3}$
		\> \pby{Weak. Type Env 2}
\\[1ex]
\>	(8) 	\> $\kiJudge
			{\Gamma[\varphi_3/a], \ b : \kappa_1[\varphi_3/a]}
			{\Sigma}
			{\tau_1[\varphi_3/a]}
			{\kappa_2[\varphi_3/a]}$
		\> \pby{IH, Def. Subst 5 7 3}
\\[1ex]
\>	(9) 	\> $b \notin \ifv(\Gamma[\varphi_3/a])$
		\> \pby{Uniqueness of Vars}
\\[1ex]
\>	(10) 	\> $\kiJudge
			{\Gamma[\varphi_3/a]}
			{\Sigma}
			{(\tyForall{b}{\kappa_1}{\tau_1})[\varphi_3/a]}
			{\kappa_2[\varphi_3/a]}$
		\> \pby{KiAll, Def. Sub. 6 8 9}
\end{tabbing}

\bigskip
% --------------------
\pCase{$\varphi = \varphi_1 \ \varphi_2$ / KiApp}
$$
\qq	\infer
	{ (\un{1}) \
	  \kiJudge
		{\Gamma, \ a : \kappa_4}
		{\Sigma}
		{\varphi_1 \ \varphi_2}
		{\kappa_{12} [\varphi_2/a]}
	}
	{ (4) \
	  \kiJudge
		{\Gamma, \ a : \kappa_4}
		{\Sigma}
		{\varphi_1}
		{\kiPi{b}{\kappa_{11}}{\kappa_{12}}}
	  \quad
	  (5) \
	  \kiJudge
		{\Gamma, \ a : \kappa_4}
		{\Sigma}
		{\varphi_2}
		{\kappa_{11}}
	}
$$
\begin{tabbing}
MM \= MMx \= MMMMMMMMMMMMMMMMMMMMMMMMMMM \= MMMMM  \kill
\>	(\un{2}) 
		\> $\kiJudgeGS{\varphi_3}{\kappa_3}$
		\> \pby{assume}
\\[1ex]
\>	(\un{3}) 
		\> $\kappa_4 \sims \kappa_3$
		\> \pby{assume}
\\[1ex]
\>	(6) 	\> $\kiJudge
			{\Gamma[\varphi_3/a]}
			{\Sigma}
			{\varphi_1[\varphi_3/a]}
			{\kiPi	{b}
				{\kappa_{11}[\varphi_3/a]}
				{(\kappa_{12}[\varphi_3/a])}$
			}
		\> \pby{IH, Def. Sub. 4 2 3}
\\[1ex]
\>	(7) 	\> $\kiJudge
			{\Gamma[\varphi_3/a]}
			{\Sigma}
			{\varphi_2[\varphi_3/a]}
			{\kappa_{11}[\varphi_3/a]}$
		\> \pby{IH 5 2 3}
\\[1ex]
\>	(8) 	\> ${\Gamma[\varphi_3/a]}
			\ \vert \ 			{\Sigma}
			\ \vdash_{\trm{\tiny{T}}} \	{\varphi_1[\varphi_3/a] \ \varphi_2[\varphi_3/a]}$
\\[0.2ex]
\>		\> \hspace{12ex} $:: \ \ {(\kappa_{12}[\varphi_3/a])[\varphi_2[\varphi_3/a]/b]}$
		\> \pby{KiApp 6 7}
\\[1ex]
\>	(9) 	\> $b \notin \ifv(\varphi_3)$
		\> \pby{Uniqueness of Var}
\\[1ex]
\>	(10) 	\> $\kiJudge
			{\Gamma[\varphi_3/a]}
			{\Sigma}
			{(\varphi_1 \ \varphi_2)[\varphi_3/a]}
			{(\kappa_{12}[\varphi_2/b])[\varphi_3/a]}$
		\> \pby{Def. Sub. 8}
\end{tabbing}

\bigskip
The remaining cases are similar to the KiApp case.

\end{flushleft}








% -- Theorem: Progress
\clearpage{}
\begin{flushleft}
\textbf{Theorem: (Progress)} \\
Suppose we have a state $\heap \with t$ with store $\heap$ and term $t$. 
Let $\Sigma$ be a store typing which models $\heap$. 
If $\heap$ is well typed, and $t$ is closed and well typed, and $t$ contains no fabricated region witnesses,
then either $t$ is a value or $\heap \with t$ can transition to the next state.

\medskip
\begin{tabular}{ll}
	If 	& $\tyJudge {\emptyset} {\Sigma} {t} {\tau} {\sigma}$  \\
	\ and	& $\Sigma \models \heap$ \\
	\ and	& $\Sigma \vdash  \heap$ \\
	\ and 	& $\pnofab{t}$  
	\\[1ex]
	then	& $t$ $\in$ Value \\
	\ or 	& {for some $\heap'$, $t'$ we have} \\
		& \ ($\trEval {\heap} {t} {\heap'} {t'}$ \ and \ $\pnofab{t'}$) 
\end{tabular}	
		


\bigskip
\trb{Proof:} By induction over the derivation of \ \ $\tyJudge {\emptyset} {\Sigma} {t} {\tau} {\sigma}$

Let $(\pstep{\heap}{t}) \equiv$ 
	(for some $\heap$, $t$ we have $\trEval {\heap} {t} {\heap'} {t'}$ and $\pnofab{t'}$)

We will not formally prove $\pnofab{t'}$ in the conclusion of each case. This property
can be verified by inspecting (EvLetRegion) and noting that unevaluated applications
of witness constructors are not substituted into the body of the term.

\begin{tabbing}
MM \= MMx \= MMMMMMMMMMMMMMMMMMMMMMM \= MMMMM  \kill
	\> (IH) 
	\> Progress holds for all subterms of $t$.
	\> (assume) 
\end{tabbing}


\bigskip
% ---------------------
\pCase{$t$ is one of
	$x$, 
	\ $\Lambda(a :: \kappa). \ t'$, 
	\ $\lambda (x :: \tau). \ t'$, 
	\ (),
	\ $\underline{l}$}

\medskip
$t \in \rValue$	
\bigskip

\bigskip
% ---------------------
\pCase{$t = (t_1 \ \varphi_2)$ / TyAppT}
$$
\qq 
	\frac	
	{ (5) \ 
	  \tyJudgeES{t_1} {\tyForall {a} {\kappa_{11}} {\varphi_{12}}} {\sigma} 
	  \tspace
	  (6) \ 
	  \kiJudgeES{\varphi_2} {\kappa_2}
	  \tspace
	  (7) \ 
	  \kappa_{11} \sims \kappa_2
	}
	{ (\un{1}) \
	  \tyJudgeES
	  	{t_1 \ \varphi_2}
		{\varphi_{12}[\varphi_2/a]}
		{\sigma[\varphi_2/a]}
	} 
$$
\begin{tabbing}
MM \= MMx \= MMMMMMMMMMMMMMMMMMMMMMM \= MMMMM  \kill
\>	(\un{2..4}) 
	 	\> $\Sigma \models \heap, \ \Sigma \vdash \heap, \ \pnofab{t}$
 		\> \pby{assume} 
\\[1ex]
\>	(5) 	\> $t_1 \in \rValue \por \pstep{\heap}{t_1}$	
		\> \pby{IH 5 2..4} 	
\\
\\
\>	(6) 	\> \emph{Case:} $\ \ t_1 \in \rValue$
\\[1ex]
\>	(7) 	\> \quad $t_1 = \teLAM{a}{\kappa_{11}}{t_{12}} $			
		\> \pby{Forms of Terms 6 5} 
\\[1ex]
\>	(8) 	\> \quad $\pstep{\heap}{t}$				
		\> \pby{EvTAppAbs 6} 
\\
\\
\>	(8) 	\> \emph{Case:} $\ \ \pstep{\heap}{t_1}$
\\[1ex]
\>	(9) 	\> \quad $\pstep{\heap}{t}$				
		\> \text{(EvTApp1 7)}
\end{tabbing}


\clearpage{}
% ---------------------
\pCase{$t = t_1 \ t_2$ / TyApp}
$$
\qq \qq	\frac	
	{ (5) \ \tyJudgeES {t_1} {\tau_{11} \toa{\sigma} \tau_{12}} {\sigma_1} 
	  \qq
	  (6) \ \tyJudgeES {t_2} {\tau_2} {\sigma_2} 
	  \qq
	  (7) \ \tau_{11} \sims \tau_2
	}
	{ (\un{1}) 
	  \ \tyJudgeES {t_1 \ t_2} {\tau_{12}} {\sigma_1 \lor \sigma_2 \lor \sigma} }
$$	  
\begin{tabbing}
MM \= MMMM \= MMMMMMMMMMMMMMMMMMMMMMMMMM \= MMMMM  \kill
\>	(\un{2..4})
		\> $\Sigma \models \heap, \ \Sigma \vdash \heap, \ \pnofab{t}$
		\> \pby{assume} 
\\[1ex]
\>	(8)	\> $t_1 \in \rValue \por \pstep{\heap}{t_1}$		
		\> \pby{IH 5 2..4} 
\\[1ex]
\>	(9) 	\> $t_2 \in \rValue \por \pstep{\heap}{t_2}$		
		\> \pby{IH 6 2..4} 
\\
\\
\>	(10) 	\> $\pcase {\ \pstep{\heap}{t_1}} $
\\[1ex]
\>	(11) 	\> \quad $\pstep{\heap}{t}$					
		\> \pby{EvApp1 10} 
\\
\\
\>	(12, 13)\> $\pcase {\ t_1 \in \rValue, \ \pstep{\heap}{t_2}}  $ 
\\[1ex]
\>	(14) 	\> \quad $\pstep{\heap}{t}$				
		\> \pby{EvApp2 12 13} 
\\
\\
\>	(15, 16)\> $\pcase {\ t_1 \in \rValue, \ t_2 \in \rValue}$
\\[1ex]
\>	(17) 	\> \quad $t_1 = \teLam{x}{\tau_{11}}{t_{12}}$			
		\> \pby{Forms of Terms 15 5} 
\\[1ex]
\>	(18) 	\> \quad $\pstep{\heap}{t}$				
		\> \pby{EvAppAbs 17 16} 
\end{tabbing}
\bigskip


\bigskip
% -------------------
\pCase{$t = (\teLet{x}{t_1}{t_2})$ / TyLet} 

Similarly to TyApp case.

\bigskip
\bigskip
% -------------------
\pCase{$t = (\teLetR{r}{\ov{w_i = \delta_i}}{t_1})$ / TyLetRegion}
\begin{tabbing}
MM \= MMMM \= MMMMMMMMMMMMMMMMMMMMMMMMMM \= MMMMM  \kill
\>	(\un{1}) \> $\tyJudgeES{(\teLetR{r}{\ov{w_i = \delta_i}}{t_1})}
				{\tau_1}
				{\sigma_1}$
		\> \pby{assume}
\\[1ex]
\>	(\un{2..4}) 	
		\> $\Sigma \models \heap, \ \Sigma \vdash \heap, \ \pnofab{t}$
		\> \pby{assume} 
\\[1ex]
\>	(5) 	\> $\heap, \ \ov{\rpropOf(\Delta_i)} 
				\with \ov{\delta_i \leadsto \Delta_i}$
		\> \pby{Region Wit. Assert} 
\\[1ex]
\>	(6) 	\> $\pstep{\heap}{t}$
		\> \pby{EvLetRegion 5}
\end{tabbing}


\clearpage{}
% ------------------
\pCase{$t = (\teIf{t_1}{t_2}{t_3})$ / TyIf}
$$ 
\qq \qq	\frac	
	{ \begin{aligned}
	  	\\
	  	(5) \ \tyJudgeES {t_1} {\iBool \ \varphi} {\sigma_1}
	  \end{aligned}
	  \tspace
	  \begin{aligned}
		(6) \ \tyJudge {\emptyset} {\Sigma&} {t_2} {\tau_2} {\sigma_2}  \\
   	  	(7) \ \tyJudge {\emptyset} {\Sigma&} {t_3} {\tau_3} {\sigma_3}
	  \end{aligned}
	  \tspace
	  \begin{aligned}
		\\
	  	(8) \ \tau_2 \sims \tau_3
	  \end{aligned}
	}
	{ (\un{1}) \ 
	  \tyJudgeES
	  	{(\teIf{t_1}{t_2}{t_3})} 
		{\tau_2} 
		{(\sigma_1 \lor \sigma_2 \lor \sigma_3 \lor \iRead \ \varphi)}
	}
$$
\begin{tabbing}
MM \= MMMM \= MMMMMMMMMMMMMMMMMMM \= MMMMM  \kill
\>	(\un{2..4}) 	
		\> $\Sigma \models \heap, \ \Sigma \vdash \heap, \ \pnofab{t}$
		\> \pby{assume} 
\\[1ex]
\>	(9) 	\> $t_1 \in \rValue \por \pstep{\heap}{t_1}$	
		\> \pby{IH 5 2..4} 
\\
\\[1ex]
\>	(10) 	\> $\pcase {t_1 \in \rValue}$
\\[1ex]
\>	(11) 	\> \quad $t_1 = \un{l}$						
		\> \pby{Forms of Terms 10 5} 
\\[1ex]
\>	(12) 	\> \quad $\tyJudgeES{\un{l}} {\iBool \ \varphi} {\bot}$
		\> \pby{5 11}
\\[1ex]
\>	(13) 	\> \quad $\un{l} : \iBool \ r \in \Sigma$			
		\> \pby{TyLoc 12} 
\\[1ex]
\>	(14) 	\> \quad $l \mapstoa{\rho} \trm{V} \in {\heap} \
				\ \trm{for some} \ \ \trm{V} \in \{ \trm{T}, {\trm{F}} \}$
		\> \pby{Def. Store Model 2 13} 
\\[1ex]
\>	(15) 	\> \quad $\pstep {H}{t}$						
		\> \pby{EvIfThen, EvIfElse 14} 
\\
\\[1ex]
\>	(16)	\> $\pcase {\pstep{H}{t_1}}$ 
\\[1ex]
\>	(17) 	\> \quad $\pstep {H}{t}$						
		\> \pby{EvIf 16}
\end{tabbing}


\bigskip
\bigskip
% ------------------
\pCase{$t = (\iTrue \ \varphi_1)$ / TyTrue}
$$
\qq \qq	\frac	
	{ (5) \ \kiJudge{\emptyset} {\Sigma} {\varphi_1} {\%} }
	{ (\un{1}) 
	  \ \tyJudge{\emptyset} {\Sigma} {\iTrue \ \varphi_1} {\iBool \ \varphi_1} {\bot}}
$$
\begin{tabbing}
MM \= MMMM \= MMMMMMMMMMMMMMMMMMM \= MMMMM  \kill
\>	(\un{2..4}) 	
		\> $\Sigma \models \heap, \ \Sigma \vdash \heap, \ \pnofab{t}$
		\> \pby{assume} 
\\[1ex]
\>	(6) 	\> $\varphi_1 = \un{\rho}$	
		\> \pby{Forms of Types, $t$ is closed, 1} 
\\[1ex]
\>	(7) 	\> $\kiJudge{\emptyset} {\Sigma} {\un{\rho}} {\%}$
		\> \pby{5 6} 
\\[1ex]
\>	(8) 	\> $\un{\rho} \in \Sigma$
		\> \pby{KiHandle 7} 
\\[1ex]
\>	(9) 	\> $\rho \in \heap$	
		\> \pby{Def. Store Model 2 8} 
\\[1ex]
\>	(10)	\> $\pstep{H}{t}$	
		\> \pby{EvTrue 9 6}
\end{tabbing}

\bigskip
\bigskip
% ------------------
\pCase{$t = (\iFalse \ \varphi_1)$ / TyFalse} 

Similarly to TyTrue case.
\bigskip

\clearpage{}


% ------------------
\pCase{$t = (\iupdate \ \delta \ t_1 \ t_2)$ / TyUpdate}
$$
\qq \qq	
	\frac	
	{ \begin{aligned}
	  \\
	  (5) \ \kiJudgeES{\delta} {\iMutable \ \varphi_1}
	  \end{aligned}
	  \tspace
	  \begin{aligned}
	  (6) \ \tyJudgeES{t_1} {\iBool \ \varphi_1} {\sigma_1} \\
	  (7) \ \tyJudgeES{t_2} {\iBool \ \varphi_2} {\sigma_2} \\
	  \end{aligned}
	}
	{ (\un{1}) 
	   \ \tyJudgeES
		{(\iupdate \ \delta \ t_1 \ t_2)}
		{()}
		{(\sigma_1 \lor \sigma_2 \lor \iRead \ \varphi_2 \lor \iWrite \ \varphi_1)} 
	}
$$
\begin{tabbing}
MM \= MMMM \= MMMMMMMMMMMMMMMMMMMMM \= MMMMM  \kill
\>	(\un{2..4}) 
		\> $\Sigma \models \heap, \ \Sigma \vdash \heap, \ \pnofab{t}$
		\> \pby{assume} 
\\
\\
\>	(8, 9) 	\> $\pcase{t_1 \in \rValue, \ \ t_2 \in \rValue}$
\\[1ex]
\>	(10)	\> \quad $\delta = \un{\rmutable \ \rho_1}$			
		\> \pby{Forms of Types 4 5} 
\\[1ex]
\>	(11)	\> \quad $\un{\rmutable \ \rho_1} \in \Sigma$		
		\> \pby{KiMutable 5 10} 
\\[1ex]
\>	(12)	\> \quad $\rmutable \ \rho_1 \in \heap$				
		\> \pby{Def. Store Model 2 11} 
\\[1ex]
\>	(13, 14)\> \quad $t_1 = \un{l_1}, \ t_2 = \un{l_2}$			
		\> \pby{Forms of Terms 8 9 6 7} 
\\[1ex]
\>	(15) 	\> \quad $l_1 \mapstoa{\rho_1'} \trm{V}_1' \in {\heap} \
				\trm{for some} \ \trm{V}_1' \in \{ \trm{T}, \trm{F} \}$
		\> \pby{as per TyIf case 2 6 13} 
\\[1ex]
\>	(16) 	\> \quad $l_2 \mapstoa{\rho_2} \trm{V}_2 \in {\heap} \
				\trm{for some} \ \trm{V}_2 \in \{ \trm{T}, \trm{F} \}$
		\> \pby{as per TyIf case 2 7 14}  
\\[1ex]
\>	(17) 	\> \quad $\un{\rho_1} = \varphi_1$
		\> \pby{KiMutable 5 10}
\\[1ex]
\>	(18) 	\> \quad $\tyJudgeES{\un{l_1}}{\iBool \ \un{\rho_1}}{\bot}$	
		\> \pby{6 13 17}
\\[1ex]
\>	(19) 	\> \quad $\un{l_1} : \iBool \ \un{\rho_1} \in \Sigma$	
		\> \pby{Tyloc 18}
\\[1ex]
\>	(20) 	\> \quad $l_1 \mapstoa{\rho_1} \trm{V}_1 \in {\heap} \
				\trm{for some} \ \trm{V}_1 \in \{ \trm{T}, \trm{F} \}$
		\> \pby{Def. Store Model 2 19} 
\\[1ex]
\>	(21) 	\> \quad $\un{\rho_1} = \un{\rho_1'}$
		\> \pby{Def. Store 15 20}
\\[1ex]		
\>	(22) 	\> \quad $\pstep{\heap}{t}$					
		\> \pby{EvUpdate3, 12 20 16 10 13 14}
\end{tabbing}

\smallskip
Other cases via EvUpdate1 or EvUpdate2 as per TyApp case.

\bigskip
% ---------------------
\pCase{$t = (\isuspend \ \delta \ t_1 \ t_2)$ / TySuspend}
$$
\qq \qq \frac	
	{ \begin{aligned}
	  (5) \ \tau_{11} \sims \tau_2 \\
	  (6) \ \kiJudgeES {\delta} { \iPure \ \sigma }
	  \end{aligned}
	  \tspace
	  \begin{aligned}
	  (7) \ \tyJudge {\emptyset} {\Sigma&} {t_1} { \tau_{11} \toa{\sigma} \tau_{12} } {\sigma_1} \\
	  (8) \ \tyJudge {\emptyset} {\Sigma&} {t_2} { \tau_2 } {\sigma_2} \\
	  \end{aligned}
	}
	{ (\un{1}) \ \tyJudge{\emptyset}
			{\Sigma}
			{(\isuspend \ \delta \ t_1 \ t_2)}
			{\tau_{12}}
			{\sigma_1 \lor \sigma_2} 
	}
$$
\begin{tabbing}
MM \= MMMx \= MMMMMMMMMMMMMMMMMMMMMMM \= MMMMM  \kill
\>	$(\un{2..4})$
		\> $\Sigma \models \heap, \ \Sigma \vdash \heap, \ \pnofab{t}$
		\> \pby{assume} 
\\[1ex]
\>	(9) 	\> \quad $\delta \in \{ 
				\iMkPurify \ \un{\rho} \ \delta_2, \
				\iMkPureJoin \ \sigma_3 \ \sigma_4 \ \delta_3 \ \delta_4,$ 
\\[1ex]
\>		\> $\qq \qq	\iMkPure \ \bot, \ 
				\un{\rpure \ \sigma} \}	$			
		\> \pby{Forms of Types 6} 
\\
\\
\>	(10) 	\> $\pcase{\delta_1 \in \{ 
				\iMkPurify \ \un{\rho} \ \delta_2, \
				\iMkPureJoin \ \sigma_3 \ \sigma_4 \ \delta_3 \ \delta_4, \ \iMkPure \ \bot \} }$
\\[1ex]
\>	(11) 	\> \quad $\heap \ ; \ \delta_1 \leadsto \delta_1'$		
		\> \pby{Progress of Purity 6 3 10} 	
\\[1ex]
\>	(12) 	\> \quad $\pstep{\heap}{t}$					
		\> \pby{EvSuspend1 11}		
\\
\\
\>	(13..15) 	
		\> $\pcase{\delta_1 = \un{\rpure \ \sigma}, \ t_1 \in \rValue, \ t_2 \in \rValue}$
\\[1ex]
\>	(16) 	\> \quad $t_1 = \teLam{x}{\tau}{t_3}$				
		\> \pby{Forms of Terms 6 16} 
\\[1ex]
\>	(17) 	\> \quad $\pstep{\heap}{t_1}$				
		\> \pby{EvSuspend4 13 16 15} 
\end{tabbing}

\smallskip
Other cases via EvSuspend2 or EvSuspend3 as per TyApp case.

\bigskip

\end{flushleft}

% -- Theorem: Preservation
\clearpage{}
\begin{flushleft}
\textbf{Theorem: (Preservation)} \\
Suppose we have a state $\heap \with t$ with store $\heap$ and term $t$.
	Let $\Sigma$ be a store typing which models $\heap$. 
	If $\heap$ and $t$ are well typed, 
		and $\heap \with t$ can transition to a new state $\heap' \with t'$
		then for some $\Sigma'$ which models $\heap'$, 
		$\heap'$ is well typed, 
		$t'$ has a similar type to $t$,
		and the effect $\sigma'$ of $t'$ is no greater than the effect $\sigma$ 
		of $t$.

\medskip
\begin{tabular}{ll}
	If	& $\tyJudge{\Gamma}{\Sigma}{t}{\tau}{\sigma}$ \\
	\ and	& $\trEval{\heap}{t}{\heap'}{t'}$ \\
	\ and	& $\Sigma \vdash \heap$ \ and \ $\Sigma \models \heap$ 
	\\[1ex]
	\multicolumn{2}{l}{then for some $\Sigma'$, $\tau'$, $\sigma'$ we have} \\
		& $\tyJudge{\Gamma}{\Sigma'}{t'}{\tau'}{\sigma'}$ \\
	\ and	& $\Sigma' \supseteq \Sigma$ \ \ and \ $\Sigma' \models \heap'$ \ and $\Sigma' \vdash  \heap'$  \\
	\ and	& $\tau' \sim_{\Sigma'} \tau$ \  and \ $\sigma' \sqsubseteq_{\Sigma'} \sigma$ 
\end{tabular}

\bigskip
\trb{Proof:} By induction over the derivation of $\tyJudge{\Gamma}{\Sigma}{t}{\tau}{\sigma}$.
\begin{tabbing}
MM \= MMMM \= MMMMMMMMMMMMMMMMMMMMM \= MMMMM  \kill
\>	(IH) 
	\> Progress holds for all subterms of $t$.
	\> (assume)
\end{tabbing}

\bigskip
\bigskip
% ---------------------
\pCase{$t$ is one of
	$x$, 
	\ $\Lambda(a :: \kappa). \ t'$, 
	\ $\lambda (x :: \tau). \ t'$, 
	\ $()$, 
	\ $\underline{l}$}

Can't happen. There is no transition rule for $H \with t$


\bigskip
\bigskip
% ---------------------
\pCase{$t = t_1 \ \varphi_2$ / TyAppT / EvApp1}
$$ 
\qq 
	\frac
	{ (5) \ \tyJudge {\Gamma} {\Sigma} {t_1} {\tyForall{a}{\kappa_{11}}{\varphi_{12}}} {\sigma}
	  \qq
	  (6) \ \kiJudge {\Gamma} {\Sigma} {\varphi_2} {\kappa_{2}} 
	  \qq
	  (7) \ \kappa_{11} \sims \kappa_2
	}
	{ (\un{1}) \ \tyJudge {\Gamma} {\Sigma} {t_1 \ \varphi_2} 
				{\varphi_{12}[\varphi_2/a]} 
				{\sigma[\varphi_2/a]}
	}
$$
$$
\qq \qq
	\frac
	{ (8) \ \trEval {\heap} {t_1} {\heap'} {t_1'} }
	{ (\un{2}) \ \trEval {\heap} {t_1 \ \varphi_2} {\heap'} {t_1' \ \varphi_2} }
$$
\begin{tabbing}
MM \= MMMM \= MMMMMMMMMMMMMMMMMMMMM \= MMMMM  \kill
\>	(\un{3}, \un{4}) 	
		\> $\Sigma \models \heap, \ \Sigma \vdash \heap$
		\> \pby{assume}
\\[1ex]
\>	(9..14) \> $\tyJudge{\Gamma}{\Sigma'}{t_1'}
				{ \tyForall{a} {\kappa_{11}'} {\varphi_{12}'}} {\sigma'},$ 
\\[0.2ex]
\>		\> \qq $\Sigma' \supseteq \Sigma, \ \
				\sigma' \sqsubseteq_{\Sigma'} \sigma,$
\\[0.2ex]
\>		\> \qq	$\tyForall{a}{\kappa_{11}'}{\varphi_{12}'}
					\sim_{\Sigma'} \tyForall{a}{\kappa_{11}}{\varphi_{12}}$ 
\\[0.2ex]
\>		\> \qq	$\Sigma' \vdash \heap', \ \
				\Sigma' \models \heap'$
		\> \pby {IH 5 8 3 4}
\\[1ex]
\>	(15)	\> $\kiJudge {\Gamma} {\Sigma'} {\varphi_2} {\kappa_2}$	
		\> \pby{Weak. Store Typing 6 10}
\\[1ex]
\>	(16)	\> $\kappa_{11}' \sim_{\Sigma'} \kappa_{11}$		
		\> \pby{SimAll 12}
\\[1ex]
\>	(17)	\> $\kappa_{11}  \sim_{\Sigma'} \kappa_{2}$		
		\> \pby{Weak. $(\sims)$ 7 10}
\\[1ex]
\>	(18)	\> $\kappa_{11}' \sim_{\Sigma'} \kappa_{2}$		
		\> \pby{16 17}
\\[1ex]
\>	(19)	\> $\tyJudge {\Gamma} {\Sigma'} {t_1' \ \varphi_2} 
				{\varphi_{12}'[\varphi_2/a]} 
				{\sigma'[\varphi_2/a]}$
		\> \pby {TyAppT 9 13 18}
\end{tabbing}
\bigskip

\clearpage{}
% ---------------------
\pCase{$t = t_1 \ \varphi_2$ / TyAppT / EvAppAbs}
$$
\qq \qq
	\infer
	{ (\un{1}) \ \tyJudge{\Gamma} {\Sigma} 
			{ (\teLAM{a}{\kappa_{11}}{t_{12}}) \ \varphi_2}
			{ \varphi_{12}[\varphi_2/a] }
			{ \sigma[\varphi_2/a]}
	}
	{ 	\begin{aligned}
		\infer 
		{ (5) \ \tyJudge{\Gamma}{\Sigma}
			{ \teLAM{a}{\kappa_{11}}{t_{12}} }
			{ \tyForall{a}{\kappa_{11}}{\varphi_{12}}}
			{ \sigma}
		}
		{ (8) \ \tyJudge{\Gamma, \ a : \kappa_{11}}{\Sigma}
				{t_{12}}{\varphi_{12}}{\sigma} 
		}
		\end{aligned}
	  \qq
	  	\begin{aligned}
		  (7) 	\ \kappa_{11} \sims \kappa_2
		  \\
		  (6)	\ \kiJudge{\Gamma}{\Sigma}
			{ \varphi_2}
			{ \kappa_2 }
		\end{aligned}
	}
$$
$$ (\un{2}) \
	 \heap \with (\Lambda (a :: \kappa_{11}). \ t_{12}) \ \varphi_2 \eto 
	 \heap \with t_{12}[\varphi_2/a] 
$$
\begin{tabbing}
MM \= MMMM \= MMMMMMMMMMMMMMMMMMMMMM \= MMMMM  \kill
\>	$(\un{3}, \un{4})$	
		\> $\Sigma \models \heap, \ \Sigma \vdash \heap$		
		\> \pby{assume}
\\[1ex] 
\>	(9) 	\> $\tyJudge
			{\Gamma[\varphi_2/a]}
			{\Sigma}
			{t_{12} [\varphi_2/a]}
			{\varphi_{12} [\varphi_2/a] }
			{\sigma [\varphi_2/a] }$				
		\> \pby{Sub. Type/Value 8 6 7}
\\[1ex]
\>	(10) 	\> $a \notin \Gamma$					
		\> \pby{No Var Capture 1}
\\[1ex]
\>	(11) 	\> $\Gamma[\varphi_2/a] \equiv \Gamma$			
		\> \pby{Def. Sub. 10}
\end{tabbing}

\bigskip
\bigskip
\bigskip
% ---------------------
\pCase{$t = t_1 \ t_2$ / TyApp / EvApp1}
$$
\qq \qq
	\frac	
	{ (5) \ \tyJudge{\Gamma}{\Sigma}
		{t_1}
		{\tau_{11} \funa{\sigma} \tau_{12}}
		{\sigma_1}
	  \qq
	  (6) \ \tyJudge{\Gamma}{\Sigma}
		{t_2}
		{\tau_2}
		{\sigma_2}
	  \qq
	  (7) \	\tau_{11} \sim_\Sigma \tau_2{}
	}
	{ (\un{1})
		\ \tyJudge{\Gamma}{\Sigma}
		{t_1 \ t_2}
		{\tau_{12}}
		{\sigma_1 \lor \sigma_2 \lor \sigma}
	}
$$
$$
\qq \qq
	\frac
	{ (8) \ \trEval	{\heap}{t_1}
			{\heap'}{t_1'}
	}
	{ (\un{2})
		 \ \trEval	
		 	{\heap}{t_1 \ t_2}
			{\heap'}{t_1' \ t_2}
	}
$$
\begin{tabbing}
MM \= MMMM \= MMMMMMMMMMMMMMMMMMMMMM \= MMMMM  \kill
\>	(\un{3}, \un{4})
		\> $\Sigma \models \heap, \ \Sigma \vdash \heap$
		\> \pby{assume}
\\[1ex]
\>	(9..14) \> $\tyJudge{\Gamma}{\Sigma'}
				{t_1'}
				{\tau_{11}' \funa{\sigma'} \tau_{12}'}
				{\sigma_1'}$
\\[0.2ex]
\>		\> \qq $\Sigma' \supseteq \Sigma, \ \Sigma' \vdash \heap', \ \ \Sigma' \models \heap'$
\\[0.2ex]
\>		\> \qq $(\tau_{11}' \funa{\sigma'} \tau_{12}')  \sim_{\Sigma'}
			      (\tau_{11} \funa{\sigma} \tau_{12}) $
\\[0.2ex]
\>		\> \qq $\sigma_1'  \sim_{\Sigma'} \sigma_1$ 		
		\> \pby{IH 5 8 3 4}
\\[1ex]
\>	(15) 	\> $\tyJudge{\Gamma}{\Sigma'}{t_2}{\tau_2}{\sigma_2}$
		\> \pby{Weak. Store Typing 6 10}
\\[1ex]
\>	(16) 	\> $\tau_{11}' \sim_{\Sigma'} \tau_{11}$
		\> \pby{SimApp 13}
\\[1ex]
\>	(17) 	\> $\tau_{11} \sim_{\Sigma'} \tau_{2}$
		\> \pby{Weak. $(\sims)$ 7 10}
\\[1ex]
\>	(18) 	\> $\tau_{11}' \sim_{\Sigma'} \tau_{2}$
		\> \pby{16 17}
\\[1ex] 
\>	(19) 	\> $\tyJudge{\Gamma}{\Sigma'}
				{t_1' \ t_2}
				{\tau_{12}'}
				{\sigma_1' \lor \sigma_2 \lor \sigma'}$
		\> \pby{TyApp 9 15 18}
\\[1ex]
\>	(20) 	\> $\sigma' \sim_{\Sigma'} \sigma$		
		\> \pby{SimApp 13}
\\[1ex]
\>	(21) 	\> $\sigma_1' \lor \sigma_2 \lor \sigma'
				\sqsubseteq_{\Sigma'}
				\sigma_1 \lor \sigma_2 \lor \sigma$
		\> \pby{14 20}
\end{tabbing}



\clearpage{}

% ---------------------
\pCase{$t = t_1 \ t_2$ / TyApp / EvApp2}
		
Similarly to TyApp/EvApp1 case.

\bigskip
\bigskip
% ---------------------
\pCase{$t = t_1 \ t_2$ / TyApp / EvAppAbs}
$$
 	\infer	
	{ (\un{1}) \ 
	   \tyJudge
		{\Gamma}{\Sigma}
		{(\teLam{x}{\tau_{11}}{t_{12}}) \ v^\circ}
		{\tau_{12}}
		{\sigma_1 \lor \sigma_2 \lor \sigma}
	}
	{ 	\infer 
	  	{ (5) \ 
		  \tyJudge
		  	{\Gamma}{\Sigma}
			{\teLam{x}{\tau_{11}}{t_{12}}}
			{\tau_{11} \funa{\sigma} \tau_{12}}
			{\sigma_1}
		}
		{ 	(8) \ 
			\tyJudge
				{\Gamma, \ x : \tau_{11}}{\Sigma}
				{t_{12}}
				{\tau_{12}}
				{\sigma}
		}
		& \quad
		(6) \
		\tyJudge
			{\Gamma}{\Sigma}
			{v^\circ}
			{\tau_2}
			{\bot}
		& \quad
		(7) \
		\tau_{11} \sim_\Sigma \tau_2
	}
$$	
$$
\qq \qq
	(\un{2}) \
	\trEval	{\heap}
		{(\teLam{x}{\tau_{11}}{t_{12}}) \ v^\circ}
		{\heap'}
		{t_{12}[v^\circ/x]}
$$
\begin{tabbing}
MM \= MMMM \= MMMMMMMMMMMMMMMMMMMM \= MMMMM  \kill
\>	$(\un{3}, \un{4}) $
		\> $\Sigma \models \heap, \ \Sigma \vdash \heap$
		\> \pby{assume} 
\\[1ex]
\>	(9..11) \> $\tyJudge
			{\Gamma}{\Sigma}
			{t_{12}[v^\circ/x]}
			{\tau_{12}'}
			{\sigma'}$ 
\\[0.2ex]
\>		\> \qq $\tau_{12} \sim_\Sigma \tau_{12}'$
\\[0.2ex]
\>		\> \qq $\sigma \  \sim_\Sigma \sigma'$
		\> \pby{Sub. Value/Value 8 6 7} 
\\[1ex]
\>	(12) 	\> $\sigma' \tle_\Sigma \sigma_1 \lor \sigma_2 \lor \sigma$	
		\> \pby{11}
\end{tabbing}


\bigskip
\bigskip
% ---------------------
\pCase{$t = \teLet{x}{t_1}{t_2}$ / TyLet / EvLet1}
$$
\qq \qq	\infer
	{ (\un{1}) \
	  \tyJudge
	  	{\Gamma}{\Sigma}
		{\teLet{x}{t_1}{t_2}}
		{\tau_2}
		{\sigma_1 \lor \sigma_2}
	}
	{ (5) \
	  \tyJudge
	  	{\Gamma}{\Sigma}
		{t_1}
		{\tau_1}
		{\sigma_1}
	  & \quad
	  (6) \
	  \tyJudge
	  	{\Gamma, \ x : \tau_3}{\Sigma}
		{t_2}
		{\tau_2}
		{\sigma_2}
	  & \quad
	  (7) \
	 	\tau_1 \sim_\Sigma \tau_3 
	}
$$
$$
\qq \qq	\infer
	{ (\un{2}) \
	  \trEval
	  	{\heap} {\teLet{x}{t_1}{t_2}}
		{\heap'}{\teLet{x}{t_1'}{t_2}}
	}
	{ (8) \
	  \trEval
		{\heap} {t_1}
		{\heap'}{t_1'}
	}
$$
\begin{tabbing}
MM \= MMMM \= MMMMMMMMMMMMMMMMMMMM \= MMMMM  \kill
\>	(\un{3}, \un{4}) 
		\> $\Sigma \models \heap, \ \Sigma \vdash \heap$
		\> \pby{assume} 
\\[1ex]
\>	(9..14) \> $\tyJudge
			{\Gamma}{\Sigma'}
			{t_1}
			{\tau_1'}
			{\sigma_1'}$
\\[0.2ex]
\>		\> \qq 	$\Sigma' \supseteq \Sigma, \ 
			\Sigma' \vdash \heap', \ 
			\Sigma' \models \heap'$ 
\\[0.2ex]
\>		\> \qq 	$\tau_1' \sim_{\Sigma'} \tau_1, \ 
			\sigma_1' \tle_{\Sigma'} \sigma_1$			
		\> \pby{IH 5 2 3 4}
\\[1ex]
\>	(15) 	\> $\tau_1 \sim_{\Sigma'} \tau_3$		
		\> \pby{Weak. $(\sim_\Sigma)$ 7 10}
\\[1ex]
\>	(16) 	\> $\tau_1' \sim_{\Sigma'} \tau_3$
		\> \pby{13 15}
\\[1ex]
\>	(17) 	\> $\tyJudge
		  	{\Gamma, \ x : \tau_3}{\Sigma'}
			{t_2}
			{\tau_2}
			{\sigma_2}$						
		\> \pby{Weak. Store Typing 6 10}
\\[1ex]
\>	(18) 	\> $\tyJudge
			{\Gamma}{\Sigma'}
			{\teLet{x}{t_1'}{t_2}}
			{\tau_2}
			{\sigma_1' \lor \sigma_2}$	
		\> \pby{TyLet 9 17 16} 
\\[1ex]
\>	(20) 	\> $\sigma_1' \lor \sigma_2 \ 
			\tle_{\Sigma'} \ \sigma_1 \lor \sigma_2	$
		\> \pby{14}
\end{tabbing}


\bigskip
\bigskip
% ---------------------
\pCase{$t = \teLet{x}{t_1}{t_2}$ / TyLet / EvLet}

Similarly to TyApp / EvAppAbs case.


\clearpage{}
% ---------------------
\pCase{$t = \teLetR{r}{\ov{w_i = \delta_i}}{t_1}$ / TyLetRegion / EvLetRegion}
$$
\qq	\infer
	{ (\un{1}) \
	  \tyJudge
	  	{\Gamma}{\Sigma}
	  	{(\teLetR{r}{\ov{w_i = \delta_i}}{t_1})}
		{\tau}
		{\sigma}
	}
	{ \begin{aligned}
		  \\
		  (5) \ 
		  \tyJudge
			{\Gamma, \ r : \%, \ \ov{w_i : \kappa_i}} 
			{\Sigma} 
			{t_1} 
			{\tau} 
			{\sigma}
	  \end{aligned}
	  \quad
	  \begin{aligned}
		(6) \ 
		\ksJudge{\Gamma &}
			{\Sigma}
			{\kappa_i}
			{\Diamond}
		\\
		(7) \ 
		\kiJudge
		  	{\Gamma &}
			{\Sigma}
			{\delta_i}
			{\kappa_i}
	  \end{aligned}
	  & \quad
	  \begin{array}{cc}
		  \\
	  	  (8) \ \ov{\delta_i} \ \trm{well formed} \\
	  \end{array}
	}
$$
$$
\qq\qq
	\infer
	{ (\un{2}) \ 
	  \trEval
		{\heap}
		{\teLetR{r}{\ov{w_i = \delta_i}}{t_1}}
		{\heap, \ \un{\rho}, 
			\ r \sim \rho, 
			\ \ov{\Delta_i} \ 
		}
		{\ t_1[\ov{\Delta_i/w_i}][\un{\rho}/r]}
	}
	{ (9) \ 
	  \heap, \ \ov{\rpropOf(\Delta_i)} \with \ov{\delta_i} \leadsto \ov{\Delta_i}
	  \qq 
	  (10) \ 
	  \un{\rho} \ \textrm{fresh}
	}
$$

\begin{tabbing}
MM \= MMM \= MMMMMMMMMMMMMMMMMMMM \= MMMMM  \kill
\>	(\un{3}, \un{4}) 
		\> $\Sigma \models \heap, \ \Sigma \vdash \heap	$		
		\> \pby{assume} 
\\[1ex]
\>	(11) 	\> ${\Gamma[\ov{\Delta_i/w_i}][\un{\rho}/r]} \ | \ {\Sigma} 
			  \vdash {t_1	[\ov{\Delta_i/w_i}][\un{\rho}/r]} $
\\[0.2ex]
\>		\> $\ \hspace{4em} :: {\tau_1 [\ov{\Delta_i/w_i}][\un{\rho}/r]} 
			  \ ;  \	    {\sigma [\ov{\Delta_i/w_i}][\un{\rho}/r]}$	
		\> \pby{Sub. Type/Value 5 7}
\\[1ex] 
\>	(12) 	\> $\Gamma[\ov{\Delta_i/w_i}][\un{\rho}/r] \equiv \Gamma$
		\> \pby{No Var Capture 1}
\\[1ex]
\>	(13) 	\> $\sigma[\ov{\Delta_i/w_i}][\un{\rho}/r] \equiv \sigma[\un{\rho}/r]$
		\> \pby{No Wit. Vars in Effects 7 6}
\\[1ex]
\>	(14) 	\> $\tau  [\ov{\Delta_i/w_i}][\un{\rho}/r] \equiv \tau[\un{\rho}/r]$
		\> \pby{No Wit. Vars in Value Types 7 6}
\\[1ex]
\>	(15) 	\> $\Sigma' = \Sigma,  
				\ \un{\rho}, 
				\ r \sim \rho, 
				\ \ov{\Delta_i} $				
			\> \pby{let}
\\[1ex]
\>	(16) 	\> $\tyJudge
			{\Gamma}{\Sigma'}
			{t_1	[\ov{\Delta_i/w_i}][\un{\rho}/r]}
			{\tau[\un{\rho}/r]}
			{\sigma[\un{\rho}/r]}$
		\> \pby{Weak. Store Typing 11..15}
\\[1ex]
\>	(17) 	\> $\tau   \sim_{\Sigma'} \tau[\un{\rho}/r]$		
		\> \pby{SimHandle 15}
\\[1ex]
\>	(18) 	\> $\sigma \tle_{\Sigma'} \sigma[\un{\rho}/r]$			
		\> \pby{SubReflSim, SimHandle 15}
\end{tabbing}


\bigskip
\bigskip
% ---------------------
\pCase{$t = \teIf{t_1}{t_2}{t_3}$ / TyIf / EvIf}

Similarly to TyApp / EvApp1 case.


\bigskip
\bigskip
% ---------------------
\pCase{$t = \teIf{t_1}{t_2}{t_3}$ / TyIf / EvIfThen}

$$
\qq	\infer	
	{ (\un{1}) \
	  \tyJudgeGS
		{\teIf{\un{l}}{t_2}{t_3}}
		{\tau_2}
		{\sigma_1 \lor \sigma_2 \lor \sigma_3 \lor \iRead \ \rho}
	}
	{ 
	  \begin{aligned}
	  \\
	  (5) \ 
	  \tyJudgeGS
		{\un{l}}
		{\iBool \ \rho}
		{\sigma_1}
	  \end{aligned}
	  \qq
	  \begin{aligned}
		  (6) \ 
		  \tyJudgeGS
			{t_2}
			{\tau_2}
			{\sigma_2} 
		 \\
		 (7) \ 
		  \tyJudgeGS
		  	{t_3}
			{\tau_3}
			{\sigma_3}
	  \end{aligned}
	  \qq
	  \begin{aligned}
		\\
	  	\tau_2 \sim_\Sigma \tau_3
	  \end{aligned}
	}
$$
$$
\qq \qq	(\un{2}) \ 
	\trEval
		{\heap, \ l \mapstoa{\rho} \trm{T}}
		{\teIf{\un{l}}{t_2}{t_3}}
		{\heap, \ l \mapstoa{\rho} \trm{T}}
		{t_2}
$$
\begin{tabbing}
MM \= MMM \= MMMMMMMMMMMMMMMMMMMM \= MMMMM  \kill
\>	(\un{3}, \un{4}) 
		\> $\Sigma \models \heap, \ \Sigma \vdash \heap$
		\> \pby{assume} 
\\[1ex]
\>	(8) 	\> $\tyJudgeGS
			{t_2}
			{\tau_2}
			{\sigma_2}$						
		\> \pby{Repeat 6}
\\[1ex]
\>	(9) 	\> $\sigma_2 \tle_\Sigma \sigma_1 \lor \sigma_2 \lor \sigma_3 \lor \iRead \ \rho$
		\> \pby{SubJoin2}
\end{tabbing}

\clearpage{}
% ---------------------
\pCase{$t = \teIf{t_1}{t_2}{t_3}$ / TyIf / EvIfElse}

Similarly to TyIf / EvIfThen case.


\bigskip

% ---------------------
\pCase{$t = \iTrue \ \varphi$ / TyTrue / EvTrue}

$$
	\infer
	{ (\un{1}) \
	  \tyJudgeGS
	  	{ \iTrue \ \un{\rho} }
		{ \iBool \ \un{\rho} }
		{ \bot }
	}
	{	\infer
		{ (5) \ 
		  \kiJudgeGS
			{ \un{\rho} }
			{ \% }
		}
		{ (6) \ 
		  \un{\rho} \in \Sigma }
	}
$$

$$
	\infer
	{	(\un{2}) \ 
		\trEval	{\heap, \ \un{\rho}}
			{\iTrue \ \un{\rho}}
			{\heap, \ \un{\rho}, \ \un{l} \mapstoa{\rho} \trm{T}}
			{\un{l}}
	}
	{ 	\un{l} \ \trm{fresh} 
	}
$$
\begin{tabbing}
MM \= MMMM \= MMMMMMMMMMMMMMMMMMMMMMMM \= MMMMM  \kill
\>	(\un{3}, \un{4}) 
		\> $\Sigma \models \heap, \ \Sigma \vdash \heap$
		\> \pby{assume} 
\\[1ex]
\>	(7) 	\> $\Sigma' = \Sigma, \ l : \iBool \ \un{\rho}$
		\> \pby{let}
\\[1ex]
\>	(8) 	\> $\tyJudge
			{\Gamma}
			{\Sigma'}
			{\un{l}}
			{\iBool \ \un{\rho}}
			{\bot}$
		\> \pby{TyLoc 7}
\end{tabbing}


\bigskip
% ---------------------
\pCase{$t = \iFalse \ \varphi$ / TyFalse / EvFalse}

Similarly to TyFalse / EvFalse case.

\bigskip
% ---------------------
\pCase{$t = \iupdate \ \delta \ t_1 \ t_2$ / TyUpdate / \{EvUpdate1, EvUpdate2\} }

Similarly to TyApp / EvApp1 case.

\bigskip
% ---------------------
\pCase{$t = \iupdate \ \delta \ t_1 \ t_2$ / TyUpdate / EvUpdate3}

$$ 
	\begin{aligned}
	  (\un{2}) \qq
	  	 & \heap	, \ \rmutable \ \rho_1
			, \ l_1 \mapstoa{\rho_1} V_1
			, \ l_2 \mapstoa{\rho_2} V_2
			\with \iupdate \
				\underline{\textrm{mutable $\rho_1$}} \ 
				\un{l_1} \ 
				\un{l_2} \\
		  	\eto  
		  & \heap , \ \rmutable \ \rho_1
			, \ l_1 \mapstoa{\rho_1} V_2
			, \ l_2 \mapstoa{\rho_2} V_2
			\with ()
	  \end{aligned}
$$
\begin{tabbing}
MM \= MMMM \= MMMMMMMMMMMMMMMMMMMMMMMM \= MMMMM  \kill
\>	(\un{1}) 
		\> $\Gamma \ \vert \ \Sigma \judge \iupdate \ \un{\rmutable \ \rho_1} \ \un{l_1} \ \un{l_2} :: ()$
\\[0.2ex]
\>		\> \hspace{6ex} $ ; {\ \sigma_1 \lor \sigma_2 \lor \iRead \ \rho_1 \ \lor \iRead \ \rho_2}$
		\> \pby{assume}
\\[1ex]
\>	(\un{3}, \un{4}) 
		\> $\Sigma \models \heap, \ \Sigma \vdash \heap$
		\> \pby{assume} 
\\[1ex]
\>	(5) 	\> $\tyJudgeGS{()}{()}{\bot}$
		\> \pby{TyUnit}
\end{tabbing}


\bigskip
% ---------------------
\pCase{$t = \isuspend \ \delta \ t_1 \ t_2$ / TySuspend / EvSuspend1}

Immediate


\bigskip
% ---------------------
\pCase{$t = \isuspend \ \delta \ t_1 \ t_2$ / TySuspend / \{EvSuspend2, EvSuspend3\} }

Similarly to TyApp / EvApp1 case.


\clearpage{}
% --------------------
\pCase{$t = \isuspend \ \delta \ t_1 \ t_2$ / TySuspend / EvSuspend}
$$
\qq 	\infer
	{ (\un{1}) \
	  \tyJudgeGS
		{\isuspend \ \ \un{\rpure \ \sigma} \ \ (\teLam{x}{\tau_{11}}{t_{12}}) \ \ v^\circ}
		{\tau_{12}}
		{\bot}
	}
	{	
		\begin{aligned}
			\\
			\infer
			{ (5) \
			  \tyJudgeGS
			  	{ \teLam{x}{\tau_{11}}{t_{12}} }
				{ \tau_{11} \funa{\sigma} \tau_{12} }
				{ \bot }
		   	}
			{ (9) \
			  \tyJudge
			  	{ \Gamma, \ x : \tau_{11} }
				{ \Sigma }
				{ t_{12} }
				{ \tau_{12} }
				{ \sigma }
			}
		\end{aligned}
		\qq
		\begin{aligned}
			(8) \ 
			\tau_{11} \sim_\Sigma \tau_{2}
			\\
			(7) \ 
			\tyJudgeGS
			  	{ v^\circ }
				{ \tau_2 }
				{ \bot }
			\\
			(6) \
			\kiJudgeGS
			  	{ \un{\rpure \ \sigma} }
				{ \iPure \ \sigma }
		\end{aligned}
	}
$$
$$
	(\un{2}) \
	\trEval
		{\heap}{\isuspend \ \ \un{\rpure \ \sigma} \ \ (\teLam{x}{\tau_{11}}{t_{12}}) \ \ v^\circ}
		{\heap}{t[v^\circ/x]}
$$
\begin{tabbing}
MM \= MMMM \= MMMMMMMMMMMMMMMMMMMMMM \= MMMMM  \kill
\>	$(\un{3}, \un{4})$
		\> $\Sigma \models \heap, \ \Sigma \vdash \heap$
		\> \pby{assume} 
\\[1ex]
\>	(10..12) \> $\tyJudgeGS{t_{12}[v^\circ/x]}{\tau_{12}'}{\sigma'},$ 
\\[0.2ex]
\>		\> $\qq \tau_{12}' \sim_\Sigma \tau_{12}, \ \sigma' \tle_\Sigma \sigma$
		\> \pby{Sub. Value/Value 9 7 8}
\\[1ex]
\>	(13) 	\> $\sigma \tle_\Sigma \bot$
		\> \pby{SubPurify 6}
\\[1ex]
\>	(14) 	\> $\sigma' \tle_\Sigma \bot$
		\> \pby{12 13}
\end{tabbing}

\end{flushleft}







