%!TEX root = ../Main.tex

% -----------------------------------------------------------------------------
\section{Processes and Machines}
\label{s:Processes}

A \emph{process} in our system is a simple imperative program with a local heap. A process pulls source values from an arbitrary number of input streams and pushes result values to at least one output stream. The process language is an intermediate representation we use when fusing the overall dataflow network. When describing the fusion transform we describe the control flow of the process as a state machine, hence Machine Fusion. 

A \emph{combinator} is a template for a process which parameterizes it over the particular input and output streams, as well as values of configuration parameters such as the worker function used in a @map@ process. Each process implements a logical \emph{operator} --- so we use ``operator'' when describing the values being computed, but ``process'' and ``machine'' when referring to the implementation. 


% -----------------------------------------------------------------------------
\subsection{Grouping}

The definition of the @group@ combinator which detects groups of successive identical elements in the input stream is given in Figure~\ref{fig:Process:Group}. The process emits the first value pulled from the stream and every value that is different from the last one that was pulled. For example, when executed on the input stream @[1,2,2,3]@, the process will produce the output @[1,2,3]@. We include the concrete representation and a diagram of the process when viewed as a state machine.

The @group@ combinator has two parameters, @sIn1@ and @sOut1@, which bind the input and output streams respectively. The \emph{nu-binders} \mbox{($\nu$ @(f: Bool) (l: Nat)@...)} indicate that each time the @group@ combinator is instantiated, fresh names must be given to @f@, @l@ and so on, that do not conflict with other instantiations. Overall, the @f@ variable tracks whether we are dealing with the first value from the stream, @l@ holds the last value pulled from the stream (or 0 if none have been read yet), and @v@ holds the current value pulled from the stream. 

The body of the combinator is a record that defines the process. The @ins@ field defines the set of input streams and the @outs@ field the set of output streams. The @heap@ field gives the initial values of each of the local variables. The @instrs@ field contains a set of labeled instructions that define the program, while the @label@ field gives the label of the initial instruction. In this form, the output stream is defined via a parameter, rather than being the result of the combinator, as in the representation of @uniquesUnion@ from \S\ref{s:Introduction}. 

The initial instruction @(pull sIn1 v A1 [])@ pulls the next element from the stream @sIn1@, writes it into the heap variable @v@ (value), then proceeds to the instruction at label @A1@. The empty list @[]@ after the target label @A1@ can be used to update heap variables, but as we do not need to update anything yet we leave it empty. 

Next, the instruction @(case (f || (l /= v)) A2 [] A3 [])@ checks whether predicate @(f || (l /= v))@ is true; if so it proceeds to the instruction at label @A2@, otherwise it proceeds to @A3@. We use the variable @l@ (last) to track the last value read from the stream, and the boolean @f@ (first) to track whether this is the first element.

% AR: Clumsy, but added "at L2 executes:" to force @(push...)@ onto its own line. Otherwise it goes off the right.
When the predicate is true, the instruction at label @A2@ executes
@(push sOut1 v A3 [ l = v, f = F ])@
which pushes the value @v@ to the output stream @sOut1@ and proceeds to the instruction at label @A3@, once the variable @l@ is set to @v@ and @f@ to @F@ (False).

Finally, the instruction @(drop sIn1 A0 [])@ signals that the current element that was pulled from stream @sIn1@ is no longer required, and goes back to the first instruction at @A0@. 


\begin{figure*}

\begin{minipage}{0.6\textwidth}
\begin{alltt}
 group 
   = \(\lambda\) (sIn1: Stream Nat) (sOut1: Stream Nat). 
     \(\nu\) (f: Bool) (l: Nat) (v: Nat) (A0..A3: Label).
\end{alltt}
\begin{code}
     process
      { ins:    { sIn1  }
      , outs:   { sOut1 }
      , heap:   { f = T, l = 0, v = 0 }
      , label:  A0
      , instrs: { A0 = pull sIn1 v          A1 []
                , A1 = case (f || (l /= v)) A2 []  A3 []
                , A2 = push sOut1 v         A3 [ l = v, f = F ]
                , A3 = drop sIn1            A0 [] } }
\end{code}
\end{minipage}
\begin{minipage}{0.29\textwidth}
\includegraphics[scale=1.0]{figures/state-group.pdf}
\end{minipage}

\caption{The group combinator}
\label{fig:Process:Group}
\end{figure*}


% -----------------------------------------------------------------------------
\subsection{Merging}
\begin{figure*}
\begin{alltt}
       merge
         = \(\lambda\) (sIn1: Stream Nat) (sIn2: Stream Nat) (sOut2: Stream Nat). 
           \(\nu\) (x1: Nat) (x2: Nat) (B0..E2: Label).
\end{alltt}
\begin{code}
           process
            { ins:    { sIn1, sIn2 }
            , outs:   { sOut2 }
            , heap:   { x1 = 0, x2 = 0 }
            , label:  B0
            , instrs: { B0 = pull sIn1  x1   B1 []             , B1 = pull sIn2  x2   C0 []
                      , C0 = case (x1 < x2)  D0 []  E0 []      , D0 = push sOut2 x1   D1 []
                      , D1 = drop sIn1       D2 []             , D2 = pull sIn1  x1   C0 []
                      , E0 = push sOut2 x2   E1 []             , E1 = drop sIn2       E2 []
                      , E2 = pull sIn2 x2    C0 [] } }
\end{code}
\includegraphics[scale=1.0]{figures/state-merge.pdf}
\caption{The merge combinator}
\label{fig:Process:Merge}
\end{figure*}

The definition of the @merge@ combinator, which merges two input streams, is given in Figure~\ref{fig:Process:Merge}. The combinator binds the two input streams to @sIn1@ and @sIn2@, while the output stream is @sOut2@. The two heap variables @x1@ and @x2@ store the last values read from each input stream. The process starts by pulling from each of the input streams. It then compares the two pulled values, and pushes the smaller of the values to the output stream. The process then drops the stream which yielded the the smaller value, then pulls from the same stream so that it can perform the comparison again.


% -----------------------------------------------------------------------------
\subsection{Fusion}

Our fusion algorithm takes two processes and produces a new one that computes the output of both. For example, suppose we need a single process that produces the output of the first two lines of our @uniquesUnion@ example back in \S\ref{s:Introduction}. The result will be a process that computes the result of both @group@ and @merge@ as if they were executed concurrently, where the first input stream of the @merge@ process is the same as the input stream of the @group@ process. In our informal description of the fusion algorithm we will instantiate the parameters of each combinator with arguments of the same names.

% -----------------------------------------------------------------------------
\subsubsection{Fusing Pulls}
\label{s:Fusion:FusingPulls}

The algorithm proceeds by considering pairs of states: one from each of the source process state machines to be fused. Both the @group@ machine and the @merge@ machine pull from the same stream as their initial instruction, so we have the situation shown in the top of Figure~\ref{fig:Fusion:Pulls}. The @group@ machine needs to transition from label @A0@ to label @A1@, and the @merge@ machine from @B0@ to @B1@. In the result machine we produce three new instructions that transition between four joint result states, @F0@ to @F3@.
Each of the joint result states represents a combination of two source states, one from each of the source machines. For example, the first result state @F0@ represents a combination of the @group@ machine being in its initial state @A0@ and the @merge@ machine being in its own initial state @B0@. 

We also associate each of the joint result states with a description of whether each source machine has already pulled a value from each of its input streams. For the @F0@ case at the top of Figure~\ref{fig:Fusion:Pulls} we have ((A0, \{sIn1 = none\}), (B0, \{sIn1 = none, sIn2 = none\})). The result state @F0@ represents a combination of the two source states @A0@ and @B0@. As both @A0@ and @B0@ are the initial states of their respective machines, those machines have not yet pulled any values from their two input streams, so both `sIn1' and `sIn2' map to `none'.

From the result state @F0@, both of the source machines then need to pull from stream @sIn1@, the @group@ machine storing the value in a variable @v@ and the @merge@ machine storing it in @x1@. In the result machine this is managed by first storing the pulled value in a fresh, shared buffer variable @b1@, and then using later instructions to copy the value into the original variables @v@ and @x1@. To perform the copies we attach updates to a @jump@ instruction, which otherwise transitions between states without affecting any of the input or output streams.

Finally, note that in the result states @F0@ through @F3@, the state of the input streams transitions from `none', to `pending' then to `have'. The `none' state means that we have not yet pulled a value from the associated stream. The `pending' state means we have pulled a value into the stream buffer variable (@b1@ in this case). The `have' state means that we have copied the pulled value from the stream buffer variable into the local variable used by each machine. 

% BL: Dropped this to fix pagination.
% In Figure~\ref{fig:Fusion:Pulls},  `sIn1' is set to `have' for the first machine in @F2@ after we have set `v = b1', while `sIn1' is set to `have' for the second machine in @F3@ after we have set `x1 = b1'. 


% -----------------------------------------------------------------------------
\subsubsection{Fusing Cases}
Once the result machine has arrived at the joint state @F3@, this is equivalent to the two source machines arriving in states @A1@ and @B1@ respectively. The lower half of Figure~\ref{fig:Fusion:Case} shows the next few transitions of the source machines. From state @A1@, the @group@ machine needs to perform a @case@ branch to determine whether to push the current value it has from its input stream @sIn1@ to output stream @sOut1@, or to just pull the next value from its input. From state @B1@, the @merge@ machine needs to pull a value from its second input stream @sIn2@. In the result machine, @F3@ performs the case analysis from @A1@, moving to either @F4@ or @F5@, corresponding to @A2@ and @A3@ respectively. From state @F4@, the push at @A2@ is executed and moves to @F5@, corresponding to @A3@. Finally, at @F5@ the @merge@ machine pulls from @sIn2@, moving from @F5@ to @F6@, corresponding to @B1@ and @C0@ respectively. As the stream @sIn2@ is only pulled from by the @merge@ machine, no coordination is required between the @merge@ and @group@ machines for this pull.


% -----------------------------------------------------------------------------
\subsection{Fused Result}

Figure~\ref{fig:Process:Fused} shows the final result of fusing @group@ and @merge@ together. There are similar rules for handling the other combinations of instructions, but we defer the details to \S\ref{s:Fusion}. The result process has two input streams, @sIn1@ and @sIn2@, and two output streams: @sOut1@ from @group@, and @sOut2@ from @merge@. The shared input @sIn1@ is pulled by @merge@ instructions at two places, and since both of these need to agree with when @group@ pulls, the @group@ instructions are duplicated at @F3@-@F5@ and @F13@-@F15@. The first set of instructions could be simplified by constant propagation to a single @push@, as @f@ is initially true.

To complete the implementation of our example from \S\ref{s:Introduction} we would now fuse this result process with a process from the final line of the example (also a @group@). Note that although the result process has a single shared heap, the heap bindings from each fused process are guaranteed not to interfere, as when we instantiate combinators to create source processes we introduce fresh names. 

% AR: Cut to make room for explanation why two copies of group pull
% The order in which pairs of processes are fused together does matter, as does the order in which instructions are interleaved --- we discuss both points further in \S\ref{s:FusionOrder} and \S\ref{s:Evaluation}.

% -----------------------------------------------------------------------------
\subsection{Breaking It Down}
We started with a pure functional program in \S\ref{s:Introduction}, reimagined it as a dataflow graph, then interleaved imperative code that implemented two of the operators in that dataflow graph. We needed to \emph{break down} the definition of each operator into imperative statements so that we could interleave their execution appropriately. We do this because the standard, single-threaded evaluation semantics of functional programs does not allow us to evaluate stream programs that contain both splits and joins in a space efficient way. Returning to the definition of @uniquesUnion@ from \S\ref{s:Introduction}, we cannot simply execute the @group@ operator on its entire input @sIn1@ before executing the @merge@ operator, as that would require us to buffer all data read from @sIn1@. Instead, during fusion we perform the job of a concurrent scheduler at compile time. In the result process the flow of control alternates between the instructions for both the @group@ and @merge@ operators, but as the instructions are interleaved directly there is no overhead due to context switching --- as there would be in a standard concurrent implementation using multiple threads.

The general approach of converting a pure functional program to a dataflow graph, then interleaving imperative statements that implement each operator was also used in prior work on Flow Fusion~\cite{lippmeier2013data}. However, in contrast to Flow Fusion and similar systems, with \mbox{Machine Fusion} we do not need to organize statements into a fixed \emph{loop anatomy} --- we simply merge them as they are. This allows us to implement a wider range of operators, including ones with nested loops that work on segmented streams.

Note that relying on lazy evaluation for @uniquesUnion@ does not eliminate the need for unbounded buffering. Suppose we converted each of the streams to lazy lists, and used definitions of @group@ and @merge@ that worked over these lists. As @uniquesUnion@ returns a pair of results, there is nothing preventing a consumer from demanding the first list (@sUnique@) in its entirety before demanding any of the elements from the second list (@sUnion@). If this were to happen then the runtime implementation would need to retain all elements of @sIn1@ before demanding any of @sIn2@, causing a space leak. Lazy evaluation is \emph{pull only} meaning that evaluation is driven by the consumer. The space efficiency of our fused program relies critically on the fact that processes can also \emph{push} their result values directly to their consumers, and that the consumers cannot defer the handling of these values.

% TODO: Blend this in.
% Note that we could construct the fused result machine in several ways. One option is to perform the case branch first and then pull from @sIn2@, another is to pull from @sIn2@ first and then perform the branch. By construction, the predicate used in the branch refers only to variables local to the @group@ machine, and the pull instruction from @B1@ stores its result in a variable local to the @merge@ machine. As the set of variables does not overlap, either ordering is correct. For this example we choose to perform the branch first, though will discuss the ramifications of this choice further in \S\ref{s:FusionOrder}. 

% TODO: Shift this later.
% As the @merge@ process merges infinite streams, if we execute it with a finite input prefix, it will arrive at an intermediate state that may not yet have pushed all available output. For example, if we execute the process with the input streams @[1,4]@ and @[2,3,100]@ then the values @[1,2,3,4]@ will be pushed to the output. After pushing the last value @4@, the process will block at instruction @E2@, waiting for the next value to become available from @sIn2@. We discuss how to handle finite streams later in ~\S\ref{s:Finite}.

% TODO: Talk about what drop is really for later.
% This @drop@ instruction is used to coordinate concurrent processes when performing fusion. The next element of a stream may only be pulled after all consumers of that stream have pulled and then and dropped the current element.

% -----------------------------------------------------------------------------
\begin{figure*}
\includegraphics[scale=1.1]{figures/fuse-pull-pull.pdf}

\vspace{2em}

\includegraphics[scale=1.1]{figures/fuse-case-pull.pdf}
\caption{Fusing pull (top) and case (bottom) instructions}
\label{fig:Fusion:Pulls}
\label{fig:Fusion:Case}
\end{figure*}

% BL: Merged these figures togather to save space.
% \begin{figure*}
% \caption{Fusing case instructions}
% \label{fig:Fusion:Case}
% \end{figure*}

%%% AR: would like to highlight which machine is performing the current instruction, eg bolding "A0" when it's group moving from A0 to A1
\begin{figure*}
\definecolor{groupc}{HTML}{308030}
\definecolor{mergec}{HTML}{800000}
\definecolor{sharec}{HTML}{000080}

\newcommand\annot[5]{
  \tiny ((#1,   \> \tiny \{sIn1 =      #2\}), 
                \> \tiny (#3, \> \tiny \{sIn1 =      #4, \> \tiny sIn2 =      #5\}))
}

\newcommand\icase[7]{
 \tt{#1} \> \tt{= #2} \> \tt{#3} \> \tt{[ #4 ]} \> \tt{#5} \> \tt{[ #6 ]} \> \tiny #1 \> \tiny = #7
}

\newcommand\instr[5]{
 \tt{#1}\>\tt{= #2} \> \tt{#3} \> \tt{[ #4 ]} \> \> \> \tiny #1 \> \tiny = #5
}

\newcommand\tctt[2]{\textcolor{#1}{\tt{#2}}}

\hspace{5em}

\begin{adjustbox}{minipage=0.8\textwidth,margin=0pt \smallskipamount,center}
\begin{alltt}
process
\string{ ins:    \string{ \tctt{sharec}{sIn1},  \tctt{mergec}{sIn2} \string}
, outs:   \string{ \tctt{groupc}{sOut1}, \tctt{mergec}{sOut2} \string}
, heap:   \string{ \tctt{groupc}{f = T, l = 0, v = 0}, \tctt{mergec}{x1 = 0, x2 = 0}, \tctt{sharec}{b1 = 0} \string}
, label:  \tctt{sharec}{F0}
\end{alltt}

\begin{tabbing}
@  @ \=
@, F17@  \= = @case (f || (l /= v)) @
         \= @F17@ \= @[ ]     @ \= @F17@ \= @[ ]@
@   @ \= \tiny @   @F17 \= \tiny = ((A0, \= \tiny \{sIn1 = pending\}), \= \tiny (B0, \= \tiny \{sIn1 = pending, \= \tiny sIn2 = pending\})) \kill

@, instrs:@

% I have no idea how this works, but you need this particular incantation to colour the whole line in a tabbing environment.
% It's not ideal that the { and , are coloured too, but that can't be helped.
\\[0pt \color{sharec}]
\> @{@
\instr{F0}{pull sIn1 b1}{F1}{}
      {\annot{A0}{none}{B0}{none}{none}}

\\[0pt \color{groupc}]
\> @,@
\instr{F1}{jump}{F2}{v  = b1}
      {\annot{A0}{pending}{B0}{pending}{none}}

\\[0pt \color{mergec}]
\> @,@
\instr{F2}{jump}{F3}{x1 = b1}
      {\annot{A1}{have}{B0}{pending}{none}}

\\[0pt \color{groupc}]
\> @,@
\icase{F3}{case (f || (l /= v))}{F4}{}{F5}{}
      {\annot{A1}{have}{B1}{have}{none}}

\\[0pt \color{groupc}]
\> @,@
\instr{F4}{push sOut1 v}{F5}{l = v, f = F}
      {\annot{A2}{have}{B1}{have}{none}}

\\[0pt \color{groupc}]
\> @,@
\instr{F5}{jump}{F6}{}
      {\annot{A3}{have}{B1}{have}{none}}

\\[0pt \color{mergec}]
\> @,@
\instr{F6}{pull sIn2 x2}{F7}{}
      {\annot{A0}{none}{B1}{have}{none}}

\\
\\[0pt \color{mergec}]
\> @,@
\icase{F7}{case (x1 < x2)}{F8}{}{F16}{}
      {\annot{A0}{none}{C0}{have}{have}}

\\
\\[0pt \color{mergec}]
\> @,@
\instr{F8}{push sOut2 x1}{F9}{}
      {\annot{A0}{none}{D0}{have}{have}}

\\[0pt \color{mergec}]
\> @,@
\instr{F9}{drop sIn1}{F10}{}
      {\annot{A0}{none}{D1}{none}{have}}

\\[0pt \color{sharec}]
\> @,@
\instr{F10}{pull sIn1 b1}{F11}{}
      {\annot{A0}{none}{D2}{none}{have}}

\\[0pt \color{groupc}]
\> @,@
\instr{F11}{jump}{F12}{v = b1}
      {\annot{A0}{pending}{D2}{pending}{have}}

\\[0pt \color{mergec}]
\> @,@
\instr{F12}{jump}{F13}{x1 = b1}
      {\annot{A1}{have}{D2}{pending}{have}}

\\[0pt \color{groupc}]
\> @,@
\icase{F13}{case (f || (l /= v))}{F14}{}{F15}{}
      {\annot{A1}{have}{C0}{have}{have}}

\\[0pt \color{groupc}]
\> @,@
\instr{F14}{push sOut1 v}{F15}{l = v, f = F}
      {\annot{A2}{have}{C0}{have}{have}}

\\[0pt \color{groupc}]
\> @,@
\instr{F15}{jump}{F7}{}
      {\annot{A3}{have}{C0}{have}{have}}

\\

\\[0pt \color{mergec}]
\> @,@
\instr{F16}{push sOut2 x2}{F17}{}
      {\annot{A0}{none}{E0}{have}{have}}

\\[0pt \color{mergec}]
\> @,@
\instr{F17}{drop sIn2}{F18}{}
      {\annot{A0}{none}{E1}{have}{have}}

\\[0pt \color{mergec}]
\> @,@
\instr{F18}{pull sIn2}{F7}{}
      {\annot{A0}{none}{E2}{have}{none}}


\\[0pt \color{black}]
@} }@
\end{tabbing}
\end{adjustbox}
\caption{Fusion of \textcolor{groupc}{group} and \textcolor{mergec}{merge}, along with \textcolor{sharec}{shared} instructions}
\label{fig:Process:Fused}
\end{figure*}
% -----------------------------------------------------------------------------

