%!TEX root = ../Main.tex


% Process
% ~~~~~~~   
%    Process  ::= process name (a : Kind)* (x : Type)* (s : Series k Type)*
%                    with  Operator*  yields  Exp

%    Operator ::= mkSel (k_inner : &) (x_sel : Sel k_outer k_inner)
%                     from k_outer s_flags  in  Operator*
%             |   s_out    <- map^n  k_in  Type*n  Exp_work  s_in*n
%             |   s_out    <- pack   k_out k_in  Type_in  x_sel  s_in
%             |   x_result <- fold   k_in  Type_in  Type_result  s_in
%                               with  Exp_work  and Exp_zero
%             |   x_vec    <- create k_in  Type_in  s_in


\clearpage{}
\section{Slurping}

\begin{alltt}
Core 
~~~~
   Prog    ::= Top*                                (Core program)
   Top     ::= let v1 : Type = Exp                 (top level binding)

   Exp     ::= x                                   (variable)
            |  \(\Lambda\)(a : Kind). Exp                    (type abstraction)
            |  \(\lambda\)(x : Type). Exp                    (value abstraction)
            |  Exp @Type                           (type application)
            |  Exp  Exp                            (value application)
            |  let Binding in Exp                  (local binding)   
                     
   Binding ::=  x : Type = Exp

Strip 
~~~~~
 !! STRIP   :: Exp -> ((Var, Kind)*, (Var, Type)*, (Var, Type)*,  Exp)
    Takes an expression with some outer-most lambda abstractions and strips
    off the type parameters, non-series value parameters, and series parameters,
    also returning the body of the inner-most abstraction.

Slurping
~~~~~~~~
 !!  SLURPP|- Top => Process

     STRIP[ X1 ] = (aks, xts, sts, X_body)   SLURPX|- nil | X_body => P_ops ; X_yield
   ----------------------------------------------------------------
     SLURPP|- let v_name : T1 = X1  
           => process  v_name aks xts sts  with  P_ops  yields  X_yield


 !!  SLURPX|- Binding* | Exp => Operator* ; Exp

     SLURPX|- BS | X2 => PS ; X0
   ------------------------------------------------------------------------------- (mkSel)
     SLURPX|- BS | mkSel @k1 @T1 s1 (\(\Lambda\)(k2 : &). \(\lambda\)(l1 : Sel k1 k2). X2) ]
           => mkSel (k2 : &) (l1 : Sel k1 k2) from  k1 s1 in PS            ; X0

     SLURPX|- BS | X2 => PS ; X0
   -------------------------------------------------------------------------------- (map)
     SLURPX|- BS | let s1 : Series k1 T2 = map  @k1 @T1 @T2 X1 s2     in X2
           =>                       (s1 <- map   k1  T1    X1 s2)    ++ PS ; X0

     SLURPX|- BS | X2 => PS ; X0
   -------------------------------------------------------------------------------- (pack)
     SLURPX|- BS | let s1 : Series k2 T2 = pack @k1 @k2 @T2 l1 s2    in X2
           =>                       (s1 <- pack  k1  k2  T2 l1 s2)   ++ PS ; X0

     SLURPX|- BS | X2 => PS ; X0
   -------------------------------------------------------------------------------- (create)
     SLURPX|- BS | let x1 : Vector T1    = create @k1 @T1 s1         in X2
           =>                       (x1 <= create k1 T1 s1)          ++ PS ; X0

     SLURPX|- BS | X2 => PS ; X0
   ------------------------------------------------------------------------------------ (fold)
     SLURPX|- BS | let x1 : T1 = fold @k1 @T1 @T2 X1 X2 s1 in X3
           =>             (x1 <= fold  k1  T1  T2 s1 with X1 and X2) ++ PS ; X0

     SLURPX|- BS ++ (x : T1 = X1) | X2 => PS ; X0      T1 /= Series k T  for any  k, T
   ------------------------------------------------------------------------------------ (let)
     SLURPX|- BS | let x : T1 = X1 in X2 => PS ; X0

   ------------------------------------------------------------------------------------ (var)
     SLURPX|- BS | x  => nil ; let BS in x
\end{alltt}
