
\documentclass[preprint]{sigplanconf}
\usepackage{pdf14}
\usepackage{graphicx}
\usepackage{style/code}
\usepackage{style/utils}

% % -----------------------------------------------------------------------------
\begin{document}
\title
{       Polarised Data Parallel Data Flow}
 
\authorinfo
{       Ben Lippmeier$^\alpha$ \and 
        Fil Mackay$^\beta$ \and
        Amos Robinson$^\gamma$
}
{ 
  \vspace{5pt}
  \shortstack{
    $^{\alpha,\beta}$Vertigo Technology \\
    \\[2pt]
    \textsf{\{benl,fil\}@vergo.co}
  }
  \shortstack{
    ~~~$^{\alpha,\gamma}$UNSW Australia \\
    \\[2pt]
    ~~~~~~ \textsf{\{benl,amosr\}@cse.unsw.edu.au}
  }
  \shortstack{
    $^\gamma$Ambiata \\
    \\[2pt]
    ~~~~~ \textsf{amos.robinson@ambiata.com}
  }
}

\maketitle
\makeatactive

% ---------------------------------------------------------------
\begin{abstract}
We present an approach to writing fused parallel data flow programs, where the library API guarantees that programs run in constant space. We support branching data flows which allow multiple independent aggregates to be computed over the same stream without hand fusing the aggregation code, as well as a Hadoop-like shuffle operator to exchange data between independent data parallel streams. Our approach is embodied in the Repa Flow Haskell library, and we provide production examples that demonstrate our technique produces code efficient enough to saturate the IO system of a current multicore machine.
\TODO{Highlight main contributions are polarisation and parallelism}
\end{abstract}


% -----------------------------------------------------------------------------
%!TEX root = ../Main.tex
\section{Introdution}

The Haskell library ecosystem is blessed with a multitude of libraries for writing streaming data flow programs. Stand out examples include iteratee CITE, enumerator CITE, conduit CITE and pipes CITE. These libraries are based around ... and more recent examples such as pipes provide a useful set of algebraic equivalences that give a clean mathematical structure to the provided mathemetical structure.

Libraries such as iteratee and enumerator are typically used to deal with data sets that do not fit in main memory, as the constant space guarantee ensures that the program will run to completion without suffering an out-of-memory error. However, current computing platforms use multi-core processors, the programming models provided by such streaming libraries do not also provide a notion of \emph{parallelism} to help deal with the implied amount of data. They also lack support for branching data flows where produced streams can be consumed by several consumers without the programmer needing to had fuse them.

We provide several techniques that increase the scope of programs that can be written in such libraries. Our target applications concern \emph{medium data}, meaning data that is large enough that it does not fit in the main memory of a normal desktop machine, but not so large that we require a cluster of multiple physical machines. For a lesser amount of data one could simply load the data into main memory and use an in-memory array library such as CITE or CITE. For greater data one needs to turn to a distributed system such as Hadoop or Spark and deal with the unreliable network and lack of shared memory. Repa Flow targets the sweet middle ground.

We make the following contributions:

\begin{itemize}
\item Our parallel data flows consist of a bundle of streams, where each stream can process a separate partition of a data set on a separate processor core.

\item Our API uses polarised flow endpoints (@Sources@ and @Sinks@) to ensure that programs run in constant space. We demonstrate how this standard technique can be extended to branching data flows, where produced flows are consumed by multiple consumers.

\item The data processed by our streams is chunked so that each operation processes several elements at a time. We show how to design the core API in a generic fashion so that chunk-at-a-time operators can interoperate smoothly with element-at-a-time operators.
\TODO{We don't support leftovers}

\item We show how to use Continuation Passing Style to provoke the Glasgow Haskell Compiler into applying stream fusion across chunks processed by independent flow operators. For example, the map-map fusion on flows arises naturally from map-map fusion rule on chunks (arrays) of elements.
\end{itemize}

Our work is embodied in Repa Flow, which is available on Hackage. \TODO{Specify the relationship to previous work on Repa}. This is a new layer on the original delayed arrays of our original Repa library.


%!TEX root = ../Main.tex

\clearpage
\section{Streams and Flows}

Our library is based around polarised streams. A stream is the usual list like structure which allows the next element to be taken from the front, but does not support random access of other elements. We imagine a stream as an array where the indexing dimension is time --- as each element is read it exists only for a moment, then is gone. In the library we manipulate stream endpoints rather than the streams themselves. The endpoints are polarised, meaning that we push (write) data in to stream sinks, but pull (read) data from stream sources. We name a \emph{bundle} several related streams a \emph{flow}. Typically, a flow consists of streams that carry separate partitions of a single large data set. Fig \ref{f:GenericFlows} shows the types we use to represent the the bundle of stream endpoints for an overall flow.

In the type @Sources i m e@, the @i@ parameter stands for type that indexes the individual streams, @m@ is a monadic constructor that sets the computational fabric, and @e@ is the type of elements pushed to or pulled from the endpoints. The paramters of the @Sinks@ constructor are similar. 

Mention conduit and pipes, manpiulates whole streams.

\begin{figure}
\begin{code}
data Sources i m e 
   = Sources { arity :: i
             , pull  :: i -> (e -> m ()) -> m () 
                                         -> m () }
data Sinks   i m e 
   = Sinks   { arity :: i
             , push  :: i -> e -> m ()
             , eject :: i -> m () }
\end{code}
\label{f:GenericFlows}
\caption{Generic Flow definitions and conversions}
\end{figure}

\begin{figure}
\begin{code}
zipWith_ii :: Monad m => (a -> b -> c)
           -> Sources i m a -> Sources i m b -> m (Sources i m c)
zipWith_ii f (Sources nA pullA) (Sources nB pullB)
 = return \$ Sources (min nA nB) pullC
 where  pullC i eatC ejectC
         = pullA i eatA ejectC
         where  eatA xA = pullB i eatB ejectC
                 where  eatB xB = eatC (f xA xB)

zipWith_io :: (Ord i, Monad m) => (a -> b -> c)
           -> Sinks i m c -> Sources i m a -> m (Sinks i m b)
zipWith_io f (Sinks nC pushC ejectC) (Sources nA pullA)
 = return \$ Sinks nB pushB ejectC
 where  nB = min nC nA
        pushB i xB 
         | i > nB       = return ()
         | otherwise    = pullA i eatA (ejectC i)
         where  eatA xA = pushC i (f xA xB)
\end{code}
\caption{Continuation style implementation of zipWith functions}
\end{figure}

\begin
{figure*}
\begin{code}
-- Conversion
fromList   :: i -> [a] -> m (Sources i m a)
toList1    :: i -> Sources i m a -> m [a]

fromLists  :: [[a]] -> m (Sources Int m a)
toLists    :: Sources Int m a -> m [[a]]

-- Computation
drainS     :: Sources i   m  a -> Sinks i   m  a -> m  ()
drainP     :: Sources Int IO a -> Sinks Int IO a -> IO ()

-- Mapping
map_i      :: (a -> b) -> Sources i m a  -> m (Sources i m b)
map_o      :: (b -> a) -> Sinks   i m b  -> m (Sinks   i m a)

zipWith_ii ::  (a -> b -> c) 
                  -> Sources i m a -> Sources i m b -> m (Sources i m c)
zipWith_io :: ... -> Sources i m a -> Sinks   i m c -> m (Sinks   i m b)
zipWith_oi :: ... -> Sinks   i m a -> Sources i m b -> m (Sinks   i m a)

-- Connection
dup_oo     ::        Sinks   i m a -> Sinks   i m a -> m (Sinks   i m a)
dup_io     ::        Sources i m a -> Sinks   i m a -> m (Sinks   i m a)
dup_oi     ::        Sinks   i m a -> Sources i m a -> m (Sinks   i m a)

connect_i  ::        Sources i m a -> m (Sources i m a, Sources i m a)

-- Projection
project_i  :: i ->   Sources i m a -> m (Sources () m a)
project_o  :: i ->   Sinks   i m a -> m (Sinks   () m a)

-- Funneling
funnel_i   ::        Sources i m a -> m (Sources () m a)
funnel_o   ::        Sinks  () m a -> m (Sinks   i  m a)

-- Elided constraints: (Monad m, States i m) => ...
\end{code}
\caption{Generic Flow operators}
\end{figure*}



%!TEX root = ../Main.tex
\clearpage{}
\section{Chunked streams}

\begin{itemize}
\item Chunked streams to reduce overhead.
\end{itemize}

% %!TEX root = ../Main.tex
\clearpage{}
\section{Every day I'm shufflin'}


\begin{figure}
\begin{code}
deal_o       :: (Int -> a -> m ()) 
             -> Sinks Int m a         -> m (Sinks () m (Array a))

distribute_o :: (Int -> Array a -> m ())
             -> Sinks Int m (Array a) -> m (Sinks () m (Array (Int, a)))

shuffleP     :: Sources Int IO (Array (Int, a))
             -> Sinks   Int IO (Array a)
             -> IO ()
\end{code}
\caption{Deal, Distribute and Shuffle}
\end{figure}



\begin{itemize}
\item Also Sieve function, like shuffle but to disk.
      This is our main synchronisation barrier.
\item Then fromTable :: Sources Int IO ...
\end{itemize}

%!TEX root = ../Main.tex
\clearpage{}
\section{Benchmarks}

\begin{figure}
\begin{code}

\end{code}
\caption{Analytic Operators}
\end{figure}

%!TEX root = ../Main.tex
\section{Related Work}

\begin{itemize}
\item Iteratee Enumerator
\item Conduit, we don't have await and yield. Flows are not monad transformers.
\item Pipes
\item Machines support multiple inputs and fanout.
\item Monad par, is event flow network using IVars.
\item Impala.
\item Hive.
\item Spark.
\item Google Tensorflow
\item Lustre, Lucid sync data flow, Kahn networks.
\item StreamIt, Brook.
\item Scala streams library.
\item FRP libraries, eg reflex.
\end{itemize}


% Repa flow fills a sweet spot between the roles functional array library and analytic database. In terms of the programming model, a key feature of Repa Flow compared with Scoobi and Scalding is that the API carefully distinguishes between operators that run in constant space and those which do not. Systems based on map-reduce make implicit use of a \emph{shuffle} operator that distributes data between the compute nodes. The \emph{shuffle} operator sends data between the nodes in a data-dependent way, which can result in a skewed workload where most data is sent to a subset of nodes while the others are starved. When all source-level queries are converted into map-reduce jobs then there is no systematic way in which skew can be avoided. Taking inspiration from the work on synchronous data flow and Khan networks, we have arranged our API so that most operators execute without buffering. With Repa Flow it is easy to write programs where both buffering and data skew are avoided by construction, or admitted only in a controlled way.



% -----------------------------------------------------------------------------
% \paragraph{Acknowledgements}
% Thanks to peeps.
% 
% \bibliographystyle{plain}
% \bibliography{Main}


\end{document}
