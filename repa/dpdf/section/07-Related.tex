%!TEX root = ../Main.tex
\section{Related Work}
% The idea of expressing data processing as a data flow graph is as old as computer science itself. The main contributions of this paper are 1) the observation that if we are careful about the polarity of our operators we can avoid introducing intermediate buffering; 2) the simple types of @Sources@ and @Sinks@ that enable data parallelism using @drainP@, and 3) the observation that using continuation passing style to define the operators, combined with the lack of buffering enables intra-chunk fusion. 

The idea that parallelism a data flow graph can be introduced via a single operator is well known in the databases community. The Volcano~\cite{Graefe:Volcano} parallel database was the first to introduce an @exchange@ operator into its query plans, which forks a child thread for the producer of some data, leaving the master thread as the consumer. The implementation of @exchange@ also introduces buffering and uses back pressure to manage the possible mismatch between the rates at which data is produced and consumed. In Repa Flow we use @drainP@ to introduce parallelism, and @drainP@ itself introduces no extra buffering. In Volcano and other database systems, communication between operators is performed with a uniform @open@, @next@, @close@ interface, similar to a streaming file API. In Repa Flow the API between operators consists of the @Sources@ and @Sinks@ type, where the next element in a given stream can be uniformly acquired via the @pull@ function.

The main difference between Repa Flow and Iteratee based Haskell libraries 
\cite{Kiselyov:iteratee, hackage:enumerator, hackage:conduit, hackage:pipes} is that Repa Flow uses the separate @Sources@ and @Sinks@ types to express the \emph{endpoints} of flows, whereas Iteratees are based around a type that represent the flows themselves. The flow type is given a monadic interface, so an @Iteratee@ is better thought of as a \emph{computation} rather than a concrete data structure. The advantage of the @Iteratee@ approach is that satisfying algebraic identities arise between iteratee computations. The disadvantage is that consuming data from two separate sources is awkward because each source is represented by its own monadic computation, and these computations must be layered using monad transformers. Repa Flow lacks the convenience of a uniform monadic interface, though writing programs that pull and push to many sources and sinks is straightforward.

Recently, Bernardy and Svenningsson describes a library~\cite{Bernardy:Duality} that defines streams with sources and sinks, where each is defined as if it were the logical negation of the other. They also define co-sources and co-sinks, where a co-source is a sink that accepts element consumers and a co-sink is a source that produces element consumers. In related work Bernardy~\emph{et al} describe a core calculus \cite{Bernardy:Composable} based on Linear Logic which guarantees fusion does not increase the cost of program execution. The system is based fundamentally around linear logic rather than lambda calculus, with evaluation being driven by cut elimiation rather than function application. They describe a compiler targeting C and some encouraging benchmark results.


% In future work would be interesting to implement a system to take a data flow graph without a polarity assignment, and assign polarities to the combinators such that the entire graph can be executed without buffering (if possible). However, so far the largest programs we have written using the library have included only 10-15 combinators in a single function, and so far assigning the polarities has not been a burden. 

% In practice there is usually a natural distinction between the \emph{input network} written using sources and the \emph{output network} using sinks, and hybrid combinators such as @dup_iio@ are used infrequently.

% \begin{itemize}
% \item Machines~\cite{hackage:machines} support multiple inputs and fanout.
% \item FRP libraries, eg reflex.
% \item Kahn networks.
% \item StreamIt, Brook.

% \item Iteratee~\cite{Kiselyov:iteratee}
% \item Enumerator~\cite{hackage:enumerator}
% \item Conduit~\cite{hackage:conduit}, we don't have await and yield. Flows are not monad transformers.
% \item Pipes~\cite{hackage:pipes}
% \item Monad par, is event flow network using IVars.
% \item Spark~\cite{Zaharia:RDDs}.
% \item Google Tensorflow
% \item Lucy~\cite{Mandel:Lucy}.
% \item On the polarity of pipelines~\cite{Kay:YouPull}.
% \item Scalaz streams library~\cite{github:scalaz-streams}
% \item Obsidian~\cite{Claessen:ExpressiveArray}.
% \item Volcano DBMS one of the first to use abstract producers and iterator pattern. Its exchange operator is similar to our drainP.~\cite{Graefe:Volcano}
% \end{itemize}


% Repa flow fills a sweet spot between the roles functional array library and analytic database. In terms of the programming model, a key feature of Repa Flow compared with Scoobi and Scalding is that the API carefully distinguishes between operators that run in constant space and those which do not. Systems based on map-reduce make implicit use of a \emph{shuffle} operator that distributes data between the compute nodes. The \emph{shuffle} operator sends data between the nodes in a data-dependent way, which can result in a skewed workload where most data is sent to a subset of nodes while the others are starved. When all source-level queries are converted into map-reduce jobs then there is no systematic way in which skew can be avoided. Taking inspiration from the work on synchronous data flow and Khan networks, we have arranged our API so that most operators execute without buffering. With Repa Flow it is easy to write programs where both buffering and data skew are avoided by construction, or admitted only in a controlled way.
